\documentclass[letterpaper,11pt,nointlimits,reqno]{amsart}

% Packages
\usepackage{accents}
\usepackage{algorithm}
\usepackage{algorithmic}
\usepackage{amsfonts}
\usepackage{amsmath}
\usepackage{amssymb}
\usepackage{cancel}
\usepackage{enumerate}
\usepackage{fancyhdr}
\usepackage{fullpage}
\usepackage{ifthen}
\usepackage{lastpage}
\usepackage{latexsym}
\usepackage{mathrsfs}
\usepackage{mathtools}
\usepackage[numbers,sort&compress]{natbib}
\usepackage{parskip}
\usepackage{pstricks}
\usepackage{setspace}
\usepackage{txfonts}
\usepackage{units}
\usepackage{varioref}
\usepackage{wrapfig}

% Hyperref package must be last otherwise the contents are jumbled
% hypertexnames disabled to fix links pointing to incorrect locations
\usepackage[hypertexnames=false,final]{hyperref}

\mathtoolsset{showonlyrefs,showmanualtags}
\allowdisplaybreaks[1] % Allow grouped equations to be split across pages

% Line Spacing
\singlespacing

% Set appropriate header/footer information on each page
\fancypagestyle{plain}{
    \fancyhf{}
    \renewcommand{\headheight}{2.0em}
    \renewcommand{\headsep}{0.75em}
    \renewcommand{\headrulewidth}{1.0pt}
    \renewcommand{\footrulewidth}{0pt}
    \lhead{
        Nondimensional, perfect gas, isothermal supersonic flat plate treatment
    }
    \rhead{
        Page \thepage{} of \pageref{LastPage}
    }
}
\pagestyle{plain}

% Document-specific commands
\newcommand{\Mach}[1][]{\mbox{Ma}_{#1}}
\newcommand{\Reynolds}[1][]{\mbox{Re}_{#1}}
\newcommand{\Cov}[2]{\ensuremath{\operatorname{Cov}\left[{#1},{#2}\right]}}
\newcommand{\Var}[1]{\ensuremath{\operatorname{Var}\left[{#1}\right]}}

\begin{document}

This document describes the treatment of the isothermal freestream boundary for
a compressible, nondimensional flat plate problem computed using Suzerain.  All
variables are nondimensional.

\section{Ideal scenario definition}

The scenario has a mean freestream characterized by a Mach number
$\Mach[\infty]$ and Reynolds number $\Reynolds[\infty]$:
\begin{align}
  \label{eq:def_mach}
  \Mach[\infty]{}
  &= \Mach{} \overline{\,\frac{u}{a}\,}
   = \Mach{} \overline{\,\frac{u}{\sqrt{T}}\,}
   = \Mach\, \overline{u T^{-1/2}}
\\
  \label{eq:def_reynolds}
  \Reynolds[\infty]{}
  &= \Reynolds{} \overline{\,\frac{\rho u \delta\,} {\mu}}
   = \Reynolds{} \overline{\,\frac{\rho u \delta\,} {T^\beta}}
   = \Reynolds\, \overline{\rho u \delta T^{-\beta}}
\end{align}
Here, $\Mach$ and $\Reynolds$ are Suzerain code parameters and $\delta$ is the
boundary layer thickness.  The constitutive laws
\begin{align}
  a &= \sqrt{T}
&\mu &= {T}^\beta
\end{align}
have been used to re-express the sound speed and viscosity using temperature
because we desire
\begin{align}
  \label{eq:ideal_T}
  \bar{T} &= T_\infty
\end{align}
for some given constant $T_\infty$.  Defining and substituting
\begin{align}
  \Mach[\infty]{} &= \Mach
& \Reynolds[\infty]{} &= \Reynolds
& \delta &= 1
\end{align}
produces two statements about mean freestream nondimensional state implied by
the scenario definition:
\begin{align}
  \label{eq:ideal_u}
  1 &= \overline{u T^{-1/2}}_\infty
\\
  \label{eq:ideal_rhou}
  1 &= \overline{\rho u T^{-\beta}}_\infty
\end{align}
Ideally, one would implement an isothermal freestream by strongly enforcing
\eqref{eq:ideal_T}, \eqref{eq:ideal_u}, and~\eqref{eq:ideal_rhou}.
Unfortunately, exactly achieving these three constraints is difficult as the
spectral nature of Suzerain and its state choices make constraining only
$\bar{\rho}_\infty$, $\overline{\rho{}u}_\infty$, and
$\overline{\rho{}E}_\infty$ simple.

\section{Approximate scenario definition}

Both \eqref{eq:ideal_u} and \eqref{eq:ideal_rhou} possess the form $\overline{c
T^p}$ for spatially-varying $c$ and constant $p$.  A second-order expansion
about $\bar{c}$ and $\bar{T}$ yields
\begin{align}
  c T^p &= \bar{c} \bar{T}^p
         + \left(c - \bar{c}\right) \bar{T}^p
         + \left(T - \bar{T}\right) \bar{c} p \bar{T}^{p-1}
\\      &+ \frac{1}{2} p \bar{T}^{p-1} \left(c - \bar{c}\right)\left(T - \bar{T}\right)
         + \frac{1}{2} \bar{c} p \left(p-1\right) \bar{T}^{p-2} \left(T - \bar{T}\right)^2
         + \dots
\end{align}
where the $\mathcal{O}\left[c - \bar{c}\right]^2$ term does not appear because it
has leading coefficient zero.  Averaging generates
\begin{align}
  \overline{c T^p} &= \bar{c} \bar{T}^p
                    + \frac{1}{2} p \bar{T}^{p-1} \Cov{c}{T}
                    + \frac{1}{2} \bar{c} p \left(p-1\right) \bar{T}^{p-2} \Var{T}
                    + \dots
.
\end{align}
Therefore, making the approximation
\begin{align}
  \label{eq:damage}
  \overline{c T^p} &\approx \bar{c} \bar{T}^p
\end{align}
does not require assuming $\Var{c} \approx 0$ but it does force neglecting
$\Cov{c}{T}$, $\Var{T}$, and all higher moments.

We apply \eqref{eq:damage} followed by~\eqref{eq:ideal_T} to conditions
\eqref{eq:ideal_u} and \eqref{eq:ideal_rhou} to obtain two approximate
relationships:
\begin{align}
  \label{eq:approx_u}
  1 &\approx \bar{u}_\infty T_\infty^{-1/2}
  & &\implies
  & \bar{u}_\infty &\approx \sqrt{T_\infty}
\\
  \label{eq:approx_rhou}
  1 &\approx \overline{\rho u}_\infty T_\infty^{-\beta}
  & &\implies
  & \overline{\rho u}_\infty &\approx T_\infty^\beta
\end{align}
These statements are not unreasonable as $\Cov{u}{T}$, $\Cov{\rho u}{T}$, and
$\Var{T}$ should be small at the freestream.

Because~\eqref{eq:approx_u} constrains something other than mean conserved
state, it is difficult to implement.  Statements $\eqref{eq:approx_u}$ and
$\eqref{eq:approx_rhou}$ may be combined into an approximate constraint for
$\bar{\rho}_\infty$:
\begin{equation}
  \label{eq:approx_rho}
  \bar{\rho}_\infty
  =
  \frac{\bar{\rho}_\infty \bar{u}_\infty}
       {\bar{u}_\infty}
  \approx
  \frac{\bar{\rho}_\infty \bar{u}_\infty + \Cov{\rho_\infty}{u_\infty}}
       {\bar{u}_\infty}
  =
  \frac{\overline{\rho u}_\infty}
       {\bar{u}_\infty}
  \approx
  \frac{T_\infty^{-1/2}}{T_\infty^{-\beta}}
  =
  T_\infty^{\beta-1/2}
\end{equation}
Notice the useful identity
\begin{equation}
  \label{eq:covxy}
  \overline{xy} = \overline{xy} - \bar{x} \bar{y} + \bar{x} \bar{y}
                = \Cov{x}{y} + \bar{x} \bar{y}
\end{equation}
was employed and $\Cov{\rho_\infty}{u_\infty}$ neglected to produce this
result.

\section{Isothermal total energy constraint}

It remains to constrain the total energy $\overline{\rho E}_\infty$ to achieve
$\bar{T}=T_\infty$.

First, consider the mean specific total energy.  It always holds that
\begin{align}
  \bar{E}
  &= \frac{\bar{T}}{\gamma\left( \gamma-1 \right)}
   + \frac{\Mach^2}{2}\left(\overline{u^2}+\overline{v^2}+\overline{w^2}\right)
\\
  &= \frac{\bar{T}}{\gamma\left( \gamma-1 \right)}
   + \frac{\Mach^2}{2}\left(\bar{u}^2+\bar{v}^2+\bar{w}^2
                            +\Var{u} +\Var{v} +\Var{w}\right)
\\
  &= \frac{\bar{T}}{\gamma\left( \gamma-1 \right)}
   + \frac{\Mach^2}{2}\left(\bar{u}^2+\bar{v}^2+\bar{w}^2+2 k\right)
\end{align}
where $k$ denotes the turbulent kinetic energy.  In a non-turbulent freestream
$k_\infty\approx{}0$ suggests
\begin{align}
  \bar{E}_\infty
  &\approx
     \frac{T_\infty}{\gamma\left(\gamma-1 \right)}
   + \frac{\Mach^2}{2}\left(\bar{u}_\infty^2+\bar{v}_\infty^2+\bar{w}_\infty^2\right)
\\
  \label{eq:approx_E}
  &\approx
     \frac{T_\infty}{\gamma\left(\gamma-1 \right)}
   + \frac{\Mach^2}{2}\left(T_\infty+\bar{v}^2+\bar{w}^2\right)
\end{align}
where the second line incorporates \eqref{eq:approx_u}.

Now, using $\eqref{eq:covxy}$, neglecting $\Cov{\rho_\infty}{E_\infty}$, and
applying \eqref{eq:approx_rho} then \eqref{eq:approx_E} shows
\begin{align}
  \overline{\rho E}_\infty
  &=
  \bar{\rho}_\infty \bar{E}_\infty + \Cov{\rho_\infty}{E_\infty}
\\
  &\approx \bar{\rho}_\infty \bar{E}_\infty
\\
  &\approx T_\infty^{\beta-1/2} \bar{E}_\infty
\\
  \label{eq:approx_rhoE}
  &=
  T_\infty^{\beta-1/2} \left(
      \frac{T_\infty}{\gamma\left( \gamma-1 \right)}
    + \frac{\Mach^2}{2}\left(T_\infty+\bar{v}^2_\infty+\bar{w}^2_\infty\right)
  \right)
.
\end{align}
This last result is the desired approximate freestream total energy constraint.

\section{Comments on the isothermal total energy constraint}

Declaring $\Cov{\rho_\infty}{E_\infty} \approx 0$ may seem heavy-handed.  A
longer, equivalent way to reach~\eqref{eq:approx_rhoE} begins
\begin{align}
  \overline{\rho E}
&=
    \frac{\bar{p}}{\gamma-1}
  + \frac{\Mach^2}{2}\left(
        \overline{\rho u^2}
      + \overline{\rho v^2}
      + \overline{\rho w^2}
    \right)
\\
&=
    \frac{\overline{\rho T}}{\gamma\left(\gamma-1\right)}
  + \frac{\Mach^2}{2}\left(
        \overline{\rho u^2}
      + \overline{\rho v^2}
      + \overline{\rho w^2}
    \right)
\\
&=
    \frac{\bar{\rho}\bar{T} + \Cov{\rho}{T}}
         {\gamma\left(\gamma-1\right)}
  + \frac{\Mach^2}{2}\left(
        \bar{\rho} \overline{u^2}
      + \bar{\rho} \overline{v^2}
      + \bar{\rho} \overline{w^2}
      + \Cov{\rho}{u^2}
      + \Cov{\rho}{v^2}
      + \Cov{\rho}{w^2}
    \right)
\\
&=
    \frac{\bar{\rho}\bar{T} + \Cov{\rho}{T}}
         {\gamma\left(\gamma-1\right)}
  + \frac{\Mach^2}{2}\left(
        \bar{\rho} \bar{u}^2
      + \bar{\rho} \bar{v}^2
      + \bar{\rho} \bar{w}^2
      + \Cov{\rho}{u^2}
      + \Cov{\rho}{v^2}
      + \Cov{\rho}{w^2}
      + 2 k
    \right)
.
\end{align}
Assuming $k_\infty \approx 0$ as before shows that declaring
$\Cov{\rho_\infty}{E_\infty} \approx 0$ is equivalent to neglecting all of
$\Cov{\rho_\infty}{T_\infty}$, $\Cov{\rho_\infty}{u^2_\infty}$, etc.

If required by appreciable differences between target and observed freestream
mean temperature, dynamically gathered $\bar{v}_\infty$ and $\bar{w}_\infty$
could be used within \eqref{eq:approx_rhoE} at the risk of introducing
artificial, numerically-dependent feedback.

Lastly, from the definition of $\Mach[\infty]{}^2$ at the freestream, similarly
to \eqref{eq:def_mach} one may incur \eqref{eq:damage} to deduce
\begin{align}
  \label{eq:def_mach2}
  1 &= \overline{u^2 T^{-1}}_\infty
  &
  &\implies
  &
  \overline{u^2}_\infty &\approx T_\infty
  .
\end{align}
Therefore assuming only $\Var{v} \approx \Var{w} \approx 0$ in lieu of the
stronger $2 k_\infty = \Var{u} + \Var{v} + \Var{w} \approx 0$ is possible
in the derivation of \eqref{eq:approx_rhoE}.

\section{Summary}

In summary, given $T_\infty$, $\Mach[\infty]{}$, and $\Reynolds[\infty]{}$ and
setting $\delta=1$, an approximately isothermal-in-the-mean freestream
condition may be implemented via nothing but the constraints
\eqref{eq:approx_rho}, \eqref{eq:approx_rhou}, and~\eqref{eq:approx_rhoE}.  As
many assumptions were required to arrive at these results, the quality of the
constraints in achieving the desired conditions should be monitored.

\end{document}
