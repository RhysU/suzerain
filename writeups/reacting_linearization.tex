\documentclass[10pt]{article}

% useful packages
\usepackage{amssymb, amsmath, amsfonts, amsthm, wasysym} % math
\usepackage{epsfig, subfigure} % graphics
\usepackage{fullpage, url, color, parskip} % misc

% commands I like
\newcommand{\mbb}[1]{\mathbb{#1}}
\newcommand{\mbf}[1]{\mathbf{#1}}
\newcommand{\sbf}[1]{\boldsymbol{#1}}
\newcommand{\mcal}[1]{\mathcal{#1}}
\newcommand{\mfk}[1]{\mathfrak{#1}}
\newcommand{\pp}[2]{\frac{\partial #1}{\partial #2}}
\newcommand{\dd}[2]{\frac{d #1}{d #2}}
\newcommand{\rarrow}{\rightarrow}
\newcommand{\Rarrow}{\Rightarrow}
\newcommand{\LRarrow}{\Leftrightarrow}
\newcommand{\jump}[1]{\llbracket #1 \rrbracket}
\newcommand{\avg}[1]{\{ #1 \}}
\def\etal{{\it et al.~}}
\newcommand{\vvvert}{|\kern-1pt|\kern-1pt|}
\newcommand{\enorm}[1]{\vvvert #1 \vvvert}
\newcommand{\ud}{\,\mathrm{d}}
\newcommand{\pdv}[2]{{\frac{\partial{#1}}{\partial{#2}}}}
\newcommand{\sa}{\nu_{\mathrm{sa}}}

%\newcommand{\red}[1]{\color{red}{#1}}
\newcommand{\myred}[1]{{\color{red} #1}}

% title and author
\title{Linearization for Reacting Equations of State}
\author{Nicholas Malaya \\
        Todd Oliver \\
        Rhys Ulerich \\ }

% \thanks{Institute for Computational
% Engineering and Sciences, The University of Texas at Austin,
% Austin, TX 78712 (nick@ices.utexas.edu)}}


% lets rock and roll
\begin{document}
\maketitle

%-------------------------------------------------
\section{Problem Statement}

We are attempting to linearize our reacting flow equations around some
reference state.  That is, decomposing
\begin{equation}
\partial_t u = R(u) = \left(R(u) - Lu\right) + Lu
\end{equation}
so that the linear operator $L$ consists of only reference coefficients times
differential operators.  For instance, supposing $R(\rho, m) = \Delta m / \rho$
we would construct $L$ per
\begin{equation}
 \frac{1}{\rho}\Delta m = (\frac{1}{\rho}-\frac{1}{\rho_0})\Delta m +
  \frac{1}{\rho_0}\Delta m
\end{equation}
We will perform this procedure on the Navier-Stokes equations. Conservation of
Mass and Momentum will be identical to the current procedure used in Suzerain.
\myred{The momentum equation results from \texttt{perfectgas.tex}
\textsection{}3.7 will need to be converted back to dimensional form per
\textsection{}1.5.3, and you'll probably want to express results using only the
thermodynamic properties listed in the \texttt{cantera\_interface.hpp}} The
species mass fraction equations and the energy equation will need to be
modified, however.

Species Equations:
\begin{equation}
 \frac{\partial}{\partial t} \rho_s + \frac{\partial}{\partial
  x_i}\left(\rho_s u_i\right) = \frac{\partial}{\partial x_i}\left(\rho D_s
  \frac{\partial c_s}{\partial x_i} \right) + \dot \omega_s
\end{equation}

Energy Equation:
\begin{equation}
\frac{\partial}{\partial t} \rho E + \frac{\partial }{\partial x_j}\left(\rho
u_j H\right) = \frac{\partial }{\partial x_i}\left(\tau_{ij}u_j\right) - \frac{\partial q_j
}{\partial x_j} + \frac{\partial }{\partial x_i}\left(\rho \sum^{N_s}_{s=1}
h_s D_s \frac{\partial c_s}{\partial x_i} \right)
\end{equation}

\myred{Might mention the role of the dilluter and why various species-related
summations have varying limits}.

\section{Species Equations}
Let us begin with the species equations. We start with the convective
term, $\nabla \cdot (\rho_s u)$, and replace $u$ with $\frac{m}{\rho}$
\begin{equation}
  \nabla \cdot (\rho_s u) = u \cdot \nabla \rho_s + \rho_s \cdot \nabla
   u 
\end{equation}
\newline
\newline
The only other term we need consider here is 
the diffusivity. This term, in Gibbs notation, is
\begin{equation}
  \nabla \cdot (\rho_s D_s \nabla c_s)
\end{equation}
Note that $c_s$, the mass fractions, are defined as:
$\frac{\rho_s}{\rho} = 
c_s$.  It can therefore be shown that,
\begin{equation}
  \nabla c_s = \nabla (\frac{\rho_s}{\rho}) = \nabla (\rho_s \rho^{-1})
             = \rho^{-1} \nabla \rho_s - \rho^{-2} \rho_s \nabla \rho
\end{equation}
In other words,
\begin{equation}
  \label{eq:speciesdiffexpansion}
  \nabla \cdot (\rho_s D_s \nabla c_s) = \nabla \cdot \rho D_s (\rho^{-1} \nabla \rho_s - \rho^{-2} \rho_s \nabla \rho)
\end{equation}
We distribute the $\rho^{-1}$ and expand several of the derivatives,
\begin{equation}
  \nabla D_s \cdot \nabla \rho_s + D_s \Delta \rho_s - \nabla \cdot (D_s \rho^{-1} \rho_s \nabla \rho) - D_s \rho^{-1} \rho_s \Delta \rho - \nabla(D_s \rho^{-1} \rho_s) \cdot \nabla \rho
\end{equation}
Thus, the second and third terms ($D_s \Delta \rho_s$ and $D_s \rho^{-1}
\rho_s \Delta \rho$) are components of the linear operator. 

\section{Energy Equation}

Let's perform a similar analysis for the Energy Equation:
\begin{equation}
\frac{\partial}{\partial t} \rho E + \frac{\partial }{\partial x_j}\left(\rho
u_j H\right) = \frac{\partial }{\partial x_i}\left(\tau_{ij}u_j\right) -
\frac{\partial }{\partial x_j}q_j + \frac{\partial }{\partial x_i}\left(\rho
\sum^{N_s}_{s=1} h_s D_s \frac{\partial}{\partial x_i} c_s\right)
\end{equation}

We will consider each term individually. Let's start with $\nabla \cdot (\rho u H)$. A simple expansion of this provides:
\begin{equation}
  \nabla \cdot (\rho u H) = H \nabla \cdot (\rho u) + \rho u \nabla H
\end{equation}
Likewise, $\nabla \cdot (\tau u)$ can be expanded as,
\begin{equation}
 \nabla \cdot (\tau u) = u \cdot \nabla \tau + \tau \cdot \nabla u
\end{equation}
where $\nabla \tau$ is defined in section 3.5 of the Suzerain documentation and
$\nabla u$ rewritten as $\nabla \frac{m}{\rho}$.

The heat flux term ($\nabla \cdot q$) is also a simple expansion of the
derivatives, using the assumption of Fourier's Law ($q=-k\nabla T$):
\begin{equation}
  \nabla \cdot q = \nabla k \cdot \nabla T + k \Delta T
\end{equation}
Finally, the more tricky operator is:
\begin{equation}
  \nabla \cdot (\rho \sum_s h_s \nabla c_s)
\end{equation}
Analogously to the expansion of $\nabla \cdot (\rho_s D_s \nabla c_s)$
in equation~\eqref{eq:speciesdiffexpansion},
\begin{align}
  \nabla \cdot (\rho \sum_s h_s \nabla c_s) &= \nabla \cdot (\rho \sum_s h_s D_s ( \rho^{-1} \nabla \rho_s - \rho^{-2} \rho_s \nabla \rho))
\intertext{
As before, we distribute $\rho$,
}
  &= \nabla \cdot (\sum_s h_s D_s \nabla \rho_s - \sum_s h_s D_s \rho^{-1} \rho_s \nabla \rho)
\intertext{
Next, we distribute the divergence operator,
}
  &= \sum_s \nabla h_s D_s \nabla \rho_s + \sum_s h_s \nabla D_s \nabla \rho_s + \sum_s h_s D_s \Delta \rho_s - \nabla \cdot (\sum_s h_s D_s \rho^{-1} \rho_s \nabla \rho)
\end{align}
Our final result is:
\begin{equation}
  = \sum_s \nabla (h_s D_s) \cdot \nabla \rho_s + \sum_s h_s D_s \Delta \rho_s - \sum_s h_s D_s \rho^{-1} \rho_s \Delta \rho - \nabla (\sum_s h_s D_s \rho^{-1} \rho_s) \cdot \nabla \rho
\end{equation}
We will use the second and third terms in the linear operator. 


\section{Full Linear Operator}

Once we are happy with the above, a summary section containing the
complete $L$ will be placed here. This way passing to the discrete
implementation need only reference the summary.


\section{Acoustics}

In order to determine the acoustic effects that are contributing to our
operator, we will linearize the Euler equations.

We start with the 1d Euler equations:
\begin{eqnarray}
 \rho_t + (\rho u)_x = 0 \\
 (\rho u)_t + (\rho u^2 + p)_x = 0 \\
 E_t + [u(E+p)]_x = 0 
\end{eqnarray}

The entropy of this flow is:
\begin{equation}
 S = c_v \text{Log}(\frac{P}{\rho^\gamma}) + C
\end{equation}
Which can be rearranged as,
\begin{equation}
 \frac{S}{c_v}= C'\text{Log}(\frac{P}{\rho^\gamma})
\end{equation}
which is equivalent to,
\begin{equation}
 k e^{S/c_v} = \frac{P}{\rho^\gamma}
\end{equation}
Or,
\begin{equation}
 P = k e^{S/c_v}\rho^\gamma 
\end{equation}
Thus, we have an equation of state such that $P = P(\rho,S)$. For an
isentropic flow, $\delta S=0$, therefore, our equation of state is: 
\begin{equation}
 P = \hat k\rho^\gamma
\end{equation}
and the Euler equations reduce to:
\begin{eqnarray}
 \rho_t + (\rho u)_x = 0 \\
 (\rho u)_t + (\rho u^2 + \hat k \rho\gamma)_x = 0
\end{eqnarray}

We now seek to linearize this system of equations. In essence we have a
base state



\section{ToDo}

\begin{itemize}
 \item What about B.C.?
 \item What about coupling? (e.g. $\frac{\partial c}{\partial t} =
       \alpha c''$)
\end{itemize}

\end{document}
