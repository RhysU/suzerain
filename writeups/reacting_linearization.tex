\documentclass[10pt]{article}

% useful packages
\usepackage{amssymb, amsmath, amsfonts, amsthm, wasysym} % math
\usepackage{epsfig, subfigure} % graphics
\usepackage{fullpage, url, color} % misc

% commands I like
\newcommand{\mbb}[1]{\mathbb{#1}}
\newcommand{\mbf}[1]{\mathbf{#1}}
\newcommand{\sbf}[1]{\boldsymbol{#1}}
\newcommand{\mcal}[1]{\mathcal{#1}}
\newcommand{\mfk}[1]{\mathfrak{#1}}
\newcommand{\pp}[2]{\frac{\partial #1}{\partial #2}}
\newcommand{\dd}[2]{\frac{d #1}{d #2}}
\newcommand{\rarrow}{\rightarrow}
\newcommand{\Rarrow}{\Rightarrow}
\newcommand{\LRarrow}{\Leftrightarrow}
\newcommand{\jump}[1]{\llbracket #1 \rrbracket}
\newcommand{\avg}[1]{\{ #1 \}}
\def\etal{{\it et al.~}}
\newcommand{\vvvert}{|\kern-1pt|\kern-1pt|}
\newcommand{\enorm}[1]{\vvvert #1 \vvvert}
\newcommand{\ud}{\,\mathrm{d}}
\newcommand{\pdv}[2]{{\frac{\partial{#1}}{\partial{#2}}}}
\newcommand{\sa}{\nu_{\mathrm{sa}}}

%\newcommand{\red}[1]{\color{red}{#1}}
\newcommand{\myred}[1]{{\color{red} #1}}

% title and author
\title{Linearization for Reacting Equations of State}
\author{Nicholas Malaya \\
        Todd Oliver \\ 
	Rhys Ulerich \\ }

%	\thanks{Institute for Computational
%    	Engineering and Sciences, The University of Texas at Austin,
%    	Austin, TX 78712 (nick@ices.utexas.edu)}}


% lets rock and roll
\begin{document}
\maketitle

%-------------------------------------------------
\section{Problem Statement}

We are attempting to linearize our system of equations around a
reference state for reacting flow. e.g. $(R(u)-Lu) + Lu$
The linear piece, Lu, is designated as the reference state coefficients
multiplied by the operator acting on the state variables. For instance,
\begin{equation}
 \frac{\Delta m}{\rho} = (\frac{1}{\rho}-\frac{1}{\rho_0})\Delta m +
  \frac{\Delta m}{\rho_o}
\end{equation}
We will perform this procedure on the Navier-Stokes
equations. Conservation of Mass and Momentum will be identical to the
current procedure used in Suzerain. The species mass
fraction equations and the energy equation will need to be modified, however. 
\newline
\newline
Species Equations:
\begin{equation}
 \frac{\partial \rho_s}{\partial t} + \frac{\partial}{\partial
  x_i}(\rho_s u_i) = \frac{\partial}{\partial x_i}(\rho D_s
  \frac{\partial c_s}{\partial x_i}) + \dot \omega_s
\end{equation}
\newline
\newline
Energy Equation:
\begin{equation}
\frac{\partial \rho E}{\partial t} + \frac{\partial }{\partial x_j}(\rho
u_j H) = \frac{\partial }{\partial x_j}(\tau_{ji}u_i) - \frac{\partial
q_j}{\partial x_j} + \frac{\partial }{\partial x_i}(\rho \sum^{ns}_{s=1}
h_s D_s \frac{\partial c_s}{\partial x_i})
\end{equation}

\section{Species Equations}
Let us begin with the species equations. The only term we need consider here is the diffusivity. 
This term, in Gibbs notation, is 
\begin{equation}
  \nabla \cdot (\rho_s D_s \nabla c_s)
\end{equation}
Note that $c_s$, the mass fractions, are defined as: $\frac{\rho_s}{\rho} = c_s$. 
It can therefore be shown that, 
\begin{eqnarray}
  \nabla c_s = \nabla (\frac{\rho_s}{\rho}) &= \nabla (\rho_s \rho^{-1})\\
             &= \rho^{-1} \nabla \rho_s - \rho^{-2} \rho_s \nabla \rho
\end{eqnarray}
In other words, 
\begin{equation}
  \nabla \cdot (\rho_s D_s \nabla c_s) = \nabla \cdot \rho D_s (\rho^{-1} \nabla \rho_s - \rho^{-2} \rho_s \nabla \rho)
\end{equation}
We distribute the $\rho^{-1}$ and expand several of the derivatives,
\begin{equation}
  \nabla D_s \cdot \nabla \rho_s + D_s \Delta \rho_s - \nabla \cdot (D_s \rho^{-1} \rho_s \nabla \rho) - D_s \rho^{-1} \rho_s \nabla \rho - \nabla(D_s \rho^{-1} \rho_s) \cdot \nabla \rho
\end{equation}
Thus, the first and last terms ($\nabla D_s \cdot \nabla \rho_s$ and $\nabla(D_s \rho^{-1} \rho_s) \cdot \nabla \rho$) are components of the linear operator. 

\section{Energy Equation}

Let's perform a similar analysis for the Energy Equation:
\begin{equation}
\frac{\partial \rho E}{\partial t} + \frac{\partial }{\partial x_j}(\rho
u_j H) = \frac{\partial }{\partial x_j}(\tau_{ji}u_i) - \frac{\partial
q_j}{\partial x_j} + \frac{\partial }{\partial x_i}(\rho \sum^{ns}_{s=1}
h_s D_s \frac{\partial c_s}{\partial x_i})
\end{equation}

We will consider each term individually. Let's start with $\nabla \cdot (\rho u H)$. A simple expansion of this provides:
\begin{equation}
  \nabla \cdot (\rho u H) = H \nabla \cdot (\rho u) + \rho u \nabla H 
\end{equation}
Likewise, $\nabla \cdot (\tau u)$ can be expanded as,
\begin{equation}
 \nabla \cdot (\tau u) = u \cdot \nabla \tau + \tau \cdot \nabla u 
\end{equation}
Where $\nabla \tau$ is defined in section 3.5 of the suzerain
documentation. We express the term $\nabla u$ as $\nabla
\frac{m}{\rho}$. 
\newline
\newline
The heat flux term ($\nabla \cdot q$) is also a simple expansion of the derivatives:
\begin{equation}
  \nabla \cdot q = \nabla k \nabla T + k \Delta T
\end{equation}
Finally, the more tricky operator is:
\begin{equation}
  \nabla \cdot (\rho \sum_s h_s \nabla c_s)
\end{equation}
Remember that $c_s$, the mass fractions, can be expressed as: $\nabla \cdot (\rho_s D_s \nabla c_s) = \nabla \cdot \rho D_s (\rho^{-1} \nabla \rho_s - \rho^{-2} \rho_s \nabla \rho)$. 
Therefore, 
\begin{equation}
  \nabla \cdot (\rho \sum_s h_s \nabla c_s) = \nabla \cdot (\rho \sum_s h_s D_s ( \rho^{-1} \nabla \rho_s - \rho^{-2} \rho_s \nabla \rho))
\end{equation}
As before, we distribute $\rho$,
\begin{equation}
  = \nabla \cdot (\sum_s h_s D_s \nabla \rho_s - \sum_s h_s D_s \rho^{-1} \rho_s \nabla \rho)
\end{equation}
Next, we distribute the divergence operator,
\begin{equation}
  = \sum_s \nabla h_s D_s \nabla \rho_s + \sum_s h_s \nabla D_s \nabla \rho_s + \sum_s h_s D_s \Delta \rho_s - \nabla \cdot (\sum_s h_s D_s \rho^{-1} \rho_s \nabla \rho)
\end{equation}
Our final result is:
\begin{equation}
  = \sum_s \nabla h_s D_s \nabla \rho_s + \sum_s h_s \nabla D_s \nabla \rho_s + \sum_s h_s D_s \Delta \rho_s - \sum_s h_s D_s \rho^{-1} \rho_s \Delta \rho - \nabla (\sum_s h_s D_s \rho^{-1} \rho_s) \cdot \nabla \rho 
\end{equation}

\end{document}