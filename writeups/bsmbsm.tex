\documentclass[letterpaper,11pt,nointlimits,reqno]{amsart}

% Packages
\usepackage{amsfonts}
\usepackage{amsmath}
\usepackage{amssymb}
\usepackage{fullpage}
\usepackage{mathtools}
\usepackage{setspace}

% Hyperref package must be last otherwise the contents are jumbled
% hypertexnames disabled to fix links pointing to incorrect locations
\usepackage[hypertexnames=false,final]{hyperref}

\mathtoolsset{showonlyrefs,showmanualtags}
\allowdisplaybreaks[1] % Allow grouped equations to be split across pages

% Line Spacing
\singlespacing

\pagestyle{plain}

\begin{document}
Matrix $A$ is a blocked square matrix with banded submatrices (BSMBSM)
when
\[A = \begin{pmatrix}
    B^{0,0}  & \cdots & B^{0,S-1}   \\
    \vdots    & \ddots & \vdots       \\
    B^{S-1,0} & \cdots & B^{S-1,S-1}
\end{pmatrix}\]
where every $B^{i,j}$ is an \texttt{n}x\texttt{n} banded submatrix containing
\texttt{kl} subdiagonals and \texttt{ku} superdiagonals.  We set the convention
that lowercase identifiers indicate submatrix details while uppercase
identifiers indicate global matrix details.  The structure of a BSMBSM is
defined completely by the parameters \texttt{S}, \texttt{n}, \texttt{kl}, and
\texttt{ku}.  The number of rows and columns is \texttt{N := S*n}.

Building $A$ from individually contiguous, banded submatrices $B_{i,j}$ within
$A$ is both convenient and efficient.  For example, banded matrix accumulation
operations and boundary condition imposition are simple in such a storage
format.  However, using individually contiguous, banded submatrices is grossly
inefficient for solving linear equations.

With appropriate renumbering of $A$, solving linear equations can be done
efficiently.  The zero-indexed permutation vector \[q(i) =
\left(i\bmod{}S\right)n + \lfloor{}i/S\rfloor{}\] may always be used to convert
a BSMBSM into a globally banded \texttt{N}x\texttt{N} matrix with minimum
bandwidth.  More concretely, the permutation matrix $P$ uniquely defined by
vector $q$ causes $P A P^{\mbox{T}}$ to have \texttt{KL := S*(kl+1)-1}
subdiagonals and \texttt{KU := S*(ku+1)-1} superdiagonals summing to overall
bandwidth \texttt{KL + 1 + KU = S*(kl + ku + 2)-1}.  The reverse permutation
vector has a simple closed form \[q^{-1}(i) = \left(i\bmod{}n\right)S +
\lfloor{}i/n\rfloor{}.\] With $A_{i,j}$ in hand, the banded renumbering may
formed using the relationships
\begin{align*}
       \left.A\right|_{i,j}
    &= \left.P A P^{\mbox{T}}\right|_{q^{-1}(i),q^{-1}(j)}
    &
       \left.P A P^{\mbox{T}}\right|_{i,j}
    &= \left.A\right|_{q(i),q(j)}.
\end{align*}
This renumbering can be LU factorized in order
\verb|N*(KL + 1 + KU)^2 = S*n*(S*(kl + ku + 2)-1)^2|
floating point operations to find $LU = P A
P^{\mbox{T}}$.  The linear equation $AX=B$, which is equivalent to $LUPX=PB$,
has the solution \[X = A^{-1}B = P^{\mbox{T}}\left(LU\right)^{-1}PB\] where
inversion has been used as a notational convenience representing triangular
back substitution.

\end{document}
