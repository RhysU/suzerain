\documentclass[letterpaper,11pt,nointlimits,reqno]{amsart}

% Packages
\usepackage{accents}
\usepackage{algorithm}
\usepackage{algorithmic}
\usepackage{amsfonts}
\usepackage{amsmath}
\usepackage{amssymb}
\usepackage{cancel}
\usepackage{enumerate}
\usepackage{fancyhdr}
\usepackage{fullpage}
\usepackage{ifthen}
\usepackage{lastpage}
\usepackage{latexsym}
\usepackage{mathrsfs}
\usepackage{mathtools}
\usepackage{parskip}
\usepackage{pstricks}
\usepackage{setspace}
\usepackage{txfonts}
\usepackage{units}
\usepackage{varioref}
\usepackage{wrapfig}

% Hyperref package must be last otherwise the contents are jumbled
% hypertexnames disabled to fix links pointing to incorrect locations
\usepackage[hypertexnames=false,final]{hyperref}

\mathtoolsset{showonlyrefs,showmanualtags}
\allowdisplaybreaks[1] % Allow grouped equations to be split across pages

% Line Spacing
\singlespacing

% Set appropriate header/footer information on each page
\fancypagestyle{plain}{
    \fancyhf{}
    \renewcommand{\headheight}{2.0em}
    \renewcommand{\headsep}{0.75em}
    \renewcommand{\headrulewidth}{1.0pt}
    \renewcommand{\footrulewidth}{0pt}
    \lhead{
        Possible Approaches to Linearization of Acoustics
    }
    \rhead{
        Page \thepage{} of \pageref{LastPage}
    }
}
\pagestyle{plain}

% Document-specific commands
\newcommand{\ii}{\ensuremath{\mathrm{i}}}
\newcommand{\htrans}[1]{{#1}^{\ensuremath{\mathsf{H}}}}
\newcommand{\OO}[1]{\operatorname{O}\left(#1\right)}

\begin{document}

This is the full matrix, M
\begin{equation}
\tiny \left(
\begin{array}{cccccc}
 v & -\frac{v \rho _{\alpha }}{\rho } & 0 & \frac{\rho _{\alpha }}{\rho } & 0 & 0 \\
 0 & v & 0 & 1 & 0 & 0 \\
 0 & -u v & v & u & 0 & 0 \\
 \left(\frac{1}{2} \left(u^2+v^2+w^2\right)-e\right) (\gamma -1)+T \left(R_{\alpha }-R\right) & -v^2+R T+\left(\frac{1}{2}
   \left(u^2+v^2+w^2\right)-e\right) (\gamma -1) & -u (\gamma -1) & 2 v-v (\gamma -1) & -w (\gamma -1) & \gamma -1 \\
 0 & -v w & 0 & w & v & 0 \\
 v \left(\left(\frac{1}{2} \left(u^2+v^2+w^2\right)-e\right) (\gamma -1)+T \left(R_{\alpha }-R\right)\right) & v \left(R T+\left(\frac{1}{2}
   \left(u^2+v^2+w^2\right)-e\right) (\gamma -1)\right)-\text{vH} & -u v (\gamma -1) & H-v^2 (\gamma -1) & -v w (\gamma -1) & (\gamma -1) v+v
\end{array}
\right)
\end{equation}

\section{No Species Equation}

Let's first consider a linearized matrix that only decouples the species equations.

\begin{equation}
\left(
\begin{array}{cccccc}
 0 & 0 & 0 & 0 & 0 & 0 \\
 0 & 0 & 0 & 1 & 0 & 0 \\
 0 & -u v & 0 & u & 0 & 0 \\
 0 & -v^2+R T+\left(\frac{1}{2} \left(u^2+v^2+w^2\right)-e\right) (\gamma -1) & -u (\gamma -1) & v-v (\gamma -1) & -w (\gamma -1) & \gamma -1 \\
 0 & -v w & 0 & w & 0 & 0 \\
 0 & v \left(R T+\left(\frac{1}{2} \left(u^2+v^2+w^2\right)-e\right) (\gamma -1)\right)-\text{vH} & -u v (\gamma -1) & H-v^2 (\gamma -1) & -v w
   (\gamma -1) & v (\gamma -1)
\end{array}
\right)
\end{equation}

Already, this has perturbed the eigenvalues
\begin{align}
v \pm \frac{\sqrt{-\rho  \rho _{\alpha } \left(2 (\gamma -1) e_\alpha-2 T
 R_{\alpha}+2 R T-(\gamma -1) \left(u^2+v^2+w^2\right)\right)}}{\sqrt{2}
 \rho },\\
\end{align}

\section{No Species Equation or Continuity}

Now we additionally remove the coupling of species equation and
continuity. 
\begin{equation}
\left(
\begin{array}{cccccc}
 0 & 0 & 0 & 0 & 0 & 0 \\
 0 & 0 & 0 & 0 & 0 & 0 \\
 0 & -u v & 0 & u & 0 & 0 \\
 0 & -v^2+R T+\left(\frac{1}{2} \left(u^2+v^2+w^2\right)-e\right) (\gamma -1) & -u (\gamma -1) & v-v (\gamma -1) & -w (\gamma -1) & \gamma -1 \\
 0 & -v w & 0 & w & 0 & 0 \\
 0 & v \left(R T+\left(\frac{1}{2} \left(u^2+v^2+w^2\right)-e\right) (\gamma -1)\right)-\text{vH} & -u v (\gamma -1) & H-v^2 (\gamma -1) & -v w
   (\gamma -1) & v (\gamma -1)
\end{array}
\right)
\end{equation}

% \section{No Species Equation or $\rho u$}

% \begin{equation}
% \left(
% \begin{array}{cccccc}
%  0 & 0 & 0 & 0 & 0 & 0 \\
%  0 & 0 & 0 & 1 & 0 & 0 \\
%  0 & 0 & 0 & u & 0 & 0 \\
%  0 & -v^2+R T+\left(\frac{1}{2} \left(u^2+v^2+w^2\right)-e\right) (\gamma -1) & 0 & v-v (\gamma -1) & -w (\gamma -1) & \gamma -1 \\
%  0 & -v w & 0 & w & 0 & 0 \\
%  0 & v \left(R T+\left(\frac{1}{2} \left(u^2+v^2+w^2\right)-e\right) (\gamma -1)\right)-\text{vH} & 0 & H-v^2 (\gamma -1) & -v w (\gamma -1) & v
%    (\gamma -1)
% \end{array}
% \right)
% \end{equation}
% This creates a system such that:
% \begin{equation}
% v \pm \frac{\sqrt{-\rho  \rho _{\alpha } \left(2 (\gamma -1) e_{\alpha }-2 T R_{\alpha }+2 R T-(\gamma -1)
%    \left(u^2+v^2+w^2\right)\right)}}{\sqrt{2} \rho }
% \end{equation}
% Interestingly enough, $\rho u$ has no effect on the eigenvalues here and
% clearly should be decoupled. 

\section{No Species Equation or $\rho u$, $\rho w$}

\begin{equation}
\left(
\begin{array}{cccccc}
 0 & 0 & 0 & 0 & 0 & 0 \\
 0 & 0 & 0 & 1 & 0 & 0 \\
 0 & 0 & 0 & 0 & 0 & 0 \\
 0 & -v^2+R T+\left(\frac{1}{2} \left(u^2+v^2+w^2\right)-e\right) (\gamma -1) & 0 & v-v (\gamma -1) & 0 & \gamma -1 \\
 0 & 0 & 0 & 0 & 0 & 0 \\
 0 & v \left(R T+\left(\frac{1}{2} \left(u^2+v^2+w^2\right)-e\right) (\gamma -1)\right)-\text{vH} & 0 & H-v^2 (\gamma -1) & 0 & v (\gamma -1)
\end{array}
\right)
\end{equation}

\begin{equation}
v-\frac{\sqrt{-\rho  \left(\rho _{\alpha } \left(2 (\gamma -1)
					    e_{\alpha}-2 T R_{\alpha }+2
					    R T-(\gamma -1)
					    \left(u^2+v^2+w^2\right)\right)
{\color{red} +2 (\gamma -1) \rho} \left( {\color{red} u^2+w^2} \right) \right) }}{\sqrt{2} \rho }  
\end{equation}

\end{document}