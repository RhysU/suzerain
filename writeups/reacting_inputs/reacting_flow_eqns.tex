This section details the equations that are to be discretized and
solved to simulate reacting flow in Suzerain.

\subsection{Conservation laws}
This section details the relevant conservation laws for chemically
  reacting flows.  Following~\cite{Anderson_hypersonics,
  Kirk_2009_FINS_model_doc,
  Topalian_2011_temporal_slow_growth_reacting}, the conservation of
  mass, momentum, and total energy for a compressible fluid composed
  of $N_s$ constitutive components may be written as
%
\begin{align*}
  \pp{\rho_{\alpha}}{t} & + \pp{}{x_j} (\rho_{\alpha} u_j + \rho_{\alpha} v_{\alpha, j}) = \dot{\omega}_{\alpha}, \\
  \pp{\rho}{t} & + \pp{}{x_j} (\rho u_j) = 0, \\
  \pp{\rho u_i}{t} & + \pp{}{x_j} (\rho u_j u_i + p \delta_{ji} - \tau_{ji}) = 0, \\
  \pp{\rho E}{t} & + \pp{}{x_j} \left(\rho u_j H + \sum_{\alpha=1}^{N_s} \rho_{\alpha} v_{\alpha, j} h_{\alpha}  - \tau_{ji} u_i + q_j \right) = 0,\\
\end{align*}
% 
where $\rho_{\alpha}$ is the density of species $\alpha$, $\rho=\sum_{\alpha} \rho_{\alpha}$
is the mixture density, $u_i$ is the mixture velocity in the $i$th
direction, $v_{\alpha, i}$ is the diffusion velocity of species $\alpha$ in the
$i$th direction, $E$ is the total energy per unit mass, $p$ is the
pressure, $H$ is the total enthalpy per unit mass, $\tau_{ji}$ is the
viscous stress tensor, and $q_j$ is the heat flux vector.  Note that
Roman indices ($i$, $j$) indicate spatial directions.  For these,
repeated indices imply summation.  Greek indices ($\alpha$) indicate
species, and repeated Greek indices do not imply summation.

Further note that there are more governing equations here than
unknowns.  This is resolved by the consistency between the
conservation of mass equation and the species conservation equations.
However, we must choose what set of equations to model and discretize.
Here, we choose to include the conservation of mass equation and $N_s
-1$ species conservation equations.  Further, the state variables will
be $\rho$, $\rho_{\alpha}$ for $\alpha \in 2, \ldots, N_s$, $\rho u_i$
for $i = 1, 2, 3$, and $\rho E$.  Thus, we have $N_s + 4$ state
variables, and, by convention, species $\alpha = 1$ is the diluter
(i.e., the species that is not explicitly tracked).

\subsection{Constitutive relations and other assumptions}
\label{sec:constitutive}

\subsubsection{Mass Diffusion}
Species diffusion is modeled using Fick's law.
Specifically, Fick's law is given by
%
\begin{equation*}
\rho_{\alpha} v_{\alpha, i} = - \rho \mcal{D}_{\alpha} \pp{c_{\alpha}}{x_i},
\end{equation*}
% 
where $\mcal{D}_{\alpha}$ is the mass diffusivity for species $\alpha$
and $c_{\alpha}$ is the mass fraction $\rho_{\alpha} / \rho$.
Suzerain uses a constant Lewis number model to compute the species
diffusivities:
%
\begin{equation*}
\mcal{D}_{\alpha} = \frac{Le \kappa}{\rho C_{p,\mathrm{mix}}},
\end{equation*}
%
where $Le$ is the Lewis number, $\kappa$ is the mixture thermal
conductivity, and $C_{p,\mathrm{mix}}$ is the mixture specific heat at
constant pressure.  

Note that, since $\mcal{D}_{\alpha}$ is the same for all $\alpha$ in
the mixture, the constant $Le$ model automatically satisfies
conservation of mass.

However, for complex models, the $\mcal{D}_{\alpha}$ values are in
general different for each $\alpha$.  In this case, the Fickian model
is not guaranteed to satisfy mass conservation.  That is,
%
\begin{equation*}
\sum_{\alpha = 1}^{N_s} \rho_{\alpha} v_{\alpha, i} = - \rho \sum_{\alpha = 1}^{N_s} \mcal{D}_{\alpha} \pp{c_{\alpha}}{x_i} \neq 0,
\end{equation*}
% 
leading to an extra term in the implied conservation of mass
equation.

To alleviate this problem, Ramshaw~\cite{?} devised the self-consistent
effective binary diffusion model.  In this model, each flux is
corrected as follows:
%
\begin{equation*}
\rho_{\alpha} v_{\alpha, i} = - \rho \mcal{D}_{\alpha} \pp{c_{\alpha}}{x_i} + c_{\alpha} \sum_{\beta = 1}^{N_s} \rho \mcal{D}_{\beta} \pp{c_{\beta}}{x_i}.
\end{equation*}
% 
Since $\sum_{\alpha} c_{\alpha} = 1$, this form clearly gives
$\sum_{\alpha} \rho_{\alpha} v_{\alpha,i} = 0$.

\subsubsection{Chemical Reactions}
The source terms $\dot{\omega}_{\alpha}$ in the species equations
appear due to chemical reactions.  At a given point in space and time,
these sources depend only on the state at that point in space and
time.  These terms will be evaluated from reaction mechanisms and
reaction rate models implemented in Antioch~\cite{?}.  The details of
these models will not be discussed further here.

\subsubsection{Viscous Stress}
The viscous stress is given by
%
\begin{equation*}
\tau_{ij} =  2 \mu \left[ \frac{1}{2} \left(\pp{u_i}{x_j} + \pp{u_j}{x_i} \right) - \frac{1}{3} \pp{u_k}{x_k} \delta_{ij} \right].
\end{equation*}
% 
Velocity derivatives are easily computed from the state derivatives,
and the viscosity $\mu$ is computed via a call to Antioch.

\subsubsection{Heat Flux}
The heat flux is given by
%
\begin{equation*}
q_j = - \kappa \pp{T}{x_j}.
\end{equation*}
% 
The temperature gradient is not easy to compute from derivatives of
state because an explicit functional form for the temperature given
the state is not available.  Thus, we will proceed by computing the
temperature field in physical space, transforming back to wave space,
differentiating, and transforming to physical space again.

The thermal conductivity is computed via a call to Antioch.

\subsubsection{Pressure and Temperature}
The mixture pressure is a sum of the species partial pressures:
%
\begin{equation*}
p = \sum_{\alpha = 1}^{N_s} p_{\alpha}.
\end{equation*}
%
Each partial pressure is computed from the ideal gas law,
%
\begin{equation*}
p_{\alpha} = \rho_{\alpha} R_{\alpha} T,
\end{equation*}
% 
where $R_{\alpha}$ is the gas constant for the species $\alpha$.  Thus,
%
\begin{equation*}
p = \rho R_{\mathrm{mix}} T,
\end{equation*}
% 
where $R_{\mathrm{mix}} = \sum_{s=1}^{N_s} c_s R_s$.  

Both $R_{\mathrm{mix}}$ and $T$ are obtained via calls to Antioch.

\subsubsection{Species Enthalpies}
Species enthalpies are determined using Antioch.


\subsection{Slow growth models}
\label{sec:slowgrowthmodels}
\todo{Document slow growth.  This will mainly be reference to Topalian's model docs.}






