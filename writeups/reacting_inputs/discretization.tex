This section describes the discretization of the conservation laws
described in \S\ref{sec:gov_eq}.  It draws heavily from the model
document for the perfect gas case~\cite{?} where appropriate.

\subsection{Spatial discretization}
For generality, write the continuous system of interest as
%
\begin{equation*}
\pp{u}{t} = \mathscr{L}u + \mathscr{N}\!\left(u\right) \quad \mathrm{for} \,\, (x,y,z) \in \Omega,
\end{equation*}
% 
where $u$ denotes the full state vector---i.e., $u = [\rho_{\alpha},
  \rho, \rho u_i, \rho E]^T$ for the reacting flow equations presented
in \S\ref{sec:gov_eq}--- $\mathscr{L}$ and $\mathscr{N}$ denote the
linear and non-linear parts of the complete right hand side operator,
and $\Omega = \left[-\frac{L_x}{2},\frac{L_x}{2}\right] \times{}
[0,L_y] \times{} \left[-\frac{L_z}{2},\frac{L_z}{2}\right]$ is the
spatial domain of interest.  For brevity, this section suppresses any
explicit time dependence in the operators.

To begin spatially discretizing the system, its finite dimensional
analog is introduced
%
\begin{align}
  \frac{\partial}{\partial{}t} u^h
  &=
  \mathscr{L}u^h + \mathscr{N}\!\left(u^h\right) + R^h
  \label{eq:discrete_system_with_residual}
\end{align}
where continuous $u = u\!\left(x,y,z,t\right)$ has been replaced by
discrete $u^h = u^h\!\left(x,y,z,t\right)$ with
$N_x\times{}N_y\times{}N_z$ degrees of freedom.  Here, $R^h$ is the
residual that arises because the discrete solution cannot satisfy the
continuous equations everywhere in space.  Selecting Fourier
expansions for the periodic $x$ and $z$ directions and a B-spline
expansion for the non-periodic $y$ direction we have the following
discrete representation:
\begin{align}
u^h(x,y,z,t)
&=
  \sum_{l=0}^{N_y - 1}
  \sum_{m=-\frac{N_x}{2}}^{\frac{N_x}{2}-1}
  \sum_{n=-\frac{N_z}{2}}^{\frac{N_z}{2}-1}
  \hat{u}_{l m n}(t)
  B_l\!\left(y\right)
  e^{\ii\frac{2\pi{}m}{L_x}x}
  e^{\ii\frac{2\pi{}n}{L_z}z}
  \\
&=
  \sum_{l}\sum_{m}\sum_{n}
  \hat{u}_{l m n}(t)B_l\!\left(y\right)e^{\ii k_m x}e^{\ii k_n z},
  \label{eq:u_h_expansion}
\end{align}
where $k_m = 2\pi{}m/L_x$, $k_n = 2\pi{}n/L_z$, and
$B_l\!\left(y\right)$ are a B-spline basis for some order and knot
selection.  For additional details of the B-spline basis,
see~\cite{?}.

Within the method of weighted residuals framework, we choose a mixed
Galerkin/collocation approach (often called a ``pseudospectral'' technique)
employing the $L_{2}$ inner product and test ``functions'' like
$\delta(y-y_{l'}) e^{\ii k_{m'} x}e^{\ii k_{n'} z}$ where $l'$, $m'$, and $n'$
range over the same values as $l$, $m$, and $n$, respectively.  The fixed
collocation points $y_{l'}$ depend on the B-spline basis details.  Three
relevant results are
\begin{align}
   \int_0^{L_y} \varphi(y) \, \delta(y-y_{l'}) \,d\!y
&= \varphi(y_{l'}),
&
   \int_{-\frac{L_x}{2}}^{\frac{L_x}{2}} e^{\ii k_m x} e^{-\ii k_{m'} x} \,d\!x
&= L_x \delta_{m m'}, \text{ and}
&
   \int_{-\frac{L_z}{2}}^{\frac{L_z}{2}} e^{\ii k_n z} e^{-\ii k_{n'} z} \,d\!z
&= L_z \delta_{n n'}
\end{align}
where the inner product's conjugate operation is accounted for by introducing a
negative sign into the latter two exponentials.  The weighted residual is
forced to be zero in the sense that
\begin{align}
  \int_0^{L_y}
  \int_{-\frac{L_x}{2}}^{\frac{L_x}{2}}
  \int_{-\frac{L_z}{2}}^{\frac{L_z}{2}}
  R^h\!\left(x,y,z\right) \delta(y-y_{l'}) e^{-\ii k_{m'} x}e^{-\ii k_{n'} z}
  \,d\!z \,d\!x \,d\!y
  &=
  0
  \label{eq:R_h_weighted_residual_zero}
\end{align}
holds for all $l'$, $m'$, and $n'$.  Inserting \eqref{eq:u_h_expansion} into
\eqref{eq:discrete_system_with_residual}, testing with our test functions,
applying \eqref{eq:R_h_weighted_residual_zero}, and simplifying each remaining
term separately one obtains the following:
\begin{align}
 \int_0^{L_y}
 \int_{-\frac{L_x}{2}}^{\frac{L_x}{2}}
 \int_{-\frac{L_z}{2}}^{\frac{L_z}{2}}
 \,
 &\frac{\partial}{\partial{}t}
  \left(
    \sum_{l}\sum_{m}\sum_{n}
    \hat{u}_{l m n}(t)B_l\!\left(y\right)e^{\ii k_m x}e^{\ii k_n z}
  \right)
  \left(
    \delta(y-y_{l'}) e^{-\ii k_{m'} x}e^{-\ii k_{n'} z}
  \right)
  \, dz \, dx \, dy
\\
  &=
  L_x L_z \sum_{l} B_l\!\left(y_{l'}\right)
  \frac{\partial}{\partial{}t} \hat{u}_{l m n}(t)
\\
 \int_0^{L_y}
 \int_{-\frac{L_x}{2}}^{\frac{L_x}{2}}
 \int_{-\frac{L_z}{2}}^{\frac{L_z}{2}}
 \,
 &\mathscr{L}
  \left(
    \sum_{l}\sum_{m}\sum_{n}
    \hat{u}_{l m n}(t)B_l\!\left(y\right)e^{\ii k_m x}e^{\ii k_n z}
  \right)
  \left(
    \delta(y-y_{l'}) e^{-\ii k_{m'} x}e^{-\ii k_{n'} z}
  \right)
  \, dz \, dx \, dy
\\
  &=
  L_x L_z
  \mathscr{L}\left(
     \sum_{l}
      B_l\!\left(y_{l'}\right)
     \hat{u}_{l m n}(t)
   \right)
\intertext{} % Allow soft break
  \int_0^{L_y}
  \int_{-\frac{L_x}{2}}^{\frac{L_x}{2}}
  \int_{-\frac{L_z}{2}}^{\frac{L_z}{2}}
  &\mathscr{N}\left(
     \sum_{l}\sum_{m}\sum_{n}
     \hat{u}_{l m n}(t)B_l\!\left(y\right)e^{\ii k_m x}e^{\ii k_n z}
   \right)
   \left(
     \delta(y-y_{l'}) e^{-\ii k_{m'} x}e^{-\ii k_{n'} z}
   \right)
   \, dz \, dx \, dy
\\
  &=
  \int_{-\frac{L_x}{2}}^{\frac{L_x}{2}}
  \int_{-\frac{L_z}{2}}^{\frac{L_z}{2}}
  \mathscr{N}\left(
    \sum_{m}\sum_{n}
    \left(
      \sum_{l} B_l\!\left(y_{l'}\right)
      \hat{u}_{l m n}(t)
    \right)
    e^{\ii k_m x}e^{\ii k_n z}
  \right)
  \left(
    e^{-\ii k_{m'} x}e^{-\ii k_{n'} z}
  \right)
  \, dz \, dx
\intertext{
  Reequating the terms,
}
  L_x L_z
  \sum_{l} B_l\!\left(y_{l'}\right)
  \frac{\partial}{\partial{}t} \hat{u}_{l m n}(t)
  &=
  L_x L_z
  \mathscr{L}\left(
    \sum_{l}
     B_l\!\left(y_{l'}\right)
    \hat{u}_{l m n}(t)
  \right)
\\
  &{}+
  \int_{-\frac{L_x}{2}}^{\frac{L_x}{2}}
  \int_{-\frac{L_z}{2}}^{\frac{L_z}{2}}
  \mathscr{N}\left(
    \sum_{m}\sum_{n}
    \left(
      \sum_{l} B_l\!\left(y_{l'}\right)
      \hat{u}_{l m n}(t)
    \right)
    e^{\ii k_m x}e^{\ii k_n z}
  \right)
  \left(
    e^{-\ii k_{m'} x}e^{-\ii k_{n'} z}
  \right)
  \, dz \, dx
  .
 \end{align}

Finally, approximating the two integrals by discrete sums and dividing
by $L_x$ and $L_z$ leaves
\begin{align}
  \sum_{l} B_l\!\left(y_{l'}\right)
  \frac{\partial}{\partial{}t} \hat{u}_{l m n}(t)
  &\approx
  \mathscr{L}\left(
    \sum_{l}
     B_l\!\left(y_{l'}\right)
    \hat{u}_{l m n}(t)
  \right)
\\
  &{}+
  \frac{1}{N_x N_z}
  \sum_{m'} \sum_{n'}
  \mathscr{N}\left(
    \sum_{m}
    \sum_{n}
    \left(
      \sum_{l} B_l\!\left(y_{l'}\right)
      \hat{u}_{l m n}(t)
    \right)
    e^{\ii k_m x_{m'}}e^{\ii k_n z_{n'}}
  \right)
  \!\!
  \left(
    e^{-\ii k_{m'} x_m}e^{-\ii k_{n'} z_n}
  \right)
  \label{eq:spatial_discretization}
\end{align}
where $x_{m'}=L_x m' / N_x$ and $z_{n'}=L_z n' / N_z$.  This
approximation introduces a quadrature
error \citep[see][theorem~19]{Boyd2001} which, With knowledge of the
nonlinear operator $\mathscr{N}$, can be mitigated via an appropriate
dealiasing technique \citep[see][]{Canuto2006}.  We are left with
$N_x\times{}N_z$ time-dependent systems containing $N_y$ equations
coupled in the $x$ and $z$ directions only through discrete Fourier
transforms and the requirements of the $\mathscr{L}$ and $\mathscr{N}$
operators.

A key issue now is how to evaluate the nonlinear term.  In the perfect
gas case, Ulerich~\cite{?} takes great care to avoid repeated
application of the B-spline first derivative operator when evaluating
second derivatives.  This care is appropriate given the well-known
fact that repeated application of the first derivative operator leads
to a spectrum for the second derivative that is dramatically incorrect
at high wavenumbers.  However, this procedure requires explicit
calculation of derivatives (e.g., $\nabla \mu$, $\nabla T$, and
$\nabla^2 T$) in terms of conserved state (see Ulerich~\cite{} chapter
3, section 3).  While this is possible in the reacting case, it is
significantly more difficult given the complexity of the constitutive
laws involved.  Further, given that these constitutive laws are not
implemented within Suzerain, requiring calculation of these
derivatives would place a very high burden on the third-party packages
that could be used to provide them.

Thus, here we seek a different formulation that does not require
explicit differentiation through the constitutive laws.  Specifically,
we take a ``flux-based'' approach.  Realizing that the nonlinear term
is composed of the divergence of a flux combined with source terms,
the procedure can be outlined as follows:
%
\begin{enumerate}
\item Perform $\mathcal{O}\!\left(m\times{}n\right)$ matrix-vector
  products like $\sum_{l} B_l\!\left(y_{l'}\right) \hat{u}_{l m
    n}(t)$ to evaluate the state and state derivatives;
\item Use an inverse fast Fourier transform across the $x$ and $z$
  directions to convert state and derivative information from wave
  space to physical space;
\item Form the Navier-Stokes fluxes (omitting the heat flux, which
  requires spatial derivatives of the temperature), the
  temperature field, and the reaction source terms in physical space;
\item Using an inverse mass matrix application and forward fast
  Fourier transform across the $x$ and $z$ directions, convert
  temperature from physical space to wave space;
\item Compute the temperature gradient and transform from wave to
  physical space;
\item Evaluate the heat flux vector in physical space and add it to
  the energy equation flux vector;
\item Using an inverse mass matrix application and forward fast
  Fourier transform across the $x$ and $z$ directions, convert flux
  components and reaction terms from physical space to wave space;
\item Accumulate the discrete divergence of the flux vector into the
  right hand side where the reaction terms are already stored.
\end{enumerate}

Tables~\ref{tbl:w2p} and~\ref{tbl:p2w} summarize the total number of
inverse and forward Fourier transforms to accomplish the steps
described above.
%
\begin{table}[ht]
\caption{Wave to physical.  NOTE: This count is tentative.}
\begin{tabular}{|c|c|}
\hline
Quantity & \# of Fields \\
\hline
\hline
$\rho$ & 1 \\
$\rho_{\alpha}$ & $N_s - 1$ \\
$\rho u_i$ & 3 \\
$\rho E$ & 1 \\
$\pp{\rho}{x_j}$ & 3 \\
$\pp{\rho_{\alpha}}{x_j}$ & $3 (N_s - 1)$ \\
$\pp{\rho u_i}{x_j}$ & 9 \\
$\pp{T}{x_j}$ & 3 \\
\hline
Total & $4(N_s + 4)$\\
\hline
\end{tabular}
\label{tbl:w2p}
\end{table}
% 
%
\begin{table}[ht]
\caption{Physical to wave.  NOTE: This count is tentative.}
\begin{tabular}{|c|c|}
\hline
Quantity & \# of Fields \\
\hline
\hline
Sources ($\dot{\omega}_{\alpha}$ \& SG) & $N_s + 4$ \\
Fluxes & $3 (N_s +4)$ \\
T & 1 \\
\hline
Total & $4(N_s + 4) + 1$\\
\hline
\end{tabular}
\label{tbl:p2w}
\end{table}
% 

While this scheme achieves the goal of not requiring Jacobian
information for quantities like the viscosity, it has the distinct
drawback that it uses repeated application of the B-spline first
derivative operator.  Filtering strategies to avoid the damage done by
this operation are described in \S\ref{sec:filter}.


\subsection{Temporal discretization}
\label{sec:temporal_discretization}
The temporal discretization is the hybrid implicit-explict scheme
described by Spalart, Moser, and Rogers~\cite{spalart_lowstoragerk}.
Details of the scheme and it's implementation in Suzerain can be found
in Ulerich~\cite{?}.  For brevity, these details are not given here.

\subsection{Time step stability criteria}
\label{sec:stabilitycriteria}
The time step stability calculations are exactly as described by
Ulerich~\cite{} except that 1.) the dimensional forms are used and 2.)
species diffusion is accounted for.  Thus, the following replaces the
$\max$ in Ulerich, Section 5.2, Equation 25:
%
\begin{equation*}
\max \left(
        \left|\frac{\kappa}{\rho c_v} - \left( \frac{\kappa}{\rho c_v} \right)_{\mathrm{ref}} \right|,
        \left|\nu-\nu_{\mathrm{ref}} \right|,
        \left|\nu_{B}-\nu_{B,\mathrm{ref}}\right|,
        \left|D_s-D_{s,\mathrm{ref}}\right|
     \right),
\end{equation*}
% 
where $(\cdot)_{\mathrm{ref}}$ denote the reference quantities used
to form the linear implicit operator (which are zero when using the
fully explicit scheme).

Note that stability limits due to the reaction source terms are
\emph{not} computed.  Preliminary analysis shows that in the target
problems of interest, these constraints are less restrictive than
others~\cite{}.

