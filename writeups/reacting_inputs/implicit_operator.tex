This section describes the formulation of the implicit operator
$\tilde{L}$ in the SMR91 scheme.  See~\cite{Ulerich_SZPerfect} for
details of the temporal discretization scheme.

\subsection{Motivating Concerns}
Given the wall-normal grid stretching required to adequately resolve
the near-wall features of the solution, it is expected that, in a
fully explicit treatment, the near-wall, wall-normal acoustic velocity
and viscous terms would lead to the most severe time step
restrictions.  Thus, the goal of this formulation is to bring these
effects into the implicit operator.  Of course, since the SMR91 scheme
requires that the implicit operator be linear in the state, we cannot
exactly capture the full behavior in the linear operator.  Further, to
reduce the cost of the required implicit solves, we will pursue a
formulation that treats only wall-normal portions of the Navier-Stokes
right hand sides implicitly.  This choice implies that the linear
operator $\tilde{L}$ is independence of wave-number, meaning that we
require only one matrix factorization per Runge-Kutta step.
Finally, to avoid factorizing an $N_y (N_s + 4) \times N_y (N_s + 4)$
matrix, we require that the formulation decouples the species
equations both from the flow and each other.  In this case, the
species equations have their own solvers, which each require
factorization of one $N_y \times N_y$ matrix per step.  The flow
solver requires factorization of one $5 N_y \times 5 N_y$ matrix per
step.

For clarity in developing the formulation, the inviscid and viscous
contributions are discussed separately in
\S\ref{sec:inviscid_implicit_operator} and
\S\ref{sec:viscous_implicit_operator}, respectively.  In
\S\ref{sec:complete_implicit_operator} these contributions are combined
to form the complete implicit operator.

\subsection{Inviscid Contribution}
\label{sec:inviscid_implicit_operator}
First, we develop the inviscid contribution to the implicit operator.
This development is split into two pieces: the derivation of the full
flux Jacobian and the choice of the implicit operator.

\subsubsection{The Full Wall-Normal Inviscid Flux Jacobian}
The analysis begins from the full inviscid flux Jacobian for the
$y$-component (i.e., wall-normal) flux.  For completeness, this flux
Jacobian is derived here.

To be clear, the conserved state vector is given by
%
\begin{equation*}
U = \left[ \begin{array}{c}
\rho_{\alpha} \\
\rho \\
\rho u\\
\rho v \\
\rho w \\
\rho E
\end{array} \right],
\end{equation*}
%
where $\rho_{\alpha}$ is the density of species $\alpha$, $\rho$ is
the mixture density, $\rho u$ is the $x$-momentum, $\rho v$ is the
$y$-momentum, $\rho w$ is the $z$-momentum, and $\rho E$ is the total
energy.  Note that $N_s-1$ species densities and the mixture density
are tracked.  In this work, we adopt the convention that $\alpha \in
\{2, \ldots, N_s\}$ such that the species $\alpha = 1$ is the diluter.

The flux of interest is the $y$ component of the inviscid flux:
%
\begin{equation*}
G = \left[ \begin{array}{c}
\rho_{\alpha} v \\
\rho v \\
\rho u v\\
\rho v^2 + p \\
\rho w v \\
\rho v H
\end{array} \right],
\end{equation*}
%
where $p$ is the mixture pressure and $H$ is the total enthalpy.

The Jacobian matrix derived below is similar to the standard inviscid
flux Jacobian for a calorically perfect gas except that the equation
of state is more complicated.  In particular, closed-form expressions
for the temperature and pressure in terms of conserved state variables
are not available.  Instead, the temperature is computed by
numerically inverting the following equation:
%
\begin{equation}
\label{eqn:etot_T}
e_{tot}(c_s, T) = \frac{1}{\rho} \left( \rho E - \frac{1}{2} \rho (u^2 + v^2 + w^2) \right),
\end{equation}
% 
where the mixture internal energy $e_{tot}$ is computed from the
species energies
%
\begin{equation*}
e_{tot}(c_s, T) = \sum_{s=1}^{N_s} c_s e_{s,tot}(T).
\end{equation*}
%
Once the temperature has been found, the pressure is computed from the
ideal gas law:
%
\begin{equation*}
p = \sum_{s=1}^{N_s} p_s = \sum_{s=1}^{N_s} \rho_s R_s T = \rho R_{\text{mix}} T,
\end{equation*}
%
where $R_{\text{mix}} = \sum_{s=1}^{N_s} c_s R_s$ and $R_s$ is the gas
constant for species $s$.

To make the derivation as clean as possible, we break the flux into
two parts: a ``pressure'' part $G_p$ that has explicit dependence on
the pressure and everything else, referred to here as the
``convective'' part $G_c$.  Specifically,
%
\begin{equation*}
G = G_c + G_p,
\end{equation*}
%
where
%
\begin{equation*}
G_c = \left[ \begin{array}{c}
\rho_{\alpha} v \\
\rho v \\
\rho u v\\
\rho v^2 \\
\rho w v \\
\rho v E 
\end{array} \right]
= \left[ \begin{array}{c}
\rho_{\alpha} (\rho v) / \rho\\
\rho v \\
\rho u (\rho v) / \rho \\
(\rho v)^2 / \rho \\
\rho w (\rho v)/ \rho \\
\rho v (\rho E) / \rho 
\end{array} \right], \quad
%
G_p = \left[ \begin{array}{c}
0 \\
0 \\
0 \\
p \\
0 \\
v p
\end{array} \right], = \left[ \begin{array}{c}
0 \\
0 \\
0 \\
p \\
0 \\
(\rho v) p / \rho
\end{array} \right], 
\end{equation*}
%
Because it is trivial to write the convective piece explicitly in
terms of the conserved state, one can immediately write the Jacobian
for the convective piece:
%
\begin{equation*}
\pp{G_C}{U} = \left[ \begin{array}{cccccc}
v I & -\rho_{\alpha} v / \rho & 0 & \rho_{\alpha} / \rho & 0 & 0 \\
0 & 0 & 0 & 1 & 0 & 0 \\
0 & -uv & v & u & 0 & 0 \\
0 & -v^2 & 0 & 2v & 0 & 0 \\
0 & -wv & 0 & w & v & 0 \\
0 & -vE & 0 & E & 0 & v
\end{array} \right].
\end{equation*}
%
The pressure contribution is somewhat more difficult.  To begin, write
the Jacobian in terms of derivatives of the pressure:
%
\begin{equation*}
\pp{G_p}{U} = \left[ \begin{array}{cccccc}
0 & 0 & 0 & 0 & 0 & 0 \\
0 & 0 & 0 & 0 & 0 & 0 \\
0 & 0 & 0 & 0 & 0 & 0 \\
\pp{p}{\rho_{\alpha}} & \pp{p}{\rho} & \pp{p}{\rho u} & \pp{p}{\rho v} & \pp{p}{\rho w} & \pp{p}{\rho E} \\
0 & 0 & 0 & 0 & 0 & 0 \\
v \pp{p}{\rho_{\alpha}} & -v \frac{p}{\rho} + v \pp{p}{\rho} & v \pp{p}{\rho u} & \frac{p}{\rho} + v \pp{p}{\rho v} & v \pp{p}{\rho w} & v \pp{p}{\rho E} \\
\end{array} \right].
\end{equation*}
%
To find the necessary derivatives of the pressure, begin by examining
the ideal gas law.  We have
%
\begin{equation*}
p = \rho R_{\text{mix}} T = \sum_{s=1}^{N_s} \rho_s R_s T = \rho_1 R_1 T + \sum_{s=2}^{N_2} \rho_s R_s T
\end{equation*}
%
Since
%
\begin{equation*}
\rho = \sum_{s=1}^{N_s} \rho_s \quad \Rarrow \quad \rho_1 = \rho - \sum_{s=2}^{N_s} \rho_s,
\end{equation*}
%
it is clear that
%
\begin{align*}
p & = \left( \rho - \sum_{s=2}^{N_s} \rho_s \right) R_1 T + \sum_{s=2}^{N_s} \rho_s R_s T \\
& = \rho R_1 T + \sum_{s=2}^{N_s} \rho_s (R_s - R_1) T. \\
\end{align*}
%
Thus,
%
\begin{equation*}
\pp{p}{\rho_{\alpha}} = (R_{\alpha} - R_1) T + \rho R_{\text{mix}} \pp{T}{\rho_{\alpha}},
\end{equation*}
%
and
%
\begin{equation*}
\pp{p}{\rho} = R_1 T + \rho R_{\text{mix}} \pp{T}{\rho}.
\end{equation*}
%
Since $\rho R_{\text{mix}}$ only depends on the densities, the remaining
partial derivatives are
%
\begin{equation*}
\pp{p}{\rho u} = \rho R_{\text{mix}} \pp{T}{\rho u}, \quad
\pp{p}{\rho v} = \rho R_{\text{mix}} \pp{T}{\rho v}, \quad
\pp{p}{\rho w} = \rho R_{\text{mix}} \pp{T}{\rho w}, \quad
\pp{p}{\rho E} = \rho R_{\text{mix}} \pp{T}{\rho E}.
\end{equation*}
%
Thus, to complete the Jacobian, we require the derivatives of the
temperature with respect to the conserved state.  To find these
derivatives, we use implicit differentiation on \eqref{eqn:etot_T}.

To make this procedure slightly cleaner, first multiply through by
$\rho$ to get
%
\begin{equation*}
\rho e_{tot}(c_s, T) = \sum_{s=1}^{N_s} \rho_s e_{s,tot}(T) = \rho E - \frac{1}{2 \rho } \left( (\rho u)^2 + (\rho v)^2 + (\rho w)^2 \right).
\end{equation*}
%
Eliminating the diluter density gives
%
\begin{equation*}
\rho e_{1,tot}(T) + \sum_{s=2}^{N_s} \rho_s \left[ e_{s,tot}(T) - e_{1,tot}(T) \right] = \rho E - \frac{1}{2 \rho } \left( (\rho u)^2 + (\rho v)^2 + (\rho w)^2 \right).
\end{equation*}
%

Differentiating with respect to $\rho_{\alpha}$ gives
%
\begin{equation*}
\rho \pp{e_{1,tot}}{T} \pp{T}{\rho_{\alpha}} + \left[ e_{\alpha, tot}(T) - e_{1,tot}(T) \right] + \sum_{s=2}^{N_s} \rho_s \left[ \pp{e_{s,tot}}{T} - \pp{e_{1,tot}}{T} \right] \pp{T}{\rho_{\alpha}} = 0.
\end{equation*}
%
Thus,
%
\begin{equation*}
\rho C_{v,\text{mix}} \pp{T}{\rho_{\alpha}} = e_{1,tot}(T) - e_{\alpha,tot}(T) \quad \Rarrow \quad \pp{T}{\rho_{\alpha}} = \frac{e_{1,tot} - e_{\alpha,tot}}{ \rho C_{v,\text{mix}}},
\end{equation*}
%
where
%
\begin{equation*}
C_{v,\text{mix}} = \sum_{s=1}^{N_s} c_s C_{v,s} = \sum_{s=1}^{N_s} c_s \pp{e_{s,tot}}{T}
\end{equation*}
%

Differentiating with respect to $\rho$ gives
%
\begin{equation*}
e_{1,tot}(T) + \rho \pp{e_{1,tot}}{T} \pp{T}{\rho} + \sum_{s=2}^{N_s} \rho_s \left[ \pp{e_{s,tot}}{T} - \pp{e_{1,tot}}{T} \right] \pp{T}{\rho} = \frac{1}{2 \rho^2} \left[ (\rho u)^2 + (\rho v)^2 + (\rho w)^2 \right].
\end{equation*}
%
Thus,
%
\begin{equation*}
e_{1,tot}(T) + \rho C_{v,\text{mix}} \pp{T}{\rho} = \frac{1}{2} (u^2 + v^2 + w^2),
\end{equation*}
%
and
%
\begin{equation*}
\pp{T}{\rho} = \frac{ -e_{1,tot}(T) + \frac{1}{2} (u^2 + v^2 + w^2) }{\rho C_{v,\text{mix}} }
\end{equation*}
%

Differentiating with respect to $\rho u$ gives
%
\begin{equation*}
\rho \pp{e_{1,tot}}{T} \pp{T}{\rho u} + \sum_{s=2}^{N_s} \rho_s \left[ \pp{e_{s,tot}}{T} - \pp{e_{1,tot}}{T} \right] \pp{T}{\rho u} = -u
\end{equation*}
%
Thus,
%
\begin{equation*}
\pp{T}{\rho u} = \frac{-u}{\rho C_{v,\text{mix}}}
\end{equation*}
%

Differentiating with respect to $\rho v$ gives
%
\begin{equation*}
\rho \pp{e_{1,tot}}{T} \pp{T}{\rho v} + + \sum_{s=2}^{N_s} \rho_s \left[ \pp{e_{s,tot}}{T} - \pp{e_{1,tot}}{T} \right] \pp{T}{\rho v} = -v
\end{equation*}
%
Thus,
%
\begin{equation*}
\pp{T}{\rho v} = \frac{-v}{\rho C_{v,\text{mix}}}
\end{equation*}
%

Differentiating with respect to $\rho w$ gives
%
\begin{equation*}
\rho \pp{e_{1,tot}}{T} \pp{T}{\rho w} + \sum_{s=2}^{N_s} \rho_s \left[ \pp{e_{s,tot}}{T} - \pp{e_{1,tot}}{T} \right] \pp{T}{\rho w} = -w
\end{equation*}
%
Thus,
%
\begin{equation*}
\pp{T}{\rho w} = \frac{-w}{\rho C_{v,\text{mix}}}
\end{equation*}
%

Differentiating with respect to $\rho E$ gives
%
\begin{equation*}
\rho \pp{e_{1,tot}}{T} \pp{T}{\rho E} + + \sum_{s=2}^{N_s} \rho_s \left[ \pp{e_{s,tot}}{T} - \pp{e_{1,tot}}{T} \right] \pp{T}{\rho E} = 1
\end{equation*}
%
Thus,
%
\begin{equation*}
\pp{T}{\rho E} = \frac{1}{\rho C_{v,\text{mix}}}
\end{equation*}
%

We can now collect all the terms and write the Jacobian of the
pressure component.  We have
%
\begin{gather*}
\pp{p}{\rho_{\alpha}} = (R_{\alpha} - R_1) T + \rho R_{\text{mix}} \frac{(e_{1,tot} - e_{\alpha, tot})}{\rho C_{v,\text{mix}}} = (R_{\alpha} - R_1) T + (\gamma_{\text{mix}} - 1) (e_{1,tot} - e_{\alpha, tot}), \\
\pp{p}{\rho} = R_1 T + \rho R_{\text{mix}} \left[ \frac{-e_{1,tot} + \frac{1}{2} (u^2 + v^2 + w^2)}{\rho C_{v,\text{mix}}} \right] = R_1 T + (\gamma_{\text{mix}} - 1) \left[ -e_{1,tot} + \frac{1}{2} (u^2 + v^2 + w^2) \right], \\
\pp{p}{\rho u} = \rho R_{\text{mix}} \frac{-u}{\rho C_{v,\text{mix}}} = (1 - \gamma_{\text{mix}}) u, \\
\pp{p}{\rho v} = \rho R_{\text{mix}} \frac{-v}{\rho C_{v,\text{mix}}} = (1 - \gamma_{\text{mix}}) v, \\
\pp{p}{\rho w} = \rho R_{\text{mix}} \frac{-w}{\rho C_{v,\text{mix}}} = (1 - \gamma_{\text{mix}}) w, \\
\pp{p}{\rho E} = \rho R_{\text{mix}} \frac{1}{\rho C_{v,\text{mix}}} = (\gamma_{\text{mix}} - 1). \\
\end{gather*}
%

Thus, the full flux Jacobian is
%
\begin{equation*}
\pp{G}{U} = \left[ \begin{array}{cccccc}
v I & -\rho_{\alpha} v / \rho & 0 & \rho_{\alpha} / \rho & 0 & 0 \\
0 & 0 & 0 & 1 & 0 & 0 \\
0 & -uv & v & u & 0 & 0 \\
\pp{p}{\rho_{\alpha}} & -v^2 +  \pp{p}{\rho} & \pp{p}{\rho u} & 2v + \pp{p}{\rho v} & \pp{p}{\rho w} & \pp{p}{\rho E} \\
0 & -wv & 0 & w & v & 0 \\
v \pp{p}{\rho_{\alpha}} & -v \left(E + \frac{p}{\rho} - \pp{p}{\rho}\right) & v \pp{p}{\rho u} & E + \frac{p}{\rho} + v \pp{p}{\rho v}  & v \pp{p}{\rho w} & v \left( 1+ \pp{p}{\rho E} \right)
\end{array} \right],
\end{equation*}
%
where the necessary pressure derivatives are given above.

% Two species case output from mathematica... saved for posterity
%The full flux Jacobian is
%%
%\begin{equation}
%\tiny
%M = 
%\left(
%\begin{array}{cccccc}
% v & -\frac{v \rho _{\alpha }}{\rho } & 0 & \frac{\rho _{\alpha }}{\rho } & 0 & 0 \\
% 0 & v & 0 & 1 & 0 & 0 \\
% 0 & -u v & v & u & 0 & 0 \\
% (\gamma -1) \left(\frac{1}{2} \left(u^2+v^2+w^2\right)-e_{\alpha }\right)+T \left(R_{\alpha }-R\right) & -v^2+R T+\left(\frac{1}{2} \left(u^2+v^2+w^2\right)-e\right) (\gamma
%   -1) & -u (\gamma -1) & 2 v-v (\gamma -1) & -w (\gamma -1) & \gamma -1 \\
% 0 & -v w & 0 & w & v & 0 \\
% v \left((\gamma -1) \left(\frac{1}{2} \left(u^2+v^2+w^2\right)-e_{\alpha }\right)+T \left(R_{\alpha }-R\right)\right) & v \left(R T+\left(\frac{1}{2}
%   \left(u^2+v^2+w^2\right)-e\right) (\gamma -1)\right)-\text{vH} & -u v (\gamma -1) & H-v^2 (\gamma -1) & -v w (\gamma -1) & (\gamma -1) v+v
%\end{array}
%\right)
%\end{equation}

\subsubsection{The Implicit Operator: Species/Flow Decoupling}
The goal is to construct an implicit operator such that the inviscid
flux Jacobian of the part that is treated explicitly has eigenvalues
that do not depend on the speed of sound while the implicit part does
not have any coupling between the species equations and the flow.
This can be accomplished if the inviscid flux Jacobian of the explicit
part takes the following form:
%
\begin{equation*}
A_{\mathrm{exp}} = \left[ \begin{array}{cccccc}
v I & -\rho_{\alpha} v / \rho & 0 & \rho_{\alpha} / \rho & 0 & 0 \\
0 & 0 & 0 & 1 & 0 & 0 \\
0 & 0 & v & 0 & 0 & 0 \\
\pp{p}{\rho_{\alpha}} & \xi & 0 & v  & 0 & 0 \\
0 & 0 & 0 & 0 & v & 0 \\
v \pp{p}{\rho_{\alpha}} & 0 & 0 & 0  & 0 & v 
\end{array} \right],
\end{equation*}
%
where
%
\begin{equation*}
\xi = -\sum_{\alpha=2}^{N_s} c_{\alpha} \pp{p}{\rho_{\alpha}}.
\end{equation*}
% 
A straightforward but tedious analysis shows that the eigenvalues of
$A_{\mathrm{exp}}$, denoted $\lambda_{\mathrm{exp}}$ satisfy
%
\begin{equation*}
-\lambda_{\exp} \left( v - \lambda_{\mathrm{exp}} \right)^5 = 0.
\end{equation*}
%
Thus,
%
\begin{equation*}
\lambda_{\mathrm{exp}} = v \quad \mathrm{or} \quad \lambda_{\mathrm{exp}} = 0,
\end{equation*}
% 
which clearly achieves the goal of removing acoustics from the
explicit operator.

Thus, the required implicit operator is given by
%
\begin{equation*}
A_{\mathrm{imp}} = \pp{G}{U} - A_{\mathrm{exp}} = \left[ \begin{array}{cccccc}
0 & 0 & 0 & 0 & 0 & 0 \\
0 & 0 & 0 & 0 & 0 & 0 \\
0 & -uv & 0 & u & 0 & 0 \\
0 & -v^2 +  \pp{p}{\rho} - \xi & \pp{p}{\rho u} & v + \pp{p}{\rho v} & \pp{p}{\rho w} & \pp{p}{\rho E} \\
0 & -wv & 0 & w & 0 & 0 \\
0 & -v \left(E + \frac{p}{\rho} - \pp{p}{\rho}\right) & v \pp{p}{\rho u} & E + \frac{p}{\rho} + v \pp{p}{\rho v}  & v \pp{p}{\rho w} & v \pp{p}{\rho E}
\end{array} \right].
\end{equation*}
%

Note that if further decoupling is required (to speed up the hybrid
implicit/explicit time step), it is possible to decouple the
streamwise and spanwise momentum equations without introducing the
speed of sound into the eigenvalues of the linearized explicit
operator.  However, this decoupling does introduce a $u$ and $w$
dependence.  For brevity, this analysis is omitted here because this
additional decoupling has not been implemented.

%From this, we construct a linearized acoustic matrix of the form,
%\begin{equation}
%\tiny
%L_{\text{acoustic}} = 
%\left(
%\begin{array}{cccccc}
% 0 & 0 & 0 & 0 & 0 & 0 \\
% 0 & v & 0 & 0 & 0 & 0 \\
% 0 & -u v & 0 & u & 0 & 0 \\
% 0 & -v^2+R T+\left(\frac{1}{2} \left(u^2+v^2+w^2\right)-e\right) (\gamma -1)+\frac{\left((\gamma -1) \left(\frac{1}{2} \left(u^2+v^2+w^2\right)-e_{\alpha }\right)+T
%   \left(R_{\alpha }-R\right)\right) \rho _{\alpha }}{\rho } & -u (\gamma -1) & v-v (\gamma -1) & -w (\gamma -1) & \gamma -1 \\
% 0 & -v w & 0 & w & 0 & 0 \\
% 0 & v \left(R T+\left(\frac{1}{2} \left(u^2+v^2+w^2\right)-e\right) (\gamma -1)\right)-\text{vH} & -u v (\gamma -1) & H-v^2 (\gamma -1) & -v w (\gamma -1) & v (\gamma -1)
%\end{array}
%\right)
%\end{equation}
%This system has eigenvalues of the following form,
%\begin{equation}
% \lambda  \left(-(v-\lambda )^5\right)
%\end{equation}
%Which is precisely what we desire. No acoustic effects are present, just
%convective velocities or zero.


%\subsection{Reference quantities}
%The reference quantities required are $u,v,w,\rho,\gamma$ and,
%\begin{equation}
%v \left(R T+\left(\frac{1}{2} \left(u^2+v^2+w^2\right)-e\right) (\gamma -1)\right)-\text{vH}
%\end{equation}
%and,
%\begin{equation}
%-v^2+R T+\left(\frac{1}{2} \left(u^2+v^2+w^2\right)-e\right) (\gamma -1)+\frac{\left((\gamma -1) \left(\frac{1}{2} \left(u^2+v^2+w^2\right)-e_{\alpha }\right)+T
%   \left(R_{\alpha }-R\right)\right) \rho _{\alpha }}{\rho }
%\end{equation}
%and finally,
%\begin{equation}
%H-v^2 (\gamma -1)
%\end{equation}


%\subsubsection{Further Decoupling Streamwise and Spanwise Momentum}
%Now we only consider treating $\rho e$, $\rho v$ and $\rho$ implicitly
%(neglecting both species equations as well as $\rho u$ and $\rho
%w$). The matrix has the following form:
%\begin{equation}
%\tiny
%\left(
%\begin{array}{cccccc}
% 0 & 0 & 0 & 0 & 0 & 0 \\
% 0 & v & 0 & 0 & 0 & 0 \\
% 0 & 0 & 0 & 0 & 0 & 0 \\
% 0 & -v^2+R T+\left(\frac{1}{2} \left(u^2+v^2+w^2\right)-e\right) (\gamma -1)+\frac{\left((\gamma -1) \left(\frac{1}{2} \left(u^2+v^2+w^2\right)-e_{\alpha }\right)+T \left(R_{\alpha
%   }-R\right)\right) \rho _{\alpha }}{\rho } & 0 & v-v (\gamma -1) & 0 & \gamma -1 \\
% 0 & 0 & 0 & 0 & 0 & 0 \\
% 0 & v \left(R T+\left(\frac{1}{2} \left(u^2+v^2+w^2\right)-e\right) (\gamma -1)\right)-\text{vH} & 0 & H-v^2 (\gamma -1) & 0 & v (\gamma -1)
%\end{array}
%\right)
%\end{equation}
%
%This gives eigenvalues of the form, 
%\begin{equation}
%(v-\lambda )^4 \left(\lambda ^2+(\gamma -1) \left(u^2+w^2\right)-\lambda  v\right)
%\end{equation}


\subsection{Viscous Contribution}
\label{sec:viscous_implicit_operator}
The wall-normal component of the viscous flux is given by
%
\begin{equation*}
G_{\mathrm{visc}} = \left[ \begin{array}{c}
\rho \mcal{D}_{\alpha} \pp{c_{\alpha}}{y} \\
0 \\
\tau_{21} \\
\tau_{22} \\
\tau_{23} \\
\sum_{s=1}^{N_s} \rho D_{\alpha} h_{\alpha} \pp{c_{\alpha}}{y} + \tau_{2i} u_i - \kappa \pp{T}{y}
\end{array} \right].
\end{equation*}
% 
Note that the signs are correct for the viscous term appearing on the
RHS of the Navier-Stokes equations.

To form the appropriate viscous contribution to the implicit
operator, we are interested in the Jacobian of $G_{visc}$ with respect
to $\partial U / \partial y$.  We have
%
\begin{equation*}
\pp{G_{\mathrm{visc}}}{(\partial U/ \partial y)} = \left[ \begin{array}{cccccc}
\mcal{D}_{\alpha} & -c_{\alpha} \mcal{D}_{\alpha} & 0 & 0 & 0 & 0 \\
0 & 0 & 0 & 0 & 0 & 0 \\
0 & - \mu \frac{u}{\rho} & \frac{\mu}{\rho} & 0 & 0 & 0 \\
0 & - \left(\alpha + \frac{4}{3} \right) \mu \frac{v}{\rho} & 0 & \left(\alpha + \frac{4}{3} \right) \frac{\mu}{\rho} & 0 & 0 \\
0 & - \mu \frac{u}{\rho} & 0 & 0 & \frac{\mu}{\rho} & 0 \\
* & * & * & * & * & \kappa \pp{T}{\rho E} \\
\end{array} \right],
\end{equation*}
% 
where the $*$ entries are non-zero, but, as will be shown below,
irrelevant to the current analysis, and
%
\begin{equation*}
\pp{T}{\rho E} = \frac{1}{\rho C_{v,\text{mix}}}.
\end{equation*}
% 
It is straightforward but tedious to show that the eigenvalues
$\lambda_{\mathrm{visc}}$ of this Jacobian matrix satisfy
%
\begin{equation*}
(D_{\alpha} - \lambda_{\mathrm{visc}}) (-\lambda_{\mathrm{visc}}) (\nu - \lambda_{\mathrm{visc}})^2  \left( \left( \alpha + \frac{4}{3} \right) \nu - \lambda_{\mathrm{visc}} \right) \left( \frac{\kappa}{\rho C_{v,\text{mix}}} - \lambda_{\mathrm{visc}} \right) = 0,
\end{equation*}
% 
where $\nu = \mu/\rho$.

Thus, these eigenvalues can be moved into the implicit operator with
a simple, diagonal treatment.  Specifically, the viscous part of the
implicit operator is given by
%
\begin{equation}
L_{\mathrm{visc}} = 
\left[
\begin{array}{cccccc}
 D_{\alpha} & 0 & 0 & 0 & 0 & 0 \\
 0 & 0 & 0 & 0 & 0 & 0 \\
 0 & 0 & \frac{\mu}{\rho} & 0 & 0 & 0 \\
 0 & 0 & 0 & \left(\alpha+\frac{4}{3}\right)\frac{\mu}{\rho} & 0 & 0 \\
 0 & 0 & 0 & 0 & \frac{\mu}{\rho} & 0 \\
 0 & 0 & 0 & 0 & 0 & \frac{\kappa}{\rho c_v} \\
\end{array}
\right]
\end{equation}


\subsection{The Complete, Discrete Operator}
\label{sec:complete_implicit_operator}
Loosely speaking, the complete implicit operator is given by
%
\begin{equation*}
M + \varphi \tilde{L} = \varphi \left( \frac{1}{\varphi} M - A_{\mathrm{imp}} D^{(1)} + L_{\mathrm{visc}} D^{(2)} \right),
\end{equation*}
% 
where $M$, $D^{(1)}$, and $D^{(2)}$ are the B-spline mass, first
derivative, and second derivative operators, respectively, and the
multiplications of $A_{\mathrm{imp}}$ and $L_{\mathrm{visc}}$ with the
appropriate derivative operators are interpreted correctly.
Specifically, the $N_y \times N_y$ derivative operators are scaled by
$N_y \times N_y$ diagonal matrices formed from reference profiles of
each entry in $A_{\mathrm{imp}}$ and $L_{\mathrm{visc}}$.  This
operator is further depicted in Figure~\ref{fig:discreteimplicitop}.

\begin{sidewaysfigure}
\newcommand{\entry}[1]{}          % Provides comments for subblocks
\newcommand{\C}[2]{C^{#1}_{#2}}   % For brevity below
\newcommand{\D}[1]{D^{(#1)}}      % ditto
\newcommand{\M}{M}                % ditto
\newcommand{\mx}{m_{x}}           % ditto
\newcommand{\my}{m_{y}}           % ditto
\newcommand{\mz}{m_{z}}           % ditto
\newcommand{\vp}{\varphi}         % ditto
\newcommand{\subcoeff}[3]{{       % ditto
   \renewcommand{\arraystretch}{2.0}
   \begin{Bmatrix}{#1}\\{#2}\\{#3}\end{Bmatrix}
}}
\hspace{-.04\textwidth}
{\resizebox{1.08\textwidth}{!}{\begin{minipage}[c]{\textwidth}  % SCALE-TO-FIT
\begin{align*}
\bm{\vp}
\renewcommand{\arraystretch}{1.0} % Adds whitespace between rows
\addtolength{\arraycolsep}{-.1em}
\begin{bmatrix}
%
% species density row
%
% is this here?
  \entry{\rho_{\alpha}\rho_{\alpha}}
  \subcoeff{
 \frac{1}{\vp}
  }{
  }{
  D_{\alpha}
  }
& \entry{\rho\rho_{\alpha} }
 0
& \entry{\rho\mx }
 0
& \entry{\rho\my }
 0
& \entry{\rho\mz }
 0
& \entry{\rho{}e }
 0
%
% Density row
%
 \\\entry{\rho_{\alpha}\rho}
 0
& \entry{\rho\rho }
  \subcoeff{
 \frac{1}{\vp}
  }{
 0
  }{
 0
  }
& \entry{\rho\mx }
 0
& \entry{\rho\my }
 0
& \entry{\rho\mz }
 0
& \entry{\rho{}e }
 0
% Streamwise momentum row
\\\entry{\mx\rho_{\alpha} }
 0
& \entry{\mx\rho }
  \subcoeff{
  }{
  uv
  }{
  }
& \entry{\mx\mx  }
 \subcoeff{
 \bm{\frac{1}{\vp}} % M
  }{
  }{
   \frac{\mu}{\rho}
  }
& \entry{\mx\my  }
  \subcoeff{
  }{
  -u
  }{
  }
& \entry{\mx\mz  }
 0
& \entry{\mx{}e  }
 0
% Wall-normal momentum row
\\\entry{\my\rho_{\alpha} }
 0
& 
\entry{\my\rho }
  \subcoeff{
  }{
%-v^2+R T+\left(\frac{1}{2} \left(u^2+v^2+w^2\right)-e\right) (\gamma -1)+\frac{\left((\gamma -1) \left(\frac{1}{2} \left(u^2+v^2+w^2\right)-e_{\alpha }\right)+T \left(R_{\alpha }-R\right)\right) \rho _{\alpha }}{\rho }\
  v^2 - \pp{p}{\rho} - \sum_{\alpha=2}^{N_s} c_{\alpha} \pp{p}{\rho_{\alpha}}
  }{
  }
& \entry{\my\mx  }
  \subcoeff{
  }{
    -\pp{p}{\rho u}
  }{
  }
& \entry{\my\my  }
  \subcoeff{
 \bm{\frac{1}{\vp}} % M
  }{
    -v -\pp{p}{\rho v}
  }{
 (\alpha +\frac{4}{3}) \frac{\mu}{\rho}
  }
& \entry{\my\mz  }
  \subcoeff{
  }{
 -\pp{p}{\rho w}
  }{
  }
& \entry{\my{}e  }
  \subcoeff{
  }{
 - \pp{p}{\rho E}
  }{
  }
% Spanwise momentum row
\\\entry{\mz\rho_{\alpha} }
0
& 
\entry{\mz\rho}
  \subcoeff{
  }{
    vw
  }{
  }
& \entry{\mz\mx }
 0
& \entry{\mz\my }
  \subcoeff{
  }{
    -w
  }{
  }
& \entry{\mz\mz }
  \subcoeff{
 \bm{\frac{1}{\vp}} % M
  }{
 0 
  }{
    \frac{\mu}{\rho}
  }
& \entry{\mz{}e }
 0
% Total energy row
\\\entry{e\rho_{\alpha} }
 0 
& 
\entry{e\rho  }
  \subcoeff{
  }{
    %v \left(R T+\left(\frac{1}{2} \left(u^2+v^2+w^2\right)-e\right) (\gamma -1)\right)-\text{vH}
    v \left( E + \frac{p}{\rho} - \pp{p}{\rho} \right)
  }{
  }
& \entry{e\mx   }
  \subcoeff{
  }{
    -v \pp{p}{\rho u}
  }{
  }
& \entry{e\my   }
  \subcoeff{
  }{
  -E - \frac{p}{\rho} - v \pp{p}{\rho v}
  }{
  }
& \entry{e\mz   }
  \subcoeff{
  }{
 -v \pp{p}{\rho w}
  }{
  }
& \entry{ee     }
  \subcoeff{
 \bm{\frac{1}{\vp}} % M
  }{
 - v \pp{p}{\rho E}
  }{
 \frac{\kappa}{\rho c_v}
  }
\end{bmatrix}
\renewcommand{\arraystretch}{0.6}
\begin{bmatrix}
  \hat{\rho_{\alpha}}_{\left(0,\,m,\,n\right)} \\
  \vdots \\
  \hat{\rho_{\alpha}}_{\left(N_y-1,\,m,\,n\right)} \\
\\%
\\%
\\%
\\%
  \hat{\rho}_{\left(0,\,m,\,n\right)} \\
  \vdots \\
  \hat{\rho}_{\left(N_y-1,\,m,\,n\right)} \\
\\%
\\%
\\%
\\%
  \hat{\mx}_{\left(0,\,m,\,n\right)} \\
  \vdots \\
  \hat{\mx}_{\left(N_y-1,\,m,\,n\right)} \\
\\%
\\%
\\%
\\%
  \hat{\my}_{\left(0,\,m,\,n\right)} \\
  \vdots \\
  \hat{\my}_{\left(N_y-1,\,m,\,n\right)} \\
\\%
\\%
\\%
\\%
  \hat{\mz}_{\left(0,\,m,\,n\right)} \\
  \vdots \\
  \hat{\mz}_{\left(N_y-1,\,m,\,n\right)} \\
\\%
\\%
\\%
\\%
  \hat{e}_{\left(0,\,m,\,n\right)} \\
  \vdots \\
  \hat{e}_{\left(N_y-1,\,m,\,n\right)} \\
%
\end{bmatrix}
\end{align*}
\end{minipage}}}  % END SCALE-TO-FIT!
\vspace{2em}
\\
\caption[The discrete operator $M+\varphi{}L$ used for implicit time advance]
{
    The complete discrete operator $M+\varphi{}L$ used for implicit time advance is
    depicted.  Notice the leftmost scalar factor $\bm{\vp}$.  The $3 N_y \times
    N_y$ blocked vectors surrounded by curly braces are to be ``dotted'' against
    the blocked vector $ \trans{\begin{bmatrix} \M & \D{1} & \D{2} \end{bmatrix}} $
    to form $N_y \times N_y$ subblocks.  Each of $M$, $\D{1}$, and $\D{2}$ is a
    $N_y \times N_y$ banded matrix. 
}
\label{fig:discreteimplicitop}
\end{sidewaysfigure}


