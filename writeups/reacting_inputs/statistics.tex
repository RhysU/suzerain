
\section{In support of Favre-averaged Navier--Stokes modeling}
\label{sec:supportFANS}

\todo{Edit the following for dimensional, reacting formulation}

The Favre-averaged Navier--Stokes (FANS) equations are often used to estimate
the mean effects of turbulence.  The unclosed FANS equations require modeling
approximations to be solvable.  Statistics gathered from Suzerain's solution of
the Navier--Stokes equations may be used to inform the development and
application of FANS closures.  Extensive background may be found in
\citet{Chassaing2010} or \citet{SmitsDussauge2005}.

The material in this section borrows liberally (and often literally) from
\citet{OliverFANSModels2011}.  It departs from that particular document in that
it employs Suzerain's constitutive relationships, avoids introducing customary
assumptions about the relative importance of unclosed terms, accounts for
forcing, and nondimensionalizes the results.

\subsection{Reynolds- and Favre-averages}
\label{sec:averaging}

The Reynolds average is simply the usual mean of a random variable.  Consider a
generic flow variable $q$.  The value, $q(x, y, z, t)$, of this variable at a
particular point in space, $(x, y, z)$, and time, $t$, is a random variable.
Assuming that the probability density function for $q(x, y, z, t)$ is given by
$\pi_q(V; x, y, z, t)$, the Reynolds average is defined by
%
\begin{equation}
\label{eqn:reynoldsAvg}
\bar{q}(x, y, z, t) \equiv \int V \pi_q(V; x, y, z, t) \,\mathrm{d} V.
\end{equation}
%
The Favre average is defined as the density-weighted average.  Thus,
denoting the fluid density by $\rho(x,y,z, t)$, the Favre average of
$q(x,y,z, t)$ is
%
\begin{equation*}
\tilde{q}(x,y,z, t) \equiv \frac{ \overline{\rho q}(x,y,z, t) }{ \bar{\rho}(x,y,z, t) }.
\end{equation*}
%
For the remainder of this section to make sense mathematically, it is
assumed that both the Reynolds and Favre averages are well-defined for
any required flow variable, $q$.  That is, the integral on the
right-hand side of (\ref{eqn:reynoldsAvg}) exists whenever required,
and the Reynolds-averaged density, $\bar{\rho}$, is positive
everywhere.

In the following, the flow variables will be decomposed into mean and
fluctuating parts.  Specifically, the fluctuations about the
mean---denoted by $(\cdot)'$ and $(\cdot)''$ for the Reynolds and
Favre averages, respectively---are defined by the following
relationships:
%
\begin{align*}
q' &\equiv q - \bar{q}, \\
q'' &\equiv q - \tilde{q}.
\end{align*}
%
Using the linearity of the Reynolds average and the fact that
$\bar{q}$ and $\tilde{q}$ are deterministic, not random, variables, it
is straightforward to see that
%
\begin{gather*}
\overline{q'} = \overline{q - \bar{q}} = \bar{q} - \bar{q} =  0, \\
\widetilde{q''} = \widetilde{q - \tilde{q}} = \tilde{q} - \tilde{q} = 0.
\end{gather*}
%
Furthermore,
%
\begin{equation*}
\overline{\rho q''} = \bar{\rho} \widetilde{q''} = 0.
\end{equation*}
%
However, in general,
%
\begin{equation*}
\overline{q''} = \overline{q - \tilde{q}} = \bar{q} - \tilde{q} \neq 0.
\end{equation*}
%

Wherever necessary, realizations of random fields of flow quantities
are assumed to be differentiable in both time and space so that Reynolds
averaging and differentiation commute.  For example,
%
\begin{equation*}
\overline{ \nabla{}u } = \nabla\bar{u}.
\end{equation*}
%
This commutativity is used to develop the FANS equations.  In contrast, Favre
averaging and differentiation do not, in general, commute:
\begin{align}
  \rho \nabla q &= \rho \nabla q
\\
   \rho \widetilde{\nabla{}q} + \rho \left(\nabla{}q\right)''
&=
   \rho \nabla \tilde{q} + \rho \nabla{}q''
\\
     \bar{\rho} \widetilde{\nabla{}q}
&=
     \bar{\rho} \nabla{\tilde{q}}
   + \overline{\rho \nabla{}q''}
\\
&=
     \bar{\rho} \nabla{\tilde{q}}
   - \overline{q''\nabla\rho}
\end{align}
Here the common convention that taking Favre fluctuations,
$\left(\cdot\right)''$, has higher precedence than differentiation,
$\nabla\left(\cdot\right)$, has been adopted.  Rearranging to better examine
the difference between $\widetilde{\nabla{}q}$ and $\nabla\tilde{q}$ in terms
of mean quantities,
\begin{align}
  \label{eq:favremeancommute}
  \widetilde{\nabla{}q}
  -
  \nabla{\tilde{q}}
&=
  \widetilde{\nabla{}q''}
= - \frac{{\overline{q''\nabla\rho}}}{\bar{\rho}}
= \frac{\tilde{q}\nabla\bar{\rho}}{\bar{\rho}}
  - \frac{\overline{q\nabla\rho}}{\bar{\rho}}
.
\end{align}
This lack of commutativity not problematic as it is not required to derive the
FANS equations.  It does, however, slightly complicate the mean constitutive
relationships.  The fluctuating gradient and the gradient of the fluctuations
differ according to
\begin{align}
  \label{eq:favrefluctcommute}
  \left(\nabla{}q\right)'' - \nabla{}q'' &= - \widetilde{\nabla{}q''}
.
\end{align}
In some circumstances, the difference between quantities written using a
fluctuating gradient and the gradient of the fluctuations can vanish.
One useful example is
\begin{align}
  \label{eq:favrefluctexample}
\widetilde{f''\left(\nabla{}g\right)''}
&=
\overline{\rho{}f''\left(\nabla{}g\right)''}
=
\overline{\rho{}f''\left(\nabla{}g'' - \widetilde{\nabla{}g''}\right)}
=
\overline{\rho{}f''\nabla{}g''}
- \overline{\rho{}f''}\widetilde{\nabla{}g''}
=
\widetilde{f''\nabla{}g''}
.
\end{align}

\subsection{The dimensional Favre-averaged Navier--Stokes equations}

\subsubsection{Derivation}

Recall that equations~\eqref{eq:dim_continuity}, \eqref{eq:dim_momentum},
and~\eqref{eq:dim_energy} may be written as
\begin{align}
    \frac{\partial}{\partial{}t}\rho
&=
  - \nabla\cdot\rho{}u
  + \Ssd_{\rho{}}
\\
    \frac{\partial{}}{\partial{}t}\rho{}u
&=
  - \nabla\cdot(u\otimes{}\rho{}u)
  - \nabla{}p + \nabla\cdot{}\tau + f
  + \Ssd_{\rho{} u}
\\
    \frac{\partial}{\partial{}t} \rho{}E
&=
  - \nabla\cdot{}\rho{}Hu
  + \nabla\cdot{}\tau{}u
  - \nabla\cdot{}q_{s}
  + f\cdot{}u
  + q_b
  + \Ssd_{\rho{} E}
\end{align}
where using total enthalpy $H$ reduces the number of terms in the energy
equation.

A lengthy algebraic procedure \citep[\textsection{}2]{OliverFANSModels2011}
produces exact equations governing the evolution of mean conserved quantities
$\bar{\rho}$, $\overline{\rho{}u}= \bar{\rho}\tilde{u}$, and
$\overline{\rho{}E} = \bar{\rho}\tilde{E}$:
\begin{subequations}\label{eq:unclosedfansequations}
\begin{align}
    \frac{\partial}{\partial{}t}\bar{\rho}
 =
 &- \nabla\cdot\bar{\rho}\tilde{u}
  + \overline{\Ssd_{\rho{}}}
\\
    \frac{\partial{}}{\partial{}t}\bar{\rho}\tilde{u}
 =
 &- \nabla\cdot(\tilde{u}\otimes\bar{\rho}\tilde{u})
  - \nabla{}\bar{p}
  + \nabla\cdot\left(
        \bar{\tau}
      - \bar{\rho}\widetilde{u''\otimes{}u''}
    \right)
  + \bar{f}
  + \overline{\Ssd_{\rho{} u}}
\\
  \frac{\partial}{\partial{}t} \bar{\rho}\tilde{E}
 =
 &- \nabla\cdot{}\bar{\rho}\tilde{H}\tilde{u}
  + \nabla\cdot\left(
        \left(
            \bar{\tau}
          - \bar{\rho} \widetilde{u''\otimes{}u''}
        \right) \tilde{u}
      - \frac{1}{2}\bar{\rho}\widetilde{{u''}^{2}u''}
      + \overline{\tau{}u''}
    \right)
\\
 &- \nabla\cdot\left(
        \bar{q}_s
      + \bar{\rho} \widetilde{h''u''}
    \right)
  + \bar{f}\cdot\tilde{u}
  + \overline{f\cdot{}u''}
  + \bar{q}_{b}
  + \overline{\Ssd_{\rho{} E}}
\end{align}
\end{subequations}
Several correlations impact the evolution of mean quantities: the Reynolds
stress $-\bar{\rho}\widetilde{u''\otimes{}u''}$, the Reynolds heat flux
$\bar{\rho} \widetilde{h''u''}$, turbulent transport
$-\frac{1}{2}\bar{\rho}\widetilde{{u''}^{2}u''}$, turbulent work
$\overline{\tau{}u''}$, and the forcing-velocity correlation
$\overline{f\cdot{}u''}$.  The Reynolds stress and heat flux augment the
viscous stress and heat flux, respectively.  The turbulent transport and work
terms represent transport of the turbulent kinetic energy density $k$, defined
below, and viscous stress work due to turbulent velocity fluctuations,
respectively.

We now average the perfect gas relations from section~\ref{sec:constitutive}.
The Reynolds average of~\eqref{eq:perfectgaseos} gives
\begin{align}
  \bar{p} &= R\overline{\rho{}T} = \bar{\rho}R\tilde{T}
\end{align}
while the Favre average of~\eqref{eq:perfectgasenthalpy} gives both
\begin{align}
 \tilde{H} &= \tilde{E} + R \tilde{T}
&
 \tilde{h} &= \frac{\gamma{}R\tilde{T}}{\gamma-1}.
\end{align}
The turbulent kinetic energy density
\begin{align}
  k &= \frac{1}{2}\widetilde{{u''}^2}
 \end{align}
arises from averaging the total energy given by
\eqref{eq:perfectgastotalenergy}:
\begin{align}
  \rho{} E
&=
  \frac{R}{\gamma-1} \rho{}T + \frac{1}{2}\rho{} u^{2}
\\
&=
  \frac{R}{\gamma-1} \rho{}\left( \tilde{T} + T'' \right)
+ \frac{1}{2}\rho{} \left( \tilde{u} + u'' \right)^2
\\
  \overline{\rho{}E}
&=
  \frac{R}{\gamma-1} \bar{\rho} \tilde{T}
+ \frac{1}{2}\bar{\rho} \tilde{u}^2
+ \frac{1}{2}\overline{\rho{}{u''}^2}
\\
  \tilde{E}
&=
  \frac{R}{\gamma-1} \tilde{T}
+ \frac{1}{2} \tilde{u}^2
+ k
\end{align}

An exact equation may be derived for the evolution of $k$
\citep[\textsection{}5]{OliverFANSModels2011}
\begin{align}
\label{eq:fanstke1}
    \frac{\partial{}}{\partial{}t}\bar{\rho}k
 =
 &- \nabla\cdot\bar{\rho}k\tilde{u}
  - \bar{\rho} \widetilde{u''\otimes{}u''} : \nabla\tilde{u}
  - \bar{\rho} \epsilon
  + \nabla\cdot\left(
        -\frac{1}{2}\bar{\rho}\widetilde{{u''}^{2}u''}
      + \overline{\tau{}u''}
    \right)
\\
 &- \overline{u''}\cdot\nabla\bar{p}
  - \nabla\cdot\overline{p' u''}
  + \overline{p' \nabla\cdot{}u''}
  + \overline{f\cdot{}u''}
  + \overline{\Ssd_{\rho{} u}\cdot{}u''}
\end{align}
where $A:B$ denotes $\trace \left(\trans{A} B\right)$, and the contribution of
the slow growth terms is being accounted for. The dissipation rate
density $\epsilon$, which governs the conversion rate from $k$ to mean internal
energy, is defined by
\begin{align}
  \bar{\rho} \epsilon &= \overline{\tau : \nabla{}u''}
.
\end{align}
As \citet[page 216]{Lele1994Compressibility} suggests, expanding $h$,
averaging, removing the mean state from both sides, and applying perfect gas
assumptions demonstrates the exact relationship
\begin{align}
  \overline{u''}
&=
  \frac{\widetilde{T''u''}}{\tilde{T}} - \frac{\overline{p'u''}}{\bar{p}}
.
\end{align}
Substituting $h''$ everywhere for $T''$, noting $\bar{p}/\tilde{h} =
\frac{\gamma-1}{\gamma}\bar{\rho}$, and differentiating one obtains
\begin{align}
  \overline{p'u''}
&=
  \frac{\gamma-1}{\gamma} \bar{\rho} \widetilde{h''u''}
- \bar{p} \overline{u''}
\\
  \nabla\cdot \overline{p'u''}
&=
  \frac{\gamma-1}{\gamma} \nabla\cdot \bar{\rho} \widetilde{h''u''}
- \bar{p}\nabla\cdot\overline{u''}
- \overline{u''}\cdot\nabla{}\bar{p}
.
\end{align}
Rearranging the above result to mimic terms within~\eqref{eq:fanstke1}
\begin{align}
  - \overline{u''}\cdot\nabla\bar{p}
  - \nabla\cdot\overline{p'u''}
&=
  \bar{p}\nabla\cdot\overline{u''}
- \frac{\gamma-1}{\gamma} \nabla\cdot \bar{\rho} \widetilde{h''u''}
\end{align}
allows trading an occurrence of $\overline{p'u''}$ for the Reynolds heat
flux in the exact $k$ equation:
\begin{align}
\label{eq:fanstke}
    \frac{\partial{}}{\partial{}t}\bar{\rho}k
 =
 &- \nabla\cdot\bar{\rho}k\tilde{u}
  - \bar{\rho} \widetilde{u''\otimes{}u''} : \nabla\tilde{u}
  - \bar{\rho} \epsilon
  + \nabla\cdot\left(
        -\frac{1}{2}\bar{\rho}\widetilde{{u''}^{2}u''}
      + \overline{\tau{}u''}
    \right)
\\
 &+ \bar{p}\nabla\cdot\overline{u''}
  - \frac{\gamma-1}{\gamma} \nabla\cdot\bar{\rho} \widetilde{h''u''}
  + \overline{p' \nabla\cdot{}u''}
  + \overline{f\cdot{}u''}
  + \overline{\Ssd_{\rho{} u}\cdot{}u''}
\end{align}
The trade reduces by one the number of correlations appearing in the $k$
equation which do not appear in the mean continuity, momentum, or energy
equations.  It also, as Lele suggests, encourages thermodynamic consistency
when working with pressure correlation information.

Returning to the constitutive relations, combining~\eqref{eq:tauSmub}
and~\eqref{eq:secondviscosityclaw} one obtains
\begin{align}
  \tau
&= 2 \mu{} S + \alpha \mu \left( \nabla\cdot{}u \right) I.
\end{align}
Using the kinematic viscosity $\nu = \mu / \rho$ and averaging,
\begin{align}
   \tilde{S}
&=
     \frac{1}{2}\left(
       \widetilde{\nabla{}u} + \trans{\widetilde{\nabla{}u}}
     \right)
   - \frac{1}{3}\left(\widetilde{\nabla\cdot{}u}\right) I
\\
  \bar{\tau}
&=
    2 \bar{\mu}\tilde{S}
  + 2 \bar{\rho} \widetilde{\nu''S''}
  + \alpha \bar{\mu} \widetilde{\nabla\cdot{}u} I
  + \alpha \bar{\rho} \widetilde{\nu''\left(\nabla\cdot{}u\right)''} I
.
\end{align}
By~\eqref{eq:favrefluctexample},
$\widetilde{\nu''\left(\nabla\cdot{}u\right)''}$ may also be written
$\widetilde{\nu''\nabla\cdot{}u''}$ while $\widetilde{\nu''S''}$ is equivalent
to a version using the deviatoric part of the strain rate of the fluctuating
velocity field.  Many FANS closure approximations neglect correlations between
the kinematic viscosity and velocity derivatives.  Many assume $\alpha=0$.
Accepting those approximations would eliminate the second through fourth terms
in $\bar{\tau}$.  Making the ubiquitous closure approximations
$\widetilde{\nabla{}u} + \trans{\widetilde{\nabla{}u}} \approx \nabla\tilde{u}
+ \trans{\nabla\tilde{u}}$ and
$\widetilde{\nabla{}\cdot{}u}\approx\nabla\cdot\tilde{u}$ are equivalent to
neglecting $\widetilde{\nabla{}u''} + \trans{\widetilde{\nabla{}u''}}$ and
$\widetilde{\nabla{}\cdot{}u''}$ per~\eqref{eq:favremeancommute}.

To find $\bar{q}_s$, combine~\eqref{eq:fourierlaw} and our assumption of a
constant Prandtl number
\begin{align}
  q_{s} &= - \kappa \nabla{} T
     = - \frac{\kappa}{C_p} \nabla{}h
     = - \frac{\mu}{\Prandtl} \nabla{}h
\end{align}
and again employ $\nu$ when averaging to obtain
\begin{align}
  \bar{q}_s
&= - \frac{1}{\Prandtl}\left(
                \bar{\mu}\widetilde{\nabla{}h}
              + \bar{\rho} \widetilde{\nu''\left(\nabla{}h\right)''}
            \right)
.
\end{align}
Again, by~\eqref{eq:favrefluctexample},
$\widetilde{\nu''\left(\nabla{}h\right)''}$ may also be written
$\widetilde{\nu''\nabla{}h''}$.  Again, making the ubiquitous closure
assumption $\widetilde{\nabla{}h}\approx\nabla\tilde{h}$ is equivalent to
neglecting $\widetilde{\nabla{}h''}$ per~\eqref{eq:favremeancommute}.
Straightforward averaging applied to~\eqref{eq:powerlawviscosity} produces
\begin{align}
   \bar{\rho}\tilde{\nu}
 = \bar{\mu}
&= \mu_0 \overline{\left(\frac{T}{T_0}\right)^\beta}
\end{align}
which is not computable given only Favre-averaged state.  One commonly accepted
simplification is taking $\overline{\mu\left(T\right)} \approx
\mu\left(\tilde{T}\right)$.

\subsubsection{Summary}

The Favre-averaged equations of interest are:
\begin{subequations}
\begin{align}
    \frac{\partial}{\partial{}t}\bar{\rho}
=
 &- \nabla\cdot\bar{\rho}\tilde{u}
  + \overline{\Ssd_{\rho{}}}
\\
    \frac{\partial{}}{\partial{}t}\bar{\rho}\tilde{u}
 =
 &- \nabla\cdot(\tilde{u}\otimes\bar{\rho}\tilde{u})
  - \nabla{}\bar{p}
  + \nabla\cdot\left(
        \bar{\tau}
      - \bar{\rho} \widetilde{u''\otimes{}u''}
    \right)
  + \bar{f}
  + \overline{\Ssd_{\rho{} u}}
\\
    \frac{\partial}{\partial{}t} \bar{\rho}\tilde{E}
 =
 &- \nabla\cdot{}\bar{\rho}\tilde{H}\tilde{u}
  + \nabla\cdot\left(
        \left(
            \bar{\tau}
          - \bar{\rho} \widetilde{u''\otimes{}u''}
        \right) \tilde{u}
      - \frac{1}{2}\bar{\rho}\widetilde{{u''}^{2}u''}
      + \overline{\tau{}u''}
    \right)
\\
 &- \nabla\cdot\left(
        \bar{q}_s
      + \bar{\rho} \widetilde{h''u''}
    \right)
  + \bar{f}\cdot\tilde{u}
  + \overline{f\cdot{}u''}
  + \bar{q}_b
  + \overline{\Ssd_{\rho{} E}}
\\
    \frac{\partial{}}{\partial{}t}\bar{\rho}k
=
 &- \nabla\cdot\bar{\rho}k\tilde{u}
  - \bar{\rho} \widetilde{u''\otimes{}u''} : \nabla\tilde{u}
  - \bar{\rho} \epsilon
  + \nabla\cdot\left(
        -\frac{1}{2}\bar{\rho} \widetilde{{u''}^{2}u''}
      + \overline{\tau{}u''}
    \right)
\\
 &+ \bar{p}\nabla\cdot\overline{u''}
  - \frac{\gamma-1}{\gamma} \nabla\cdot\bar{\rho} \widetilde{h''u''}
  + \overline{p' \nabla\cdot{}u''}
  + \overline{f\cdot{}u''}
  + \overline{\Ssd_{\rho{} u}\cdot{}u''}
\end{align}
The equations are augmented by the following relationships:
\begin{align}
  \bar{p} &= \bar{\rho}R\tilde{T}
&
   \bar{\rho}\tilde{\nu} =
   \bar{\mu}
&= \mu_0 \overline{\left(\frac{T}{T_0}\right)^\beta}
&
  k &= \frac{1}{2}\widetilde{{u''}^2}
&
  \bar{\rho} \epsilon &= \overline{\tau : \nabla{}u''}
\end{align}
\begin{align}
  \tilde{E}
&=
  \frac{R}{\gamma-1} \tilde{T}
+ \frac{1}{2} \tilde{u}^2
+ k
&
  \tilde{H}
&=
  \tilde{E}
+ R \tilde{T}
&
  \tilde{h} &= \frac{\gamma{}R\tilde{T}}{\gamma-1}
&
  \bar{q}_s
&= - \frac{1}{\Prandtl}\left(
                \bar{\mu}\widetilde{\nabla{}h}
              + \bar{\rho} \widetilde{\nu''\left(\nabla{}h\right)''}
            \right)
\end{align}
\begin{align}
   \tilde{S}
&=
     \frac{1}{2}\left(
       \widetilde{\nabla{}u} + \trans{\widetilde{\nabla{}u}}
     \right)
   - \frac{1}{3}\left(\widetilde{\nabla\cdot{}u}\right) I
&
   \bar{\tau}
&=  2 \bar{\mu}\tilde{S}
  + 2 \bar{\rho} \widetilde{\nu''S''}
  + \alpha \bar{\mu} \widetilde{\nabla\cdot{}u} I
  + \alpha \bar{\rho} \widetilde{\nu''\left(\nabla\cdot{}u\right)''} I
\end{align}
\end{subequations}
One may exactly compute the mean state evolution given the following
information:
\begin{align}
&\bar{\rho}
&
&\tilde{u}
&
&\tilde{E}
&
&\bar{\mu}
&
&\bar{f}
&
&\bar{q}_b
&
&k
&
&\epsilon
&
&\overline{u''}
&
&\symmetricpart{\widetilde{\nabla{}u}}
\end{align}
\begin{align}
&\overline{f\cdot{}u''}
&
&\overline{\tau{}u''}
&
&\overline{p'\nabla\cdot{}u''}
&
&-\widetilde{u''\otimes{}u''}
&
&-\frac{1}{2}\widetilde{{u''}^{2}u''}
&
&\widetilde{h''u''}
&
&\widetilde{\nu''S''}
&
&\widetilde{\nu''\left(\nabla\cdot{}u\right)''}
&
&\widetilde{\nu''\left(\nabla{}h\right)''}
\end{align}
\begin{align}
&\overline{\Ssd_{\rho{}}}
&
&\overline{\Ssd_{\rho{} u}}
&
&\overline{\Ssd_{\rho{} E}}
&
&\overline{\Ssd_{\rho{} u}\cdot{}u''}
\end{align}
Other ways to minimally capture the mean thermodynamic state, e.g. $\bar{\rho}$
and $\tilde{T}$, could have been chosen.  The information above are natural
choices following section~\ref{state_variable_selection}.  Favre-fluctuating
correlation densities (e.g.  $\widetilde{h''u''}$ are adopted for notational
brevity instead of their informationally equivalent Reynolds-averaged
correlations (e.g.  $\overline{\rho{}h''u''}$).

\subsection{The nondimensional Favre-averaged Navier--Stokes equations}
\label{sec:nondimfans}

The dimensional FANS equations from the last section need to be
nondimensionalized.   The reference quantity selections made in
section~\ref{sec:nondimrefq} are used and are augmented by
\begin{align}
  k_0 &= u_{0}^2
&
  \epsilon_0 &= \frac{u_{0}^2}{t_0}
\end{align}
Superscript star notation is suppressed as all terms
are dimensionless.  The results are:
\begin{subequations}
\begin{align}
    \frac{\partial}{\partial{}t}\bar{\rho}
=
 &- \nabla\cdot\bar{\rho}\tilde{u}
  + \overline{\Ssd_{\rho{}}}
\\
    \frac{\partial{}}{\partial{}t}\bar{\rho}\tilde{u}
=
 &- \nabla\cdot(\tilde{u}\otimes\bar{\rho}\tilde{u})
  - \frac{1}{\Mach^2}\nabla{}\bar{p}
  + \nabla\cdot\left(
        \frac{\bar{\tau}}{\Reynolds}
      - \bar{\rho} \widetilde{u''\otimes{}u''}
    \right)
  + \bar{f}
  + \overline{\Ssd_{\rho{} u}}
\\
  \frac{\partial}{\partial{}t} \bar{\rho}\tilde{E}
=
 &- \nabla\cdot\bar{\rho}\tilde{H}\tilde{u}
  + \Mach^{2} \nabla\cdot\left(
        \left(
            \frac{\bar{\tau}}{\Reynolds}
          - \bar{\rho} \widetilde{u''\otimes{}u''}
        \right) \tilde{u}
      - \frac{1}{2}\bar{\rho}\widetilde{{u''}^{2}u''}
      + \frac{\overline{\tau{}u''}}{\Reynolds}
    \right)
\\
 &+ \frac{1}{\gamma-1} \nabla\cdot\left(
      \frac{
         \bar{\mu} \widetilde{\nabla{}T}
       + \bar{\rho} \widetilde{\nu'' \left(\nabla{}T\right)''}
      }{\Reynolds\Prandtl}
      - \bar{\rho} \widetilde{T''u''}
    \right)
  + \Mach^{2} \left(
        \bar{f}\cdot\tilde{u}
      + \overline{f\cdot{}u''}
    \right)
  + \bar{q}_b
  + \overline{\Ssd_{\rho{} E}}
\\
    \frac{\partial{}}{\partial{}t}\bar{\rho}k
=
 &- \nabla\cdot\bar{\rho}k\tilde{u}
  - \bar{\rho} \widetilde{u''\otimes{}u''} : \nabla\tilde{u}
  - \frac{\bar{\rho} \epsilon}{\Reynolds}
  + \nabla\cdot\left(
        -\frac{1}{2}\bar{\rho} \widetilde{{u''}^{2}u''}
      + \frac{\overline{\tau{}u''}}{\Reynolds}
    \right)
\\
 &+ \frac{1}{\Mach^2} \left(
        \bar{p}\nabla\cdot\overline{u''}
      + \overline{p' \nabla\cdot{}u''}
      - \frac{1}{\gamma} \nabla\cdot\bar{\rho} \widetilde{T''u''}
    \right)
  + \overline{f\cdot{}u''}
  + \overline{\Ssd_{\rho{} u}\cdot{}u''}
\end{align}
The equations are augmented by the following nondimensional relationships:
\begin{align}
  \bar{p} &= \frac{\bar{\rho} \tilde{T}}{\gamma}
&
   \bar{\rho}\tilde{\nu} =
   \bar{\mu}
&= \overline{T^\beta}
&
  k &= \frac{1}{2}\widetilde{{u''}^2}
&
  \bar{\rho} \epsilon &= \overline{\tau : \nabla{}u''}
\end{align}
\begin{align}
  \tilde{E}
&=
  \frac{\tilde{T}}{\gamma\left(\gamma-1\right)}
  + \Mach^2 \left( \frac{1}{2}\tilde{u}^2 + k
  \right)
&
  \tilde{H}
&=
  \tilde{E} + \frac{\tilde{T}}{\gamma}
&
  \tilde{h} &= \frac{\tilde{T}}{\gamma-1}
\end{align}
\begin{align}
   \tilde{S}
&=
     \frac{1}{2}\left(
       \widetilde{\nabla{}u} + \trans{\widetilde{\nabla{}u}}
     \right)
   - \frac{1}{3}\left(\widetilde{\nabla\cdot{}u}\right) I
&
   \bar{\tau}
&=  2 \bar{\mu}\tilde{S}
  + 2 \bar{\rho} \widetilde{\nu''S''}
  + \alpha \bar{\mu} \widetilde{\nabla\cdot{}u} I
  + \alpha \bar{\rho} \widetilde{\nu''\left(\nabla\cdot{}u\right)''} I
\end{align}
\end{subequations}
where $\Reynolds$, $\Mach$, and $\Prandtl$ are defined as in
section~\ref{sec:nondimrefq}.  One may exactly compute the nondimensional mean
state evolution given the following statistical quantities:
\begin{align}
&\bar{\rho}
&
&\tilde{u}
&
&\tilde{E}
&
&\bar{\mu}
&
&\bar{f}
&
&\bar{q}_b
&
&k
&
&\epsilon
&
&\overline{u''}
&
&\symmetricpart{\widetilde{\nabla{}u}}
\end{align}
\begin{align}
&\overline{f\cdot{}u''}
&
&\overline{\tau{}u''}
&
&\overline{p'\nabla\cdot{}u''}
&
&-\widetilde{u''\otimes{}u''}
&
&-\frac{1}{2}\widetilde{{u''}^{2}u''}
&
&\widetilde{T''u''}
&
&\widetilde{\nu''S''}
&
&\widetilde{\nu''\left(\nabla\cdot{}u\right)''}
&
&\widetilde{\nu''\left(\nabla{}T\right)''}
\end{align}
\begin{align}
&\overline{\Ssd_{\rho{}}}
&
&\overline{\Ssd_{\rho{} u}}
&
&\overline{\Ssd_{\rho{} E}}
&
&\overline{\Ssd_{\rho{} u}\cdot{}u''}
\end{align}

Notice that the heat flux $\bar{q}_s$ has been merged into the mean energy
equation to better mimic \eqref{eq:nondim_energy}.  Enthalpy-based correlations
have been replaced by temperature-based correlations.  Notice also that the
choice of $p_0$ implies nondimensional $p$ includes a factor of $\gamma$
relative to the dimensional quantity.  Where possible, nondimensional
coefficients have been pulled out of the constitutive relationships and pushed
into the equations (for example, the factor $1/\Reynolds$ arising from the
dimensional definition of $\bar{\rho}\epsilon$).

Some locally computed quantities, e.g. the local Mach number, local Reynolds
number, or local eddy viscosity, require rescaling within to account for
nondimensionalization.  Other quantities, e.g. the local turbulent Prandtl
number, do not.  Writing the dimensional definition and re-expressing all
dimensional terms as a nondimensional value multiplied by a reference value
allows determining any appropriate multiplicative factors.

\subsection{Sampling logistics}

Statistical quantities are obtained by sampling from a well-resolved,
stationary simulation.  Mean quantity samples may be computed on-the-fly.
Fluctuating quantity samples, because they must be taken relative to an unknown
true mean, are computed by combining mean quantity samples following the rules
in section~\ref{sec:averaging}.

Sampling the following mean quantities is sufficient to compute
the statistical quantities listed in section~\ref{sec:nondimfans}:
\begin{align}
&\bar{\rho}
&
&\overline{\rho{}u}
&
&\overline{\rho{}E}
&
&\bar{\mu}
&
&\bar{f}
&
&\bar{q}_b
&
&\bar{u}
&
&\symmetricpart{\overline{\rho\nabla{}u}}
&
&\overline{\rho\nabla{}T}
&
&\overline{\tau:\nabla{}u}
&
&\overline{f\cdot{}u}
&
&\bar{\tau}
\end{align}
\begin{align}
&\overline{\tau{}u}
&
&\overline{p\nabla\cdot{}u}
&
&\overline{\rho{}u\otimes{}u}
&
&\overline{\rho{}u\otimes{}u\otimes{}u}
&
&\overline{\rho{}Tu}
&
&\overline{\mu{}S}
&
&\overline{\mu\nabla\cdot{}u}
&
&\overline{\mu\nabla{}T}
\end{align}
\begin{align}
&\overline{\Ssd_{\rho{}}}
&
&\overline{\Ssd_{\rho{} u}}
&
&\overline{\Ssd_{\rho{} E}}
&
&\overline{\Ssd_{\rho{} u}\cdot{}u}
\end{align}
These additional quantities are of interest and their so-far-unlisted
dependencies are also gathered:
\begin{align}
  \overline{\mu{}''}&= f\left(
      \bar{\rho}, \bar{\mu}, \overline{\rho\mu}
  \right)
  &
  \overline{\nu{}''} &= f\left(
      \bar{\rho}, \bar{\mu}, \bar{\nu}
  \right)
\end{align}
After averaging across the homogeneous streamwise $x$ and spanwise $z$
directions, each sample (ignoring tensor order) is a one-dimensional,
instantaneous profile varying only along the wall-normal B-spline direction.
The amount of data gathered may be reduced by exploiting symmetries in
$\bar{\tau}$, $\overline{\rho{}u\otimes{}u}$,
$\overline{\rho{}u\otimes{}u\otimes{}u}$, and $\overline{\mu{}S}$.

The instantaneous samples are combined according to the following ordered
sequence of computations to obtain the desired quantities:
{ \allowdisplaybreaks[1]
\begin{align}
  \tilde{u} &= \bar{\rho}^{-1} \overline{\rho{}u}
\\
  \tilde{E} &= \bar{\rho}^{-1} {\overline{\rho{}E}}
\\
  \widetilde{u\otimes{}u} &= \bar{\rho}^{-1} \overline{\rho{}u\otimes{}u}
\\
  \widetilde{u''\otimes{}u''} &=
  \widetilde{u\otimes{}u} - \tilde{u}\otimes\tilde{u}
\\
  k &= \frac{1}{2} \trace \widetilde{u''\otimes{}u''}
\\
  \tilde{T} &= \gamma\left(\gamma-1\right)\left(
      \tilde{E} - \Mach^2\left(\frac{1}{2}\tilde{u}^2 + k\right)
  \right)
\\
  \tilde{H} &= \tilde{E} + \frac{\tilde{T}}{\gamma}
\\
  \bar{p} &= \frac{\bar{\rho}\tilde{T}}{\gamma}
\\
  \overline{\tau:\nabla{}u''} &=
  \overline{\tau:\nabla{}u} - \bar{\tau}:\nabla\tilde{u}
\\
  \epsilon &= \bar{\rho}^{-1} \overline{\tau:\nabla{}u''}
\\
  \overline{u''} &= \bar{u} - \tilde{u}
\\
  \overline{f\cdot{}u''} &= \overline{f\cdot{}u} - \bar{f}\cdot{}\tilde{u}
\\
  \overline{\tau{}u''} &= \overline{\tau{}u} - \bar{\tau}\tilde{u}
\\
  \overline{p'\nabla\cdot{}u''}
  &= \overline{p\nabla\cdot{}u}
   - \bar{p}\nabla\cdot\bar{u}
\\
  \widetilde{u\otimes{}u\otimes{}u}
  &= \bar{\rho}^{-1} \overline{\rho{}u\otimes{}u\otimes{}u}
\\
  \overline{\Ssd_{\rho{} u}\cdot{}u''} 
  &= \overline{\Ssd_{\rho{} u}\cdot{}u} 
   - \overline{\Ssd_{\rho{} u}}\cdot{}\tilde{u}
\end{align}
}

Expressions for computing $\widetilde{u''\otimes{}u''\otimes{}u''}$ and
$-\frac{1}{2}\widetilde{{u''}^{2}u''}$ are derived in stages using index
notation and the summation convention.  Using the identities
\begin{align}
  \widetilde{u_{i}u_{j}''}
&=
  \widetilde{u_{i}u_{j}} - \tilde{u}_{i}\tilde{u}_{j}
\\
  \widetilde{u_{i}u_{j}u_{k}''}
&=
  \widetilde{u_{i}u_{j}u_{k}} - \widetilde{u_{i}u_{j}}\tilde{u}_{k}
\shortintertext{
and ignoring any symmetry-related simplifications,
}
  \widetilde{u_{i}''u_{j}''u_{k}''}
  &= \bar{\rho}^{-1} \overline{\rho{}\left(u_{i}-\tilde{u}_{i}\right)
                                     \left(u_{j}-\tilde{u}_{j}\right)
                                     u_{k}''}
\\
  &= \widetilde{u_{i}u_{j}u_{k}''}
   - \tilde{u}_i \widetilde{u_{j}u_{k}''}
   - \tilde{u}_j \widetilde{u_{i}u_{k}''}
%%\\
%%  &= \widetilde{u_{i}u_{j}u_{k}}
%%   - \widetilde{u_{i}u_{j}}\tilde{u}_{k}
%%   - \tilde{u}_{i} \widetilde{u_{j} u_{k}}
%%   + \tilde{u}_{i} \tilde{u}_{j}\tilde{u}_{k}
%%   - \tilde{u}_{j} \widetilde{u_{i}u_{k}}
%%   + \tilde{u}_{j} \tilde{u}_{i}\tilde{u}_{k}
\\
  &=   \widetilde{u_{i}u_{j}u_{k}}
   -   \widetilde{u_{i}u_{j}}\tilde{u}_{k}
   -   \tilde{u}_{i} \widetilde{u_{j} u_{k}}
   -   \tilde{u}_{j} \widetilde{u_{i}u_{k}}
   + 2 \tilde{u}_{i} \tilde{u}_{j}\tilde{u}_{k}
.
\shortintertext{
Contracting the first two indices and relabelling the third index,
}
  \widetilde{u_{i}''u_{i}''u_{j}''}
  &= \widetilde{u_{i}u_{i}u_{j}}
   - \widetilde{u_{i}u_{i}}\tilde{u}_{j}
   - 2 \tilde{u}_{i} \widetilde{u_{i} u_{j}}
   + 2 \tilde{u}_{i} \tilde{u}_{i}\tilde{u}_{j}
   .
\shortintertext{
Reverting to direct notation and employing the symmetry of
$\widetilde{u\otimes{}u}$,
}
  \widetilde{{u''}^{2}u''}
&=
      \widetilde{u^{2}u}
  -   \widetilde{u^{2}}\tilde{u}
  - 2 \widetilde{u\otimes{}u}\tilde{u}
  + 2 \tilde{u}^2 \tilde{u}
\end{align}
where $\widetilde{u^{2}u}$ and $\widetilde{u^2}$ may be computed by contracting
$\widetilde{u\otimes{}u\otimes{}u}$ and $\widetilde{u\otimes{}u}$,
respectively.

Continuing the ordered sequence of computations:
{ \allowdisplaybreaks[1]
\begin{align}
  \widetilde{Tu} &= \bar{\rho}^{-1} \overline{\rho{}Tu}
\\
  \widetilde{T''u''} &= \widetilde{Tu} - \tilde{T}\tilde{u}
\\
  \tilde{\mu} &= \bar{\rho}^{-1} \overline{\rho\mu}
\\
  \overline{\mu''} &= \bar{\mu} - \tilde{\mu}
\\
  \tilde{\nu} &= \bar{\rho}^{-1} \bar{\mu}
\\
  \overline{\nu''} &= \bar{\nu} - \tilde{\nu}
\\
  \symmetricpart{\widetilde{\nabla{}u}}
  &= \bar{\rho}^{-1} \symmetricpart{\overline{\rho\nabla{}u}}
\\
  \tilde{S} &= \symmetricpart{\widetilde{\nabla{}u}}
   - \frac{1}{3} \trace\symmetricpart{\widetilde{\nabla{}u}} I
\\
  \widetilde{\nu''S''}
  &= \bar{\rho}^{-1} \overline{\mu{}S} - \tilde{\nu}\tilde{S}
\\
  \widetilde{\nu''\left(\nabla\cdot{}u\right)''}
  &= \bar{\rho}^{-1} \overline{\mu\nabla\cdot{}u}
   - \tilde{\nu}\trace\symmetricpart{\widetilde{\nabla{}u}}
\\
  \widetilde{\nu''\left(\nabla{}T\right)''}
  &= \bar{\rho}^{-1} \overline{\mu\nabla{}T}
   - \tilde{\nu} \bar{\rho}^{-1} \overline{\rho\nabla{}T}
\end{align}
}

\subsection{Quantifying statistical quantity convergence}
\label{sec:quantconvergence}

TODO
