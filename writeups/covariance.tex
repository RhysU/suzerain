\documentclass[letterpaper,11pt,nointlimits,reqno]{amsart}

% Packages
\usepackage{algorithm}
\usepackage{algorithmic}
\usepackage{amsfonts}
\usepackage{amsmath}
\usepackage{amssymb}
\usepackage{fancyhdr}
\usepackage{fullpage}
\usepackage{lastpage}
\usepackage{mathrsfs}
\usepackage{mathtools}
\usepackage{parskip}
\usepackage{setspace}
\usepackage{txfonts}
\usepackage{varioref}

% Hyperref package must be last otherwise the contents are jumbled
% hypertexnames disabled to fix links pointing to incorrect locations
\usepackage[hypertexnames=false,final]{hyperref}

\mathtoolsset{showonlyrefs,showmanualtags}
\allowdisplaybreaks[1] % Allow grouped equations to be split across pages

% Line Spacing
\singlespacing

% Set appropriate header/footer information on each page
\fancypagestyle{plain}{
    \fancyhf{}
    \renewcommand{\headheight}{2.0em}
    \renewcommand{\headsep}{0.75em}
    \renewcommand{\headrulewidth}{1.0pt}
    \renewcommand{\footrulewidth}{0pt}
    \lhead{
        A covariance estimation procedure for nonuniformly
        sampled, autocorrelated processes.
    }
    \rhead{
        Page \thepage{} of \pageref{LastPage}
    }
}
\pagestyle{plain}

% Document-specific commands
\newcommand{\trans}[1]{{#1}^{\ensuremath{\mathsf{T}}}}
\newcommand{\OO}[1]{\operatorname{O}\left(#1\right)}
\DeclareMathOperator{\cov}{cov}

\begin{document}

\subsection*{Problem}

Assume the existence of two stationary, vector-valued stochastic processes
$\vec{\mathscr{X}}$ and $\vec{\mathscr{Y}}$ with finite, but unknown and
possibly distinct, first and second moments and decaying, but unknown and
possibly distinct, autocorrelations.  Estimate
$\cov\left(\vec{\mathscr{X}},\vec{\mathscr{Y}}\right)$ given two length $N$
sequences of samples $\vec{X}_{\iota}$ and $\vec{Y}_{\iota}$ where indices
$\iota\in{}I\subset\left[0,t\right)$ are strictly increasing but not
necessarily separated by a fixed $\Delta{}t$.

\subsection*{Assumptions}

Assume that $t$ is appreciably larger than the decay time of the
autocorrelation of both processes and that $I$ ``covers $\left[0,t\right)$
well-enough'' that interrogating $\OO{\sqrt{N}}$ consecutive samples will provide
adequate autocorrelation structure without running afoul of the
Nyquist sampling criterion.  $N$ must be large enough that standard estimates
of $\operatorname{E}\left[\mathscr{X}\right]$ and
$\operatorname{E}\left[\mathscr{Y}\right]$ may be taken as known means without
incurring too much bias as Percival~\cite{Percival1993Three} suggests they
might.  All-in-all, these imprecise restrictions are meant to suggest that
$\vec{X}_\iota$ and $\vec{Y}_\iota$ are in situ samples from a converged,
turbulence simulation.  The nonuniform sampling is intended to account for
possibly missing samples or sample time drift across a long-running simulation.

\subsection*{Approach}

The estimation process proceeds in four steps.  First, use a nonuniform
discrete Fourier transform followed by an inverse, uniform discrete Fourier
transform to project the nonuniform samples onto a uniform temporal grid.
Second, estimate the cross-spectra of the two projected sample sets in a way
maximally reducing the variance of these estimates given limited data.  Third,
compute the projection covariances from the projection cross-spectra estimates.
Finally, compute the unprojected covariances corresponding to the original
sample spacing.  For now, efficiency of the procedure is a secondary concern.

\subsection*{Projection onto uniform samples}

Define
$$
T = \frac{t N}{N-1}
$$
to be the sampling window for the purposes of a non-uniform discrete Fourier
transform (NDFT).  $T$ is chosen so a subsequent inverse fast Fourier transform
(IFFT) produces signals with the same physical wavelengths.  Define the
projection matrix $P$ by
$$
    P \vec{X}
    = 
    \frac{1}{N}
    \sum_j \exp\left(-2\pi \sqrt{-1} k t_j T^{-1} \right) X_j
$$
where $j$ indexes the strictly increasing entries of $I\subset\left[0,t\right)$

\subsection*{Cross-spectra estimation from uniform samples}

\subsection*{Covariance computation from cross-spectra estimates}

\subsection*{Reverse projection of the covariances}

\subsection*{Checking stationarity}

This procedure should be amenable to applying the Geweke
diagnostic~\cite{Geweke1992Evaluating} as a way to assess convergence.


%%%%%%%%%%%%%%%%%%%%%%%%%%%%%%%%%%%%%%%%%%%%%%%%%%%%%%%%%%%%%%%%%%%%
%%%%%%%%%%%%%%%%%%%%%%%%%%% Bibliography %%%%%%%%%%%%%%%%%%%%%%%%%%%
%%%%%%%%%%%%%%%%%%%%%%%%%%%%%%%%%%%%%%%%%%%%%%%%%%%%%%%%%%%%%%%%%%%%
\bibliographystyle{amsplain}
\bibliography{references}

\end{document}
