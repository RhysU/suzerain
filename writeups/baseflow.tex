\documentclass[letterpaper,11pt,nointlimits,reqno]{amsart}

% Packages
\usepackage{accents}
\usepackage{algorithm}
\usepackage{algorithmic}
\usepackage{amsfonts}
\usepackage{amsmath}
\usepackage{amssymb}
\usepackage{cancel}
\usepackage{enumerate}
\usepackage{fancyhdr}
\usepackage{fullpage}
\usepackage{ifthen}
\usepackage{lastpage}
\usepackage{latexsym}
\usepackage{mathrsfs}
\usepackage{mathtools}
\usepackage[numbers,sort&compress]{natbib}
\usepackage{parskip}
\usepackage{pstricks}
\usepackage{setspace}
\usepackage{txfonts}
\usepackage{units}
\usepackage{varioref}
\usepackage{wrapfig}

% Hyperref package must be last otherwise the contents are jumbled
% hypertexnames disabled to fix links pointing to incorrect locations
\usepackage[hypertexnames=false,final]{hyperref}

\mathtoolsset{showonlyrefs,showmanualtags}
\allowdisplaybreaks[1] % Allow grouped equations to be split across pages

% Line Spacing
\singlespacing

% Set appropriate header/footer information on each page
\fancypagestyle{plain}{
    \fancyhf{}
    \renewcommand{\headheight}{2.0em}
    \renewcommand{\headsep}{0.75em}
    \renewcommand{\headrulewidth}{1.0pt}
    \renewcommand{\footrulewidth}{0pt}
    \lhead{
        Favorable pressure gradient base flow computations
    }
    \rhead{
        Page \thepage{} of \pageref{LastPage}
    }
}
\pagestyle{plain}

% Document-specific commands

\begin{document}

This document discusses how a favorable pressure gradient base flow is obtained
for use within the temporally homogenized boundary layer formulation.  First, a
compressible potential flow subproblem is formulated for an axisymmetric source
or sink flow.  The boundary conditions for this subproblem are taken to match
some freestream state of interest.  Next, the resulting one-dimensional problem
is numerically solved using \citet{chebfunv4} and the solution transformed into
conserved variables.  Finally, the conserved state is mapped onto the first
quadrant of a Cartesian domain with state and derivatives collected from rays
oriented in the positive $y$ direction.

\section{The compressible potential flow equations}

Here we recall section II.A of \citet{Saad2011Coordinate} which derives a
coordinate independent velocity potential formulation of the compressible
potential flow equations.

TODO

\newcommand*{\doi}[1]{\href{http://dx.doi.org/\detokenize{#1}}{doi: #1}}
\bibliographystyle{plainnat}
\bibliography{references}

\end{document}
