\documentclass[letterpaper,11pt,nointlimits,reqno]{amsart}

% Packages
\usepackage{accents}
\usepackage{algorithm}
\usepackage{algorithmic}
\usepackage{amsfonts}
\usepackage{amsmath}
\usepackage{amssymb}
\usepackage{cancel}
\usepackage{enumerate}
\usepackage{fancyhdr}
\usepackage{fullpage}
\usepackage{ifthen}
\usepackage{lastpage}
\usepackage{latexsym}
\usepackage{mathrsfs}
\usepackage{mathtools}
\usepackage[numbers,sort&compress]{natbib}
\usepackage{parskip}
\usepackage{pstricks}
\usepackage{setspace}
\usepackage{txfonts}
\usepackage{units}
\usepackage{varioref}
\usepackage{wrapfig}

% Hyperref package must be last otherwise the contents are jumbled
% hypertexnames disabled to fix links pointing to incorrect locations
\usepackage[hypertexnames=false,final]{hyperref}

\mathtoolsset{showonlyrefs,showmanualtags}
\allowdisplaybreaks[1] % Allow grouped equations to be split across pages

% Line Spacing
\singlespacing

% Set appropriate header/footer information on each page
\fancypagestyle{plain}{
    \fancyhf{}
    \renewcommand{\headheight}{2.0em}
    \renewcommand{\headsep}{0.75em}
    \renewcommand{\headrulewidth}{1.0pt}
    \renewcommand{\footrulewidth}{0pt}
    \lhead{
        Favorable pressure gradient base flow computations
    }
    \rhead{
        Page \thepage{} of \pageref{LastPage}
    }
}
\pagestyle{plain}

% Document-specific commands
\newcommand{\Mach}[1][]{\ensuremath{\mbox{Ma}_{#1}}}

\begin{document}

This document discusses how a favorable pressure gradient base flow is obtained
for use within the temporally homogenized boundary layer formulation.  First, a
compressible potential flow subproblem is formulated for an axisymmetric source
or sink flow.  The boundary conditions for this subproblem are taken to match
some freestream state of interest.  Next, the resulting one-dimensional problem
is numerically solved using \texttt{Chebfun} by \citet{chebfunv4} and the
solution transformed into conserved variables.  Finally, the conserved state is
mapped onto the first quadrant of a Cartesian domain with state and derivatives
collected from rays oriented in the positive $y$ direction.

\section{The nondimensional, axisymmetric compressible potential flow equations}

A particularly crisp derivation of the coordinate-independent, velocity-potential
formulation for the compressible potential flow equations appears in section
II.A of \citet{Saad2011Coordinate}.  Begin by decomposing the dimensional
velocity $\vec{u}$ into a scalar and vector potential and then assume
that the flow is irrotational, \emph{viz.}
\begin{align}
  \vec{u} &= \vec{\nabla}\phi + \vec{\nabla}\times\vec{A} =  \vec{\nabla}\phi
.
\end{align}
Inserting the scalar potential into the continuity equation and assuming
smoothness,
\begin{align}
  0 &= \vec{\nabla}\cdot\rho\vec{u}
     = \vec{\nabla}\cdot\rho\vec{\nabla}\phi
     = \rho\vec{\nabla}^2\phi
     + \vec{\nabla}\rho\cdot\vec{\nabla}\phi
.
\end{align}
From the definition of the speed of sound $a$ notice
\begin{align}
    \frac{\mathrm{d}p}{\mathrm{d}\rho} = a^2
    &\implies
    \vec{\nabla}\rho = \frac{1}{a^2} \vec{\nabla} p
.
\end{align}
In a steady, inviscid, irrotational context the momentum equation yields
\begin{align}
    \vec{\nabla}p
    &= -\rho \vec{u}\cdot\vec{\nabla}\vec{u}
     = -\rho \left(   \frac{1}{2}\vec{\nabla}\vec{u}^2
                    - \vec{u}\times\vec{\nabla}\times\vec{u}
        \right)
     = - \frac{1}{2} \rho \vec{\nabla}\vec{u}^2
.
\end{align}
Rearranging the continuity result followed by two straightforward substitutions
shows
\begin{align}
    \vec{\nabla}^2\phi
    &= -\vec{\nabla}\phi\cdot\frac{1}{\rho}\vec{\nabla}\rho
     = -\vec{\nabla}\phi\cdot\frac{1}{\rho{}a^2}\vec{\nabla}p
     =  \vec{\nabla}\phi\cdot\frac{1}{2a^2} \vec{\nabla}\vec{u}^2
.
\end{align}
As expected, one obtains the Laplace equation for incompressible, irrotational
flow on taking $a \to \infty$.  Moving the denominator to the left
hand side and inserting the scalar potential,
\begin{align}
    2 a^2 \vec{\nabla}^2\phi
    &= \vec{\nabla}\phi\cdot \vec{\nabla}\vec{u}^2
     = \vec{\nabla}\phi\cdot \vec{\nabla}\left(\vec{\nabla}\phi\right)^2
\label{eq:cpfgibbs_nothermo_dim}
.
\end{align}

Next, \citeauthor{Saad2011Coordinate} employ that stagnation energy stays
constant throughout such a flow to connect the local speed of sound to
reference parameters and the local velocity.  That is, for flow enthalpy $h =
a^2 / \left(\gamma-1\right)$, reference enthalpy $h_0 = a_0^2 /
\left(\gamma_0-1\right)$, and reference velocity $\vec{u}_0$
\begin{align}
        \frac{a^2  }{\gamma  -1} + \frac{1}{2} \vec{u}^2
     &= \frac{a_0^2}{\gamma_0-1} + \frac{1}{2} \vec{u}_0^2
\end{align}
holds everywhere. From this energy constraint
\begin{align}
        2a^2
     &=   2a_0^2 \frac{\gamma-1}{\gamma_0-1}
        + \left(\gamma-1\right)\vec{u}_0^2
        - \left(\gamma-1\right)\vec{u}^2
\end{align}
may be used within~\eqref{eq:cpfgibbs_nothermo_dim} to produce the
\begin{align}
    \left[
          2a_0^2 \frac{\gamma-1}{\gamma_0-1}
        + \left(\gamma-1\right)
          \left(\vec{u}_0^2 - \left(\vec{\nabla}\phi\right)^2\right)
    \right] \vec{\nabla}^2\phi
     = \vec{\nabla}\phi\cdot \vec{\nabla}\left(\vec{\nabla}\phi\right)^2
\label{eq:cpfgibbs_dim}
.
\end{align}

To nondimensionalize, chose some reference length $l_0$ and declare
\begin{align}
    x     &= x^\ast l_0
&   u     &= u^\ast u_0
&   a     &= a^\ast a_0
&   \phi  &= \phi^\ast u_0 l_0
&   \rho  &= \rho^\ast \rho_0
&   p     &= p^\ast \rho_0 a_0^2
\end{align}
where starred quantities are dimensionless.  Inserting these definitions
into~\eqref{eq:cpfgibbs_dim},
\begin{align}
    \left[
          2a_0^2 \frac{\gamma-1}{\gamma_0-1}
        + u_0^2 \left(\gamma-1\right)
          \left(1 - \left(\vec{\nabla}^\ast\phi^\ast\right)^2\right)
    \right] \frac{u_0}{l_0} {\vec{\nabla}^\ast}^2\phi^\ast
     =       \frac{u_0^3}{l_0} \vec{\nabla}^\ast\phi^\ast
       \cdot \vec{\nabla}^\ast\left(\vec{\nabla}^\ast\phi^\ast\right)^2
.
\end{align}
Multiplying through by $\frac{l_0}{a_0^2 u_0}$, identifying $\Mach =
\frac{u_0}{a_0}$, and dropping the star notation,
\begin{align}
    \left[
          2 \frac{\gamma-1}{\gamma_0-1}
        + \Mach^2 \left(\gamma-1\right)
          \left(1 - \left(\vec{\nabla}\phi\right)^2\right)
    \right] {\vec{\nabla}}^2\phi
     =       \Mach^2 \vec{\nabla}\phi
       \cdot \vec{\nabla}\left(\vec{\nabla}\phi\right)^2
\label{eq:cpfgibbs_nondim}
.
\end{align}


\newcommand*{\doi}[1]{\href{http://dx.doi.org/\detokenize{#1}}{doi: #1}}
\bibliographystyle{plainnat}
\bibliography{references}

\end{document}
