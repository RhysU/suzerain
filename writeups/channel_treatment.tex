\documentclass[letterpaper,11pt,nointlimits,reqno]{amsart}

% Packages
\usepackage{accents}
\usepackage{algorithm}
\usepackage{algorithmic}
\usepackage{amsfonts}
\usepackage{amsmath}
\usepackage{amssymb}
\usepackage{cancel}
\usepackage{enumerate}
\usepackage{fancyhdr}
\usepackage{fullpage}
\usepackage{ifthen}
\usepackage{lastpage}
\usepackage{latexsym}
\usepackage{mathrsfs}
\usepackage{mathtools}
\usepackage[numbers,sort&compress]{natbib}
\usepackage{parskip}
\usepackage{pstricks}
\usepackage{setspace}
\usepackage{txfonts}
\usepackage{units}
\usepackage{varioref}
\usepackage{wrapfig}

% Hyperref package must be last otherwise the contents are jumbled
% hypertexnames disabled to fix links pointing to incorrect locations
\usepackage[hypertexnames=false,final]{hyperref}

\mathtoolsset{showonlyrefs,showmanualtags}
\allowdisplaybreaks[1] % Allow grouped equations to be split across pages

% Line Spacing
\singlespacing

% Set appropriate header/footer information on each page
\fancypagestyle{plain}{
    \fancyhf{}
    \renewcommand{\headheight}{2.0em}
    \renewcommand{\headsep}{0.75em}
    \renewcommand{\headrulewidth}{1.0pt}
    \renewcommand{\footrulewidth}{0pt}
    \lhead{
        Suzerain explicit isothermal channel treatment
    }
    \rhead{
        Page \thepage{} of \pageref{LastPage}
    }
}
\pagestyle{plain}

% Document-specific commands

\begin{document}

This document describes possible treatments for Coleman et al.'s isothermal,
compressible channel within Suzerain's numerical framework.

\section{Setting}

We advance nondimensional density $\rho$, momentum $\vec{m}=\rho{}\vec{u}$, and
total energy $e = \rho{}{E}$ according to
\begin{subequations}
\begin{align}
  \partial_{t} \rho{} &= R_{\rho}\!\left(\rho,\vec{m},e\right)
  \\
  \partial_{t} m      &= \vec{R}_{m}\!\left(\rho,\vec{m},e\right)
                       + \vec{f}_{m}(t)
  \\
  \partial_{t} e      &= R_{e}\!\left(\rho,\vec{m},e\right)
                       + \mbox{Ma}^{2} \vec{f}_{m}(t) \cdot \vec{u}
\end{align}
\end{subequations}
where $R_{\rho}$, $\vec{R}_{m}$, and $R_{e}$ are the spatial portion of the
compressible Navier--Stokes operator appropriate for a nondimensionalization
where $\mbox{Ma}=u_{0}/a_\text{wall}$.  We will determine the space-invariant
form of the uniform force $\vec{f}_{m}$ necessary to drive an isothermal
channel problem similar to Coleman et al.~\cite{Coleman1995Numerical}.

We assume a Fourier basis in the streamwise $x$ and spanwise $z$ direction as
well as an inhomogeneous treatment for wall-normal
$y\in\left[0,2\delta\right]$.  We assume that mean quantities can be obtained
as a function of wall-normal position using the constant (``zero zero'')
Fourier modes.  Under this ergodic assumption, we denote
$\rho\!\left(\cdot,y,\cdot\right) = \left<\rho\left(x,y,z\right)\right>_{x,z}$,
and likewise $\vec{m}\!\left(\cdot,y,\cdot\right) \coloneqq
\left<\vec{m}\left(x,y,z\right)\right>_{x,z}$.

We will work with the
nondimensional equation of state
\begin{align}
  p  = \frac{\rho{}T}{\gamma}
    &= \left(\gamma-1\right) \left(
          \rho{E} - \mbox{Ma}^2\frac{\vec{m}\cdot\vec{m}}{2\rho}
        \right)
\end{align}
where temperature $T$ has been nondimensionalized so that $T_\text{wall}=1$
though other choices should have a minimal impact on the analysis that follows.

Coleman et al.'s isothermal channel is characterized by
\begin{enumerate}
  \item A time-invariant bulk density $\bar{\rho} \coloneqq \frac{1}{2\delta}
        \int_{0}^{2\delta} \rho\!\left(\cdot,y,\cdot\right)\,dy $
  \item The no slip, isothermal boundary condition
        $\vec{u}=0$, $T = 1$ at $y = 0,2\delta$
  \item A time-invariant bulk streamwise momentum
        $\bar{m}_{x} \coloneqq \frac{1}{2\delta}
        \int_{0}^{2\delta} \bar{m}_{x}\!\left(\cdot,y,\cdot\right)\,dy $
\end{enumerate}


\section{Enforcing nonevolution via the nonlinear operator}

\subsection{Analysis}

The problem as presented should provide a stationary bulk density $\bar{\rho}$
when solved using conservative numerics.  The boundary condition $\vec{u} = 0$
at $y = 0,2\delta$ implies that at the wall
\begin{align}
  \partial_{t} \vec{m}
  =   \rho \cancel{\partial_{t} \vec{u}}
    + \cancel{\vec{u}} \partial_{t} \rho
  = 0.
\end{align}
The condition $T=1$ combined with the no-slip condition is equivalent to fixing
${E}=\frac{1}{\gamma\left(\gamma-1\right)}$.  This demands that
\begin{align}
  \partial_{t} e
  &=  \rho \cancel{\partial_{t} {E}}
    + {E} \partial_{t} \rho
  = \frac{1}{\gamma\left(\gamma-1\right)} \partial_{t} \rho
\end{align}
also holds at the wall.  Finally, we find $\vec{f}_{m} =
\left(f_{m_x},0,0\right)^{\textrm{T}}$ such that the bulk streamwise momentum
$\bar{m}_{x}$ remains constant in time:
\begin{equation}
  \partial_{t} \bar{m}_x
   = \frac{1}{2\delta} \int_{0}^{2\delta}
     \partial_{t} m_{x} \!\left(\cdot,y,\cdot\right) \, dy
   = \frac{1}{2\delta} \int_{0}^{2\delta}
     \left[  R_{m_x}\!\left(\cdot,y,\cdot\right) + f_{m_x} \right] \, dy
   = \frac{1}{2\delta} \int_{0}^{2\delta}
     R_{m_x}\!\left(\cdot,y,\cdot\right) \, dy + f_{m_x}
   = 0
\end{equation}
Rearranging we find $f_{m_x} = - \frac{1}{2\delta}
\int_{0}^{2\delta} R_{m_x}\!\left(\cdot,y,\cdot\right) \, dy$.

\subsection{Implementation}

We use that for
$\phi\in\left\{\rho,m_{x}\right\}$
\begin{align}
  \frac{1}{2\delta} \int_{0}^{2\delta} \phi\!\left(\cdot,y,\cdot\right)\,dy
  &=
  c \cdot \phi\!\left(\cdot,y,\cdot\right)
\end{align}
where $c$ is a precomputable coefficient vector depending on the chosen
wall-normal basis.

Assuming that the prior state satisfies the isothermal channel conditions with
bulk density $\bar{\rho}$, one (admittedly suboptimal) way to treat the
problem looks as follows:
\begin{enumerate}
  \item Compute $R_{\rho}$, $R_{\vec{m}}$, $R_{e}$,
        and $u_{x}\!\left(\cdot,y,\cdot\right)$ in wavespace.
  \item Set $\left.R_{\vec{m}}\right|_{y=0,2\delta} = 0$.
  \item Set $\left.R_{e}\right|_{y=0,2\delta} =
        \frac{1}{\gamma\left(\gamma-1\right)}
        \left.R_{\rho}\right|_{y=0,2\delta}$.
  \item Compute temporary value $\alpha =
        c \cdot R_{m_{x}}\!\left(\cdot,y,\cdot\right)$.
  \item Apply $f_{m_x}$ by updating
        $R_{m_x}\!\left(\cdot,y,\cdot\right) \verb!-=!\, \alpha$
        for $y\neq0,2\delta$.
  \item Apply $\mbox{Ma}^{2}\vec{f}_{m}\cdot\vec{u} = \mbox{Ma}^{2} f_{m_x}
        u_x$ by updating $R_{e}\!\left(\cdot,y,\cdot\right) \verb!-=!\,
        \mbox{Ma}^{2} \alpha \, u_{x}\!\left(\cdot,y,\cdot\right)$ for
        $y\neq0,2\delta$.
\end{enumerate}
One then advances $\partial_{t}\rho = R_{\rho}$, $\partial_{t} \vec{m} =
R_{\vec{m}}$, and $\partial_{t} e = R_{e}$ using regular time integration
techniques.  The above updates will cause the next state to also satisfy the
isothermal channel conditions.  Note that $f_{m_x}$ is not applied at
$y=0,2\delta$ but that these points may be included in the momentum integral
computation ($\alpha$) as they are identically zero from earlier boundary
condition enforcement.  Each of these computations may be performed in
wavespace as they are linear in at most one state variable.


\section{Enforcing a target bulk momentum via the linear operator}

\subsection{Analysis}

Suzerain's low-storage time advancement scheme may be modified to include a
streamwise momentum forcing with unknown leading coefficient $\phi$:
\begin{align}
  \left(M - \Delta{}t\beta_{i}L\right) {m_x}^{i+1}
  &=
  \left(M + \Delta{}t\alpha_{i}L\right) {m_x}^{i}
  + \Delta{}t\gamma_{i}\chi{}N\left({m_x}^{i}\right)
  + \Delta{}t\zeta_{i-1}\chi{}N\left({m_x}^{i-1}\right)
  + \phi{} f
  .
\end{align}
At each substep $\phi{} f$ can be chosen such that $\bar{m}_x$ obtains a target
value $\Gamma$.  This approach differs from the prior nonevolution one in that
$\bar{m}_x$ cannot drift.  Moreover, the target bulk momentum can be changed
easily between simulations.  Note that $\phi$ is implicitly a function of
$\Delta{}t$ and is only active during the ``$\alpha$'' portion of each substep.
That is,
\begin{align} % See personal notes dated 27 Sept 2012
  \phi{} f &= \alpha_i \Delta{}t \left(\frac{\phi{}f}{\alpha_i\Delta{}t}\right)
\end{align}
which has an impact on sampling the mean driving force for ensemble
averaging.

Specifically, the final implicit solve in each low storage substep resembles
${m_x}^{i+1} = A^{-1}\left(\vec{n} + \phi\vec{f}\right)$ where $A = \left(M -
\Delta{}t\beta_{i}L\right)$, $\vec{n}$ is an intermediate result, and $\vec{f}$
contains the value $1$ for $y\neq{}0,2\delta$.  Then
\begin{align}
\Gamma &= \bar{m}_x^{i+1}
        = c \cdot m_x^{i+1}
        = c \cdot{} A^{-1}\vec{n} + c \cdot{} \phi{}A^{-1}\vec{f}
\end{align}
requires selecting
\begin{align}
  \phi &= \frac{\Gamma - c \cdot{} A^{-1} \vec{n}}{c \cdot{} A^{-1} \vec{f}}
  \label{eq:phitarget}
\end{align}
to obtain the desired bulk streamwise momentum.  With $\phi$ known, the
associated substep energy contribution $f_{m_x} u_{x}$ can be determined.

Prior to this implicit forcing application, the wall collocation point state
must be modified to enforce a no-slip, isothermal condition by setting the
momentum to zero and the total energy $e = \rho {E} =
\frac{\rho}{\gamma\left(\gamma-1\right)}$.  As in the nonevolution approach,
the bulk density should be stationary without any special treatment.

\subsection{Implementation}

Following advice from Moser, this approach may be implemented as follows:
\begin{enumerate}
  \item During the nonlinear portion of each substep, compute and save
        $u_x\!\left(\cdot,y,\cdot\right)$ at all non-wall collocation points.
  \item Prior to computing $A^{-1}$, place the values $1$ and
        $u_x\!\left(\cdot,y,\cdot\right)$ into the otherwise unused imaginary
        components of the non-wall $m_x$ and $e$ ``zero zero'' collocation
        points, respectively. The wall values should be set to zero as we do
        not wish to apply forcing there.
  \item Compute $A^{-1}$ as usual.  The necessary implicit solves for the
        momentum and energy forcing are performed for free during the usual
        ``zero zero'' mode work.
  \item Compute $c\cdot{}A^{-1}m_x\!\left(\cdot,y,\cdot\right)$.
        The real and imaginary portions of the result are
        $c\cdot{}A^{-1}\vec{n}$ and $c\cdot{}A^{-1}\vec{f}$.
  \item Compute $\phi$ per \eqref{eq:phitarget}.
  \item Add $\phi{}c\cdot{}A^{-1}\vec{f}$ to $m_x\!\left(\cdot,y,\cdot\right)$
        and then clear the imaginary part.
  \item Add $\phi \, \mbox{Ma}^{2}$ times the imaginary part of
        $A^{-1}e\!\left(\cdot,y,\cdot\right)$ to the real part of
        $A^{-1}e\!\left(\cdot,y,\cdot\right)$ and then clear the imaginary
        part.
  \item Set $m_x$, $m_y$ and $m_z$ to be zero for $y=0,2\delta$
        taking care to account for working in coefficient versus collocation
        point space.  When $L = 0$ this can be done most simply prior
        to the solve.
  \item Set $e = \frac{\rho}{\gamma\left(\gamma-1\right)}$ for $y = 0,\delta$
        taking care to account for working in coefficient versus collocation
        point space.  Again, when $L = 0$ this can be done prior to the
        solve.
\end{enumerate}

\section{Enforcing a target bulk momentum via the linear operator (Redux)}

The prior approach, which includes directly accounting for no-slip, isothermal
walls by not forcing collocation point values is incorrect when operator $L
\neq 0$.  Hiding the solution cost for one integral constraint requires
applying special boundary condition logic to the ``zero zero'' mode as
real-valued boundary conditions must be separately applied to the real and
imaginary part of the vector.  Further, as the B-splines used in the
wall-normal direction are not discretely conservative it is desirable to add an
integral constraint on the bulk density $\bar{\rho}$.  Handling two distinct
constraints prevents hiding the cost entirely using the imaginary part of the
``zero zero'' collocation points.  Instead, separate constraint-solving logic
is necessary.  And, because $L$ may couple all equations, we must
simultaneously solve for all constraint coefficients.

The ideas leading up to \eqref{eq:phitarget} continue to hold.  With possible
coupling amongst all equations, the time step advance is
\begin{align}
  \left(M - \Delta{}t\beta_{i}L\right)
  \begin{bmatrix} e \\ m_x \\ m_y \\ m_z \\ \rho \end{bmatrix}^{i+1}
&=
  \left(M + \Delta{}t\alpha_{i}L\right)
  \begin{bmatrix} e \\ m_x \\ m_y \\ m_z \\ \rho \end{bmatrix}^{i}
  + \Delta{}t\gamma_{i}\chi{}N\left(
  \begin{bmatrix} e \\ m_x \\ m_y \\ m_z \\ \rho \end{bmatrix}^{i}
  \right)
  + \Delta{}t\zeta_{i-1}\chi{}N\left(
  \begin{bmatrix} e \\ m_x \\ m_y \\ m_z \\ \rho \end{bmatrix}^{i-1}
  \right)
  + \phi_\rho
  \begin{bmatrix} 0 \\ 0 \\ 0 \\ 0 \\ 1 \end{bmatrix}
  + \phi_{m_x}
  \begin{bmatrix} u \\ 1 \\ 0 \\ 0 \\ 0 \end{bmatrix}
  + \phi_{e}
  \begin{bmatrix} 1 \\ 0 \\ 0 \\ 0 \\ 0 \end{bmatrix}
\end{align}
where an additional bulk total energy constraint enforced through multiplier
$\phi_{e}$ has been introduced.  Constraining total energy can be thought of as
adding an automatically controlled heating term $q_b$ to the flow.  It can be
useful, e.g., to manually move past slow thermal transients prior to
stationarity.

Let $\Gamma_\rho$, $\Gamma_{m_x}$, and $\Gamma_{e}$ be the target values for
$\bar{\rho}$, $\overline{\rho{}u}$, and $\overline{\rho{}E}$, respectively.
Denote by $c_\rho$ the coefficient vector that, when dotted against the state
vector, will compute $\bar{\rho}$.  Similarly for $c_{m_x}$ and $c_e$.  Again,
$A = M - \Delta{}t \beta_i L$ and $\vec{n}$ is the intermediate right hand
side.

It follows that
\begin{align}
  \begin{bmatrix}
    c_\rho \cdot A^{-1} \begin{bmatrix} 0 \\ 0 \\ 0 \\ 0 \\ 1 \end{bmatrix}
    &
    c_\rho \cdot A^{-1} \begin{bmatrix} u \\ 1 \\ 0 \\ 0 \\ 0 \end{bmatrix}
    &
    c_\rho \cdot A^{-1} \begin{bmatrix} 1 \\ 0 \\ 0 \\ 0 \\ 0 \end{bmatrix}
    \\
    c_{m_x} \cdot A^{-1} \begin{bmatrix} 0 \\ 0 \\ 0 \\ 0 \\ 1 \end{bmatrix}
    &
    c_{m_x} \cdot A^{-1} \begin{bmatrix} u \\ 1 \\ 0 \\ 0 \\ 0 \end{bmatrix}
    &
    c_{m_x} \cdot A^{-1} \begin{bmatrix} 1 \\ 0 \\ 0 \\ 0 \\ 0 \end{bmatrix}
    \\
    c_{e} \cdot A^{-1} \begin{bmatrix} 0 \\ 0 \\ 0 \\ 0 \\ 1 \end{bmatrix}
    &
    c_{e} \cdot A^{-1} \begin{bmatrix} u \\ 1 \\ 0 \\ 0 \\ 0 \end{bmatrix}
    &
    c_{e} \cdot A^{-1} \begin{bmatrix} 1 \\ 0 \\ 0 \\ 0 \\ 0 \end{bmatrix}
  \end{bmatrix}
  \begin{bmatrix}
    \phi_\rho
    \\
    \phi_{m_x}
    \\
    \phi_{e}
  \end{bmatrix}
&=
  \begin{bmatrix}
    \Gamma_\rho - c_\rho \cdot{} A^{-1} \vec{n}
    \\
    \Gamma_{m_x} - c_{m_x} \cdot{} A^{-1} \vec{n}
    \\
    \Gamma_{e} - c_{e} \cdot{} A^{-1} \vec{n}
  \end{bmatrix}
.
\end{align}
This system determines the constraint coefficients $\phi_\rho$, $\phi_{m_x}$,
and $\phi_{e}$.  To be robust in the case where one or the other forcing is not
desired or where the forcing constraints otherwise become degenerate in some
fashion, a least squares solution to these constraints may be used.  Least
squares is also beneficial from an implementation perspective because it
permits zeroing the, say, third row and column in the matrix to force no
heating when it is not desired.

% FIXME May change when wall-transpiration employed as adding
%       multiple integral constraints could impact no-slip

%%%%%%%%%%%%%%%%%%%%%%%%%%%%%%%%%%%%%%%%%%%%%%%%%%%%%%%%%%%%%%%%%%%%
%%%%%%%%%%%%%%%%%%%%%%%%%%% Bibliography %%%%%%%%%%%%%%%%%%%%%%%%%%%
%%%%%%%%%%%%%%%%%%%%%%%%%%%%%%%%%%%%%%%%%%%%%%%%%%%%%%%%%%%%%%%%%%%%
\newcommand*{\doi}[1]{\href{http://dx.doi.org/\detokenize{#1}}{doi: #1}}
\bibliographystyle{plainnat}
\bibliography{references}


\end{document}
