\documentclass[letterpaper,11pt,nointlimits,reqno]{amsart}

% Packages
\usepackage{accents}
\usepackage{algorithm}
\usepackage{algorithmic}
\usepackage{amsfonts}
\usepackage{amsmath}
\usepackage{amssymb}
\usepackage{cancel}
\usepackage{enumerate}
\usepackage{fancyhdr}
\usepackage{fullpage}
\usepackage{ifthen}
\usepackage{lastpage}
\usepackage{latexsym}
\usepackage{mathrsfs}
\usepackage{mathtools}
\usepackage[numbers,sort&compress]{natbib}
\usepackage{parskip}
\usepackage{pstricks}
\usepackage{setspace}
\usepackage{txfonts}
\usepackage{units}
\usepackage{varioref}
\usepackage{wrapfig}

% Hyperref package must be last otherwise the contents are jumbled
% hypertexnames disabled to fix links pointing to incorrect locations
\usepackage[hypertexnames=false,final]{hyperref}

\mathtoolsset{showonlyrefs,showmanualtags}
\allowdisplaybreaks[1] % Allow grouped equations to be split across pages

% Line Spacing
\singlespacing

% Set appropriate header/footer information on each page
\fancypagestyle{plain}{
    \fancyhf{}
    \renewcommand{\headheight}{2.0em}
    \renewcommand{\headsep}{0.75em}
    \renewcommand{\headrulewidth}{1.0pt}
    \renewcommand{\footrulewidth}{0pt}
    \lhead{
        Suzerain explicit isothermal channel treatment
    }
    \rhead{
        Page \thepage{} of \pageref{LastPage}
    }
}
\pagestyle{plain}

% Document-specific commands

\begin{document}

This document describes possible treatments for Coleman et al.'s isothermal,
compressible channel within Suzerain's numerical framework.

\section{Setting}

We advance nondimensional density $\rho$, momentum $\vec{m}=\rho{}\vec{u}$, and
total energy $e = \rho{}\tilde{e}$ according to
\begin{subequations}
\begin{align}
  \partial_{t} \rho{} &= R_{\rho}\!\left(\rho,\vec{m},e\right)
  \\
  \partial_{t} m      &= \vec{R}_{m}\!\left(\rho,\vec{m},e\right)
                       + \vec{f}_{m}(t)
  \\
  \partial_{t} e      &= R_{e}\!\left(\rho,\vec{m},e\right)
                       + \mbox{Ma}^{2} \vec{f}_{m}(t) \cdot \vec{u}
\end{align}
\end{subequations}
where $R_{\rho}$, $\vec{R}_{m}$, and $R_{e}$ are the spatial portion of the
compressible Navier--Stokes operator appropriate for a nondimensionalization
where $\mbox{Ma}=u_{0}/a_\text{wall}$.  We will determine the space-invariant
form of the uniform force $\vec{f}_{m}$ necessary to drive an isothermal
channel problem similar to Coleman et al.~\cite{Coleman1995Numerical}.

We assume a Fourier basis in the streamwise $x$ and spanwise $z$ direction as
well as an inhomogeneous treatment for wall-normal
$y\in\left[0,2\delta\right]$.  We assume that mean quantities can be obtained
as a function of wall-normal position using the constant (``zero zero'')
Fourier modes.  Under this ergodic assumption, we denote
$\rho\!\left(\cdot,y,\cdot\right) = \left<\rho\left(x,y,z\right)\right>_{x,z}$,
and likewise $\vec{m}\!\left(\cdot,y,\cdot\right) \coloneqq
\left<\vec{m}\left(x,y,z\right)\right>_{x,z}$.

We will work with the
nondimensional equation of state
\begin{align}
  p  = \frac{\rho{}T}{\gamma}
    &= \left(\gamma-1\right) \left(
          \rho\tilde{e} - \mbox{Ma}^2\frac{\vec{m}\cdot\vec{m}}{2\rho}
        \right)
\end{align}
where temperature $T$ has been nondimensionalized so that $T_\text{wall}=1$
though other choices should have a minimal impact on the analysis that follows.

Coleman et al.'s isothermal channel is characterized by
\begin{enumerate}
  \item A time-invariant bulk density $\bar{\rho} \coloneqq \frac{1}{2\delta}
        \int_{0}^{2\delta} \rho\!\left(\cdot,y,\cdot\right)\,dy $
  \item The no slip, isothermal boundary condition
        $\vec{u}=0$, $T = 1$ at $y = 0,2\delta$
  \item A time-invariant bulk streamwise momentum
        $\bar{m}_{x} \coloneqq \frac{1}{2\delta}
        \int_{0}^{2\delta} \bar{m}_{x}\!\left(\cdot,y,\cdot\right)\,dy $
\end{enumerate}


\section{Enforcing nonevolution via the nonlinear operator}

\subsection{Analysis}

The problem as presented should provide a stationary bulk density $\bar{\rho}$
when solved using conservative numerics.  The boundary condition $\vec{u} = 0$
at $y = 0,2\delta$ implies that at the wall
\begin{align}
  \partial_{t} \vec{m}
  =   \rho \cancel{\partial_{t} \vec{u}}
    + \cancel{\vec{u}} \partial_{t} \rho
  = 0.
\end{align}
The condition $T=1$ combined with the no-slip condition is equivalent to fixing
$\tilde{e}=\frac{1}{\gamma\left(\gamma-1\right)}$.  This demands that
\begin{align}
  \partial_{t} e
  &=  \rho \cancel{\partial_{t} \tilde{e}}
    + \tilde{e} \partial_{t} \rho
  = \frac{1}{\gamma\left(\gamma-1\right)} \partial_{t} \rho
\end{align}
also holds at the wall.  Finally, we find $\vec{f}_{m} =
\left(f_{m_x},0,0\right)^{\textrm{T}}$ such that the bulk streamwise momentum
$\bar{m}_{x}$ remains constant in time:
\begin{equation}
  \partial_{t} \bar{m}_x
   = \frac{1}{2\delta} \int_{0}^{2\delta}
     \partial_{t} m_{x} \!\left(\cdot,y,\cdot\right) \, dy
   = \frac{1}{2\delta} \int_{0}^{2\delta}
     \left[  R_{m_x}\!\left(\cdot,y,\cdot\right) + f_{m_x} \right] \, dy
   = \frac{1}{2\delta} \int_{0}^{2\delta}
     R_{m_x}\!\left(\cdot,y,\cdot\right) \, dy + f_{m_x}
   = 0
\end{equation}
Rearranging we find $f_{m_x} = - \frac{1}{2\delta}
\int_{0}^{2\delta} R_{m_x}\!\left(\cdot,y,\cdot\right) \, dy$.

\subsection{Implementation}

We use that for
$\phi\in\left\{\rho,m_{x}\right\}$
\begin{align}
  \frac{1}{2\delta} \int_{0}^{2\delta} \phi\!\left(\cdot,y,\cdot\right)\,dy
  &=
  c \cdot \phi\!\left(\cdot,y,\cdot\right)
\end{align}
where $c$ is a precomputable coefficient vector depending on the chosen
wall-normal basis.

Assuming that the prior state satisfies the isothermal channel conditions with
bulk density $\bar{\rho}$, one (admittedly suboptimal) way to treat the
problem looks as follows:
\begin{enumerate}
  \item Compute $R_{\rho}$, $R_{\vec{m}}$, $R_{e}$,
        and $u_{x}\!\left(\cdot,y,\cdot\right)$ in wavespace.
  \item Set $\left.R_{\vec{m}}\right|_{y=0,2\delta} = 0$.
  \item Set $\left.R_{e}\right|_{y=0,2\delta} =
        \frac{1}{\gamma\left(\gamma-1\right)}
        \left.R_{\rho}\right|_{y=0,2\delta}$.
  \item Compute temporary value $\alpha =
        c \cdot R_{m_{x}}\!\left(\cdot,y,\cdot\right)$.
  \item Apply $f_{m_x}$ by updating
        $R_{m_x}\!\left(\cdot,y,\cdot\right) \verb!-=!\, \alpha$
        for $y\neq0,2\delta$.
  \item Apply $\mbox{Ma}^{2}\vec{f}_{m}\cdot\vec{u} = \mbox{Ma}^{2} f_{m_x}
        u_x$ by updating $R_{e}\!\left(\cdot,y,\cdot\right) \verb!-=!\,
        \mbox{Ma}^{2} \alpha \, u_{x}\!\left(\cdot,y,\cdot\right)$ for
        $y\neq0,2\delta$.
\end{enumerate}
One then advances $\partial_{t}\rho = R_{\rho}$, $\partial_{t} \vec{m} =
R_{\vec{m}}$, and $\partial_{t} e = R_{e}$ using regular time integration
techniques.  The above updates will cause the next state to also satisfy the
isothermal channel conditions.  Note that $f_{m_x}$ is not applied at
$y=0,2\delta$ but that these points may be included in the momentum integral
computation ($\alpha$) as they are identically zero from earlier boundary
condition enforcement.  Each of these computations may be performed in
wavespace as they are linear in at most one state variable.


\section{Enforcing a target bulk momentum via the linear operator}

\subsection{Analysis}

Suzerain's low-storage time advancement scheme may be modified to include a
streamwise momentum forcing with unknown leading coefficient $\phi$:
\begin{align}
  \left(M - \Delta{}t\beta_{i}L\right) {m_x}^{i+1}
  &=
  \left(M + \Delta{}t\alpha_{i}L\right) {m_x}^{i}
  + \Delta{}t\gamma_{i}\chi{}N\left({m_x}^{i}\right)
  + \Delta{}t\zeta_{i-1}\chi{}N\left({m_x}^{i-1}\right)
  + \phi{} f
  .
\end{align}
At each substep $\phi{} f$ can be chosen such that $\bar{m}_x$ obtains a target
value $\Gamma$.  Note that $\phi$ is implicitly a function of $\Delta{}t$.
This approach differs from the prior nonevolution one in that $\bar{m}_x$
cannot drift.  Moreover, the target bulk momentum can be changed easily between
simulations.

Specifically, the final implicit solve in each low storage substep resembles
${m_x}^{i+1} = A^{-1}\left(\vec{n} + \phi\vec{f}\right)$ where $A = \left(M -
\Delta{}t\beta_{i}L\right)$, $\vec{n}$ is an intermediate result, and $\vec{f}$
contains the value $1$ for $y\neq{}0,2\delta$.  Then $\Gamma = \bar{m}_x^{i+1}
= c \cdot m_x^{i+1} = c \cdot{} A^{-1}\vec{n} + \phi{}A^{-1}\vec{f}$ requires
selecting
\begin{align}
  \phi &= \frac{\Gamma - c \cdot{} A^{-1} \vec{n}}{c \cdot{} A^{-1} \vec{f}}
  \label{eq:phitarget}
\end{align}
to obtain the desired bulk streamwise momentum.  With $\phi$ known, the
associated substep energy contribution $f_{m_x} u_{x}$ can be determined.

Prior to this implicit forcing application, the wall collocation point state
must be modified to enforce a no-slip, isothermal condition by setting the
momentum to zero and the total energy $e = \rho \tilde{e} =
\frac{\rho}{\gamma\left(\gamma-1\right)}$.  As in the nonevolution approach,
the bulk density should be stationary without any special treatment.

\subsection{Implementation}

Following advice from Moser, this approach may be implemented as follows:
\begin{enumerate}
  \item During the nonlinear portion of each substep, compute and save
        $u_x\!\left(\cdot,y,\cdot\right)$ at all non-wall collocation points.
  \item Prior to computing $A^{-1}$, place the values $1$ and
        $u_x\!\left(\cdot,y,\cdot\right)$ into the otherwise unused imaginary
        components of the non-wall $m_x$ and $e$ ``zero zero'' collocation
        points, respectively. The wall values should be set to zero as we do
        not wish to apply forcing there.
  \item Compute $A^{-1}$ as usual.  The necessary implicit solves for the
        momentum and energy forcing are performed for free during the usual
        ``zero zero'' mode work.
  \item Compute $c\cdot{}A^{-1}m_x\!\left(\cdot,y,\cdot\right)$.
        The real and imaginary portions of the result are
        $c\cdot{}A^{-1}\vec{n}$ and $c\cdot{}A^{-1}\vec{f}$.
  \item Compute $\phi$ per \eqref{eq:phitarget}.
  \item Add $\phi{}c\cdot{}A^{-1}\vec{f}$ to $m_x\!\left(\cdot,y,\cdot\right)$
        and then clear the imaginary part.
  \item Add $\phi \, \mbox{Ma}^{2}$ times the imaginary part of
        $A^{-1}e\!\left(\cdot,y,\cdot\right)$ to the real part of
        $A^{-1}e\!\left(\cdot,y,\cdot\right)$ and then clear the imaginary
        part.
  \item Set $m_x$, $m_y$ and $m_z$ to be zero for $y=0,2\delta$
        taking care to account for working in coefficient versus collocation
        point space.  When $L = 0$ this can be done most simply prior
        to the solve.
  \item Set $e = \frac{\rho}{\gamma\left(\gamma-1\right)}$ for $y = 0,\delta$
        taking care to account for working in coefficient versus collocation
        point space.  Again, when $L = 0$ this can be done prior to the
        solve.
\end{enumerate}

%%%%%%%%%%%%%%%%%%%%%%%%%%%%%%%%%%%%%%%%%%%%%%%%%%%%%%%%%%%%%%%%%%%%
%%%%%%%%%%%%%%%%%%%%%%%%%%% Bibliography %%%%%%%%%%%%%%%%%%%%%%%%%%%
%%%%%%%%%%%%%%%%%%%%%%%%%%%%%%%%%%%%%%%%%%%%%%%%%%%%%%%%%%%%%%%%%%%%
\newcommand*{\doi}[1]{\href{http://dx.doi.org/\detokenize{#1}}{doi: #1}}
\bibliographystyle{plainnat}
\bibliography{references}


\end{document}
