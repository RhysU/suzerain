\documentclass[letterpaper,11pt,nointlimits,reqno]{amsart}

% Packages
\usepackage{accents}
\usepackage{algorithm}
\usepackage{algorithmic}
\usepackage{amsfonts}
\usepackage{amsmath}
\usepackage{amssymb}
\usepackage{cancel}
\usepackage{enumerate}
\usepackage{fancyhdr}
\usepackage{fullpage}
\usepackage{ifthen}
\usepackage{lastpage}
\usepackage{latexsym}
\usepackage{mathrsfs}
\usepackage{mathtools}
\usepackage{parskip}
\usepackage{pstricks}
\usepackage{setspace}
\usepackage{txfonts}
\usepackage{units}
\usepackage{varioref}
\usepackage{wrapfig}

% Hyperref package must be last otherwise the contents are jumbled
% hypertexnames disabled to fix links pointing to incorrect locations
\usepackage[hypertexnames=false,final]{hyperref}

\mathtoolsset{showonlyrefs,showmanualtags}
\allowdisplaybreaks[1] % Allow grouped equations to be split across pages

% Line Spacing
\singlespacing

% Set appropriate header/footer information on each page
\fancypagestyle{plain}{
    \fancyhf{}
    \renewcommand{\headheight}{2.0em}
    \renewcommand{\headsep}{0.75em}
    \renewcommand{\headrulewidth}{1.0pt}
    \renewcommand{\footrulewidth}{0pt}
    \lhead{
        Linearization of the Acoustics in the Inviscid Operator
    }
    \rhead{
        Page \thepage{} of \pageref{LastPage}
    }
}
\pagestyle{plain}

% Document-specific commands
\newcommand{\ii}{\ensuremath{\mathrm{i}}}
\newcommand{\htrans}[1]{{#1}^{\ensuremath{\mathsf{H}}}}
\newcommand{\OO}[1]{\operatorname{O}\left(#1\right)}

\begin{document}

In this document we detail the process by which we capture the linear
and non-linear (acoustic) terms of the hyperbolic Euler equations.

Define state as
\begin{equation}
 \phi = \{\rho, \rho u, \rho v, \rho w, \rho E \}
\end{equation}
from which one can write the 3D Euler Equations
\begin{equation}
 \partial_t \phi + \partial_x F(\phi) + \partial_y G(\phi) + \partial_z
  H(\phi) = 0
  .
\end{equation}
We linearize about a reference state $\phi_0 = \{\rho_0, {\rho u}_0,
{\rho v}_0, {\rho w}_0, {\rho E}_0 \}$ such that
\begin{equation}
 \partial_t \phi +  F_0 \partial_x \phi + G_0 \partial_y \phi + H_0 \partial_z
  \phi = 0
  .
\end{equation}
Let's consider only the wall-normal components for further analysis, as
these are expected to create the strictest criterion.  We assume that our
linearized flux Jacobian can be diagonalized, such that $G_0 = R \lambda
R^{-1}$, and so
\begin{equation}
 \partial_t \phi +  F_0 \partial_x \phi + R \lambda R^{-1} \partial_y \phi + H_0 \partial_z
  \phi = 0
\end{equation}


\end{document}
