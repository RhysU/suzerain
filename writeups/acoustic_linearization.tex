\documentclass[letterpaper,11pt,nointlimits,reqno]{amsart}

% Packages
\usepackage{accents}
\usepackage{algorithm}
\usepackage{algorithmic}
\usepackage{amsfonts}
\usepackage{amsmath}
\usepackage{amssymb}
\usepackage{cancel}
\usepackage{enumerate}
\usepackage{fancyhdr}
\usepackage{fullpage}
\usepackage{ifthen}
\usepackage{lastpage}
\usepackage{latexsym}
\usepackage{mathrsfs}
\usepackage{mathtools}
\usepackage{parskip}
\usepackage{pstricks}
\usepackage{setspace}
\usepackage{txfonts}
\usepackage{units}
\usepackage{varioref}
\usepackage{wrapfig}

% Hyperref package must be last otherwise the contents are jumbled
% hypertexnames disabled to fix links pointing to incorrect locations
\usepackage[hypertexnames=false,final]{hyperref}

\mathtoolsset{showonlyrefs,showmanualtags}
\allowdisplaybreaks[1] % Allow grouped equations to be split across pages

% Line Spacing
\singlespacing

% Set appropriate header/footer information on each page
\fancypagestyle{plain}{
    \fancyhf{}
    \renewcommand{\headheight}{2.0em}
    \renewcommand{\headsep}{0.75em}
    \renewcommand{\headrulewidth}{1.0pt}
    \renewcommand{\footrulewidth}{0pt}
    \lhead{
        Linearization of the Acoustics in the Inviscid Operator
    }
    \rhead{
        Page \thepage{} of \pageref{LastPage}
    }
}
\pagestyle{plain}

% Document-specific commands
\newcommand{\ii}{\ensuremath{\mathrm{i}}}
\newcommand{\htrans}[1]{{#1}^{\ensuremath{\mathsf{H}}}}
\newcommand{\OO}[1]{\operatorname{O}\left(#1\right)}

\begin{document}

This document details the process by which we isolate the linearized
acoustic terms within the hyperbolic Euler equations.

Define state as
\begin{equation}
 \phi = \{\rho, \rho u, \rho v, \rho w, \rho E \}
\end{equation}
from which one can write the 3D Euler Equations
\begin{equation}
 \partial_t \phi + \partial_x F(\phi) + \partial_y G(\phi) + \partial_z
  H(\phi) = 0
  .
\end{equation}
We linearize about a reference state $\phi_0 = \{\rho_0, {\rho u}_0,
{\rho v}_0, {\rho w}_0, {\rho E}_0 \}$ such that
\begin{equation}\label{eq:system}
 \partial_t \phi +  F_0 \partial_x \phi + G_0 \partial_y \phi + H_0 \partial_z
  \phi = 0
  .
\end{equation}
Let's consider only the wall-normal components for further analysis, as these
are expected to create the strictest criterion.  The linearized flux Jacobian
can be diagonalized per $G_0 = R \Lambda R^{-1}$ where $\Lambda$ is a diagonal
matrix containing the eigenvalues of $G_0$.  Without loss of generality, one
may write
\begin{equation}
  \Lambda = \operatorname{diag}\left[v+c,v-c,v,v,v\right]
\end{equation}
where it is the eigenvalues $v\pm{}c$ that lead to the most restrictive
CFL-related limitations for system~\eqref{eq:system}.  These are the acoustic
modes traveling at the speed of sound $c$ relative to the wall-normal velocity
$v$.  Decomposing
\begin{equation}\label{eq:eigdecomp}
  \Lambda = \Lambda_v + \Lambda_c =
  \operatorname{diag}\left[v,v,v,v,v\right]
  +
  \operatorname{diag}\left[c,-c,0,0,0\right]
\end{equation}
so that $ G_0 = R \left(\Lambda_v + \Lambda_c\right) R^{-1} $ permits recasting
\eqref{eq:system} as
\begin{equation}
    \partial_t \phi
 +  F_0 \partial_x \phi
 +  \underbrace{
      G_0 \partial_y \phi - R \Lambda_c R^{-1} \partial_y \phi
    }_{R \Lambda_v R^{-1}}
 +  \underbrace{R \Lambda_c R^{-1}}_{\text{Implicit}} \partial_y \phi
 +  H_0 \partial_z \phi
  = 0
  .
\end{equation}
One may approximately remove the wall-normal acoustic CFL restriction by
treating operator $R \Lambda_c R^{-1}$ in a linearly implicit fashion as these
are nothing but the terms giving rise to the $\pm{}c$ wave speeds in $\Lambda$.

Linearly-implicit treatment of $R \Lambda_c R^{-1}$ may be undesirably
expensive due to cross-equation coupling.  For example, the density will appear
within the total energy equation in a way decreasing the sparsity of the
resulting discrete operator.  Instead, one may choose to linearly treat a
perturbed operator $ R \tilde\Lambda_c R^{-1} $ with reduced coupling.  This
can be successful provided that the remaining, explicitly treated linear
portion of the Euler spatial operator, $ R \tilde\Lambda_v R^{-1} = G_0 - R
\tilde\Lambda_c R^{-1} $, has no eigenvalues markedly different from $v$ as
measured relative to $c$.

Notice \eqref{eq:eigdecomp} implies
\begin{align}
G_0 &= R \left(\Lambda_v + \Lambda_c\right) R^{-1}    \\
    &= R \left(v I + \Lambda_c\right) R^{-1}          \\
    &= R \left(v I\right) R^{-1} + R \Lambda_c R^{-1} \\
    &= vI + R \Lambda_c R^{-1}
\intertext{
so that
}
  \label{eq:G0Lambdac}
  R \Lambda_c R^{-1} &= G_0 - vI
  .
\end{align}
Though the particular form of $R$ and $G_0$ depends on our choice for $\phi$, a
result analogous to \eqref{eq:G0Lambdac} exists for any state variable
selection.

Temporarily consider the primitive state
\begin{equation}
  \psi = \left\{\rho, u, v, w, p\right\}
\end{equation}
for which the wall-normal Jacobian is
\begin{equation}
 G_{\psi,0} =
 \begin{bmatrix}
   v  &  0  &  \rho       & 0 &  0          \\
   0  &  v  &  0          & 0 &  0          \\
   0  &  0  &  v          & 0 &  \rho^{-1}  \\
   0  &  0  &  0          & v &  0          \\
   0  &  0  &  \gamma{}p  & 0 &  v          \\
 \end{bmatrix}
\end{equation}
By \eqref{eq:G0Lambdac},
\begin{equation}
 R_\psi \Lambda_C R_\psi^{-1} =
 \begin{bmatrix}
   0  &  0  &  \rho       & 0 &  0          \\
   0  &  0  &  0          & 0 &  0          \\
   0  &  0  &  0          & 0 &  \rho^{-1}  \\
   0  &  0  &  0          & 0 &  0          \\
   0  &  0  &  \gamma{}p  & 0 &  0          \\
 \end{bmatrix}
\end{equation}
which has eigenvalues $\left\{c, -c, 0, 0, 0\right\}$ using that
$c=\sqrt{\gamma p / \rho}$.  Decoupling the first equation by neglecting the
$\rho$ does not perturb the eigenvalues.  Neglecting either the $\rho^{-1}$ or
$\gamma p$ terms causes the reduced operator to no longer capture acoustics.

\end{document}
