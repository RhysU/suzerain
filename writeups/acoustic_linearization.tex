\documentclass[letterpaper,11pt,nointlimits,reqno]{amsart}

% Packages
\usepackage{accents}
\usepackage{algorithm}
\usepackage{algorithmic}
\usepackage{amsfonts}
\usepackage{amsmath}
\usepackage{amssymb}
\usepackage{cancel}
\usepackage{enumerate}
\usepackage{fancyhdr}
\usepackage{fullpage}
\usepackage{ifthen}
\usepackage{lastpage}
\usepackage{latexsym}
\usepackage{mathrsfs}
\usepackage{mathtools}
\usepackage{parskip}
\usepackage{pstricks}
\usepackage{setspace}
\usepackage{txfonts}
\usepackage{units}
\usepackage{varioref}
\usepackage{wrapfig}

% Hyperref package must be last otherwise the contents are jumbled
% hypertexnames disabled to fix links pointing to incorrect locations
\usepackage[hypertexnames=false,final]{hyperref}

\mathtoolsset{showonlyrefs,showmanualtags}
\allowdisplaybreaks[1] % Allow grouped equations to be split across pages

% Line Spacing
\singlespacing

% Set appropriate header/footer information on each page
\fancypagestyle{plain}{
    \fancyhf{}
    \renewcommand{\headheight}{2.0em}
    \renewcommand{\headsep}{0.75em}
    \renewcommand{\headrulewidth}{1.0pt}
    \renewcommand{\footrulewidth}{0pt}
    \lhead{
        Linearization of the acoustics in the inviscid operator
    }
    \rhead{
        Page \thepage{} of \pageref{LastPage}
    }
}
\pagestyle{plain}

% Document-specific commands
\newcommand{\ii}{\ensuremath{\mathrm{i}}}
\newcommand{\htrans}[1]{{#1}^{\ensuremath{\mathsf{H}}}}
\newcommand{\OO}[1]{\operatorname{O}\left(#1\right)}

\begin{document}

This document describes how to isolate the linearized
acoustic terms within the hyperbolic Euler equations.

Define state as
\begin{equation}
 \phi = \{\rho, \rho u, \rho v, \rho w, \rho E \}
\end{equation}
from which one can write the 3D Euler equations
\begin{equation}
 \partial_t \phi + \partial_x F(\phi) + \partial_y G(\phi) + \partial_z
  H(\phi) = 0
  .
\end{equation}
We linearize about a reference state $\phi_0 = \{\rho_0, \left\{\rho
u\right\}_0, \left\{\rho v\right\}_0, \left\{\rho w\right\}_0, \left\{\rho
E\right\}_0 \}$ such that
\begin{equation}\label{eq:system}
 \partial_t \phi +  F_0 \partial_x \phi + G_0 \partial_y \phi + H_0 \partial_z
  \phi = 0
  .
\end{equation}
Let's consider only the wall-normal components for further analysis, as these
are expected to create the strictest criterion.  The linearized flux Jacobian
can be diagonalized per $G_0 = R \Lambda R^{-1}$ where $\Lambda$ is a diagonal
matrix containing the eigenvalues of $G_0$.  Without loss of generality, one
may write
\begin{equation}
  \Lambda = \operatorname{diag}\left[v+c,v-c,v,v,v\right]
\end{equation}
where it is the eigenvalues $v\pm{}c$ that lead to the most restrictive
CFL-related limitations for system~\eqref{eq:system}.  These are the acoustic
modes traveling at the speed of sound $c$ relative to the wall-normal velocity
$v$.  Decomposing
\begin{equation}\label{eq:eigdecomp}
  \Lambda = \Lambda_v + \Lambda_c =
  \operatorname{diag}\left[v,v,v,v,v\right]
  +
  \operatorname{diag}\left[c,-c,0,0,0\right]
\end{equation}
so that $ G_0 = R \left(\Lambda_v + \Lambda_c\right) R^{-1} $ permits recasting
\eqref{eq:system} as
\begin{equation}
    \partial_t \phi
 +  F_0 \partial_x \phi
 +  \underbrace{
      G_0 \partial_y \phi - R \Lambda_c R^{-1} \partial_y \phi
    }_{R \Lambda_v R^{-1}}
 +  \underbrace{R \Lambda_c R^{-1}}_{\text{Linearly Implicit}} \partial_y \phi
 +  H_0 \partial_z \phi
  = 0
  .
\end{equation}
One may approximately remove the wall-normal acoustic CFL restriction by
treating operator $R \Lambda_c R^{-1}$ in a linearly implicit fashion as these
are nothing but the terms giving rise to the $\pm{}c$ wave speeds in $G_0$.

Notice \eqref{eq:eigdecomp} implies
\begin{align}
G_0 &= R \left(\Lambda_v + \Lambda_c\right) R^{-1}    \\
    &= R \left(v I + \Lambda_c\right) R^{-1}          \\
    &= R \left(v I\right) R^{-1} + R \Lambda_c R^{-1} \\
    &= vI + R \Lambda_c R^{-1}
\intertext{
so that
}
  \label{eq:G0Lambdac}
  R \Lambda_c R^{-1} &= G_0 - vI
  .
\end{align}
Though the particular form of $R$ and $G_0$ depends on our choice for $\phi$, a
result analogous to \eqref{eq:G0Lambdac} exists for any state variable
selection.

Recall the primitive variable state vector
\begin{equation}
  \psi = \left\{\rho, u, v, w, p\right\}
\end{equation}
for which the complete wall-normal Jacobian and, by \eqref{eq:G0Lambdac}, the
acoustic operator are as follows:
\begin{align}\label{eq:prim}
 G_{\psi,0} &=
 \begin{bmatrix}
   v  &  0  &  \rho       & 0 &  0          \\
   0  &  v  &  0          & 0 &  0          \\
   0  &  0  &  v          & 0 &  \rho^{-1}  \\
   0  &  0  &  0          & v &  0          \\
   0  &  0  &  \gamma{}p  & 0 &  v          \\
 \end{bmatrix}
&
 R_\psi \Lambda_c R_\psi^{-1} &=
 \begin{bmatrix}
   0  &  0  &  \rho       & 0 &  0          \\
   0  &  0  &  0          & 0 &  0          \\
   0  &  0  &  0          & 0 &  \rho^{-1}  \\
   0  &  0  &  0          & 0 &  0          \\
   0  &  0  &  \gamma{}p  & 0 &  0          \\
 \end{bmatrix}
\end{align}
As expected, the eigenvalues of $R_\psi \Lambda_c R_\psi^{-1}$ are $\left\{c,
-c, 0, 0, 0\right\}$ using that $c=\sqrt{\gamma p / \rho}$.  Employing a
similarity transformation permits mapping~\eqref{eq:prim} to our chosen state
variables $\phi$.\footnote{We could have simply written down $G_0-vI$ for state
$\phi$ but instead introduce the mapping as it will be reused momentarily.}
That is, we use
\begin{align}
  \frac{\partial \phi}{\partial \psi}
  &=
 \begin{bmatrix}
  1                      & 0      & 0      & 0      &  0                  \\
  u                      & \rho   & 0      & 0      &  0                  \\
  v                      & 0      & \rho   & 0      &  0                  \\
  w                      & 0      & 0      & \rho   &  0                  \\
  \frac{q}{2}  & \rho u & \rho v & \rho w &  \frac{1}{\gamma-1} \\
 \end{bmatrix}
&
  \frac{\partial \psi}{\partial \phi}
  &=
 \begin{bmatrix}
 1 & 0 & 0 & 0 & 0 \\
 -\frac{u}{\rho } & \frac{1}{\rho } & 0 & 0 & 0 \\
 -\frac{v}{\rho } & 0 & \frac{1}{\rho } & 0 & 0 \\
 -\frac{w}{\rho } & 0 & 0 & \frac{1}{\rho } & 0 \\
 \frac{\gamma-1}{2} q  & u(1-\gamma) & v(1-\gamma) & w(1-\gamma) & \gamma-1
 \end{bmatrix}
\end{align}
where $q=u^2+v^2+w^2$.  Then
\begin{align}
  R \Lambda_c R^{-1}
  &=
  \frac{\partial \phi}{\partial \psi}
  R_\psi \Lambda_c R_\psi^{-1}
  \frac{\partial \psi}{\partial \phi}
\\ \label{eq:cons}
 &=
\begin{bmatrix}
 -v & 0 & 1 & 0 & 0 \\
 -u v & 0 & u & 0 & 0 \\
 \frac{\gamma-1}{2}q -v^2 & u(1-\gamma)  & v(2-\gamma) & w(1-\gamma) & \gamma-1 \\
 -v w & 0 & w & 0 & 0 \\
 \frac{\gamma-2}{2} q v+\frac{v c^2}{1-\gamma} & u v (1-\gamma) & \frac{1}{2}q + (1-\gamma)v^2 + \frac{c^2}{\gamma-1} & v w (1-\gamma ) & v (\gamma-1)
\end{bmatrix}
.
\end{align}
This last statement seemingly implies that, for a conservative formulation,
fully treating linearized, wall-normal acoustics in an implicit fashion
requires some form of five-equation treatment with at least a three-equation
coupled solve.

The three-equation coupling required to implicitly treat \eqref{eq:cons} may be
undesirably due to cross-equation coupling.  Instead, one may choose to
linearly treat a perturbed operator $ R \tilde\Lambda_c R^{-1} $ with reduced
coupling.  This can be successful provided that the remaining, explicitly
treated linear portion of the Euler spatial operator, $ R \tilde\Lambda_v
R^{-1} = G_0 - R \tilde\Lambda_c R^{-1} $, has no eigenvalues markedly
different from $v$ as measured relative to $c$ and likewise no appreciably
different eigenvectors.

Neglecting either the $\rho^{-1}$ or $\gamma p$ terms within \eqref{eq:prim}
causes the reduced operator to no longer capture acoustic-like eigenvalues and
so we avoid considering those two options any further.  Decoupling the first
equation by neglecting the $\rho$ term in $R_\psi \Lambda_c R_\psi^{-1}$ to
obtain
\begin{equation}\label{eq:approxprim}
 R_\psi \tilde\Lambda_c R_\psi^{-1} =
 \begin{bmatrix}
   0  &  0  &  0          & 0 &  0          \\
   0  &  0  &  0          & 0 &  0          \\
   0  &  0  &  0          & 0 &  \rho^{-1}  \\
   0  &  0  &  0          & 0 &  0          \\
   0  &  0  &  \gamma{}p  & 0 &  0          \\
 \end{bmatrix}
\end{equation}
does not perturb the eigenvalues but it does perturb the eigenvectors.
Specifically, the $\pm{}a$ eigenvectors for $R_\psi \Lambda_c R_\psi^{-1}$,
defined by \eqref{eq:prim}, are
\begin{equation}\label{eq:primev}
\begin{bmatrix} \frac{1}{c^2} & 0 & -\frac{1}{c\rho} & 0 & 1 \end{bmatrix}^\top
\end{equation}
while for $R_\psi \tilde\Lambda_c R_\psi^{-1}$, depicted in
\eqref{eq:approxprim}, they are
\begin{equation}\label{eq:approxprimev}
\begin{bmatrix} 0 & 0 & -\frac{1}{c\rho} & 0 & 1 \end{bmatrix}^\top
.
\end{equation}
Whether this difference matters depends on the relative magnitudes of $c$,
$\rho$, and $1$ for some given reference state.  Under the decomposition
strategy \eqref{eq:eigdecomp}, we believe \eqref{eq:approxprim} decouples the
linearized acoustics to the maximum extent possible but its success is governed
by the relative error in the perturbed eigenvectors \eqref{eq:approxprimev}
relative to the true ones \eqref{eq:primev}.  Regardless of its suitability,
when mapped to use conservative state $\phi$ the perturbed operator becomes
\begin{align}
  R \tilde\Lambda_c R^{-1}
  &=
  \frac{\partial \phi}{\partial \psi}
  R_\psi \tilde\Lambda_c R_\psi^{-1}
  \frac{\partial \psi}{\partial \phi}
\\
 &=
\begin{bmatrix}
 0 & 0 & 0 & 0 & 0 \\
 0 & 0 & 0 & 0 & 0 \\
 \frac{1}{2} q (\gamma-1) & u(1-\gamma)  & v(1-\gamma)  & w(1-\gamma)  & \gamma-1  \\
 0 & 0 & 0 & 0 & 0 \\
 \frac{\gamma-1}{2} q v +\frac{v c^2}{1-\gamma} & u v (1-\gamma ) & v^2 (1-\gamma)+\frac{c^2}{\gamma-1} & v w (1-\gamma ) & v (\gamma-1 )
\end{bmatrix}
.
\end{align}
This approximate linearized acoustic treatment requires at least a two-equation
coupled solve.  It is the lowest cost treatment which could reasonably be
expected to succeed in specialized circumstances.

\end{document}
