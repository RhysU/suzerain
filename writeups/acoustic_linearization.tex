\documentclass[letterpaper,11pt,nointlimits,reqno]{amsart}

% Packages
\usepackage{accents}
\usepackage{algorithm}
\usepackage{algorithmic}
\usepackage{amsfonts}
\usepackage{amsmath}
\usepackage{amssymb}
\usepackage{cancel}
\usepackage{enumerate}
\usepackage{fancyhdr}
\usepackage{fullpage}
\usepackage{ifthen}
\usepackage{lastpage}
\usepackage{latexsym}
\usepackage{mathrsfs}
\usepackage{mathtools}
\usepackage{parskip}
\usepackage{pstricks}
\usepackage{setspace}
\usepackage{txfonts}
\usepackage{units}
\usepackage{varioref}
\usepackage{wrapfig}

% Hyperref package must be last otherwise the contents are jumbled
% hypertexnames disabled to fix links pointing to incorrect locations
\usepackage[hypertexnames=false,final]{hyperref}

\mathtoolsset{showonlyrefs,showmanualtags}
\allowdisplaybreaks[1] % Allow grouped equations to be split across pages

% Line Spacing
\singlespacing

% Set appropriate header/footer information on each page
\fancypagestyle{plain}{
    \fancyhf{}
    \renewcommand{\headheight}{2.0em}
    \renewcommand{\headsep}{0.75em}
    \renewcommand{\headrulewidth}{1.0pt}
    \renewcommand{\footrulewidth}{0pt}
    \lhead{
        $L^2$ computations using Suzerain's discretization
    }
    \rhead{
        Page \thepage{} of \pageref{LastPage}
    }
}
\pagestyle{plain}

% Document-specific commands
\newcommand{\ii}{\ensuremath{\mathrm{i}}}
\newcommand{\htrans}[1]{{#1}^{\ensuremath{\mathsf{H}}}}
\newcommand{\OO}[1]{\operatorname{O}\left(#1\right)}

\begin{document}

In this document we detail how to compute $L^2$ norm-related quantities using
Suzerain's discretization.  In this discretization, any instantaneous,
real-valued field $u\!\left(x,y,z\right)$ on the spatial domain
$\left[-\frac{L_x}{2},\frac{L_x}{2}\right] \times{} [0,L_y] \times{}
\left[-\frac{L_z}{2},\frac{L_z}{2}\right]$ is discretized as
\begin{align}
  u^h(x,y,z)
&=
  \sum_{l=0}^{N_y - 1}
  \sum_{m=-\frac{N_x}{2}}^{\frac{N_x}{2}-1}
  \sum_{n=-\frac{N_z}{2}}^{\frac{N_z}{2}-1}
  \hat{u}_{l m n}
  B_l\!\left(y\right)
  e^{\ii\frac{2\pi{}m}{L_x}x}
  e^{\ii\frac{2\pi{}n}{L_z}z}
=
  \sum_{l}\sum_{m}\sum_{n}
  \hat{u}_{l m n}B_l\!\left(y\right)e^{\ii k_m x}e^{\ii k_n z}
\end{align}
where $k_m = 2\pi{}m/L_x$, $k_n = 2\pi{}n/L_z$, and $B_l\!\left(y\right)$ are a
B-spline basis for some order and some knot selection.

\section{$L^2$ norm of a discrete field}

We wish to compute the $L^2$ norm of a discrete field $u^h$.  By direct
computation we find
\begin{align}
  \left|\left|
    u^h
  \right|\right|^{2}_{L^{2}_{xyz}}
&=
  \int_0^{L_y}
  \int_{-\frac{L_x}{2}}^{\frac{L_x}{2}}
  \int_{-\frac{L_z}{2}}^{\frac{L_z}{2}}
  u^h \overline{u^h}
  \,d\!z \,d\!x \,d\!y
\\ &\approx
  \int_0^{L_y}
  \int_{-\frac{L_x}{2}}^{\frac{L_x}{2}}
  \int_{-\frac{L_z}{2}}^{\frac{L_z}{2}}
  \left(
    \sum_{l}\sum_{m}\sum_{n}
    \hat{u}_{l m n}B_l\!\left(y\right)e^{\ii k_m x}e^{\ii k_n z}
  \right)
  \left(
    \sum_{l^\prime}\sum_{m^\prime}\sum_{n^\prime}
    \overline{\hat{u}_{l^\prime m^\prime n^\prime}}
    B_{l^\prime}\!\left(y\right)e^{-\ii k_{m^\prime} x}e^{-\ii k_{n^\prime} z}
  \right)
  \,d\!z \,d\!x \,d\!y
\\ &=
  \sum_{l}
  \sum_{l^\prime}
  \int_0^{L_y}
  \left(
    B_l\!\left(y\right)
    B_{l^\prime}\!\left(y\right)
    \sum_{m}
    \sum_{m^\prime}
    \int_{-\frac{L_x}{2}}^{\frac{L_x}{2}}
    \left(
      e^{\ii k_m x}
      e^{-\ii k_{m^\prime} x}
      \sum_{n}
      \sum_{n^\prime}
      \int_{-\frac{L_z}{2}}^{\frac{L_z}{2}}
      \left(
        e^{\ii k_n z}
        e^{-\ii k_{n^\prime} z}
        \hat{u}_{l m n}
        \overline{\hat{u}_{l^\prime m^\prime n^\prime}}
      \right)
      \,d\!z
    \right)
    \,d\!x
  \right)
  \,d\!y
\\ &=
  \sum_{l}
  \sum_{l^\prime}
  \int_0^{L_y}
  \left(
    B_l\!\left(y\right)
    B_{l^\prime}\!\left(y\right)
    \sum_{m}
    \sum_{m^\prime}
    \int_{-\frac{L_x}{2}}^{\frac{L_x}{2}}
    \left(
      e^{\ii k_m x}
      e^{-\ii k_{m^\prime} x}
      L_z \sum_{n} \hat{u}_{l m n} \overline{\hat{u}_{l^\prime m^\prime n}}
    \right)
    \,d\!x
  \right)
  \,d\!y
\\ &=
  \sum_{l}
  \sum_{l^\prime}
  \int_0^{L_y}
  \left(
    B_l\!\left(y\right)
    B_{l^\prime}\!\left(y\right)
    L_x L_z \sum_{m} \sum_{n} \hat{u}_{l m n} \overline{\hat{u}_{l^\prime m n}}
  \right)
  \,d\!y
\\ &=
   L_x L_z \sum_{m} \sum_{n}
  \sum_{l}
  \hat{u}_{l m n}
  \sum_{l^\prime}
  \overline{\hat{u}_{l^\prime m n}}
  \int_0^{L_y}
  B_l\!\left(y\right)
  B_{l^\prime}\!\left(y\right)
  \,d\!y
\\ &=
   L_x L_z \sum_{m} \sum_{n} \htrans{\hat{u}_{m n}} M \hat{u}_{m n}
   \quad\text{where}\quad
   M = \int_0^{L_y} B_l\!\left(y\right) B_{l^\prime}\!\left(y\right) \,d\!y
   \label{eq:L2uh}
   .
\end{align}
In the above expression the three-dimensional field of transform coefficients
$\hat{u}_{l m n}$ is treated as a two-dimensional collection of length $N_y$
vectors indexed by $m$ and $n$.  For a B-spline basis with piecewise polynomial
order $k - 1$ giving rise to a real-valued, symmetric positive definite matrix
$M$ with bandwidth $2k-1$, the computational cost to find $\left|\left| u^h
\right|\right|^{2}_{L^{2}_{xyz}}$ scales as $\OO{k N_x N_y^2 N_z}$.

The $y$-varying norm over the $x$ and $z$ directions can be computed via
\begin{align}
  \left|\left|
    u^h
  \right|\right|^{2}_{L^{2}_{xz}}
  \!(y)
&=
  \int_{-\frac{L_x}{2}}^{\frac{L_x}{2}}
  \int_{-\frac{L_z}{2}}^{\frac{L_z}{2}}
  u^h
  \overline{u^h}
  \,d\!z \,d\!x
\\ &\approx
  \int_{-\frac{L_x}{2}}^{\frac{L_x}{2}}
  \int_{-\frac{L_z}{2}}^{\frac{L_z}{2}}
  \left(
    \sum_{l}\sum_{m}\sum_{n}
    \hat{u}_{l m n}B_l\!\left(y\right)e^{\ii k_m x}e^{\ii k_n z}
  \right)
  \left(
    \sum_{l^\prime}\sum_{m^\prime}\sum_{n^\prime}
    \overline{\hat{u}_{l^\prime m^\prime n^\prime}}
    B_{l^\prime}\!\left(y\right)e^{-\ii k_{m^\prime} x}e^{-\ii k_{n^\prime} z}
  \right)
  \,d\!z \,d\!x
\\ &=
  \sum_{l}
  B_l\!\left(y\right)
  \sum_{l^\prime}
  B_{l^\prime}\!\left(y\right)
  \sum_{m}
  \sum_{m^\prime}
  \int_{-\frac{L_x}{2}}^{\frac{L_x}{2}}
  \left(
    e^{\ii k_m x}
    e^{-\ii k_{m^\prime} x}
    \sum_{n}
    \sum_{n^\prime}
    \int_{-\frac{L_z}{2}}^{\frac{L_z}{2}}
    \left(
      e^{\ii k_n z}
      e^{-\ii k_{n^\prime} z}
      \hat{u}_{l m n}
      \overline{\hat{u}_{l^\prime m^\prime n^\prime}}
    \right)
    \,d\!z
  \right)
  \,d\!x
\\ &=
  L_x L_z
  \sum_{l}
  B_l\!\left(y\right)
  \sum_{l^\prime}
  B_{l^\prime}\!\left(y\right)
  \sum_{m} \sum_{n} \hat{u}_{l m n} \overline{\hat{u}_{l^\prime m n}}
  .
\end{align}
Using the limited, symmetric support of the products $B_l\left(y\right)
B_{l^\prime}\left(y\right)$ to reduce the required operations, one can compute
this fully functional representation of $\left|\left| u^h
\right|\right|^{2}_{L^{2}_{xz}} \!(y)$ at a cost proportional to $\OO{k N_x
N_y^2 N_z}$.

When $\left|\left|u^h\right|\right|^{2}_{L^{2}_{xz}} \!(y)$ is
only of interest at points $y_j$, the previous expression yields
\begin{align}
  \left|\left|
    u^h
  \right|\right|^{2}_{L^{2}_{xz}}
  \!(y_j)
&\approx
  L_x L_z
  \sum_{m} \sum_{n}
  \left(
    \sum_{l}
    B_l\!\left(y_j\right)
    \hat{u}_{l m n}
  \right)
  \left(
    \sum_{l^\prime}
    B_{l^\prime}\!\left(y_j\right)
    \overline{\hat{u}_{l^\prime m n}}
  \right)
\\ &=
  L_x L_z
  \sum_{m} \sum_{n}
  \left|
    \sum_{l}
    B_l\!\left(y_j\right)
    \hat{u}_{l m n}
  \right|^{2}
\\ &=
  L_x L_z
  \sum_{m} \sum_{n}
  \left|
      \left(M \hat{u}_{m n}\right)\bigr|_{y=y_j}
  \right|^{2}
  \quad\text{where}\quad
  M = B_l\!\left(y_j\right)
\end{align}
where again the three-dimensional field is treated as as a two-dimensional
field of vectors.  The computational cost for this operation is much lower than
the others at $\OO{\left(2k-1\right) N_x N_y N_z}$.

These three results include summations over all coefficients in the homogeneous
$x$ and $z$ directions, namely $m \in \left\{-N_x/2, \dots, 0, \dots,
N_x/2-1\right\}$ and $n \in \left\{-N_z/2, \dots, 0, \dots, N_z/2-1\right\}$.
Because $u\!\left(x,y,z\right)$ is real-valued, its transform coefficients
$\hat{u}_{lmn}$ exhibit conjugate symmetry in one of the homogeneous
directions.  Often, not all of these coefficients are stored.  Consequently,
care must be exercised when evaluating summations like $\sum_m \sum_n$ within
such norm calculations.

More concretely, say one employs conjugate symmetry in the $x$ direction
when computing $f\!\left(\hat{u}_{lmn}\right)$.  Then
\begin{align}
  \sum_{m=-\frac{N_x}{2}}^{\frac{N_x}{2}-1}
  \sum_{n=-\frac{N_z}{2}}^{\frac{N_z}{2}-1}
  f\!\left(\hat{u}_{lmn}\right)
&=
  \sum_{n=-\frac{N_z}{2}}^{\frac{N_z}{2}-1}
  \left[
    \sum_{m=0}^{\frac{N_x}{2}-1}
    f\!\left(\hat{u}_{lmn}\right)
    +
    \sum_{m=-\frac{N_x}{2}}^{-1}
    f\!\left(\hat{u}_{lmn}\right)
  \right]
\\ &=
  \sum_{n=-\frac{N_z}{2}}^{\frac{N_z}{2}-1}
  \left[
    \sum_{m=0}^{\frac{N_x}{2}-1}
    f\!\left(\hat{u}_{lmn}\right)
    +
    \sum_{m=1}^{\frac{N_x}{2}}
    f\!\left(\overline{\hat{u}_{lmn}}\right)
  \right]
\\ &=
  \sum_{n=-\frac{N_z}{2}}^{\frac{N_z}{2}-1}
  \left[
    \left.f\!\left(\hat{u}_{lmn}\right)\right|_{m=0}
    +
    \sum_{m=1}^{\frac{N_x}{2}-1}
    \left[
      f\!\left(\hat{u}_{lmn}\right)
      +
      f\!\left(\overline{\hat{u}_{lmn}}\right)
    \right]
    +
    \left.f\!\left(\overline{\hat{u}_{lmn}}\right)\right|_{m=\frac{N_x}{2}}
  \right]
  .
\intertext{
When $f\!\left(\hat{u}_{lmn}\right) = f\!\left(\overline{\hat{u}_{lmn}}\right)$
holds, as it does for $\left|\left| u^h \right|\right|^{2}_{L^{2}_{xyz}}$,
$\left|\left| u^h \right|\right|^{2}_{L^{2}_{xz}} \!(y_j)$, and $\left|\left|
u^h \right|\right|^{2}_{L^{2}_{xz}} \!(y_j)$, the summands further simplify to
yield
}
  \sum_{m=-\frac{N_x}{2}}^{\frac{N_x}{2}-1}
  \sum_{n=-\frac{N_z}{2}}^{\frac{N_z}{2}-1}
  f\!\left(\hat{u}_{lmn}\right)
&=
  \sum_{n=-\frac{N_z}{2}}^{\frac{N_z}{2}-1}
  \left[
    \left.f\!\left(\hat{u}_{lmn}\right)\right|_{m=0}
    +
    2
    \sum_{m=1}^{\frac{N_x}{2}-1}
      f\!\left(\hat{u}_{lmn}\right)
    +
    \left.f\!\left(\hat{u}_{lmn}\right)\right|_{m=\frac{N_x}{2}}
  \right]
  .
\end{align}

% Not useful so commented out...
%%\section{$L^2$ approximation error of a discrete field}
%%
%%To measure the $L^2$ approximation error of a discrete field $u^h$ against a
%%known, analytic field $u$, we wish to compute $\left|\left| u - u^h
%%\right|\right|^{2}_{L^{2}_{xyz}}$.  Employing inner product notation we find
%%\begin{align}
%%  \left|\left| u - u^h \right|\right|^{2}_{L^{2}_{xyz}}
%%&=
%%  \left( u-u^h, u-u^h \right)
%%\\ &=
%%    \left( u, u \right)
%%  - \left( u, u^h \right)
%%  - \left( u^h, u \right)
%%  + \left(u^h, u^h\right)
%%\\ &=
%%    \left|\left| u   \right|\right|^{2}_{L^{2}_{xyz}}
%%  - 2 \left( u, u^h \right)_{L^{2}_{xyz}}
%%  + \left|\left| u^h \right|\right|^{2}_{L^{2}_{xyz}}
%%\end{align}
%%using that both $u$ and $u^h$ are real-valued.  The first term can be evaluated
%%analytically.  The third term can be computed following \eqref{eq:L2uh}.
%%
%%The second term may be treated as
%%\begin{align}
%%  \left( u^h, u \right)_{L^{2}_{xyz}}
%%&=
%%  \int_0^{L_y}
%%  \int_{-\frac{L_x}{2}}^{\frac{L_x}{2}}
%%  \int_{-\frac{L_z}{2}}^{\frac{L_z}{2}}
%%  u\!\left(x,y,z\right)
%%  \left(
%%    \sum_{l}\sum_{m}\sum_{n}
%%    \hat{u}_{l m n}B_l\!\left(y\right)e^{\ii k_m x}e^{\ii k_n z}
%%  \right)
%%  \,d\!z \,d\!x \,d\!y
%%% \\ &=
%%%   \sum_{n}
%%%   \left(
%%%     \int_{-\frac{L_z}{2}}^{\frac{L_z}{2}}
%%%     \sum_{m}
%%%     \left(
%%%       \int_{-\frac{L_x}{2}}^{\frac{L_x}{2}}
%%%       \sum_{l}
%%%       \hat{u}_{l m n}
%%%       \left(
%%%         \int_0^{L_y}
%%%         u\!\left(x,y,z\right)
%%%         B_l\!\left(y\right)
%%%         \,d\!y
%%%       \right)
%%%       e^{\ii k_m x}
%%%       \,d\!x
%%%     \right)
%%%     e^{\ii k_n z}
%%%     \,d\!z
%%%   \right)
%%\end{align}
%%where some flash of insight as to a convenient computational technique would be
%%quite helpful.

\end{document}
