\documentclass[letterpaper,11pt,nointlimits,reqno]{amsart}

% Packages
\usepackage{accents}
\usepackage{algorithm}
\usepackage{algorithmic}
\usepackage{amsfonts}
\usepackage{amsmath}
\usepackage{amssymb}
\usepackage{cancel}
\usepackage{enumerate}
\usepackage{fancyhdr}
\usepackage{fullpage}
\usepackage{ifthen}
\usepackage{lastpage}
\usepackage{latexsym}
\usepackage{mathrsfs}
\usepackage{mathtools}
\usepackage{pstricks}
\usepackage{setspace}
\usepackage{txfonts}
\usepackage{units}
\usepackage{varioref}
\usepackage{wrapfig}
\usepackage{yhmath}

\mathtoolsset{showonlyrefs,showmanualtags}
%%% \allowdisplaybreaks[1] % Allow grouped equations to be split across pages

% Line Spacing
\singlespacing

% Set appropriate header/footer information on each page
\fancypagestyle{plain}{
    \fancyhf{}
    \renewcommand{\headheight}{2.0em}
    \renewcommand{\headsep}{0.75em}
    \renewcommand{\headrulewidth}{1.0pt}
    \renewcommand{\footrulewidth}{0pt}
    \lhead{
        Suzerain model derivation, discretization, and numerical considerations
    }
    \rhead{
        Page \thepage{} of \pageref{LastPage}
    }
}
\pagestyle{plain}

% Paragraph spacing
\setlength{\parindent}{0em}
\setlength{\parskip}{2.0ex plus 0.75ex minus 0.75ex}

% Document-specific commands
\newcommand{\ii}{\ensuremath{\mathrm{i}}}
\newcommand{\trans}[1]{{#1}^{\ensuremath{\mathsf{T}}}}
\newcommand{\Mach}[1][]{\ensuremath{\mbox{Ma}_{#1}}}
\newcommand{\Reynolds}[1][]{\ensuremath{\mbox{Re}_{#1}}}
\newcommand{\Prandtl}[1][]{\ensuremath{\mbox{Pr}_{#1}}}
\newcommand{\reference}[1]{\ensuremath{\left\{#1\right\}_{0}}}
\newcommand{\lessreference}[1]
  {\ensuremath{\left({#1}-\reference{#1}\right)}}
\newcommand{\symmetricpart}[1]
  {\ensuremath{\operatorname{Sym}\left(#1\right)}}
\DeclareMathOperator{\trace}{tr}

\begin{document}

\section{Model derivation}
\label{sec:derivation}

Here we derive the mathematical model used in Suzerain.  Special attention is
paid to the origins of all conservation laws and constitutive relations
employed.  The model will nondimensionalized after derivation is complete.

\subsection{Conservation laws}

\subsubsection{Reynolds transport theorem}

Consider a time-varying control volume $\Omega$ with surface
$\partial\Omega$ and unit outward normal $\hat{n}$.  For any
scalar, vector, or tensor field quantity
$T$, Leibniz' theorem states
\begin{align}
  \label{eq:rtt}
  \frac{d}{dt}\int_{\Omega(t)}T(x,t)\,dV
  &=
  \int_{\Omega}\frac{\partial}{\partial{}t}T\,dV
  +
  \int_{\partial\Omega} \hat{n}\cdot{}w T\,dA
  =
  \int_{\Omega}\frac{\partial}{\partial{}t}T+\nabla\cdot{}wT\,dV
\end{align}
where $w$ is the velocity of $\partial\Omega$.  When $\Omega$ follows
a fixed set of fluid particles, $w$ becomes the fluid velocity $u$.

\subsubsection{Mass continuity}
Since mass $M=\int_{\Omega} \rho\,dV$
and mass conservation requires $\frac{d}{dt}M=0$,
\begin{align}
  0 = \frac{d}{dt}M
  = \frac{d}{dt}\int_{\Omega} \rho\,dV
  =
  \int_{\Omega}\frac{\partial}{\partial{}t}\rho+\nabla\cdot{}u\rho{}\,dV.
\end{align}
Because the result must hold for any control volume, we obtain
\begin{align}
  \label{eq:cons_mass}
  \frac{\partial}{\partial{}t}\rho+\nabla\cdot\rho{}u &= 0
  .
\end{align}

\subsubsection{Momentum equation}
Separating total force into surface forces and body forces
\begin{align}
  \sum{}F
  &=
     \int_{\partial\Omega} f_s \, dA
   + \int_{\Omega} f \, dV
  =
     \int_{\partial\Omega} \sigma \hat{n} \, dA
  +  \int_{\Omega} f \, dV
  =  \int_{\Omega} \nabla\cdot\sigma + f \, dV
\end{align}
where $\sigma$ is the Cauchy stress tensor.  Examining
momentum $I=\int_{\Omega} \rho{}u\,dV$ and its conservation
$\frac{d}{dt}I=\sum{}F$,
\begin{align}
    \int_{\Omega}\frac{\partial{}}{\partial{}t}\rho{}u
  + \nabla\cdot(u\otimes{}\rho{}u)\,dV
&= \int_{\Omega} \nabla\cdot\sigma + f \, dV
.
\end{align}
Because the control volume may be arbitrary,
\begin{align}
  \frac{\partial{}}{\partial{}t}\rho{}u + \nabla\cdot(u\otimes{}\rho{}u)
&= \nabla\cdot\sigma + f
.
\end{align}
Lastly, we separate the pressure $p$ and viscous contributions $\tau$ to
the Cauchy stress tensor so that $\sigma = -p I + \tau$,
\begin{align}
\label{eq:cons_momentum}
\frac{\partial{}}{\partial{}t}\rho{}u + \nabla\cdot(u\otimes{}\rho{}u)
&= -\nabla{}p + \nabla\cdot{}\tau + f
.
\end{align}

\subsubsection{Energy equation}
Lumping internal and kinetic energy into an intrinsic density $E$,
the energy $\mathscr{E}$ is
\begin{align}
  \mathscr{E} &= \int_{\Omega} \rho{}E \, dV
  .
\end{align}
Treating heat input $Q$ as both a surface phenomenon described by an outward
heat flux $q_{s}$ and as a volumetric phenomenon governed by a
body heating density $q_{b}$,
\begin{align}
  Q
  &=
   \int_{\Omega}\rho{}q_{b}\,dV
  -\int_{\partial\Omega}\hat{n}\cdot{}q_{s}\,dA
  =
    \int_{\Omega}\rho{}q_{b} - \nabla\cdot{}q_{s}\,dV
  .
\end{align}
Power input $P=F\cdot{}v$ accounts for surface stress work and body
force work to give
\begin{align}
  P
  &=
    \int_{\partial\Omega} \sigma{}\hat{n} \cdot{} u \, dA
  + \int_{\Omega} f \cdot{} u \, dV
  = \int_{\Omega} \nabla\cdot{}\sigma{}u + f \cdot{} u \, dV
  .
\end{align}
Demanding energy conservation $\frac{d}{dt}\mathscr{E}=Q+P$,
\begin{align}
  \int_{\Omega}\frac{\partial}{\partial{}t} \rho{}E
  +
  \nabla\cdot{}u\rho{}E
  \,dV
&=
    \int_{\Omega}\rho{}q_{b} - \nabla\cdot{}q_{s}\,dV
  + \int_{\Omega} \nabla\cdot\sigma{}u + f \cdot{} u \, dV
  .
\end{align}
Again, since the control volume was arbitrary,
\begin{align}
  \frac{\partial}{\partial{}t} \rho{}E
  +
  \nabla\cdot{}\rho{}Eu
&=
  - \nabla\cdot{}q_{s}
  + \nabla\cdot\sigma{}u
  + f \cdot{} u
  + \rho{}q_{b}
  .
\end{align}
After splitting $\sigma$'s pressure and viscous stress contributions we have
\begin{align}
  \label{eq:cons_energy}
  \frac{\partial}{\partial{}t} \rho{}E
  +
  \nabla\cdot{}\rho{}Eu
&=
  - \nabla\cdot{}q_{s}
  - \nabla\cdot{}pu
  + \nabla\cdot{}\tau{}u
  + f \cdot{} u
  + \rho{}q_{b}
  .
\end{align}

\subsection{Constitutive relations and other assumptions}
\label{sec:constitutive}

\subsubsection{Perfect gas}

We assume our fluid is a thermally and calorically perfect gas governed by
\begin{align}
  \label{eq:perfectgaseos}
  p &= \rho{} R T
\end{align}
where $R$ is the gas constant. The constant volume $C_{v}$ specific heat,
constant pressure specific heat $C_{p}$, and acoustic velocity $a$
relationships follow:
\begin{align}
  \label{eq:perfectgasrelations}
  \gamma &= \frac{C_{p}}{C_{v}}
  &
  C_{v} &= \frac{R}{\gamma - 1}
  &
  C_{p} &= \frac{\gamma{}R}{\gamma-1}
  &
  R &= C_{p} - C_{v}
  &
  a^{2} = \gamma{}RT
\end{align}
We assume $\gamma$ and therefore $C_{v}$ and $C_{p}$ are constant.
The total (internal and kinetic) energy density is
\begin{align}
  \label{eq:perfectgastotalenergy}
  E &= C_{v} T + \frac{1}{2}u^{2}
     = \frac{RT}{\gamma-1} + \frac{1}{2}u^{2}
\end{align}
where the notation $u^2 = u\cdot{}u$ is employed.
The total enthalpy density $H$ and (internal) enthalpy density $h$ are
\begin{align}
  \label{eq:perfectgasenthalpy}
  H &= E + \frac{p}{\rho}
     = C_{p} T + \frac{1}{2}u^{2}
     = \frac{\gamma{}RT}{\gamma-1} + \frac{1}{2}u^{2}
  &
  h &= H - \frac{1}{2}u^{2}
     = C_{p} T
     = \frac{\gamma{}RT}{\gamma-1}
  .
\end{align}
See a gas dynamics reference, e.g.~\cite{LiepmannRoshko2002}, for more details.

\subsubsection{Newtonian fluid}
\label{sec:newtonianfluid}

If we seek a constitutive law for the viscous stress tensor $\tau$
using only velocity information, the principle of material frame
indifference implies that uniform translation (given by velocity $u$)
and solid-body rotation (given by the skew-symmetric rotation tensor
$\omega=\frac{1}{2}\left( \nabla{}u-\trans{\nabla{}u} \right)$)
may not influence $\tau$.  Considering contributions only up to the
gradient of velocity, extensional strain (dilatation) and shear strain
effects may depend on only the symmetric strain rate tensor
$\varepsilon=\frac{1}{2}\left( \nabla{}u+\trans{\nabla{}u}\right)$
and its principal invariants.

Assuming $\tau$ is isotropic and depends linearly upon only $\varepsilon$,
we can express it as
\begin{align}
\tau_{ij}
&= c_{ijmn} \varepsilon_{mn}
\notag \\
&= \left( A \delta_{ij} \delta_{mn}
        + B \delta_{im} \delta_{jn}
        + C \delta_{in} \delta_{jm}
    \right) \varepsilon_{mn}
&
&\text{for some }A, B, C\in\mathbb{R}
\notag \\
&= A \delta_{ij} \varepsilon_{mm} + B\varepsilon_{ij} + C\varepsilon_{ji}
\notag \\
&= A \delta_{ij} \varepsilon_{mm} + \left( B+C \right)\varepsilon_{ji}
\notag \\
&= 2 \mu \varepsilon_{ij} + \lambda\delta_{ij}\nabla\cdot{}u
\end{align}
where $\mu=\frac{1}{2}\left( B + C \right)$ is the dynamic coefficient of
viscosity (shear) and $\lambda=A$ is the second coefficient of viscosity
(dilatational).  Reverting to direct notation we have
\begin{align}
\tau
&= 2 \mu \varepsilon + \lambda \left( \nabla\cdot{}u \right) I
\notag \\
\label{eq:taunewt}
&=   \mu \left( \nabla{}u + \trans{\nabla{}u} \right)
  + \lambda \left( \nabla\cdot{}u \right) I
.
\end{align}

The bulk viscosity $\mu_{B}=\lambda + \frac{2}{3}\mu$ and the deviatoric
component of the strain rate tensor $S = \varepsilon - \frac{1}{3} \trace
\varepsilon \, I$ may be used to write $\tau$ as
\begin{align}
  \tau
&= 2 \mu \varepsilon - \frac{2}{3}\mu \left(\nabla\cdot{}u\right) I
   + \left(\lambda + \frac{2}{3}\mu\right) \left(\nabla\cdot{}u\right) I
\\
\label{eq:tauSmub}
  &= 2 \mu S + \mu_b  \left( \nabla\cdot{}u \right) I
.
\end{align}

\subsubsection{Stokes' hypothesis}
\label{sec:stokeshypothesis}

We allow the bulk viscosity $\mu_{B}$ to be a fixed multiple of the dynamic
viscosity $\mu$.  One may write this relationship as either
\begin{align}
\label{eq:secondviscosityclaw}
\mu_{B} &= \alpha \mu
&
&\text{or}
&
\lambda &= \left( \alpha - \frac{2}{3} \right) \mu
\end{align}
where we have introduced a dimensionless proportionality constant $\alpha$.
Stokes' hypothesis that the bulk viscosity is negligible may be recovered by
selecting $\alpha = 0$.  Though Stokes' hypothesis is valid for most
circumstances~\cite{GadelHak1995}, we choose to separately track $\mu$ and
$\lambda$ terms in the model.

\subsubsection{Power law viscosity}

We assume that viscosity varies only with temperature according to
\begin{align}
\label{eq:powerlawviscosity}
\frac{\mu}{\mu_{0}}=\left(\frac{T}{T_{0}}\right)^{\beta}
\end{align}
where $\mu_{0}$ and $T_{0}$ are suitable reference values.  This
relationship models air well for temperatures up to several thousand
degrees Kelvin~\cite{NASA-TR-R-132}.

\subsubsection{Fourier's equation}

% TODO Radiative heat transfer negligible due to low temperature
% TODO Is heat transfer by molecular diffusion neglected due to time scale?
Neglecting the transport of energy by molecular diffusion and radiative
heat transfer, we seek a relation between the surface heat flux $q_{s}$
and the temperature $T$.  The principle of frame indifference implies
we may only use the temperature gradient so that
\begin{align}
  \label{eq:fouriertensorlaw}
  q_{s} &= \underline{\kappa} \cdot \nabla{} T
\end{align}
where $\underline{\kappa}$ is a thermal conductivity tensor.
Consistent with our assumption that $\tau$ is isotropic, we assume
$\underline{\kappa}$ is isotropic to obtain
\begin{align}
  \label{eq:fourierlaw}
  q_{s} &= - \kappa \nabla{} T
\end{align}
where $\kappa$ is the scalar thermal conductivity.  We introduce the
negative sign so that heat flows from hot to cold when $\kappa>0$.

\subsubsection{Constant Prandtl number}

We assume the Prandtl number $\Prandtl = \frac{\mu{}C_{p}}{\kappa}$ is constant.
Because $C_{p}$ is constant the ratio $\frac{\mu}{\kappa}$ must be
constant.  The viscosity and thermal conductivity must either grow at
identical rates or they must grow according to an inverse relationship.
The latter is not observed in practice for our class of fluids, and
so we assume
\begin{align}
  \frac{\mu}{\mu_{0}} = \frac{\kappa}{\kappa_{0}}
  .
  \label{eq:mukappa}
\end{align}

\subsection{Dimensional equations}

By combining the conservation laws with our constitutive relations
and assumptions, we arrive at the dimensional equations
\begin{subequations}\label{eq:dimensionalmodel}
\begin{align}
  \label{eq:dim_continuity}
  \frac{\partial}{\partial{}t}\rho
&=
  - \nabla\cdot\rho{}u
  \\
  \label{eq:dim_momentum}
  \frac{\partial{}}{\partial{}t}\rho{}u
&=
  - \nabla\cdot(u\otimes{}\rho{}u)
  -\nabla{} p
  + \nabla\cdot{} \tau
  + f
  \\
  \label{eq:dim_energy}
  \frac{\partial}{\partial{}t} \rho{}E
&=
  - \nabla\cdot{}\rho{}Eu
  + \nabla\cdot{} \frac{\kappa_{0}}{\mu_{0}} \mu \nabla{} T
  - \nabla\cdot{} p u
  + \nabla\cdot{}\tau{} u
  + f \cdot{} u
  + \rho{}q_{b}
\intertext{
  where terms in the right hand side make use of
}
  \label{eq:dim_pressure}
  p &=   \left(\gamma-1\right)\left(\rho{}E
       - \frac{1}{2}\rho{}u^{2} \right)
  \\
  \label{eq:dim_temperature}
  T &= \frac{p}{\rho{}R}
  \\
  \label{eq:dim_viscosity}
  \mu &= \mu_{0} \left( \frac{T}{T_{0}} \right)^{\beta}
  \\
  \label{eq:dim_secondviscosity}
  \lambda &= \left(\alpha- \frac{2}{3}\right) \mu
  \\
  \label{eq:dim_viscousstress}
  \tau &=   \mu \left( \nabla{}u + \trans{\nabla{}u} \right)
          + \lambda \left( \nabla\cdot{}u \right) I
  .
\end{align}
\end{subequations}

\subsection{Nondimensionalization}
\label{sec:nondim}

\subsubsection{Introduction of nondimensional variables}
\label{sec:intronondim}

We rewrite the dimensional equations using nondimensional variables
combined with arbitrary reference quantities.  For each dimensional
quantity in the dimensional model we introduce a nondimensional variable
or operator denoted by a superscript star, e.g. $\nabla^{*}$.

We introduce $t^{*}=\frac{t}{t_{0}}$ and $x^{*}=\frac{x}{l_{0}}$ for some
reference $t_{0}$ and $l_{0}$.  This induces the following relationships:
\begin{align}
  \frac{\partial{}}{\partial{}t}
  &=
  \frac{\partial{}}{\partial{}t^{*}}
  \frac{\partial{}t^{*}}{\partial{}t}
  =
  \frac{1}{t_{0}}\frac{\partial}{\partial{}t^{*}}
  &
  \frac{\partial{}}{\partial{}x}
  &=
  \frac{\partial{}}{\partial{}x^{*}}
  \frac{\partial{}x^{*}}{\partial{}x}
  =
  \frac{1}{l_{0}}\frac{\partial}{\partial{}x^{*}}
  &
  \nabla
  &=
  \hat{e}_{i} \frac{\partial{}}{\partial{}x_{i}}
  =
  \hat{e}_{i} \frac{1}{l_{0}} \frac{\partial}{\partial{}x^{*}_{i}}
  =
  \frac{1}{l_{0}} \nabla^{*}
  \label{eq:nondim_derivops}
\end{align}

We introduce more nondimensional quantities (e.g. $\rho^{*} =
\frac{\rho}{\rho_{0}}$) and use them to reexpress the model
\begin{subequations}\label{eq:dimwithref_model}
\begin{align}
  \label{eq:dimwithref_continuity}
  \frac{\rho_{0}}{t_{0}} \frac{\partial}{\partial{}t^{*}}\rho^{*}
&=
  - \frac{\rho_{0}u_{0}}{l_{0}} \nabla^{*}\cdot\rho^{*}u^{*}
  \\
  \label{eq:dimwithref_momentum}
  \frac{\rho_{0}u_{0}}{t_{0}} \frac{\partial{}}{\partial{}t^{*}}\rho^{*}u^{*}
&=
  - \frac{\rho_{0}u_{0}^{2}}{l_{0}}
    \nabla^{*}\cdot(u^{*}\otimes{}\rho^{*}u^{*})
  - \frac{p_{0}}{l_{0}} \nabla^{*} p^{*}
  + \frac{\tau_{0}}{l_{0}} \nabla^{*}\cdot{} \tau^{*}
  + f_{0} f^{*}
  \\
  \label{eq:dimwithref_energy}
  \frac{\rho_{0}E_{0}}{t_{0}}
  \frac{\partial}{\partial{}t^{*}} \rho^{*}E^{*}
&=
  - \frac{\rho_{0}E_{0}u_{0}}{l_{0}} \nabla^{*} \cdot{}\rho^{*}E^{*}u^{*}
  + \frac{\kappa_{0}T_{0}}{l_{0}^{2}}
    \nabla^{*}\cdot{} \mu^{*} \nabla^{*} T^{*}
  - \frac{p_{0}u_{0}}{l_{0}} \nabla^{*}\cdot{} p^{*} u^{*}
\notag\\
&\quad{}+ \frac{\tau_{0}u_{0}}{l_{0}} \nabla^{*}\cdot{}\tau^{*} u^{*}
  + f_{0}u_{0} f^{*} \cdot{} u^{*}
  + \rho_{0}q_{0} \rho^{*} q_{b}^{*}
\intertext{
  where terms in the right hand side are computed using
}
  \label{eq:dimwithref_pressure}
  p^{*} &= \frac{\gamma-1}{p_{0}} \left(
        \rho_{0}E_{0}\rho^{*}E^{*}
      - \rho_{0}u_{0}^{2}\,\rho^{*}\frac{u^{*}\cdot{}u^{*}}{2}
  \right)
  \\
  \label{eq:dimwithref_temperature}
  T^{*} &= \frac{p_{0}p^{*}}{\rho_{0}RT_{0}\,\rho^{*}}
  \\
  \label{eq:dimwithref_viscosity}
  \mu^{*} &= \left( T^{*} \right)^{\beta}
  \\
  \label{eq:dimwithref_secondviscosity}
  \lambda^{*} &= \left(\alpha - \frac{2}{3}\right) \mu^{*}
  \\
  \label{eq:dimwithref_viscousstress}
  \tau^{*} &= \frac{\mu_{0}u_{0}}{l_{0} \tau_{0}} \left[
      \mu^{*} \left( \nabla^{*}u^{*} + \trans{\nabla^{*}u^{*}} \right)
      + \lambda^{*} \left( \nabla^{*}\cdot{}u^{*} \right) I
    \right]
  .
\end{align}
\end{subequations}
Notice that $\lambda$ has been nondimensionalized using $\mu_{0}$.  At this
stage, we have many more reference quantities than the underlying quantity
dimensions warrant.

\subsubsection{Reference quantity selections}
\label{sec:nondimrefq}

We choose a reference density $\rho_{0}$, length $l_{0}$, velocity $u_{0}$, and
temperature $T_{0}$.  These selections fix all other dimensional reference
quantities:
\begin{align}
  a_{0} &= \sqrt{\gamma{}RT_{0}}
  &
  E_{0}, H_{0}, h_{0} &= a_{0}^{2}
  &
  p_{0} &= \rho_{0} a_{0}^{2}
  &
  t_{0} &= \frac{l_{0}}{u_{0}}
  &
  \tau_{0} &= \frac{\mu_{0}u_{0}}{l_{0}}
  &
  f_{0} &= \frac{\rho_{0}u_{0}}{t_{0}}
  &
  q_{0} &= \frac{a_{0}^{2}}{t_{0}}
\end{align}
Because we assume viscosity varies only with temperature, $\mu_{0}=\mu\!\left(
T_{0} \right)$ is fixed by $T_{0}$.  Because we assume a constant Prandtl
number, $\kappa_{0}=\kappa\!\left( \mu\!\left( T_{0} \right) \right)$ is also
fixed by $T_{0}$.  Choosing this form for $p_{0}$ in lieu of employing
equation~\eqref{eq:perfectgaseos} is customary and removes many instances of
$\gamma$ from the resulting equations.  Note that these reference choices imply
\begin{align}
a^{*}&=\sqrt{T^{*}}
&
&\text{and}
&
h^{*}&=\frac{T^{*}}{\gamma-1}
.
\end{align}

\subsubsection{Nondimensional equations}
\label{nondim_equations}

We employ the reference quantity relationships after multiplying the
continuity, momentum, and energy equations by $\frac{t_{0}}{\rho_{0}}$,
$\frac{t_{0}}{\rho_{0}u_{0}}$, and $\frac{t_{0}}{\rho_{0}E_{0}}$
respectively.  Henceforth we suppress the superscript star notation because all
terms are dimensionless.  We arrive at the following nondimensional equations
\begin{subequations}
\begin{align}
  \label{eq:nondim_continuity}
  \frac{\partial}{\partial{}t}\rho{}
&=
  - \nabla\cdot\rho{}u
  \\
  \label{eq:nondim_momentum}
  \frac{\partial}{\partial{}t}\rho{}u
&=
  - \nabla\cdot(u\otimes\rho{}u)
  - \frac{1}{\Mach^{2}} \nabla{} p
  + \frac{1}{\Reynolds} \nabla\cdot\tau
  + f
  \\
  \label{eq:nondim_energy}
  \frac{\partial}{\partial{}t} \rho{}E
&=
  - \nabla\cdot\rho{}Eu
  + \frac{1}{\Reynolds\,\Prandtl\,\left( \gamma - 1 \right)}
    \nabla\cdot\mu\nabla{} T
  - \nabla\cdot{} p u
  + \frac{\Mach^{2}}{\Reynolds} \nabla\cdot\tau{} u
  + \Mach^{2} f \cdot{} u
  + \rho{} q_{b}
\intertext{
along with the relationships
}
  \label{eq:nondim_pressure}
  p &= \left(\gamma-1\right) \left(
    \rho{}E - \frac{\Mach^{2}}{2}\rho{}u^{2}
  \right)
  \\
  \label{eq:nondim_temperature}
  T &= \gamma{} \frac{p}{\rho}
  \\
  \label{eq:nondim_viscosity}
  \mu &= T^{\beta}
  \\
  \label{eq:nondim_secondviscosity}
  \lambda &= \left(\alpha-\frac{2}{3}\right)\mu
  \\
  \label{eq:nondim_viscousstress}
  \tau &=  \mu\left(\nabla{}u+\trans{\nabla{}u}\right)
         + \lambda\left(\nabla\cdot{}u\right) I
\end{align}
\end{subequations}
where the nondimensional quantities
\begin{align}
  \Reynolds &= \frac{\rho_{0}u_{0}l_{0}}{\mu_{0}}
  &
  \Mach &= \frac{u_{0}}{a_{0}}
  &
  \Prandtl &= \frac{\mu_{0}C_{p}}{\kappa_{0}}
\end{align}
are the Reynolds, Mach, and Prandtl numbers, respectively.

Introducing a velocity reference $u_{0}$ separate from the sound speed $a_{0}$
was not necessary from a dimensional standpoint.  Doing so is the origin of the
$\Mach$ factors appearing above.  These add some minor complexity to the
equations but greatly simplify investigating the physics in the various
$\Mach\to{}0$ limits.

\section{Discretization}
\label{sec:discretization}

Here we discuss the space and time discretization techniques employed to solve
the continuous model equations.  In this section $u$ denotes an arbitrary state
vector and not fluid velocity.

\subsection{Spatial discretization}
\label{sec:spatialdiscretization}

We start with the continuous system
\begin{align}
  \frac{\partial}{\partial{}t} u &= \mathscr{L}u + \mathscr{N}\!\left(u\right)
\end{align}
on the spatial domain $\left[-\frac{L_x}{2},\frac{L_x}{2}\right] \times{}
[0,L_y] \times{} \left[-\frac{L_z}{2},\frac{L_z}{2}\right]$.  The operators
$\mathscr{L}$ and $\mathscr{N}$ are linear and nonlinear, respectively.  For
brevity, the section suppresses any time dependence in the operators.  To begin
spatially discretizing the system, we introduce its finite dimensional analog
\begin{align}
  \frac{\partial}{\partial{}t} u^h
  &=
  \mathscr{L}u^h + \mathscr{N}\!\left(u^h\right) + R^h
  \label{eq:discrete_system_with_residual}
\end{align}
where continuous $u = u\!\left(x,y,z,t\right)$ has been replaced by discrete
$u^h = u^h\!\left(x,y,z,t\right)$ with $N_x\times{}N_y\times{}N_z$ degrees of
freedom.  Here, $R^h$ is the discretization error that arises because the
discrete solution cannot satisfy the continuous equations everywhere in space.
We select Fourier expansions for the periodic $x$ and $z$ directions and a
B-spline expansion for the non-periodic $y$ direction.  That is,
\begin{align}
u^h(x,y,z,t)
&=
  \sum_{l=0}^{N_y - 1}
  \sum_{m=-\frac{N_x}{2}}^{\frac{N_x}{2}-1}
  \sum_{n=-\frac{N_z}{2}}^{\frac{N_z}{2}-1}
  \hat{u}_{l m n}(t)
  B_l\!\left(y\right)
  e^{\ii\frac{2\pi{}m}{L_x}x}
  e^{\ii\frac{2\pi{}n}{L_z}z}
  \\
&=
  \sum_{l}\sum_{m}\sum_{n}
  \hat{u}_{l m n}(t)B_l\!\left(y\right)e^{\ii k_m x}e^{\ii k_n z}
  \label{eq:u_h_expansion}
\end{align}
where $k_m = 2\pi{}m/L_x$, $k_n = 2\pi{}n/L_z$, and $B_l\!\left(y\right)$ are a
B-spline basis for some order and knot selection.

Now, within the method of weighted residuals framework, we choose a mixed
Galerkin/collocation approach (often called a ``pseudospectral'' technique)
employing the $L_{2}$ inner product and test ``functions'' like
$\delta(y-y_{l'}) e^{\ii k_{m'} x}e^{\ii k_{n'} z}$ where $l'$, $m'$, and $n'$
range over the same values as $l$, $m$, and $n$, respectively.  The fixed
collocation points $y_{l'}$ depend on the B-spline basis details.  We catalog
three relevant results
\begin{align}
   \int_0^{L_y} \varphi(y) \, \delta(y-y_{l'}) \,d\!y
&= \varphi(y_{l'}),
&
   \int_{-\frac{L_x}{2}}^{\frac{L_x}{2}} e^{\ii k_m x} e^{-\ii k_{m'} x} \,d\!x
&= L_x \delta_{m m'}, \text{ and}
&
   \int_{-\frac{L_z}{2}}^{\frac{L_z}{2}} e^{\ii k_n z} e^{-\ii k_{n'} z} \,d\!z
&= L_z \delta_{n n'}
\end{align}
where the inner product's conjugate operation is accounted for by introducing a
negative sign into the latter two exponentials.  We force the weighted residual
to be zero in the sense that
\begin{align}
  \int_0^{L_y}
  \int_{-\frac{L_x}{2}}^{\frac{L_x}{2}}
  \int_{-\frac{L_z}{2}}^{\frac{L_z}{2}}
  R^h\!\left(x,y,z\right) \delta(y-y_{l'}) e^{-\ii k_{m'} x}e^{-\ii k_{n'} z}
  \,d\!z \,d\!x \,d\!y
  &=
  0
  \label{eq:R_h_weighted_residual_zero}
\end{align}
holds for all $l'$, $m'$, and $n'$.  Inserting \eqref{eq:u_h_expansion} into
\eqref{eq:discrete_system_with_residual}, testing with our test functions,
applying \eqref{eq:R_h_weighted_residual_zero}, and simplifying each remaining
term separately we obtain the following:
\begin{align}
 \int_0^{L_y}
 \int_{-\frac{L_x}{2}}^{\frac{L_x}{2}}
 \int_{-\frac{L_z}{2}}^{\frac{L_z}{2}}
 \,
 &\frac{\partial}{\partial{}t}
  \left(
    \sum_{l}\sum_{m}\sum_{n}
    \hat{u}_{l m n}(t)B_l\!\left(y\right)e^{\ii k_m x}e^{\ii k_n z}
  \right)
  \left(
    \delta(y-y_{l'}) e^{-\ii k_{m'} x}e^{-\ii k_{n'} z}
  \right)
  \, dz \, dx \, dy
\\
  &=
  L_x L_z \sum_{l} B_l\!\left(y_{l'}\right)
  \frac{\partial}{\partial{}t} \hat{u}_{l m n}(t)
\\
 \int_0^{L_y}
 \int_{-\frac{L_x}{2}}^{\frac{L_x}{2}}
 \int_{-\frac{L_z}{2}}^{\frac{L_z}{2}}
 \,
 &\mathscr{L}
  \left(
    \sum_{l}\sum_{m}\sum_{n}
    \hat{u}_{l m n}(t)B_l\!\left(y\right)e^{\ii k_m x}e^{\ii k_n z}
  \right)
  \left(
    \delta(y-y_{l'}) e^{-\ii k_{m'} x}e^{-\ii k_{n'} z}
  \right)
  \, dz \, dx \, dy
\\
  &=
  L_x L_z
  \mathscr{L}\left(
     \sum_{l}
      B_l\!\left(y_{l'}\right)
     \hat{u}_{l m n}(t)
   \right)
\intertext{} % Allow soft break
  \int_0^{L_y}
  \int_{-\frac{L_x}{2}}^{\frac{L_x}{2}}
  \int_{-\frac{L_z}{2}}^{\frac{L_z}{2}}
  &\mathscr{N}\left(
     \sum_{l}\sum_{m}\sum_{n}
     \hat{u}_{l m n}(t)B_l\!\left(y\right)e^{\ii k_m x}e^{\ii k_n z}
   \right)
   \left(
     \delta(y-y_{l'}) e^{-\ii k_{m'} x}e^{-\ii k_{n'} z}
   \right)
   \, dz \, dx \, dy
\\
  &=
  \int_{-\frac{L_x}{2}}^{\frac{L_x}{2}}
  \int_{-\frac{L_z}{2}}^{\frac{L_z}{2}}
  \mathscr{N}\left(
    \sum_{m}\sum_{n}
    \left(
      \sum_{l} B_l\!\left(y_{l'}\right)
      \hat{u}_{l m n}(t)
    \right)
    e^{\ii k_m x}e^{\ii k_n z}
  \right)
  \left(
    e^{-\ii k_{m'} x}e^{-\ii k_{n'} z}
  \right)
  \, dz \, dx
\intertext{
  Reequating the terms we have
}
  L_x L_z
  \sum_{l} B_l\!\left(y_{l'}\right)
  \frac{\partial}{\partial{}t} \hat{u}_{l m n}(t)
  &=
  L_x L_z
  \mathscr{L}\left(
    \sum_{l}
     B_l\!\left(y_{l'}\right)
    \hat{u}_{l m n}(t)
  \right)
\\
  &{}+
  \int_{-\frac{L_x}{2}}^{\frac{L_x}{2}}
  \int_{-\frac{L_z}{2}}^{\frac{L_z}{2}}
  \mathscr{N}\left(
    \sum_{m}\sum_{n}
    \left(
      \sum_{l} B_l\!\left(y_{l'}\right)
      \hat{u}_{l m n}(t)
    \right)
    e^{\ii k_m x}e^{\ii k_n z}
  \right)
  \left(
    e^{-\ii k_{m'} x}e^{-\ii k_{n'} z}
  \right)
  \, dz \, dx
  .
 \end{align}

Finally, approximating the two integrals by discrete sums and dividing
by $L_x$ and $L_z$ leaves us with
\begin{align}
  \sum_{l} B_l\!\left(y_{l'}\right)
  \frac{\partial}{\partial{}t} \hat{u}_{l m n}(t)
  &\approx
  \mathscr{L}\left(
    \sum_{l}
     B_l\!\left(y_{l'}\right)
    \hat{u}_{l m n}(t)
  \right)
\\
  &{}+
  \frac{1}{N_x N_z}
  \sum_{m'} \sum_{n'}
  \mathscr{N}\left(
    \sum_{m}
    \sum_{n}
    \left(
      \sum_{l} B_l\!\left(y_{l'}\right)
      \hat{u}_{l m n}(t)
    \right)
    e^{\ii k_m x_{m'}}e^{\ii k_n z_{n'}}
  \right)
  \left(
    e^{-\ii k_{m'} x_m}e^{-\ii k_{n'} z_n}
  \right)
  \label{eq:spatial_discretization}
\end{align}
where $x_{m'}=L_x m' / N_x$ and $z_{n'}=L_z n' / N_z$.  This approximation is
nothing but a quadrature error (see, e.g., theorem~19 in~\cite{Boyd2001}).
With knowledge of the nonlinear operator $\mathscr{N}$, such quadrature error
can be mitigated via an appropriate dealiasing technique (see,
e.g.,~\cite{Canuto2006}).  Interestingly, our weighted residual
\eqref{eq:R_h_weighted_residual_zero} need now only be zero in this weaker,
discrete sense.  We are left with $N_x\times{}N_z$ time-dependent systems
containing $N_y$ equations coupled in the $x$ and $z$ directions only through
discrete Fourier transforms and the requirements of the $\mathscr{L}$ and
$\mathscr{N}$ operators.

Calling $\hat{u}_{l m n}(t)$ the wave space coefficients of the solution at
some time $t$, the nonlinear portion of \eqref{eq:spatial_discretization} may
be efficiently computed by
\begin{enumerate}
 \item performing $\mathcal{O}\!\left(m\times{}n\right)$ matrix-vector products
       like $\sum_{l} B_l\!\left(y_{l'}\right) \hat{u}_{l m n}(t)$,
 \item using an inverse fast Fourier transform across
       the $x$ and $z$ directions to convert state information from wave space
       to physical space,
 \item applying the nonlinear operator $\mathscr{N}$ in physical space, and
 \item using a forward fast Fourier transform across the $x$ and $z$
       directions to convert information from physical space to wave space.
\end{enumerate}
Note that the left hand side of \eqref{eq:spatial_discretization} contains a
time-independent mass matrix arising from the B-spline basis and collocation
point choices.  We retain the mass matrix on the same side as the time
derivative in anticipation of the time discretization scheme.  The
computational cost of the scaling factor $\left(N_x N_z\right)^{-1}$
can be hidden within the time advancement scheme.

Rather than time advancing the B-spline coefficients $\hat{u}_{l m n}(t)$ as
state, one could instead instead advance the wavenumber-dependent collocation
point values $\hat{u}_{y_{l'} m n}(t) = \sum_{l} B_l\!\left(y_{l'}\right)
\hat{u}_{l m n}(t)$.  Then \eqref{eq:spatial_discretization} would become
\begin{align}
  \frac{\partial}{\partial{}t} \hat{u}_{y_{l'} m n}(t)
  &\approx
  \mathscr{L}\left(\hat{u}_{y_{l'} m n}(t)\right)
  +
  \frac{1}{N_x N_z}
  \sum_{m'} \sum_{n'}
  \mathscr{N}\left(
    \sum_{m}
    \sum_{n}
    \hat{u}_{y_{l'} m n}(t)
    e^{\ii k_m x_{m'}}e^{\ii k_n z_{n'}}
  \right)
  \left(
    e^{-\ii k_{m'} x_m}e^{-\ii k_{n'} z_n}
  \right)
  .
  \label{eq:spatial_discretization_gridpoints}
\end{align}
Equation \eqref{eq:spatial_discretization_gridpoints} is obviously simpler than
equation \eqref{eq:spatial_discretization}.  In particular, it lacks a mass
matrix in the time derivative term.

\subsection{Time discretization}

We use Spalart, Moser, and Rogers'  low storage hybrid implicit/explicit
time stepper from appendix A of~\cite{spalart_lowstoragerk}.  This time stepper
advances the system
\begin{align}
 u_t &= Lu + N(u,t)
\end{align}
from $u(t)$ to $u\left( t+\Delta{}t \right)$.  Here $L$ and $N$ are a linear
and nonlinear operator, respectively, distinct from but related to the
preceding section's $\mathscr{L}$ and $\mathscr{N}$ (see
\textsection\ref{sec:combineddiscretization} for the exact relationships).  The
operator $L$ must be time-independent throughout each interval $\left[t,
t+\Delta{}t\right)$.  In contrast to the discussion
in~\cite{spalart_lowstoragerk} and following S. Yang (personal communication),
we allow $N$ to vary in time.  The ``SMR'' scheme advances according to the
following equations:
\begin{subequations}
\begin{align}
  u'
  &=
  u_{n}
  + \Delta{}t\left[
      L\left( \alpha_{1}u_{n} + \beta_{1}u' \right)
    + \gamma_{1} N\left(u_{n},t_{n}\right)
  \right]
  \tag{SMR A4a}
  \label{eq:SMR_A4a}
\\
  u''
  &=
  u'
  + \Delta{}t\left[
    L\left( \alpha_{2}u' + \beta_{2}u'' \right)
    + \gamma_{2} N\left(u',t_{n}+\eta_{2}\Delta{}t\right)
    + \zeta_{1}  N\left(u_{n},t_{n}\right)
  \right]
  \tag{SMR A4b}
  \label{eq:SMR_A4b}
\\
  u_{n+1}
  &=
  u''
  + \Delta{}t\left[
      L\left( \alpha_{3}u'' + \beta_{3}u_{n+1} \right)
    + \gamma_{3} N\left(u'',t_{n}+\eta_{3}\Delta{}t\right)
    + \zeta_{2}  N\left(u',t_{n}+\eta_{2}\Delta{}t\right)
  \right]
  \tag{SMR A4c}
  \label{eq:SMR_A4c}
\end{align}
\begin{align}
  \alpha_1 + \beta_1 &= \gamma_1 = \eta_2
  &
  \alpha_2 + \beta_2 &= \gamma_2 + \zeta_1
  &
  \alpha_3 + \beta_3 &= \gamma_3 + \zeta_2
  &
  \eta_{3} &= \eta_2 + \alpha_2 + \beta_2
  \tag{SMR A5}
\end{align}
\end{subequations}
The authors determined suitable values for the above coefficients to be
\begin{align*}
  \alpha_1, \alpha_2, \alpha_3 &= \left\{
    \frac{29}{96}, -\frac{3}{40},  \frac{1}{6}
  \right\}
  &
  \beta_1, \beta_2, \beta_3 &= \left\{
    \frac{37}{160}, \frac{5}{24}, \frac{1}{6}
  \right\}
  &
  \gamma_1, \gamma_2, \gamma_3 &= \left\{
    \frac{8}{15}, \frac{5}{12}, \frac{3}{4}
  \right\}
\end{align*}
\begin{align*}
  \zeta_0, \zeta_1, \zeta_2 &= \left\{
    0, -\frac{17}{60}, -\frac{5}{12}
  \right\}
  &
  \eta_0, \eta_1, \eta_2, \eta_3 &= \Biggl\{
    0, 0, \frac{8}{15}, \frac{2}{3}
  \Biggr\}
\end{align*}
We have added $\zeta_0$ and $\eta_0$ for notational convenience.  The
$\eta_{1}$--$\eta_{3}$ coefficients used to extend the SMR scheme for
time-dependent $N$ are discussed by Yang~\cite{ShanYang2011}.  Each substep
\eqref{eq:SMR_A4a}--\eqref{eq:SMR_A4c} has the general form
\begin{align}
  u^{i+1} &= u^i + \Delta{}t \left[
        \alpha_{i} L u^i
      + \beta_{i}  L u^{i+1}
      + \gamma_{i} N\left( u^{i}, t_{n}+\eta_{i}\Delta{}t \right)
      + \zeta_{i-1} N\left( u^{i-1}, t_{n}+\eta_{i-1}\Delta{}t \right)
  \right]
  \label{eq:generalsubstep}
\end{align}
where $i\in\left\{ 1,2,3 \right\}$ is the substep number.
We rewrite the general substep equation as
\begin{align}
  \left(I - \Delta{}t\beta_{i}L\right) u^{i+1}
  &=
  \left(I + \Delta{}t\alpha_{i}L\right) u^{i}
  + \Delta{}t\gamma_{i}N\left(u^{i}, t_{n}+\eta_{i}\Delta{}t\right)
  + \Delta{}t\zeta_{i-1}N\left(u^{i-1}, t_{n}+\eta_{i-1}\Delta{}t\right)
  \label{eq:generaloperatorsubstep}
  .
\end{align}
The scheme treats $N$ with third-order accuracy and $L$ with second-order
accuracy.  See both the top of page 323 in~\cite{spalart_lowstoragerk} and
chapter 7 in~\cite{ShanYang2011} for more accuracy-related details.

We wish to advance problems like $Mu_{t}=\tilde{L}u+\chi\tilde{N}\left( u,t
\right)$ where $\chi$ is a time-independent scalar and both $\tilde{L}$ and
$\tilde{N}$ take state values to some non-state representation which can be
converted back to state by the linear ``mass matrix'' operator $M$.  Then by
$L=M^{-1}\tilde{L}$ and $N=\chi{}M^{-1}\tilde{N}$ the general substep equation
gives
\begin{align}
  \left(I - \Delta{}t\beta_{i}M^{-1}\tilde{L}\right) u^{i+1}
  &=
  \left(I + \Delta{}t\alpha_{i}M^{-1}\tilde{L}\right) u^{i}
\\
  &+ \Delta{}t\gamma_{i}\chi{}M^{-1}
    \tilde{N}\left(u^{i}, t_{n}+\eta_{i}\Delta{}t\right)
  + \Delta{}t\zeta_{i-1}\chi{}M^{-1}
    \tilde{N}\left(u^{i-1}, t_{n}+\eta_{i-1}\Delta{}t\right)
  .
\end{align}
We multiply through by $M$ to obtain
\begin{align}
  \left(M - \Delta{}t\beta_{i}\tilde{L}\right) u^{i+1}
  &=
  \left(M + \Delta{}t\alpha_{i}\tilde{L}\right) u^{i}
  + \Delta{}t\gamma_{i}\chi{}
    \tilde{N}\left(u^{i}, t_{n}+\eta_{i}\Delta{}t\right)
  + \Delta{}t\zeta_{i-1}\chi{}
    \tilde{N}\left(u^{i-1}, t_{n}+\eta_{i-1}\Delta{}t\right)
  \label{eq:generaloperatormasssubstep}
\end{align}
which is just \eqref{eq:generaloperatorsubstep} with $M$ replacing $I$,
$\tilde{L}$ replacing $L$, and $\chi{}\tilde{N}$ replacing $N$.  We now drop
the tildes on $L$ and $N$ with the understanding that they are implied whenever
$M\neq{}I$.

The time advancement scheme in \eqref{eq:generaloperatormasssubstep} requires
implementations of $u\mapsto{}{N}\left(u\right)$,
$u\mapsto{}\left(M+\varphi{}L\right)u$, and
$u\mapsto{}\left(M+\varphi{}L\right)^{-1}u$ for a given $M$ and some arbitrary
scalar $\varphi$.  To achieve a low storage implementation, the
$N\left(u\right)$ and $\left(M+\varphi{}L\right)^{-1}$ implementations must
operate in-place while $\left(M+\varphi{}L\right)$ must operate out-of-place.
Given only two storage locations $a$ and $b$, each substep computation follows
algorithm \vref{alg:substep}.

\begin{algorithm}
\label{alg:substep}
\caption{Compute one substep in the SMR scheme following
         equation (\ref{eq:generaloperatormasssubstep}) %FIXME \eqref?
         }
\begin{algorithmic}
  \REQUIRE Storage $a = u^i$;
           storage $b = N\left(u^{i-1},t_{n}+\eta_{i-1}\Delta{}t\right)$
  \STATE $b\leftarrow{}   \left(M+\Delta{}t\alpha_{i}L\right)a
                        + \Delta{}t\zeta_{i-1}\chi{}b$
  \STATE $a\leftarrow{}N\left(a,t_{n}+\eta_{i}\Delta{}t\right)$
  \STATE $b\leftarrow{}\Delta{}t\gamma_{i}\chi{}a + b$
  \STATE $b\leftarrow{}\left(M-\Delta{}t\beta_{i}L\right)^{-1}b$
  \ENSURE Storage $a = N\left(u^{i},t_{n}+\eta_{i}\Delta{}t\right)$;
          storage $b = u^{i+1}$
\end{algorithmic}
\end{algorithm}

Two possible issues arise when repeatedly using the substep algorithm: First,
during the first substep the step size $\Delta{}t$ may need to be computed
dynamically based on a stability criterion available only during the nonlinear
operator computation.  Second, it may be important to always apply each
operator against a particular storage location.  This requirement implies that
the implementation of $\left(M+\varphi{}L\right)$ must also support in-place
application.  For the second and subsequent substeps, either
$\left(M+\varphi{}L\right)$ must support a decidedly non-BLAS-like, complicated
out-of-place-apply-and-swap operation or the underlying state storage must
support a swap operation.  We choose the latter and denote the swap operation
as $a\leftrightarrow{}b$.  Under these considerations, time step computation
follows algorithm \vref{alg:step}.

\begin{algorithm}
\caption{Compute all substeps in the SMR scheme following
         equation (\ref{eq:generaloperatormasssubstep}) % FIXME \eqref?
         }
\label{alg:step}
\begin{algorithmic}
  \renewcommand{\algorithmiccomment}[1]{\hfill{}// #1}
  \REQUIRE Storage $a = u\left(t_{n}\right) = u^{0} $;
           storage $b$ content undefined
  \STATE $b\leftarrow{}a$
  \STATE $b\leftarrow{}N\left(b,t_{n}\right)$
  \STATE Compute $\Delta{}t$ from $a=u^0$ and $b=N\left(u^0,t_{n}\right)$
  \STATE $a\leftarrow{}\left(M+\Delta{}t\alpha_{1}L\right)a$
  \STATE $a\leftarrow{}\Delta{}t \gamma_{1} \chi{} b + a$
  \STATE $a\leftarrow{}\left(M-\Delta{}t\beta_{1}L\right)^{-1}a$
  \ENSURE Storage $a = u^1$;
          storage $b = N\left(u^{0},t_{n}\right)$
  \STATE $b\leftarrow{}   \left(M+\Delta{}t\alpha_{2}L\right)a
                        + \Delta{}t\zeta_{1}\chi{}b$
  \STATE $a\leftrightarrow{}b$
  \STATE $b\leftarrow{}N\left(b,t_{n}+\eta_{2}\Delta{}t\right)$
  \STATE $a\leftarrow{}\Delta{}t \gamma_{2} \chi{} b + a$
  \STATE $a\leftarrow{}\left(M-\Delta{}t\beta_{2}L\right)^{-1}a$
  \ENSURE Storage $a = u^{2}$;
          storage $b = N\left(u^{1},t_{n}+\eta_{2}\Delta{}t\right)$
  \STATE $b\leftarrow{}   \left(M+\Delta{}t\alpha_{3}L\right)a
                        + \Delta{}t\zeta_{2}\chi{}b$
  \STATE $a\leftrightarrow{}b$
  \STATE $b\leftarrow{}N\left(b,t_{n}+\eta_{3}\Delta{}t\right)$
  \STATE $a\leftarrow{}\Delta{}t \gamma_{3} \chi{}b + a$
  \STATE $a\leftarrow{}\left(M-\Delta{}t\beta_{3}L\right)^{-1}a$
  \ENSURE Storage $a = u\left(t+\Delta{}t\right)= u^{3}$;
          storage $b = N\left(u^{2},t_{n}+\eta_{3}\Delta{}t\right)$
\end{algorithmic}
\end{algorithm}

\subsection{Combined space and time discretization}
\label{sec:combineddiscretization}

Using \eqref{eq:generaloperatormasssubstep} to advance state
$\hat{u}_{l m n}(t)$ following \eqref{eq:spatial_discretization} gives
the time-independent discrete operators
\begin{subequations}
\begin{align}
   M u\bigr|_{m n}
&= \sum_{l} B_l\!\left(y_{l'}\right)
   \hat{u}_{l m n}
\\
   \left.\tilde{L} u\right|_{m n}
&= \mathscr{L}\left(
     \sum_{l}
     B_l\!\left(y_{l'}\right)
     \hat{u}_{l m n}
   \right)
\\
   \left.\tilde{N}\!\left(u\right)\right|_{m n}
&= \underbrace{\frac{1}{N_x N_z}}_{\chi}
   \sum_{m'} \sum_{n'}
   \mathscr{N}\left(
     \sum_{m}
     \sum_{n}
     \left(
       \sum_{l} B_l\!\left(y_{l'}\right)
       \hat{u}_{l m n}
     \right)
     e^{\ii k_m x_{m'}}e^{\ii k_n z_{n'}}
   \right)
   \left(
     e^{-\ii k_{m'} x_m}e^{-\ii k_{n'} z_n}
   \right)
\end{align}
\end{subequations}
while advancing state $\hat{u}_{y_{l'} m n}(t)$ following
\eqref{eq:spatial_discretization_gridpoints} instead fixes the operators as
\begin{subequations}
\begin{align}
   M u\bigr|_{m n}
&= \hat{u}_{y_{l'} m n}
\\
   \left.\tilde{L} u\right|_{m n}
&= \mathscr{L}\left(\hat{u}_{y_{l'} m n}\right)
\\
   \left.\tilde{N}\!\left(u\right)\right|_{m n}
&= \underbrace{\frac{1}{N_x N_z}}_{\chi}
   \sum_{m'} \sum_{n'}
   \mathscr{N}\left(
     \sum_{m}
     \sum_{n}
     \hat{u}_{y_{l'} m n}
     e^{\ii k_m x_{m'}}e^{\ii k_n z_{n'}}
   \right)
   \left(
     e^{-\ii k_{m'} x_m}e^{-\ii k_{n'} z_n}
   \right)
   .
\end{align}
\end{subequations}
When following \eqref{eq:spatial_discretization} the operators $\tilde{L}$ and
$\tilde{N}$ take B-spline coefficients as input and return collocation point
values.  When following \eqref{eq:spatial_discretization_gridpoints} the
operators both take and return collocation point values.  We choose to work
with \eqref{eq:spatial_discretization} because it provides computational
advantages.  For a discussion on moving operators between these two
representations, see \textsection{}5.5 in~\cite{Boyd2001}.

\section{Numerical considerations}

Here we take the complete mathematical model from section \ref{sec:derivation}
and put it into the form which we will compute per section
\ref{sec:discretization}.  Though less clean in appearance, this section's
equations will better reflect the spectral implementation details used in
Suzerain than those given in earlier sections.

\subsection{Convective derivative operator form}

We choose to use the conservative form of the convective derivative operator,
$\nabla\cdot\left(u\otimes{}\rho{}u\right)$, instead of the skew-symmetric
form, $\frac{1}{2}u\cdot\nabla{}\rho{}u +
\frac{1}{2}\nabla\cdot{}u\otimes{}\rho{}u$.  The former is simpler to compute,
retains the conservative nature of the equations, and behaves comparably to the
latter in the incompressible case when aliasing errors are
removed~\cite{Zang1991Rotation}.  This choice may need to be revisited as the
wall-normal direction is not dealiased.

\subsection{State variable selection}
\label{state_variable_selection}

We use nondimensional density $\rho$, momentum $m=\rho{}u$, and total energy
$e=\rho{}E$ as the state variables for our computations.  Though it
eliminates division and potentially allows for fully dealiased calculations, we
do not use specific density $\sigma=1/\rho$ because it requires using a
nonconservative mass equation.  When rewritten using the state variables
the equations in section~\ref{nondim_equations} become
\begin{subequations}
\begin{align}
  \label{eq:state_continuity}
  \frac{\partial}{\partial{}t}\rho{}
&=
  - \nabla\cdot{}m
  \\
  \label{eq:state_momentum}
  \frac{\partial}{\partial{}t}m
&=
  - \nabla\cdot\left(\frac{m}{\rho}\otimes{}m\right)
  - \frac{1}{\Mach^{2}} \nabla{} p
  + \frac{1}{\Reynolds} \nabla\cdot\tau
  + f
  \\
  \label{eq:state_energy}
  \frac{\partial}{\partial{}t} e
&=
  - \nabla\cdot{}\left(e + p\right)\frac{m}{\rho}
  + \frac{1}{\Reynolds\,\Prandtl\,\left( \gamma - 1 \right)}
    \nabla\cdot\mu\nabla{} T
  + \frac{\Mach^{2}}{\Reynolds} \nabla\cdot\tau{}\frac{m}{\rho}
  + \Mach^{2} f \cdot{} \frac{m}{\rho}
  + \rho{} q_{b}
\intertext{
  where the non-state quantities are fixed by
}
  \label{eq:state_pressure}
  p &= \left(\gamma-1\right) \left( e - \Mach^{2}\frac{m^2}{2\rho} \right)
  \\
  \label{eq:state_temperature}
  T &= \gamma{} \frac{p}{\rho}
  \\
  \label{eq:state_viscosity}
  \mu &= T^{\beta}
  \\
  \label{eq:state_secondviscosity}
  \lambda &= \left(\alpha- \frac{2}{3}\right) \mu
  \\
  \label{eq:state_viscousstress}
  \tau &= 2 \mu \symmetricpart{\nabla{}\frac{m}{\rho}}
        + \lambda\left(\nabla\cdot{}\frac{m}{\rho}\right) I
\end{align}
\end{subequations}
and we have employed the notation
$\symmetricpart{A}=\frac{1}{2}\left(A+\trans{A}\right)$.

\subsection{Communications overhead}

Suzerain performs time advancement in wave space but must compute nonlinear
terms in physical space.  The communications and computation cost required to
convert state data from wave space to physical space or vice versa is very
high.  Consequently, we transform back and forth only once per time integration
substep.  This implies that we must be able to compute derived quantity
derivatives using state quantity derivatives.

\subsection{Velocity derivative expansions}
\label{velocity_derivative_expansions}

We expand the necessary velocity derivatives into a combination of terms each
containing derivative operators applied only to state quantities:
\begin{subequations}
\begin{align}
  \nabla\cdot\frac{m}{\rho}
  &=
  \rho^{-1}\left[ \nabla\cdot{}m - \rho^{-1}m\cdot\nabla\rho \right]
\\
  \nabla{}\frac{m}{\rho}
  &=
  \rho^{-1}\left[ \nabla{}m - \rho^{-1}{m}\otimes\nabla\rho  \right]
\\
  \symmetricpart{\nabla\frac{m}{\rho}}
  &=
  \rho^{-1}\left[
      \symmetricpart{\nabla{}m}
    - \symmetricpart{\rho^{-1}m\otimes\nabla\rho}
  \right]
\\
  \Delta\frac{m}{\rho}
  &=
 \rho^{-1}\left[
      \Delta{}m
    + \rho^{-1}\left[
          \left(
              2\rho^{-1}\left(\nabla\rho\right)^{2}
            - \Delta\rho
          \right) {m}
        - 2 \left(\nabla{}m\right)\nabla\rho
      \right]
 \right]
\\
  \nabla\nabla\cdot\frac{m}{\rho}
  &=
  \rho^{-1}\left[
        \nabla\nabla\cdot{}m
      + \rho^{-1}\left[
            \left(2\rho^{-1}\nabla\rho\cdot{}m-\nabla\cdot{}m\right)\nabla\rho
          - \left(\nabla\nabla\rho\right)m
          - \trans{\nabla{}m}\nabla\rho
        \right]
  \right]
\end{align}
\end{subequations}

We note some relationships amongst the information appearing in such
derivatives:
\begin{align}
  \Delta\rho
  &=
  \trace\left( \nabla\nabla\rho \right)
&
  \nabla\cdot{}m
  &=
  \trace\left(\nabla{}m\right)
  =
  \trace\symmetricpart{\nabla\frac{m}{\rho}}
\end{align}

\subsection{Separation of first and second derivative operators}
\label{separate_first_section_deriv}

Suzerain uses a B-spline basis in the wall normal direction.  Unlike a Fourier
basis, for B-splines the repeated application of a discrete first derivative
operator gives a result that differs significantly from applying a discrete
second derivative operator.  In particular, repeated first differentiation
severely abates high frequency modes (see figures 2 and 3 in ~\cite{Kwok2001}).

Second differentiation enters equations~\eqref{eq:state_continuity},
\eqref{eq:state_momentum}, and~\eqref{eq:state_energy} through the terms
$\nabla\cdot\tau$, $\nabla\cdot\tau\frac{m}{\rho}$, and
$\nabla\cdot\mu\nabla{}T$.  We wish to compute these terms in a way that keeps
first and second derivative applications wholly separate.  Doing so will help
ensure that these three terms have the most physically correct diffusive impact
on high frequency content at a given spatial resolution.

We expand the three mixed order, nonlinear terms:
\begin{subequations}
\begin{align}
  \nabla\cdot\tau
  &=
    2 \symmetricpart{\nabla\frac{m}{\rho}}\nabla\mu
  + \mu \Delta\frac{m}{\rho}
  + \left(\mu+\lambda\right)\nabla\nabla\cdot\frac{m}{\rho}
  + \left(\nabla\cdot\frac{m}{\rho}\right)\nabla\lambda
\\
  \nabla\cdot\tau{}\frac{m}{\rho}
  &=
    \frac{m}{\rho}\cdot\left(\nabla\cdot\tau\right)
  + \trace\left( \tau\,\nabla\frac{m}{\rho} \right)
\\
  \nabla\cdot\mu\nabla{}T
  &=
    \nabla\mu\cdot\nabla{}T
  + \mu \Delta{}T
\end{align}
\end{subequations}
Many of these term contains non-state quantity derivatives which we also
expand:
\begin{subequations}
\begin{align}
  \nabla{}p &= (\gamma-1)\left[
        \nabla{}e + \frac{\Mach^{2}}{\rho} \left[
            \frac{m^{2}}{2\rho} \nabla\rho
          - \trans{\nabla{}m}m
        \right]
  \right]
\\
  \nabla{}T &= \frac{\gamma}{\rho}
               \left[ \nabla{}p - \frac{p}{\rho}\nabla\rho \right]
             = \rho^{-1}\left( \gamma\nabla{}p - T\nabla\rho \right)
\\
  \nabla\mu &= \beta{}T^{\beta-1}\nabla{}T
\\
  \nabla\lambda &= \left(\alpha-\frac{2}{3}\right)\nabla\mu
\\
  \Delta{}p
  &=
  \left(\gamma-1\right)\left[
      \Delta{}e
      - \frac{\Mach^{2}}{\rho}\left[
            \trace\left( \trans{\nabla{}m}\nabla{}m \right)
          + m\cdot\Delta{}m
\right.\right. \notag\\ &\qquad\qquad\qquad\qquad \left.\left. % LINE BREAK
        {}- \rho^{-1}\left[
                2\trans{\nabla{}m}m\cdot\nabla{}\rho
              + \frac{1}{2} m^2 \Delta\rho
              - \rho^{-1} m^2 \left(\nabla\rho\right)^{2}
          \right]
      \right]
  \right]
\\
  \Delta{}T
  &=
  \gamma\rho^{-1}\left[
        \Delta{}p
      - \rho^{-1}\left[
            p\Delta{}\rho
          + 2\nabla{}\rho\cdot\left( \nabla{}p - \rho^{-1}p\nabla\rho \right)
      \right]
  \right]
\end{align}
\end{subequations}

Though we could have found these expressions for only the wall-normal
direction, writing them for the complete $\nabla$ operator allows us to reuse
them later.

\subsection{Fully explicit scheme cost tabulation}

Though Suzerain uses a hybrid implicit/explicit time stepping scheme and a
Fourier basis in the streamwise and spanwise directions where repeated first
differentiation is acceptable, it is useful to examine the communication costs
for purely a explicit time stepper when we do not mix derivative orders.

Using the information in sections~\ref{velocity_derivative_expansions}
and~\ref{separate_first_section_deriv} we tabulate in
table~\ref{tab:nofirstderivnonlinearcost} the specific state variable
derivatives necessary to compute each mixed-derivative nonlinear term and all
of its contributions.  From this table and the equations appearing in
section~\ref{state_variable_selection}, a fully explicit timestepping approach
could compute a single substep at the cost of converting 33 scalar fields from
wave space to physical space, forming 5 scalars representing the right hand
sides of equations~\eqref{eq:state_continuity}--\eqref{eq:state_energy}, and
converting 5 scalar fields back to wave space.

%%%%%%%%%%%%%%%%%%%%%%%%%%%%%%%%%%%%%%%%%%%%%%%%%%%%%%%%%%%%%%%%%%%%%%%%%%%%%%
%%%%%%%%%%%%%%%%%%%%%%%%%%%%%%%%%%%%%%%%%%%%%%%%%%%%%%%%%%%%%%%%%%%%%%%%%%%%%%
\begin{table}[p]
\centering
\vspace{1em}
\renewcommand{\arraystretch}{1.40}   % Adds whitespace between rows
\newcommand{\cm}{\checkmark}         % For brevity in the table details
\newcommand{\cd}{\ensuremath{\cdot}} % For brevity in the table details
\begin{tabular}{r|cccc|cccccc|ccc|r}
% 001 & 002 & 003 & 004 & 005 & 006 & 007 & 008 & 009 & 011 & 012 & 013 & 014
&   1 &   3 &   1 &   6 &   3 &   1 &   6 &   9 &   3 &   3 &   1 &   3 &   1
\\
& $\rho$                                              % 01
& $\nabla\rho$                                        % 02
& $\Delta\rho$                                        % 03
& $\nabla\nabla\rho$                                  % 04
& $m$                                                 % 05
& $\nabla\cdot{}m$                                    % 06
& $\symmetricpart{\nabla{}m}$                         % 07
& $\nabla{}m$                                         % 08
& $\Delta{}m$                                         % 09
& $\nabla\nabla\cdot{}m$                              % 11
& $e$                                                 % 12
& $\nabla{}e$                                         % 13
& $\Delta{}e$                                         % 14
\\ \hline
$\nabla\cdot\frac{m}{\rho}$
% 001 & 002 & 003 & 004 & 005 & 006 & 007 & 008 & 009 & 011 & 012 & 013 & 014
& \cm & \cm &     &     & \cm & \cm &     &     &     &     &     &     &
& 8 \\
$\nabla\frac{m}{\rho}$
% 001 & 002 & 003 & 004 & 005 & 006 & 007 & 008 & 009 & 011 & 012 & 013 & 014
& \cm & \cm &     &     & \cm &     &     & \cm &     &     &     &     &
& 16 \\
$\symmetricpart{\nabla\frac{m}{\rho}}$
% 001 & 002 & 003 & 004 & 005 & 006 & 007 & 008 & 009 & 011 & 012 & 013 & 014
& \cm & \cm &     &     & \cm &     & \cm &     &     &     &     &     &
& 13 \\
$\Delta\frac{m}{\rho}$
% 001 & 002 & 003 & 004 & 005 & 006 & 007 & 008 & 009 & 011 & 012 & 013 & 014
& \cm & \cm & \cm &     & \cm &     &     & \cm & \cm &     &     &     &
& 20 \\
$\nabla\nabla\cdot\frac{m}{\rho}$
% 001 & 002 & 003 & 004 & 005 & 006 & 007 & 008 & 009 & 011 & 012 & 013 & 014
& \cm & \cm &     & \cm & \cm & \cd &     & \cm &     & \cm &     &     &
& 25 \\[1.5em]
$p$, $T$, $\mu$, $\lambda$
% 001 & 002 & 003 & 004 & 005 & 006 & 007 & 008 & 009 & 011 & 012 & 013 & 014
& \cm &     &     &     & \cm &     &     &     &     &     & \cm &     &
& 5 \\
$\nabla{}p$, $\nabla{}T$, $\nabla\mu$, $\nabla\lambda$
% 001 & 002 & 003 & 004 & 005 & 006 & 007 & 008 & 009 & 011 & 012 & 013 & 014
& \cm & \cm &     &     & \cm &     &     & \cm &     &     & \cm & \cm &
& 20 \\
$\Delta{}p$
% 001 & 002 & 003 & 004 & 005 & 006 & 007 & 008 & 009 & 011 & 012 & 013 & 014
& \cm & \cm & \cm &     & \cm &     &     & \cm & \cm &     &     &     & \cm
& 21 \\
$\Delta{}T$
% 001 & 002 & 003 & 004 & 005 & 006 & 007 & 008 & 009 & 011 & 012 & 013 & 014
& \cm & \cm & \cm &     & \cm &     &     & \cm & \cm &     & \cm & \cm & \cm
& 25 \\[1.5em]
$\tau$
% 001 & 002 & 003 & 004 & 005 & 006 & 007 & 008 & 009 & 011 & 012 & 013 & 014
& \cm & \cm &     &     & \cm & \cd & \cm &     &     &     & \cm &     &
& 14 \\[1.5em]
$\symmetricpart{\nabla\frac{m}{\rho}} \nabla\mu$
% 001 & 002 & 003 & 004 & 005 & 006 & 007 & 008 & 009 & 011 & 012 & 013 & 014
& \cm & \cm &     &     & \cm &     & \cd & \cm &     &     & \cm & \cm &
& 20 \\
$\mu\Delta\frac{m}{\rho}$
% 001 & 002 & 003 & 004 & 005 & 006 & 007 & 008 & 009 & 011 & 012 & 013 & 014
& \cm & \cm & \cm &     & \cm &     &     & \cm & \cm &     & \cm &     &
& 21 \\
$\left(\mu+\lambda\right)\nabla\nabla\cdot\frac{m}{\rho}$
% 001 & 002 & 003 & 004 & 005 & 006 & 007 & 008 & 009 & 011 & 012 & 013 & 014
& \cm & \cm &     & \cm & \cm & \cd &     & \cm &     & \cm & \cm &     &
& 26 \\
$\left(\nabla\cdot\frac{m}{\rho}\right)\nabla\lambda$
% 001 & 002 & 003 & 004 & 005 & 006 & 007 & 008 & 009 & 011 & 012 & 013 & 014
& \cm & \cm &     &     & \cm & \cd &     & \cm &     &     & \cm & \cm &
& 20 \\
$\nabla\cdot\tau$
% 001 & 002 & 003 & 004 & 005 & 006 & 007 & 008 & 009 & 011 & 012 & 013 & 014
& \cm & \cm & \cd & \cm & \cm & \cd & \cd & \cm & \cm & \cm & \cm & \cm &
& 32 \\[1.5em]
$\frac{m}{\rho}\cdot\left(\nabla\cdot\tau\right)$
% 001 & 002 & 003 & 004 & 005 & 006 & 007 & 008 & 009 & 011 & 012 & 013 & 014
& \cm & \cm & \cd & \cm & \cm & \cd & \cd & \cm & \cm & \cm & \cm & \cm &
& 32 \\
$\trace\left(\tau\nabla\frac{m}{\rho}\right)$
% 001 & 002 & 003 & 004 & 005 & 006 & 007 & 008 & 009 & 011 & 012 & 013 & 014
& \cm & \cm &     &     & \cm & \cd & \cd & \cm &     &     & \cm &     &
& 20 \\
$\nabla\cdot\tau\frac{m}{\rho}$
% 001 & 002 & 003 & 004 & 005 & 006 & 007 & 008 & 009 & 011 & 012 & 013 & 014
& \cm & \cm & \cd & \cm & \cm & \cd & \cd & \cm & \cm & \cm & \cm & \cm &
& 32 \\[1.5em]
$\nabla\mu\cdot\nabla{}T$
% 001 & 002 & 003 & 004 & 005 & 006 & 007 & 008 & 009 & 011 & 012 & 013 & 014
& \cm & \cm &     &     & \cm &     &     & \cm &     &     & \cm & \cm &
& 20 \\
$\mu\Delta{}T$
% 001 & 002 & 003 & 004 & 005 & 006 & 007 & 008 & 009 & 011 & 012 & 013 & 014
& \cm & \cm & \cm &     & \cm &     &     & \cm & \cm &     & \cm & \cm & \cm
& 25 \\
$\nabla\cdot\mu\nabla{}T$
% 001 & 002 & 003 & 004 & 005 & 006 & 007 & 008 & 009 & 011 & 012 & 013 & 014
& \cm & \cm & \cm &     & \cm &     &     & \cm & \cm &     & \cm & \cm & \cm
& 25
\end{tabular}
\vspace{1em}
\caption{
    State variable derivatives and the relative computational cost required to
    completely compute quantities in physical space without using repeated
    first derivative applications.  A check (\checkmark) indicates that a
    quantity is required to compute the given term.  A dot ($\cdot$) indicates
    the quantity is required but it can be computed from other required
    quantities.  Costs are given relative to the cost of transforming a single
    scalar field from wave space to physical space and do not include floating
    point operations. The total cost for each term is found in the rightmost
    column of the table.
}
\label{tab:nofirstderivnonlinearcost}
\end{table}
%%%%%%%%%%%%%%%%%%%%%%%%%%%%%%%%%%%%%%%%%%%%%%%%%%%%%%%%%%%%%%%%%%%%%%%%%%%%%%
%%%%%%%%%%%%%%%%%%%%%%%%%%%%%%%%%%%%%%%%%%%%%%%%%%%%%%%%%%%%%%%%%%%%%%%%%%%%%%

\subsection{Linearization of second order viscous terms}

Suzerain uses a hybrid implicit/explicit time stepper to alleviate the very
restrictive time step requirements that arise if we advance time in a purely
explicit fashion.  The hybrid time stepper requires the implicit operator to be
linear in the state variables and time-independent.  Precious little of the
full operator as written in
equations~\eqref{eq:state_continuity}--\eqref{eq:state_energy} meets these
criteria.  We must carve it up.

Because second derivatives give rise to the most restrictive eigenvalues for
timestepping, we focus on how second order terms enter the full operator.  We
take the quantities from sections~\ref{velocity_derivative_expansions}
and~\ref{separate_first_section_deriv} and expand the nonlinear coefficients on
second order terms about reference states.  For example, $\rho^{-1}\Delta{}m =
\lessreference{\rho^{-1}}\Delta{}m + \reference{\rho^{-1}}\Delta{}m$ where
$\reference{\rho^{-1}}$ indicates the term $\rho^{-1}$ evaluated at some given,
fixed state.  This approach separates each quantity into an explicitly-treated
nonlinear portion plus a linear contribution that satisfies our time stepper's
implicit operator needs:
\begin{subequations}
\begin{align}
\mu\Delta\frac{m}{\rho} = &\phantom{{}+}
    \mu\rho^{-2}\left[
          2\rho^{-1}m\left(\nabla\rho\right)^{2}
        - 2 \left(\nabla{}m\right)\nabla\rho
    \right]
\\
  &{}+ \lessreference{\mu\rho^{-1}} \Delta{}m
     - \lessreference{\mu\rho^{-2}m} \Delta\rho
\\
  &{}+ \reference{\mu\rho^{-1}} \Delta{}m
     - \reference{\mu\rho^{-2}m} \Delta\rho
\\
\left(\mu+\lambda\right)\nabla\nabla\cdot\frac{m}{\rho} = &\phantom{{}+}
   \left(\mu+\lambda\right)\rho^{-2}\left[
       \left(2\rho^{-1}\nabla\rho\cdot{}m-\nabla\cdot{}m\right)\nabla\rho
     - \trans{\nabla{}m}\nabla\rho
   \right]
\\
  &{}+ \lessreference{\left(\mu+\lambda\right)\rho^{-1}} \nabla\nabla\cdot{}m
     - \nabla\nabla\rho \lessreference{\left(\mu+\lambda\right)\rho^{-2}m}
\\
  &{}+ \reference{\left(\mu+\lambda\right)\rho^{-1}} \nabla\nabla\cdot{}m
     - \nabla\nabla\rho \reference{\left(\mu+\lambda\right)\rho^{-2}m}
\\
\Delta{}p =
  &{}- \left(\gamma-1\right)\Mach^{2}\rho^{-1}\left[
             \trace\left(\trans{\nabla{}m}\nabla{}m\right)
           - \rho^{-1}\left[
               2\trans{\nabla{}m}m\cdot\nabla{}\rho
             - \rho^{-1} m^2 \left(\nabla\rho\right)^{2}
           \right]
       \right]
\\
  &{}+ \left(\gamma-1\right)\Delta{}e
     - \left(\gamma-1\right)\Mach^{2}\rho^{-1}m\cdot\Delta{}m
     + \frac{\gamma-1}{2}\Mach^{2}\rho^{-2}m^2 \Delta\rho
\\
\mu\Delta{}T =
  &{}- 2\gamma\mu\rho^{-2}\nabla\rho\cdot
       \left(\nabla{}p-\rho^{-1}p\nabla\rho\right)
     + \gamma\mu\rho^{-1}\Delta{}p
     - \gamma\mu\rho^{-2}p\Delta\rho
\\
=
  &{}- 2\gamma\mu\rho^{-2}\nabla{}\rho\cdot
       \left(\nabla{}p-\rho^{-1}p\nabla\rho\right)
\\
  &{}- \gamma\left(\gamma-1\right)\Mach^{2}\mu\rho^{-2}\left[
             \trace\left(\trans{\nabla{}m}\nabla{}m\right)
           - \rho^{-1}\left[
               2\trans{\nabla{}m}m\cdot\nabla{}\rho
             - \rho^{-1} m^2 \left(\nabla\rho\right)^{2}
           \right]
       \right]
\\
  &{}+ \gamma\left(\gamma-1\right)\lessreference{\mu\rho^{-1}}\Delta{}e
     - \gamma\left(\gamma-1\right)\Mach^{2}
       \lessreference{\mu\rho^{-2}m}\cdot\Delta{}m
\\
  &{}+ \gamma\lessreference{ \mu\rho^{-2}\left(
           \left(\gamma-1\right)\Mach^{2} \frac{m^2}{2\rho} - p
       \right) } \Delta\rho
\\
  &{}+ \gamma\left(\gamma-1\right)\reference{\mu\rho^{-1}}\Delta{}e
     - \gamma\left(\gamma-1\right)\Mach^{2}
       \reference{\mu\rho^{-2}m}\cdot\Delta{}m
   {}+ \gamma\reference{ \mu\rho^{-2}\left(
           \left(\gamma-1\right)\Mach^{2} \frac{m^2}{2\rho} - p
       \right) } \Delta\rho
\end{align}
\end{subequations}
The last line of each quantity is the linearized, implicit-ready portion.
Evaluating reference values from mean quantities taken as a function of
wall-normal position should provide a reasonable time/storage tradeoff.  Note
that we will recover an explicit-only operator if we choose identically zero
reference values.

\section{In support of Favre-averaged Navier--Stokes modeling}
\label{sec:supportFANS}

The Favre-averaged Navier--Stokes (FANS) equations are often used to estimate
the mean effects of turbulence.  The unclosed FANS equations require modeling
approximations to be solvable.  Statistics gathered from Suzerain's solution of
the Navier--Stokes equations may be used to inform the development and
application of FANS closures.  Extensive background may be found in
books by Chassaing et.~al.~\cite{Chassaing2010} or Smits and
Dussauge~\cite{SmitsDussauge2005}.

The material in this section borrows liberally (and often literally) from work
by Todd Oliver~\cite{OliverFANSModels2011}.  It departs from that particular
document in that it employs Suzerain's constitutive relationships, avoids
introducing customary assumptions about the relative importance of unclosed
terms, accounts for forcing, and nondimensionalizes the results.


\subsection{Reynolds- and Favre-averages}
\label{sec:averaging}

The Reynolds average is simply the usual mean of a random variable.  Consider a
generic flow variable $q$.  The value, $q(x, y, z, t)$, of this variable at a
particular point in space, $(x, y, z)$, and time, $t$, is a random variable.
Assuming that the probability density function for $q(x, y, z, t)$ is given by
$\pi_q(V; x, y, z, t)$, the Reynolds average is defined by
%
\begin{equation}
\label{eqn:reynoldsAvg}
\bar{q}(x, y, z, t) \equiv \int V \pi_q(V; x, y, z, t) \,\mathrm{d} V.
\end{equation}
%
The Favre average is defined as the density-weighted average.  Thus,
denoting the fluid density by $\rho(x,y,z, t)$, the Favre average of
$q(x,y,z, t)$ is
%
\begin{equation*}
\tilde{q}(x,y,z, t) \equiv \frac{ \overline{\rho q}(x,y,z, t) }{ \bar{\rho}(x,y,z, t) }.
\end{equation*}
%
For the remainder of this section to make sense mathematically, it is
assumed that both the Reynolds and Favre averages are well-defined for
any required flow variable, $q$.  That is, the integral on the
right-hand side of (\ref{eqn:reynoldsAvg}) exists whenever required,
and the Reynolds-averaged density, $\bar{\rho}$, is positive
everywhere.

In the following, the flow variables will be decomposed into mean and
fluctuating parts.  Specifically, the fluctuations about the
mean---denoted by $(\cdot)'$ and $(\cdot)''$ for the Reynolds and
Favre averages, respectively---are defined by the following
relationships:
%
\begin{align*}
q' &\equiv q - \bar{q}, \\
q'' &\equiv q - \tilde{q}.
\end{align*}
%
Using the linearity of the Reynolds average and the fact that
$\bar{q}$ and $\tilde{q}$ are deterministic, not random, variables, it
is straightforward to see that
%
\begin{gather*}
\overline{q'} = \overline{q - \bar{q}} = \bar{q} - \bar{q} =  0, \\
\widetilde{q''} = \widetilde{q - \tilde{q}} = \tilde{q} - \tilde{q} = 0.
\end{gather*}
%
Furthermore,
%
\begin{equation*}
\overline{\rho q''} = \bar{\rho} \widetilde{q''} = 0.
\end{equation*}
%
However, in general,
%
\begin{equation*}
\overline{q''} = \overline{q - \tilde{q}} = \bar{q} - \tilde{q} \neq 0.
\end{equation*}
%

Finally, wherever necessary, realizations of random fields of flow
quantities---e.g., $u$, $p$, $\tau$, etc.---are assumed to be differentiable in
both time and space so that averaging and differentiation commute.  For
example,
%
\begin{equation*}
\widetilde{ \nabla{}u } = \nabla\tilde{u}.
\end{equation*}
%
This operation is used extensively in the development of the FANS equations.

\subsection{The dimensional Favre-averaged Navier--Stokes equations}

\subsubsection{Derivation}

Recall that equations~\eqref{eq:cons_mass}, \eqref{eq:cons_momentum},
and~\eqref{eq:cons_energy} may be written as
\begin{align}
    \frac{\partial}{\partial{}t}\rho
&=
  - \nabla\cdot\rho{}u
\\
    \frac{\partial{}}{\partial{}t}\rho{}u
&=
  - \nabla\cdot(u\otimes{}\rho{}u)
  - \nabla{}p + \nabla\cdot{}\tau + f
\\
    \frac{\partial}{\partial{}t} \rho{}E
&=
  - \nabla\cdot{}\rho{}Hu
  + \nabla\cdot{}\tau{}u
  - \nabla\cdot{}q_{s}
  + f\cdot{}u
  + \rho{}q_b
\end{align}
where we have employed total enthalpy $H$ to reduce the number of terms in the
energy equation.

A lengthy algebraic procedure (detailed in section~2
of~\cite{OliverFANSModels2011}) produces exact equations governing the
evolution of mean conserved quantities $\bar{\rho}$, $\overline{\rho{}u}=
\bar{\rho}\tilde{u}$, and $\overline{\rho{}E} = \bar{\rho}\tilde{E}$:
\begin{subequations}\label{eq:unclosedfansequations}
\begin{align}
    \frac{\partial}{\partial{}t}\bar{\rho}
 =
 &- \nabla\cdot\bar{\rho}\tilde{u}
\\
    \frac{\partial{}}{\partial{}t}\bar{\rho}\tilde{u}
 =
 &- \nabla\cdot(\tilde{u}\otimes\bar{\rho}\tilde{u})
  - \nabla{}\bar{p}
  + \nabla\cdot\left(
        \bar{\tau}
      - \bar{\rho}\widetilde{u''\otimes{}u''}
    \right)
  + \bar{f}
\\
  \frac{\partial}{\partial{}t} \bar{\rho}\tilde{E}
 =
 &- \nabla\cdot{}\bar{\rho}\tilde{H}\tilde{u}
  + \nabla\cdot\left(
        \left(
            \bar{\tau}
          - \bar{\rho} \widetilde{u''\otimes{}u''}
        \right) \tilde{u}
      - \frac{1}{2}\bar{\rho}\widetilde{{u''}^{2}u''}
      + \overline{\tau{}u''}
    \right)
\\
 &- \nabla\cdot\left(
        \bar{q}_s
      + \bar{\rho} \widetilde{h''u''}
    \right)
  + \bar{f}\cdot\tilde{u}
  + \overline{f\cdot{}u''}
  + \bar{\rho}\tilde{q}_b
\end{align}
\end{subequations}
Several correlations impact the evolution of mean quantities: the Reynolds
stress $-\bar{\rho}\widetilde{u''\otimes{}u''}$, the Reynolds heat flux
$\bar{\rho} \widetilde{h''u''}$, turbulent transport
$-\frac{1}{2}\bar{\rho}\widetilde{{u''}^{2}u''}$, turbulent work
$\overline{\tau{}u''}$, and the forcing-velocity correlation
$\overline{f\cdot{}u''}$.  The Reynolds stress and heat flux augment the
viscous stress and heat flux, respectively.  The turbulent transport and work
terms represent transport of the turbulent kinetic energy density $k$, defined
below, and viscous stress work due to turbulent velocity fluctuations,
respectively.

We now average the perfect gas relations from section~\ref{sec:constitutive}.
The Reynolds average of~\eqref{eq:perfectgaseos} gives
\begin{align}
  \bar{p} &= R\overline{\rho{}T} = \bar{\rho}R\tilde{T}
\end{align}
while the Favre average of~\eqref{eq:perfectgasenthalpy} gives both
\begin{align}
 \tilde{H} &= \tilde{E} + R \tilde{T}
&
 \tilde{h} &= \frac{\gamma{}R\tilde{T}}{\gamma-1}.
\end{align}
The turbulent kinetic energy density
\begin{align}
  k &= \frac{1}{2}\widetilde{{u''}^2}
 \end{align}
arises from averaging the total energy given by
\eqref{eq:perfectgastotalenergy}:
\begin{align}
  \rho{} E
&=
  \frac{R}{\gamma-1} \rho{}T + \frac{1}{2}\rho{} u^{2}
\\
&=
  \frac{R}{\gamma-1} \rho{}\left( \tilde{T} + T'' \right)
+ \frac{1}{2}\rho{} \left( \tilde{u} + u'' \right)^2
\\
  \overline{\rho{}E}
&=
  \frac{R}{\gamma-1} \bar{\rho} \tilde{T}
+ \frac{1}{2}\bar{\rho} \tilde{u}^2
+ \frac{1}{2}\overline{\rho{}{u''}^2}
\\
  \tilde{E}
&=
  \frac{R}{\gamma-1} \tilde{T}
+ \frac{1}{2} \tilde{u}^2
+ k
\end{align}

An exact equation may be derived for the evolution of $k$ (details in section~5
of~\cite{OliverFANSModels2011})
\begin{align}
\label{eq:fanstke1}
    \frac{\partial{}}{\partial{}t}\bar{\rho}k
 =
 &- \nabla\cdot\bar{\rho}k\tilde{u}
  - \bar{\rho} \widetilde{u''\otimes{}u''} : \nabla\tilde{u}
  - \bar{\rho} \epsilon
  + \nabla\cdot\left(
        -\frac{1}{2}\bar{\rho}\widetilde{{u''}^{2}u''}
      + \overline{\tau{}u''}
    \right)
\\
 &- \overline{u''}\cdot\nabla\bar{p}
  - \nabla\cdot\overline{p' u''}
  + \overline{p' \nabla\cdot{}u''}
  + \overline{f\cdot{}u''}
\end{align}
where $A:B$ denotes $\trace \left(\trans{A} B\right)$. The dissipation rate
density $\epsilon$, which governs the conversion rate from $k$ to mean internal
energy, is defined by
\begin{align}
  \bar{\rho} \epsilon &= \overline{\tau : \nabla{}u''}
.
\end{align}
As page~216 of Lele~\cite{Lele1994Compressibility} suggests, expanding $h$,
averaging, removing the mean state from both sides, and applying perfect gas
assumptions demonstrates the exact relationship
\begin{align}
  \overline{u''}
&=
  \frac{\widetilde{T''u''}}{\tilde{T}} - \frac{\overline{p'u''}}{\bar{p}}
.
\end{align}
Substituting $h''$ everywhere for $T''$, noting $\bar{p}/\tilde{h} =
\frac{\gamma-1}{\gamma}\bar{\rho}$, and differentiating one obtains
\begin{align}
  \overline{p'u''}
&=
  \frac{\gamma-1}{\gamma} \bar{\rho} \widetilde{h''u''}
- \bar{p} \overline{u''}
\\
  \nabla\cdot \overline{p'u''}
&=
  \frac{\gamma-1}{\gamma} \nabla\cdot \bar{\rho} \widetilde{h''u''}
- \bar{p}\nabla\cdot\overline{u''}
- \overline{u''}\cdot\nabla{}\bar{p}
.
\end{align}
Rearranging the above result to mimic terms within~\eqref{eq:fanstke1}
\begin{align}
  - \overline{u''}\cdot\nabla\bar{p}
  - \nabla\cdot\overline{p'u''}
&=
  \bar{p}\nabla\cdot\overline{u''}
- \frac{\gamma-1}{\gamma} \nabla\cdot \bar{\rho} \widetilde{h''u''}
\end{align}
allows us to trade an occurrence of $\overline{p'u''}$ for the Reynolds heat
flux in the exact $k$ equation:
\begin{align}
\label{eq:fanstke}
    \frac{\partial{}}{\partial{}t}\bar{\rho}k
 =
 &- \nabla\cdot\bar{\rho}k\tilde{u}
  - \bar{\rho} \widetilde{u''\otimes{}u''} : \nabla\tilde{u}
  - \bar{\rho} \epsilon
  + \nabla\cdot\left(
        -\frac{1}{2}\bar{\rho}\widetilde{{u''}^{2}u''}
      + \overline{\tau{}u''}
    \right)
\\
 &+ \bar{p}\nabla\cdot\overline{u''}
  - \frac{\gamma-1}{\gamma} \nabla\cdot\bar{\rho} \widetilde{h''u''}
  + \overline{p' \nabla\cdot{}u''}
  + \overline{f\cdot{}u''}
\end{align}
The trade reduces by one the number of correlations appearing in the $k$
equation which do not appear in the mean continuity, momentum, or energy
equations.  It also, as Lele suggests, encourages thermodynamic consistency
when working with pressure correlation information.

Returning to the constitutive relations, combining~\eqref{eq:tauSmub}
and~\eqref{eq:secondviscosityclaw} we obtain
\begin{align}
  \tau
&= 2 \mu{} S + \alpha \mu \left( \nabla\cdot{}u \right) I.
\end{align}
Using the kinematic viscosity $\nu = \mu / \rho$ and averaging we find
\begin{align}
   \bar{\tau}
&= 2 \bar{\mu}\tilde{S} + 2 \bar{\rho} \widetilde{\nu''S''}
  + \alpha \bar{\mu} \left(\nabla\cdot\tilde{u}\right) I
  + \alpha \bar{\rho} \widetilde{\nu''\nabla\cdot{}u''} I
.
\end{align}
Some FANS closure approximations neglect correlations between the kinematic
viscosity and velocity derivatives.  Many assume $\alpha=0$.  Accepting those
approximations would eliminate the second through fourth terms in $\bar{\tau}$.
To find $\bar{q}_s$ we combine~\eqref{eq:fourierlaw} and our assumption of a
constant Prandtl number
\begin{align}
  q_{s} &= - \kappa \nabla{} T
     = - \frac{\kappa}{C_p} \nabla{}h
     = - \frac{\mu}{\Prandtl} \nabla{}h
\end{align}
and again employ $\nu$ when averaging to obtain
\begin{align}
  \bar{q}_s
&= - \frac{1}{\Prandtl}\left(
                \bar{\mu}\nabla\tilde{h}
              + \bar{\rho} \widetilde{\nu''\nabla{}h''}
            \right)
.
\end{align}
Straightforward averaging applied to~\eqref{eq:powerlawviscosity} produces
\begin{align}
   \bar{\rho}\tilde{\nu}
 = \bar{\mu}
&= \mu_0 \overline{\left(\frac{T}{T_0}\right)^\beta}
\end{align}
which is not computable given only Favre-averaged state.  One commonly accepted
simplification is taking $\overline{\mu\left(T\right)} \approx
\mu\left(\tilde{T}\right)$.

\subsubsection{Summary}

The Favre-averaged equations of interest are:
\begin{subequations}
\begin{align}
    \frac{\partial}{\partial{}t}\bar{\rho}
=
 &- \nabla\cdot\bar{\rho}\tilde{u}
\\
    \frac{\partial{}}{\partial{}t}\bar{\rho}\tilde{u}
 =
 &- \nabla\cdot(\tilde{u}\otimes\bar{\rho}\tilde{u})
  - \nabla{}\bar{p}
  + \nabla\cdot\left(
        \bar{\tau}
      - \bar{\rho} \widetilde{u''\otimes{}u''}
    \right)
  + \bar{f}
\\
    \frac{\partial}{\partial{}t} \bar{\rho}\tilde{E}
 =
 &- \nabla\cdot{}\bar{\rho}\tilde{H}\tilde{u}
  + \nabla\cdot\left(
        \left(
            \bar{\tau}
          - \bar{\rho} \widetilde{u''\otimes{}u''}
        \right) \tilde{u}
      - \frac{1}{2}\bar{\rho}\widetilde{{u''}^{2}u''}
      + \overline{\tau{}u''}
    \right)
\\
 &- \nabla\cdot\left(
        \bar{q}_s
      + \bar{\rho} \widetilde{h''u''}
    \right)
  + \bar{f}\cdot\tilde{u}
  + \overline{f\cdot{}u''}
  + \bar{\rho}\tilde{q}_b
\\
    \frac{\partial{}}{\partial{}t}\bar{\rho}k
=
 &- \nabla\cdot\bar{\rho}k\tilde{u}
  - \bar{\rho} \widetilde{u''\otimes{}u''} : \nabla\tilde{u}
  - \bar{\rho} \epsilon
  + \nabla\cdot\left(
        -\frac{1}{2}\bar{\rho} \widetilde{{u''}^{2}u''}
      + \overline{\tau{}u''}
    \right)
\\
 &+ \bar{p}\nabla\cdot\overline{u''}
  - \frac{\gamma-1}{\gamma} \nabla\cdot\bar{\rho} \widetilde{h''u''}
  + \overline{p' \nabla\cdot{}u''}
  + \overline{f\cdot{}u''}
\end{align}
The equations are augmented by the following relationships:
\begin{align}
  \bar{p} &= \bar{\rho}R\tilde{T}
&
   \bar{\rho}\tilde{\nu} =
   \bar{\mu}
&= \mu_0 \overline{\left(\frac{T}{T_0}\right)^\beta}
&
  k &= \frac{1}{2}\widetilde{{u''}^2}
&
  \bar{\rho} \epsilon &= \overline{\tau : \nabla{}u''}
\end{align}
\begin{align}
  \tilde{E}
&=
  \frac{R}{\gamma-1} \tilde{T}
+ \frac{1}{2} \tilde{u}^2
+ k
&
  \tilde{H}
&=
  \tilde{E}
+ R \tilde{T}
&
  \tilde{h} &= \frac{\gamma{}R\tilde{T}}{\gamma-1}
&
  \bar{q}_s
&= - \frac{1}{\Prandtl}\left(
                \bar{\mu}\nabla\tilde{h}
              + \bar{\rho} \widetilde{\nu''\nabla{}h''}
            \right)
\end{align}
\begin{align}
   \tilde{\varepsilon}
&=
   \frac{1}{2}\left( \nabla{}\tilde{u}+\trans{\nabla{}\tilde{u}} \right)
&
   \tilde{S}
&=
   \tilde{\varepsilon}-\frac{1}{3}\trace{\tilde{\varepsilon}}\,I
&
   \bar{\tau}
&= 2 \bar{\mu}\tilde{S} + 2 \bar{\rho} \widetilde{\nu''S''}
  + \alpha \bar{\mu} \left(\nabla\cdot\tilde{u}\right) I
  + \alpha \bar{\rho} \widetilde{\nu''\nabla\cdot{}u''} I
\end{align}
\end{subequations}
One may exactly compute the mean state evolution given the following
information:
\begin{align}
&\bar{\rho}
&
&\tilde{u}
&
&\tilde{E}
&
&\bar{\mu}
&
&\bar{f}
&
&\tilde{q}_b
&
&k
&
&\epsilon
&
&\overline{u''}
\end{align}
\begin{align}
&\overline{f\cdot{}u''}
&
&\overline{\tau{}u''}
&
&\overline{p'\nabla\cdot{}u''}
&
&-\widetilde{u''\otimes{}u''}
&
&-\widetilde{{u''}^{2}u''}
&
&\widetilde{h''u''}
&
&\widetilde{\nu''S''}
&
&\widetilde{\nu''\nabla\cdot{}u''}
&
&\widetilde{\nu''\nabla{}h''}
\end{align}

\subsection{The nondimensional Favre-averaged Navier--Stokes equations}
\label{sec:nondimfans}

The dimensional FANS equations from the last section need to be
nondimensionalized.   The reference quantity selections made in
section~\ref{sec:nondimrefq} are used and are augmented by
\begin{align}
  k_0 &= u_{0}^2
&
  \epsilon_0 &= \frac{u_{0}^2}{t_0}
\end{align}
Superscript star notation is suppressed as all terms
are dimensionless.  The results are:
\begin{subequations}
\begin{align}
    \frac{\partial}{\partial{}t}\bar{\rho}
=
 &- \nabla\cdot\bar{\rho}\tilde{u}
\\
    \frac{\partial{}}{\partial{}t}\bar{\rho}\tilde{u}
=
 &- \nabla\cdot(\tilde{u}\otimes\bar{\rho}\tilde{u})
  - \frac{1}{\Mach^2}\nabla{}\bar{p}
  + \nabla\cdot\left(
        \frac{\bar{\tau}}{\Reynolds}
      - \bar{\rho} \widetilde{u''\otimes{}u''}
    \right)
  + \bar{f}
\\
  \frac{\partial}{\partial{}t} \bar{\rho}\tilde{E}
=
 &- \nabla\cdot\bar{\rho}\tilde{H}\tilde{u}
  + \Mach^{2} \nabla\cdot\left(
        \left(
            \frac{\bar{\tau}}{\Reynolds}
          - \bar{\rho} \widetilde{u''\otimes{}u''}
        \right) \tilde{u}
      - \frac{1}{2}\bar{\rho}\widetilde{{u''}^{2}u''}
      + \frac{\overline{\tau{}u''}}{\Reynolds}
    \right)
\\
 &+ \frac{1}{\gamma-1} \nabla\cdot\left(
      \frac{
         \bar{\mu} \nabla \tilde{T}
       + \bar{\rho} \widetilde{\nu'' \nabla{}T''}
      }{\Reynolds\Prandtl}
      - \bar{\rho} \widetilde{T''u''}
    \right)
  + \Mach^{2} \left(
        \bar{f}\cdot\tilde{u}
      + \overline{f\cdot{}u''}
    \right)
  + \bar{\rho}\tilde{q}_b
\\
    \frac{\partial{}}{\partial{}t}\bar{\rho}k
=
 &- \nabla\cdot\bar{\rho}k\tilde{u}
  - \bar{\rho} \widetilde{u''\otimes{}u''} : \nabla\tilde{u}
  - \frac{\bar{\rho} \epsilon}{\Reynolds}
  + \nabla\cdot\left(
        -\frac{1}{2}\bar{\rho} \widetilde{{u''}^{2}u''}
      + \frac{\overline{\tau{}u''}}{\Reynolds}
    \right)
\\
 &+ \frac{1}{\Mach^2} \left(
        \bar{p}\nabla\cdot\overline{u''}
      + \overline{p' \nabla\cdot{}u''}
      - \frac{1}{\gamma} \nabla\cdot\bar{\rho} \widetilde{T''u''}
    \right)
  + \overline{f\cdot{}u''}
\end{align}
The equations are augmented by the following nondimensional relationships:
\begin{align}
  \bar{p} &= \frac{\bar{\rho} \tilde{T}}{\gamma}
&
   \bar{\rho}\tilde{\nu} =
   \bar{\mu}
&= \overline{T^\beta}
&
  k &= \frac{1}{2}\widetilde{{u''}^2}
&
  \bar{\rho} \epsilon &= \overline{\tau : \nabla{}u''}
\end{align}
\begin{align}
  \tilde{E}
&=
  \frac{\tilde{T}}{\gamma\left(\gamma-1\right)}
  + \Mach^2 \left( \frac{1}{2}\tilde{u}^2 + k
  \right)
&
  \tilde{H}
&=
  \tilde{E} + \frac{\tilde{T}}{\gamma}
&
  \tilde{h} &= \frac{\tilde{T}}{\gamma-1}
\end{align}
\begin{align}
   \tilde{\varepsilon}
&=
   \frac{1}{2}\left( \nabla{}\tilde{u}+\trans{\nabla{}\tilde{u}} \right)
&
   \tilde{S}
&=
   \tilde{\varepsilon}-\frac{1}{3}\trace{\tilde{\varepsilon}}\,I
&
   \bar{\tau}
&= 2 \bar{\mu}\tilde{S} + 2 \bar{\rho} \widetilde{\nu''S''}
  + \alpha \bar{\mu} \left(\nabla\cdot\tilde{u}\right) I
  + \alpha \bar{\rho} \widetilde{\nu''\nabla\cdot{}u''} I
\end{align}
\end{subequations}
where $\Reynolds$, $\Mach$, and $\Prandtl$ are defined as in
section~\ref{sec:nondimrefq}.  One may exactly compute the nondimensional mean
state evolution given the following statistical quantities:
\begin{align}
&\bar{\rho}
&
&\tilde{u}
&
&\tilde{E}
&
&\bar{\mu}
&
&\bar{f}
&
&\tilde{q}_b
&
&k
&
&\epsilon
&
&\overline{u''}
\end{align}
\begin{align}
&\overline{f\cdot{}u''}
&
&\overline{\tau{}u''}
&
&\overline{p'\nabla\cdot{}u''}
&
&-\widetilde{u''\otimes{}u''}
&
&-\widetilde{{u''}^{2}u''}
&
&\widetilde{T''u''}
&
&\widetilde{\nu''S''}
&
&\widetilde{\nu''\nabla\cdot{}u''}
&
&\widetilde{\nu''\nabla{}T''}
\end{align}

Notice that the heat flux $\bar{q}_s$ has been merged into the mean energy
equation to better mimic \eqref{eq:nondim_energy}.  Enthalpy-based correlations
have been replaced by temperature-based correlations.  Notice also that the
choice of $p_0$ implies nondimensional $p$ includes a factor of $\gamma$
relative to the dimensional quantity.  Where possible, nondimensional
coefficients have been pulled out of the constitutive relationships and pushed
into the equations (for example, the factor $1/\Reynolds$ arising from the
dimensional definition of $\bar{\rho}\epsilon$).

\subsection{Sampling logistics}

Statistical quantities are obtained by sampling from a well-resolved,
stationary simulation.  Mean quantities may be computed on-the-fly.
Fluctuating quantity samples, because they must be taken relative to an unknown
true mean, are computed by combining mean quantity samples following the rules
in section~\ref{sec:averaging}.

Sampling the following mean quantities is sufficient to compute
the statistical quantities listed in section \vref{sec:nondimfans}:
\begin{align}
&\bar{\rho}
&
&\overline{\rho{}u}
&
&\overline{\rho{}E}
&
&\bar{\mu}
&
&\bar{f}
&
&\overline{\rho{}q_b}
&
&\bar{u}
&
&\overline{\tau:\nabla{}u}
&
&\overline{f\cdot{}u}
\end{align}
\begin{align}
&\bar{\tau}
&
&\overline{\tau{}u}
&
&\overline{p\nabla\cdot{}u}
&
&-\overline{\rho{}u\otimes{}u}
&
&-\overline{\rho{}u\otimes{}u\otimes{}u}
&
&\overline{\rho{}Tu}
&
&\overline{\rho\nu{}S}
&
&\overline{\rho\nu\nabla\cdot{}u}
&
&\overline{\rho\nu\nabla{}T}
\end{align}
After averaging across the homogeneous streamwise $x$ and spanwise $z$
directions, each sample (ignoring tensor order) is a one-dimensional,
instantaneous profile varying only along the wall-normal B-spline direction.

The instantaneous samples are combined according to the following ordered
sequence of computations to obtain the desired quantities:
{ \allowdisplaybreaks[1]
\begin{align}
  \tilde{u} &= \frac{\overline{\rho{}u}}{\bar{\rho}}
\\
  \tilde{E} &= \frac{\overline{\rho{}E}}{\bar{\rho}}
\\
  \tilde{q} &= \frac{\overline{\rho{}q_{b}}}{\bar{\rho}}
\\
  \overline{\rho{}u''\otimes{}u''} &=
  \overline{\rho{}u\otimes{}u} - \bar{\rho}\tilde{u}\otimes\tilde{u}
\\
  - \widetilde{u''\otimes{}u''} &=
  - \frac{\overline{\rho{}u''\otimes{}u''}}{\bar{\rho}}
\\
  k &= \frac{1}{2} \trace \widetilde{u''\otimes{}u''}
\\
  \tilde{T} &=
  \gamma\left(\gamma-1\right)\tilde{E}
  - \Mach^2\left(\frac{1}{2}\tilde{u}^2 + k\right)
\\
  \tilde{H} &= \tilde{E} + \frac{\tilde{T}}{\gamma}
\\
  \bar{p} &= \frac{\bar{\rho}\tilde{T}}{\gamma}
\\
  \overline{\tau:\nabla{}u''} &=
  \overline{\tau:\nabla{}u} - \bar{\tau}:\nabla\tilde{u}
\\
  \epsilon &= \frac{\overline{\tau:\nabla{}u''}}{\bar{\rho}}
\\
  \overline{u''} &= \bar{u} - \tilde{u}
\\
  \overline{f\cdot{}u''} &= \overline{f\cdot{}u} - \bar{f}\cdot{}\tilde{u}
\\
  \overline{\tau{}u''} &= \overline{\tau{}u} - \bar{\tau}\tilde{u}
\\
  \overline{p'\nabla\cdot{}u''}
  &= \overline{p\nabla\cdot{}u}
   - \bar{p}\nabla\cdot\tilde{u}
   - \bar{p}\nabla\cdot\overline{u''}
\end{align}
}

\subsection{Quantifying statistical quantity convergence}
\label{sec:quantconvergence}


%%%%%%%%%%%%%%%%%%%%%%%%%%%%%%%%%%%%%%%%%%%%%%%%%%%%%%%%%%%%%%%%%%%%
%%%%%%%%%%%%%%%%%%%%%%%%%%% Bibliography %%%%%%%%%%%%%%%%%%%%%%%%%%%
%%%%%%%%%%%%%%%%%%%%%%%%%%%%%%%%%%%%%%%%%%%%%%%%%%%%%%%%%%%%%%%%%%%%
\bibliographystyle{amsplain}
\bibliography{references}


%%%\appendix
%%%
%%%\section{Miscellaneous}
%%%
%%%\subsection{The Variable Density Equations}
%%%
%%%The limits of the compressible Navier--Stokes equations as the Mach number
%%%$\Mach = u_{0} / a_{0} \to 0$ while allowing finite temperature and density
%%%fluctuations are called the variable density equations.

\end{document}
