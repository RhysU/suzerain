\documentclass[letterpaper,11pt,nointlimits,reqno,draft]{amsart}

% Packages
\usepackage{accents}
\usepackage{algorithm}
\usepackage{algorithmic}
\usepackage{amsfonts}
\usepackage{amsmath}
\usepackage{amssymb}
\usepackage{bm}
\usepackage{enumerate}
\usepackage{fancyhdr}
\usepackage{floatpag}
\usepackage{fullpage}
\usepackage{ifthen}
\usepackage{lastpage}
\usepackage{latexsym}
\usepackage{mathrsfs}
\usepackage{mathtools}
\usepackage{parskip}
\usepackage{pstricks}
\usepackage{rotating}
\usepackage{setspace}
\usepackage{txfonts}
\usepackage{units}
\usepackage{varioref}
\usepackage{wrapfig}
\usepackage{yhmath}

% Hyperref package must be last otherwise the contents are jumbled
% hypertexnames disabled to fix links pointing to incorrect locations
\usepackage[hypertexnames=false,final]{hyperref}

% Environment sidewaysfigure from rotating plays poorly with amsart class
% Fix per http://www.latex-community.org/forum/viewtopic.php?f=4&t=1742
\setlength\rotFPtop{0pt plus 1fil}

\mathtoolsset{showonlyrefs,showmanualtags}
%%% \allowdisplaybreaks[1] % Allow grouped equations to be split across pages

% Line Spacing
\singlespacing

% Increase table of contents depth
\setcounter{tocdepth}{4}

% Simplify headings on floating pages
\floatpagestyle{plain}
\rotfloatpagestyle{empty}

% Document-specific commands
\newcommand{\ii}{\ensuremath{\mathrm{i}}}
\newcommand{\trans}[1]{{#1}^{\ensuremath{\mathsf{T}}}}
\newcommand{\Mach}[1][]{\ensuremath{\mbox{Ma}_{#1}}}
\newcommand{\Reynolds}[1][]{\ensuremath{\mbox{Re}_{#1}}}
\newcommand{\Prandtl}[1][]{\ensuremath{\mbox{Pr}_{#1}}}
\newcommand{\reference}[1]{\ensuremath{\left\{#1\right\}_{0}}}
\newcommand{\lessreference}[1]
  {\ensuremath{\left({#1}-\reference{#1}\right)}}
\newcommand{\symmetricpart}[1]
  {\ensuremath{\operatorname{sym}\left(#1\right)}}
\DeclareMathOperator{\trace}{tr}

\begin{document}

\title{Suzerain model derivation,\\discretization, and numerical considerations}
\author{Rhys Ulerich}
\date{\today}
\thanks{Institute for Computational
    Engineering and Sciences, The University of Texas at Austin
    (rhys@ices.utexas.edu)}

\begin{abstract}
\end{abstract}

\maketitle
\renewcommand{\contentsname}{} % No idea why "Contents" appears in wrong location
\tableofcontents
\newpage

\section{Model derivation}
\label{sec:derivation}

Here we derive the mathematical model used in Suzerain.  Special attention is
paid to the origins of all conservation laws and constitutive relations
employed.  The model will nondimensionalized after derivation is complete.

\subsection{Conservation laws}

\subsubsection{Reynolds transport theorem}

Consider a time-varying control volume $\Omega$ with surface
$\partial\Omega$ and unit outward normal $\hat{n}$.  For any
scalar, vector, or tensor field quantity
$T$, Leibniz' theorem states
\begin{align}
  \label{eq:rtt}
  \frac{d}{dt}\int_{\Omega(t)}T(x,t)\,dV
  &=
  \int_{\Omega}\frac{\partial}{\partial{}t}T\,dV
  +
  \int_{\partial\Omega} \hat{n}\cdot{}w T\,dA
  =
  \int_{\Omega}\frac{\partial}{\partial{}t}T+\nabla\cdot{}wT\,dV
\end{align}
where $w$ is the velocity of $\partial\Omega$.  When $\Omega$ follows
a fixed set of fluid particles, $w$ becomes the fluid velocity $u$.

\subsubsection{Mass continuity}
Since mass $M=\int_{\Omega} \rho\,dV$
and mass conservation requires $\frac{d}{dt}M=0$,
\begin{align}
  0 = \frac{d}{dt}M
  = \frac{d}{dt}\int_{\Omega} \rho\,dV
  =
  \int_{\Omega}\frac{\partial}{\partial{}t}\rho+\nabla\cdot{}u\rho{}\,dV.
\end{align}
Because the result must hold for any control volume, we obtain
\begin{align}
  \label{eq:cons_mass}
  \frac{\partial}{\partial{}t}\rho+\nabla\cdot\rho{}u &= 0
  .
\end{align}

\subsubsection{Momentum equation}
Separating total force into surface forces and body forces
\begin{align}
  \sum{}F
  &=
     \int_{\partial\Omega} f_s \, dA
   + \int_{\Omega} f \, dV
  =
     \int_{\partial\Omega} \sigma \hat{n} \, dA
  +  \int_{\Omega} f \, dV
  =  \int_{\Omega} \nabla\cdot\sigma + f \, dV
\end{align}
where $\sigma$ is the Cauchy stress tensor.  From linear momentum
$I=\int_{\Omega} \rho{}u\,dV$ and its conservation $\frac{d}{dt}I=\sum{}F$,
\begin{align}
    \int_{\Omega}\frac{\partial{}}{\partial{}t}\rho{}u
  + \nabla\cdot(u\otimes{}\rho{}u)\,dV
&= \int_{\Omega} \nabla\cdot\sigma + f \, dV
.
\end{align}
Because the control volume may be arbitrary,
\begin{align}
  \frac{\partial{}}{\partial{}t}\rho{}u + \nabla\cdot(u\otimes{}\rho{}u)
&= \nabla\cdot\sigma + f
.
\end{align}
Lastly, we separate the pressure $p$ and viscous contributions $\tau$ to
the Cauchy stress tensor so that $\sigma = -p I + \tau$,
\begin{align}
\label{eq:cons_momentum}
\frac{\partial{}}{\partial{}t}\rho{}u + \nabla\cdot(u\otimes{}\rho{}u)
&= -\nabla{}p + \nabla\cdot{}\tau + f
.
\end{align}
The conservation of angular momentum implies $\sigma=\trans{\sigma}$ and
therefore $\tau=\trans{\tau}$ holds also.

\subsubsection{Energy equation}
Lumping internal and kinetic energy into an intrinsic density $E$,
the energy $\mathscr{E}$ is
\begin{align}
  \mathscr{E} &= \int_{\Omega} \rho{}E \, dV
  .
\end{align}
Treating heat input $Q$ as both a surface phenomenon described by an outward
heat flux $q_{s}$ and as a volumetric phenomenon governed by a
body heating $q_{b}$,
\begin{align}
  Q
  &=
   \int_{\Omega}\rho{}q_{b}\,dV
  -\int_{\partial\Omega}\hat{n}\cdot{}q_{s}\,dA
  =
    \int_{\Omega}q_{b} - \nabla\cdot{}q_{s}\,dV
  .
\end{align}
Power input $P=F\cdot{}v$ accounts for surface stress work and body
force work to give
\begin{align}
  P
  &=
    \int_{\partial\Omega} \sigma{}\hat{n} \cdot{} u \, dA
  + \int_{\Omega} f \cdot{} u \, dV
  = \int_{\Omega} \nabla\cdot{}\sigma{}u + f \cdot{} u \, dV
  .
\end{align}
Demanding energy conservation $\frac{d}{dt}\mathscr{E}=Q+P$,
\begin{align}
  \int_{\Omega}\frac{\partial}{\partial{}t} \rho{}E
  +
  \nabla\cdot{}u\rho{}E
  \,dV
&=
    \int_{\Omega} q_{b} - \nabla\cdot{}q_{s}\,dV
  + \int_{\Omega} \nabla\cdot\sigma{}u + f \cdot{} u \, dV
  .
\end{align}
Again, since the control volume was arbitrary,
\begin{align}
  \frac{\partial}{\partial{}t} \rho{}E
  +
  \nabla\cdot{}\rho{}Eu
&=
  - \nabla\cdot{}q_{s}
  + \nabla\cdot\sigma{}u
  + f \cdot{} u
  + q_{b}
  .
\end{align}
After splitting $\sigma$'s pressure and viscous stress contributions we have
\begin{align}
  \label{eq:cons_energy}
  \frac{\partial}{\partial{}t} \rho{}E
  +
  \nabla\cdot{}\rho{}Eu
&=
  - \nabla\cdot{}q_{s}
  - \nabla\cdot{}pu
  + \nabla\cdot{}\tau{}u
  + f \cdot{} u
  + q_{b}
  .
\end{align}

\subsection{Constitutive relations and other assumptions}
\label{sec:constitutive}

\subsubsection{Perfect gas}

We assume our fluid is a thermally and calorically perfect gas governed by
\begin{align}
  \label{eq:perfectgaseos}
  p &= \rho{} R T
\end{align}
where $R$ is the gas constant. The constant volume $C_{v}$ specific heat,
constant pressure specific heat $C_{p}$, and acoustic velocity $a$
relationships follow:
\begin{align}
  \label{eq:perfectgasrelations}
  \gamma &= \frac{C_{p}}{C_{v}}
  &
  C_{v} &= \frac{R}{\gamma - 1}
  &
  C_{p} &= \frac{\gamma{}R}{\gamma-1}
  &
  R &= C_{p} - C_{v}
  &
  a^{2} = \gamma{}RT
\end{align}
We assume $\gamma$ and therefore $C_{v}$ and $C_{p}$ are constant.
The total (internal and kinetic) energy density is
\begin{align}
  \label{eq:perfectgastotalenergy}
  E &= C_{v} T + \frac{1}{2}u^{2}
     = \frac{RT}{\gamma-1} + \frac{1}{2}u^{2}
\end{align}
where the notation $u^2 = u\cdot{}u$ is employed.
The total enthalpy density $H$ and (internal) enthalpy density $h$ are
\begin{align}
  \label{eq:perfectgasenthalpy}
  H &= E + \frac{p}{\rho}
     = C_{p} T + \frac{1}{2}u^{2}
     = \frac{\gamma{}RT}{\gamma-1} + \frac{1}{2}u^{2}
  &
  h &= H - \frac{1}{2}u^{2}
     = C_{p} T
     = \frac{\gamma{}RT}{\gamma-1}
  .
\end{align}
See a gas dynamics reference, e.g.~\cite{LiepmannRoshko2002}, for more details.

\subsubsection{Newtonian fluid}
\label{sec:newtonianfluid}

If we seek a constitutive law for the viscous stress tensor $\tau$
using only velocity information, the principle of material frame
indifference implies that uniform translation (given by velocity $u$)
and solid-body rotation (given by the skew-symmetric rotation tensor
$\omega=\frac{1}{2}\left( \nabla{}u-\trans{\nabla{}u} \right)$)
may not influence $\tau$.  Considering contributions only up to the
gradient of velocity, extensional strain (dilatation) and shear strain
effects may depend on only the symmetric strain rate tensor
$\varepsilon=\frac{1}{2}\left( \nabla{}u+\trans{\nabla{}u}\right)$
and its principal invariants.

Assuming $\tau$ is isotropic and depends linearly upon only $\varepsilon$,
we can express it as
\begin{align}
\tau_{ij}
&= c_{ijmn} \varepsilon_{mn}
\notag \\
&= \left( A \delta_{ij} \delta_{mn}
        + B \delta_{im} \delta_{jn}
        + C \delta_{in} \delta_{jm}
    \right) \varepsilon_{mn}
&
&\text{for some }A, B, C\in\mathbb{R}
\notag \\
&= A \delta_{ij} \varepsilon_{mm} + B\varepsilon_{ij} + C\varepsilon_{ji}
\notag \\
&= A \delta_{ij} \varepsilon_{mm} + \left( B+C \right)\varepsilon_{ji}
\notag \\
&= 2 \mu \varepsilon_{ij} + \lambda\delta_{ij}\nabla\cdot{}u
\end{align}
where $\mu=\frac{1}{2}\left( B + C \right)$ is the dynamic coefficient of
viscosity (shear) and $\lambda=A$ is the second coefficient of viscosity
(dilatational).  Reverting to direct notation we have
\begin{align}
\tau
&= 2 \mu \varepsilon + \lambda \left( \nabla\cdot{}u \right) I
\notag \\
\label{eq:taunewt}
&=   \mu \left( \nabla{}u + \trans{\nabla{}u} \right)
  + \lambda \left( \nabla\cdot{}u \right) I
.
\end{align}

The bulk viscosity $\mu_{B}=\lambda + \frac{2}{3}\mu$ and the deviatoric
component of the strain rate tensor
\begin{align}
  S &= \varepsilon - \frac{1}{3} \trace\left(\varepsilon\right) I
     = \frac{1}{2}\left(\nabla{}u + \trans{\nabla{}u}\right)
     - \frac{1}{3}\left(\nabla\cdot{}u\right)I
\end{align}
may be used to write $\tau$ as
\begin{align}
\label{eq:tauSmub}
  \tau &= 2 \mu S + \mu_b  \left( \nabla\cdot{}u \right) I
.
\end{align}
The kinematic viscosity and bulk kinematic viscosity
\begin{align}
 \nu &= \frac{\mu}{\rho} & \nu_{B} &= \frac{\mu_{B}}{\rho}
\end{align}
are often used to simplify notation.

\subsubsection{Stokes' hypothesis}
\label{sec:stokeshypothesis}

We allow the bulk viscosity $\mu_{B}$ to be a fixed multiple of the dynamic
viscosity $\mu$.  One may write this relationship as either
\begin{align}
\label{eq:secondviscosityclaw}
\mu_{B} &= \alpha \mu
&
&\text{or}
&
\lambda &= \left( \alpha - \frac{2}{3} \right) \mu
\end{align}
where we have introduced a dimensionless proportionality constant $\alpha$.
Stokes' hypothesis that the bulk viscosity is negligible may be recovered by
selecting $\alpha = 0$.  Though Stokes' hypothesis is valid for most
circumstances~\cite{GadelHak1995}, we choose to separately track $\mu$ and
$\lambda$ terms in the model.

\subsubsection{Power law viscosity}

We assume that viscosity varies only with temperature according to
\begin{align}
\label{eq:powerlawviscosity}
\frac{\mu}{\mu_{0}}=\left(\frac{T}{T_{0}}\right)^{\beta}
\end{align}
where $\mu_{0}$ and $T_{0}$ are suitable reference values.  This
relationship models air well for temperatures up to several thousand
degrees Kelvin~\cite{NASA-TR-R-132}.

\subsubsection{Fourier's equation}

% TODO Radiative heat transfer negligible due to low temperature
% TODO Is heat transfer by molecular diffusion neglected due to time scale?
Neglecting the transport of energy by molecular diffusion and radiative
heat transfer, we seek a relation between the surface heat flux $q_{s}$
and the temperature $T$.  The principle of frame indifference implies
we may only use the temperature gradient so that
\begin{align}
  \label{eq:fouriertensorlaw}
  q_{s} &= \underline{\kappa} \cdot \nabla{} T
\end{align}
where $\underline{\kappa}$ is a thermal conductivity tensor.
Consistent with our assumption that $\tau$ is isotropic, we assume
$\underline{\kappa}$ is isotropic to obtain
\begin{align}
  \label{eq:fourierlaw}
  q_{s} &= - \kappa \nabla{} T
\end{align}
where $\kappa$ is the scalar thermal conductivity.  We introduce the
negative sign so that heat flows from hot to cold when $\kappa>0$.

\subsubsection{Constant Prandtl number}

We assume the Prandtl number $\Prandtl = \frac{\mu{}C_{p}}{\kappa}$ is constant.
Because $C_{p}$ is constant the ratio $\frac{\mu}{\kappa}$ must be
constant.  The viscosity and thermal conductivity must either grow at
identical rates or they must grow according to an inverse relationship.
The latter is not observed in practice for our class of fluids, and
so we assume
\begin{align}
  \frac{\mu}{\mu_{0}} = \frac{\kappa}{\kappa_{0}}
  .
  \label{eq:mukappa}
\end{align}

\subsection{Dimensional equations}
\label{sec:dimensionalmodelequations}

By combining the conservation laws with our constitutive relations
and assumptions, we arrive at the dimensional equations
\begin{subequations}\label{eq:dimensionalmodel}
\begin{align}
  \label{eq:dim_continuity}
  \frac{\partial}{\partial{}t}\rho
&=
  - \nabla\cdot\rho{}u
  \\
  \label{eq:dim_momentum}
  \frac{\partial{}}{\partial{}t}\rho{}u
&=
  - \nabla\cdot(u\otimes{}\rho{}u)
  -\nabla{} p
  + \nabla\cdot{} \tau
  + f
  \\
  \label{eq:dim_energy}
  \frac{\partial}{\partial{}t} \rho{}E
&=
  - \nabla\cdot{}\rho{}Eu
  + \nabla\cdot{} \frac{\kappa_{0}}{\mu_{0}} \mu \nabla{} T
  - \nabla\cdot{} p u
  + \nabla\cdot{}\tau{} u
  + f \cdot{} u
  + q_{b}
\intertext{
  where terms in the right hand side make use of
}
  \label{eq:dim_pressure}
  p &=   \left(\gamma-1\right)\left(\rho{}E
       - \frac{1}{2}\rho{}u^{2} \right)
  \\
  \label{eq:dim_temperature}
  T &= \frac{p}{\rho{}R}
  \\
  \label{eq:dim_viscosity}
  \mu &= \mu_{0} \left( \frac{T}{T_{0}} \right)^{\beta}
  \\
  \label{eq:dim_secondviscosity}
  \lambda &= \left(\alpha- \frac{2}{3}\right) \mu
  \\
  \label{eq:dim_viscousstress}
  \tau &=   \mu \left( \nabla{}u + \trans{\nabla{}u} \right)
          + \lambda \left( \nabla\cdot{}u \right) I
  .
\end{align}
\end{subequations}

\subsection{Nondimensionalization}
\label{sec:nondim}

\subsubsection{Introduction of nondimensional variables}
\label{sec:intronondim}

We rewrite the dimensional equations using nondimensional variables
combined with arbitrary reference quantities.  For each dimensional
quantity in the dimensional model we introduce a nondimensional variable
or operator denoted by a superscript star, e.g. $\nabla^{*}$.

We introduce $t^{*}=\frac{t}{t_{0}}$ and $x^{*}=\frac{x}{l_{0}}$ for some
reference $t_{0}$ and $l_{0}$.  This induces the following relationships:
\begin{align}
  \frac{\partial{}}{\partial{}t}
  &=
  \frac{\partial{}}{\partial{}t^{*}}
  \frac{\partial{}t^{*}}{\partial{}t}
  =
  \frac{1}{t_{0}}\frac{\partial}{\partial{}t^{*}}
  &
  \frac{\partial{}}{\partial{}x}
  &=
  \frac{\partial{}}{\partial{}x^{*}}
  \frac{\partial{}x^{*}}{\partial{}x}
  =
  \frac{1}{l_{0}}\frac{\partial}{\partial{}x^{*}}
  &
  \nabla
  &=
  \hat{e}_{i} \frac{\partial{}}{\partial{}x_{i}}
  =
  \hat{e}_{i} \frac{1}{l_{0}} \frac{\partial}{\partial{}x^{*}_{i}}
  =
  \frac{1}{l_{0}} \nabla^{*}
  \label{eq:nondim_derivops}
\end{align}

We introduce more nondimensional quantities (e.g. $\rho^{*} =
\frac{\rho}{\rho_{0}}$) and use them to reexpress the model
\begin{subequations}\label{eq:dimwithref_model}
\begin{align}
  \label{eq:dimwithref_continuity}
  \frac{\rho_{0}}{t_{0}} \frac{\partial}{\partial{}t^{*}}\rho^{*}
&=
  - \frac{\rho_{0}u_{0}}{l_{0}} \nabla^{*}\cdot\rho^{*}u^{*}
  \\
  \label{eq:dimwithref_momentum}
  \frac{\rho_{0}u_{0}}{t_{0}} \frac{\partial{}}{\partial{}t^{*}}\rho^{*}u^{*}
&=
  - \frac{\rho_{0}u_{0}^{2}}{l_{0}}
    \nabla^{*}\cdot(u^{*}\otimes{}\rho^{*}u^{*})
  - \frac{p_{0}}{l_{0}} \nabla^{*} p^{*}
  + \frac{\tau_{0}}{l_{0}} \nabla^{*}\cdot{} \tau^{*}
  + f_{0} f^{*}
  \\
  \label{eq:dimwithref_energy}
  \frac{\rho_{0}E_{0}}{t_{0}}
  \frac{\partial}{\partial{}t^{*}} \rho^{*}E^{*}
&=
  - \frac{\rho_{0}E_{0}u_{0}}{l_{0}} \nabla^{*} \cdot{}\rho^{*}E^{*}u^{*}
  + \frac{\kappa_{0}T_{0}}{l_{0}^{2}}
    \nabla^{*}\cdot{} \mu^{*} \nabla^{*} T^{*}
  - \frac{p_{0}u_{0}}{l_{0}} \nabla^{*}\cdot{} p^{*} u^{*}
\notag\\
&\quad{}+ \frac{\tau_{0}u_{0}}{l_{0}} \nabla^{*}\cdot{}\tau^{*} u^{*}
  + f_{0}u_{0} f^{*} \cdot{} u^{*}
  + q_{0} q_{b}^{*}
\intertext{
  where terms in the right hand side are computed using
}
  \label{eq:dimwithref_pressure}
  p^{*} &= \frac{\gamma-1}{p_{0}} \left(
        \rho_{0}E_{0}\rho^{*}E^{*}
      - \rho_{0}u_{0}^{2}\,\rho^{*}\frac{u^{*}\cdot{}u^{*}}{2}
  \right)
  \\
  \label{eq:dimwithref_temperature}
  T^{*} &= \frac{p_{0}p^{*}}{\rho_{0}RT_{0}\,\rho^{*}}
  \\
  \label{eq:dimwithref_viscosity}
  \mu^{*} &= \left( T^{*} \right)^{\beta}
  \\
  \label{eq:dimwithref_secondviscosity}
  \lambda^{*} &= \left(\alpha - \frac{2}{3}\right) \mu^{*}
  \\
  \label{eq:dimwithref_viscousstress}
  \tau^{*} &= \frac{\mu_{0}u_{0}}{l_{0} \tau_{0}} \left[
      \mu^{*} \left( \nabla^{*}u^{*} + \trans{\nabla^{*}u^{*}} \right)
      + \lambda^{*} \left( \nabla^{*}\cdot{}u^{*} \right) I
    \right]
  .
\end{align}
\end{subequations}
Notice that $\lambda$ has been nondimensionalized using $\mu_{0}$.  At this
stage, we have many more reference quantities than the underlying quantity
dimensions warrant.

\subsubsection{Reference quantity selections}
\label{sec:nondimrefq}

We choose a reference density $\rho_{0}$, length $l_{0}$, velocity $u_{0}$, and
temperature $T_{0}$.  These selections fix all other dimensional reference
quantities:
\begin{align}
  a_{0} &= \sqrt{\gamma{}RT_{0}}
  &
  E_{0}, H_{0}, h_{0} &= a_{0}^{2}
  &
  p_{0} &= \rho_{0} a_{0}^{2}
  &
  t_{0} &= \frac{l_{0}}{u_{0}}
  &
  \tau_{0} &= \frac{\mu_{0}u_{0}}{l_{0}}
  &
  f_{0} &= \frac{\rho_{0}u_{0}}{t_{0}}
  &
  q_{0} &= \frac{\rho_{0}a_{0}^{2}}{t_{0}}
\end{align}
Because we assume viscosity varies only with temperature, $\mu_{0}=\mu\!\left(
T_{0} \right)$ is fixed by $T_{0}$.  Because we assume a constant Prandtl
number, $\kappa_{0}=\kappa\!\left( \mu\!\left( T_{0} \right) \right)$ is also
fixed by $T_{0}$.  Choosing this form for $p_{0}$ in lieu of employing
equation~\eqref{eq:perfectgaseos} is customary and removes many instances of
$\gamma$ from the resulting equations.  Note that these reference choices imply
\begin{align}
a^{*}&=\sqrt{T^{*}}
&
&\text{and}
&
h^{*}&=\frac{T^{*}}{\gamma-1}
.
\end{align}

\subsubsection{Nondimensional equations}
\label{nondim_equations}

We employ the reference quantity relationships after multiplying the
continuity, momentum, and energy equations by $\frac{t_{0}}{\rho_{0}}$,
$\frac{t_{0}}{\rho_{0}u_{0}}$, and $\frac{t_{0}}{\rho_{0}E_{0}}$
respectively.  Henceforth we suppress the superscript star notation because all
terms are dimensionless.  We arrive at the following nondimensional equations
\begin{subequations}
\begin{align}
  \label{eq:nondim_continuity}
  \frac{\partial}{\partial{}t}\rho{}
&=
  - \nabla\cdot\rho{}u
  \\
  \label{eq:nondim_momentum}
  \frac{\partial}{\partial{}t}\rho{}u
&=
  - \nabla\cdot(u\otimes\rho{}u)
  - \frac{1}{\Mach^{2}} \nabla{} p
  + \frac{1}{\Reynolds} \nabla\cdot\tau
  + f
  \\
  \label{eq:nondim_energy}
  \frac{\partial}{\partial{}t} \rho{}E
&=
  - \nabla\cdot\rho{}Eu
  + \frac{1}{\Reynolds\,\Prandtl\,\left( \gamma - 1 \right)}
    \nabla\cdot\mu\nabla{} T
  - \nabla\cdot{} p u
  + \frac{\Mach^{2}}{\Reynolds} \nabla\cdot\tau{} u
  + \Mach^{2} f \cdot{} u
  + q_{b}
\intertext{
along with the relationships
}
  \label{eq:nondim_pressure}
  p &= \left(\gamma-1\right) \left(
    \rho{}E - \frac{\Mach^{2}}{2}\rho{}u^{2}
  \right)
  \\
  \label{eq:nondim_temperature}
  T &= \gamma{} \frac{p}{\rho}
  \\
  \label{eq:nondim_viscosity}
  \mu &= T^{\beta}
  \\
  \label{eq:nondim_secondviscosity}
  \lambda &= \left(\alpha-\frac{2}{3}\right)\mu
  \\
  \label{eq:nondim_viscousstress}
  \tau &=  \mu\left(\nabla{}u+\trans{\nabla{}u}\right)
         + \lambda\left(\nabla\cdot{}u\right) I
\end{align}
\end{subequations}
where the nondimensional quantities
\begin{align}
  \Reynolds &= \frac{\rho_{0}u_{0}l_{0}}{\mu_{0}}
  &
  \Mach &= \frac{u_{0}}{a_{0}}
  &
  \Prandtl &= \frac{\mu_{0}C_{p}}{\kappa_{0}}
\end{align}
are the Reynolds, Mach, and Prandtl numbers, respectively.

Introducing a velocity reference $u_{0}$ separate from the sound speed $a_{0}$
was not necessary from a dimensional standpoint.  Doing so is the origin of the
$\Mach$ factors appearing above.  These add some minor complexity to the
equations but greatly simplify investigating the physics in the various
$\Mach\to{}0$ limits.

\section{Discretization}
\label{sec:discretization}

Here we discuss the space and time discretization techniques employed to solve
the continuous model equations.  In this section $u$ denotes an arbitrary state
vector and not fluid velocity.

\subsection{Spatial discretization}
\label{sec:spatialdiscretization}

We start with the continuous system
\begin{align}
  \frac{\partial}{\partial{}t} u &= \mathscr{L}u + \mathscr{N}\!\left(u\right)
\end{align}
on the spatial domain $\left[-\frac{L_x}{2},\frac{L_x}{2}\right] \times{}
[0,L_y] \times{} \left[-\frac{L_z}{2},\frac{L_z}{2}\right]$.  The operators
$\mathscr{L}$ and $\mathscr{N}$ are linear and nonlinear, respectively.  For
brevity, the section suppresses any time dependence in the operators.  To begin
spatially discretizing the system, we introduce its finite dimensional analog
\begin{align}
  \frac{\partial}{\partial{}t} u^h
  &=
  \mathscr{L}u^h + \mathscr{N}\!\left(u^h\right) + R^h
  \label{eq:discrete_system_with_residual}
\end{align}
where continuous $u = u\!\left(x,y,z,t\right)$ has been replaced by discrete
$u^h = u^h\!\left(x,y,z,t\right)$ with $N_x\times{}N_y\times{}N_z$ degrees of
freedom.  Here, $R^h$ is the discretization error that arises because the
discrete solution cannot satisfy the continuous equations everywhere in space.
We select Fourier expansions for the periodic $x$ and $z$ directions and a
B-spline expansion for the non-periodic $y$ direction.  That is,
\begin{align}
u^h(x,y,z,t)
&=
  \sum_{l=0}^{N_y - 1}
  \sum_{m=-\frac{N_x}{2}}^{\frac{N_x}{2}-1}
  \sum_{n=-\frac{N_z}{2}}^{\frac{N_z}{2}-1}
  \hat{u}_{l m n}(t)
  B_l\!\left(y\right)
  e^{\ii\frac{2\pi{}m}{L_x}x}
  e^{\ii\frac{2\pi{}n}{L_z}z}
  \\
&=
  \sum_{l}\sum_{m}\sum_{n}
  \hat{u}_{l m n}(t)B_l\!\left(y\right)e^{\ii k_m x}e^{\ii k_n z}
  \label{eq:u_h_expansion}
\end{align}
where $k_m = 2\pi{}m/L_x$, $k_n = 2\pi{}n/L_z$, and $B_l\!\left(y\right)$ are a
B-spline basis for some order and knot selection.

Now, within the method of weighted residuals framework, we choose a mixed
Galerkin/collocation approach (often called a ``pseudospectral'' technique)
employing the $L_{2}$ inner product and test ``functions'' like
$\delta(y-y_{l'}) e^{\ii k_{m'} x}e^{\ii k_{n'} z}$ where $l'$, $m'$, and $n'$
range over the same values as $l$, $m$, and $n$, respectively.  The fixed
collocation points $y_{l'}$ depend on the B-spline basis details.  We catalog
three relevant results
\begin{align}
   \int_0^{L_y} \varphi(y) \, \delta(y-y_{l'}) \,d\!y
&= \varphi(y_{l'}),
&
   \int_{-\frac{L_x}{2}}^{\frac{L_x}{2}} e^{\ii k_m x} e^{-\ii k_{m'} x} \,d\!x
&= L_x \delta_{m m'}, \text{ and}
&
   \int_{-\frac{L_z}{2}}^{\frac{L_z}{2}} e^{\ii k_n z} e^{-\ii k_{n'} z} \,d\!z
&= L_z \delta_{n n'}
\end{align}
where the inner product's conjugate operation is accounted for by introducing a
negative sign into the latter two exponentials.  We force the weighted residual
to be zero in the sense that
\begin{align}
  \int_0^{L_y}
  \int_{-\frac{L_x}{2}}^{\frac{L_x}{2}}
  \int_{-\frac{L_z}{2}}^{\frac{L_z}{2}}
  R^h\!\left(x,y,z\right) \delta(y-y_{l'}) e^{-\ii k_{m'} x}e^{-\ii k_{n'} z}
  \,d\!z \,d\!x \,d\!y
  &=
  0
  \label{eq:R_h_weighted_residual_zero}
\end{align}
holds for all $l'$, $m'$, and $n'$.  Inserting \eqref{eq:u_h_expansion} into
\eqref{eq:discrete_system_with_residual}, testing with our test functions,
applying \eqref{eq:R_h_weighted_residual_zero}, and simplifying each remaining
term separately we obtain the following:
\begin{align}
 \int_0^{L_y}
 \int_{-\frac{L_x}{2}}^{\frac{L_x}{2}}
 \int_{-\frac{L_z}{2}}^{\frac{L_z}{2}}
 \,
 &\frac{\partial}{\partial{}t}
  \left(
    \sum_{l}\sum_{m}\sum_{n}
    \hat{u}_{l m n}(t)B_l\!\left(y\right)e^{\ii k_m x}e^{\ii k_n z}
  \right)
  \left(
    \delta(y-y_{l'}) e^{-\ii k_{m'} x}e^{-\ii k_{n'} z}
  \right)
  \, dz \, dx \, dy
\\
  &=
  L_x L_z \sum_{l} B_l\!\left(y_{l'}\right)
  \frac{\partial}{\partial{}t} \hat{u}_{l m n}(t)
\\
 \int_0^{L_y}
 \int_{-\frac{L_x}{2}}^{\frac{L_x}{2}}
 \int_{-\frac{L_z}{2}}^{\frac{L_z}{2}}
 \,
 &\mathscr{L}
  \left(
    \sum_{l}\sum_{m}\sum_{n}
    \hat{u}_{l m n}(t)B_l\!\left(y\right)e^{\ii k_m x}e^{\ii k_n z}
  \right)
  \left(
    \delta(y-y_{l'}) e^{-\ii k_{m'} x}e^{-\ii k_{n'} z}
  \right)
  \, dz \, dx \, dy
\\
  &=
  L_x L_z
  \mathscr{L}\left(
     \sum_{l}
      B_l\!\left(y_{l'}\right)
     \hat{u}_{l m n}(t)
   \right)
\intertext{} % Allow soft break
  \int_0^{L_y}
  \int_{-\frac{L_x}{2}}^{\frac{L_x}{2}}
  \int_{-\frac{L_z}{2}}^{\frac{L_z}{2}}
  &\mathscr{N}\left(
     \sum_{l}\sum_{m}\sum_{n}
     \hat{u}_{l m n}(t)B_l\!\left(y\right)e^{\ii k_m x}e^{\ii k_n z}
   \right)
   \left(
     \delta(y-y_{l'}) e^{-\ii k_{m'} x}e^{-\ii k_{n'} z}
   \right)
   \, dz \, dx \, dy
\\
  &=
  \int_{-\frac{L_x}{2}}^{\frac{L_x}{2}}
  \int_{-\frac{L_z}{2}}^{\frac{L_z}{2}}
  \mathscr{N}\left(
    \sum_{m}\sum_{n}
    \left(
      \sum_{l} B_l\!\left(y_{l'}\right)
      \hat{u}_{l m n}(t)
    \right)
    e^{\ii k_m x}e^{\ii k_n z}
  \right)
  \left(
    e^{-\ii k_{m'} x}e^{-\ii k_{n'} z}
  \right)
  \, dz \, dx
\intertext{
  Reequating the terms we have
}
  L_x L_z
  \sum_{l} B_l\!\left(y_{l'}\right)
  \frac{\partial}{\partial{}t} \hat{u}_{l m n}(t)
  &=
  L_x L_z
  \mathscr{L}\left(
    \sum_{l}
     B_l\!\left(y_{l'}\right)
    \hat{u}_{l m n}(t)
  \right)
\\
  &{}+
  \int_{-\frac{L_x}{2}}^{\frac{L_x}{2}}
  \int_{-\frac{L_z}{2}}^{\frac{L_z}{2}}
  \mathscr{N}\left(
    \sum_{m}\sum_{n}
    \left(
      \sum_{l} B_l\!\left(y_{l'}\right)
      \hat{u}_{l m n}(t)
    \right)
    e^{\ii k_m x}e^{\ii k_n z}
  \right)
  \left(
    e^{-\ii k_{m'} x}e^{-\ii k_{n'} z}
  \right)
  \, dz \, dx
  .
 \end{align}

Finally, approximating the two integrals by discrete sums and dividing
by $L_x$ and $L_z$ leaves us with
\begin{align}
  \sum_{l} B_l\!\left(y_{l'}\right)
  \frac{\partial}{\partial{}t} \hat{u}_{l m n}(t)
  &\approx
  \mathscr{L}\left(
    \sum_{l}
     B_l\!\left(y_{l'}\right)
    \hat{u}_{l m n}(t)
  \right)
\\
  &{}+
  \frac{1}{N_x N_z}
  \sum_{m'} \sum_{n'}
  \mathscr{N}\left(
    \sum_{m}
    \sum_{n}
    \left(
      \sum_{l} B_l\!\left(y_{l'}\right)
      \hat{u}_{l m n}(t)
    \right)
    e^{\ii k_m x_{m'}}e^{\ii k_n z_{n'}}
  \right)
  \!\!
  \left(
    e^{-\ii k_{m'} x_m}e^{-\ii k_{n'} z_n}
  \right)
  \label{eq:spatial_discretization}
\end{align}
where $x_{m'}=L_x m' / N_x$ and $z_{n'}=L_z n' / N_z$.  This approximation is
nothing but a quadrature error (see, e.g., theorem~19 in~\cite{Boyd2001}).
With knowledge of the nonlinear operator $\mathscr{N}$, such quadrature error
can be mitigated via an appropriate dealiasing technique (see,
e.g.,~\cite{Canuto2006}).  Interestingly, our weighted residual
\eqref{eq:R_h_weighted_residual_zero} need now only be zero in this weaker,
discrete sense.  We are left with $N_x\times{}N_z$ time-dependent systems
containing $N_y$ equations coupled in the $x$ and $z$ directions only through
discrete Fourier transforms and the requirements of the $\mathscr{L}$ and
$\mathscr{N}$ operators.

Calling $\hat{u}_{l m n}(t)$ the wave space coefficients of the solution at
some time $t$, the nonlinear portion of \eqref{eq:spatial_discretization} may
be efficiently computed by
\begin{enumerate}
 \item performing $\mathcal{O}\!\left(m\times{}n\right)$ matrix-vector products
       like $\sum_{l} B_l\!\left(y_{l'}\right) \hat{u}_{l m n}(t)$,
 \item using an inverse fast Fourier transform across
       the $x$ and $z$ directions to convert state information from wave space
       to physical space,
 \item applying the nonlinear operator $\mathscr{N}$ in physical space, and
 \item using a forward fast Fourier transform across the $x$ and $z$
       directions to convert information from physical space to wave space.
\end{enumerate}
Note that the left hand side of \eqref{eq:spatial_discretization} contains a
time-independent mass matrix arising from the B-spline basis and collocation
point choices.  We retain the mass matrix on the same side as the time
derivative in anticipation of the time discretization scheme.  The
computational cost of the scaling factor $\left(N_x N_z\right)^{-1}$
can be hidden within the time advancement scheme.

Rather than time advancing the B-spline coefficients $\hat{u}_{l m n}(t)$ as
state, one could instead instead advance the wavenumber-dependent collocation
point values $\hat{u}_{y_{l'} m n}(t) = \sum_{l} B_l\!\left(y_{l'}\right)
\hat{u}_{l m n}(t)$.  Then \eqref{eq:spatial_discretization} would become
\begin{align}
  \frac{\partial}{\partial{}t} \hat{u}_{y_{l'} m n}(t)
  &\approx
  \mathscr{L}\left(\hat{u}_{y_{l'} m n}(t)\right)
  +
  \frac{1}{N_x N_z}
  \sum_{m'} \sum_{n'}
  \mathscr{N}\left(
    \sum_{m}
    \sum_{n}
    \hat{u}_{y_{l'} m n}(t)
    e^{\ii k_m x_{m'}}e^{\ii k_n z_{n'}}
  \right)
  \left(
    e^{-\ii k_{m'} x_m}e^{-\ii k_{n'} z_n}
  \right)
  .
  \label{eq:spatial_discretization_gridpoints}
\end{align}
Equation \eqref{eq:spatial_discretization_gridpoints} is obviously simpler than
equation \eqref{eq:spatial_discretization}.  In particular, it lacks a mass
matrix in the time derivative term.

\subsection{Time discretization}
\label{sec:timediscretization}

We use Spalart, Moser, and Rogers' low storage hybrid implicit/explicit
time stepper from appendix A of~\cite{spalart_lowstoragerk}.  This time stepper
advances the system
\begin{align}
\label{eq:timediscretization}
 u_t &= Lu + N(u,t)
\end{align}
from $u(t)$ to $u\left( t+\Delta{}t \right)$.  Here $L$ and $N$ are a linear
and nonlinear operator, respectively, distinct from but related to the
preceding section's $\mathscr{L}$ and $\mathscr{N}$ (see
\textsection\ref{sec:combineddiscretization} for the exact relationships).  The
operator $L$ must be time-independent throughout each interval $\left[t,
t+\Delta{}t\right)$.  In contrast to the discussion
in~\cite{spalart_lowstoragerk} and following S. Yang (personal communication),
we allow $N$ to vary in time.  The ``SMR91'' scheme advances according to the
following equations:
\begin{subequations}
\begin{align}
  u'
  &=
  u_{n}
  + \Delta{}t\left[
      L\left( \alpha_{1}u_{n} + \beta_{1}u' \right)
    + \gamma_{1} N\left(u_{n},t_{n}\right)
  \right]
  \tag{SMR91 A4a}
  \label{eq:SMR91_A4a}
\\
  u''
  &=
  u'
  + \Delta{}t\left[
    L\left( \alpha_{2}u' + \beta_{2}u'' \right)
    + \gamma_{2} N\left(u',t_{n}+\eta_{2}\Delta{}t\right)
    + \zeta_{1}  N\left(u_{n},t_{n}\right)
  \right]
  \tag{SMR91 A4b}
  \label{eq:SMR91_A4b}
\\
  u_{n+1}
  &=
  u''
  + \Delta{}t\left[
      L\left( \alpha_{3}u'' + \beta_{3}u_{n+1} \right)
    + \gamma_{3} N\left(u'',t_{n}+\eta_{3}\Delta{}t\right)
    + \zeta_{2}  N\left(u',t_{n}+\eta_{2}\Delta{}t\right)
  \right]
  \tag{SMR91 A4c}
  \label{eq:SMR91_A4c}
\end{align}
\begin{align}
  \alpha_1 + \beta_1 &= \gamma_1 = \eta_2
  &
  \alpha_2 + \beta_2 &= \gamma_2 + \zeta_1
  &
  \alpha_3 + \beta_3 &= \gamma_3 + \zeta_2
  &
  \eta_{3} &= \eta_2 + \alpha_2 + \beta_2
  \tag{SMR91 A5}
\end{align}
\end{subequations}
The authors determined suitable values for the above coefficients to be
\begin{align*}
  \alpha_1, \alpha_2, \alpha_3 &= \left\{
    \frac{29}{96}, -\frac{3}{40},  \frac{1}{6}
  \right\}
  &
  \beta_1, \beta_2, \beta_3 &= \left\{
    \frac{37}{160}, \frac{5}{24}, \frac{1}{6}
  \right\}
  &
  \gamma_1, \gamma_2, \gamma_3 &= \left\{
    \frac{8}{15}, \frac{5}{12}, \frac{3}{4}
  \right\}
\end{align*}
\begin{align*}
  \zeta_0, \zeta_1, \zeta_2 &= \left\{
    0, -\frac{17}{60}, -\frac{5}{12}
  \right\}
  &
  \eta_0, \eta_1, \eta_2, \eta_3 &= \Biggl\{
    0, 0, \frac{8}{15}, \frac{2}{3}
  \Biggr\}
\end{align*}
We have added $\zeta_0$ and $\eta_0$ for notational convenience.  The
$\eta_{1}$--$\eta_{3}$ coefficients used to extend the SMR91 scheme for
time-dependent $N$ are discussed by Yang~\cite{ShanYang2011}.  Each substep
\eqref{eq:SMR91_A4a}--\eqref{eq:SMR91_A4c} has the general form
\begin{align}
  u^{i+1} &= u^i + \Delta{}t \left[
        \alpha_{i} L u^i
      + \beta_{i}  L u^{i+1}
      + \gamma_{i} N\left( u^{i}, t_{n}+\eta_{i}\Delta{}t \right)
      + \zeta_{i-1} N\left( u^{i-1}, t_{n}+\eta_{i-1}\Delta{}t \right)
  \right]
  \label{eq:generalsubstep}
\end{align}
where $i\in\left\{ 1,2,3 \right\}$ is the substep number.
We rewrite the general substep equation as
\begin{align}
  \left(I - \Delta{}t\beta_{i}L\right) u^{i+1}
  &=
  \left(I + \Delta{}t\alpha_{i}L\right) u^{i}
  + \Delta{}t\gamma_{i}N\left(u^{i}, t_{n}+\eta_{i}\Delta{}t\right)
  + \Delta{}t\zeta_{i-1}N\left(u^{i-1}, t_{n}+\eta_{i-1}\Delta{}t\right)
  \label{eq:generaloperatorsubstep}
  .
\end{align}
The scheme treats $N$ with third-order accuracy and $L$ with second-order
accuracy.  See both the top of page 323 in~\cite{spalart_lowstoragerk} and
chapter 7 in~\cite{ShanYang2011} for more accuracy-related details.

We wish to advance problems like $Mu_{t}=\tilde{L}u+\chi\tilde{N}\left( u,t
\right)$ where $\chi$ is a time-independent scalar and both $\tilde{L}$ and
$\tilde{N}$ take state values to some non-state representation which can be
converted back to state by the linear ``mass matrix'' operator $M$.  Then by
$L=M^{-1}\tilde{L}$ and $N=\chi{}M^{-1}\tilde{N}$ the general substep equation
gives
\begin{align}
  \left(I - \Delta{}t\beta_{i}M^{-1}\tilde{L}\right) u^{i+1}
  &=
  \left(I + \Delta{}t\alpha_{i}M^{-1}\tilde{L}\right) u^{i}
\\
  &+ \Delta{}t\gamma_{i}\chi{}M^{-1}
    \tilde{N}\left(u^{i}, t_{n}+\eta_{i}\Delta{}t\right)
  + \Delta{}t\zeta_{i-1}\chi{}M^{-1}
    \tilde{N}\left(u^{i-1}, t_{n}+\eta_{i-1}\Delta{}t\right)
  .
\end{align}
We multiply through by $M$ to obtain
\begin{align}
  \left(M - \Delta{}t\beta_{i}\tilde{L}\right) u^{i+1}
  &=
  \left(M + \Delta{}t\alpha_{i}\tilde{L}\right) u^{i}
  + \Delta{}t\gamma_{i}\chi{}
    \tilde{N}\left(u^{i}, t_{n}+\eta_{i}\Delta{}t\right)
  + \Delta{}t\zeta_{i-1}\chi{}
    \tilde{N}\left(u^{i-1}, t_{n}+\eta_{i-1}\Delta{}t\right)
  \label{eq:generaloperatormasssubstep}
\end{align}
which is just \eqref{eq:generaloperatorsubstep} with $M$ replacing $I$,
$\tilde{L}$ replacing $L$, and $\chi{}\tilde{N}$ replacing $N$.  We now drop
the tildes on $L$ and $N$ with the understanding that they are implied whenever
$M\neq{}I$.

The time advancement scheme in \eqref{eq:generaloperatormasssubstep} requires
implementations of $u\mapsto{}{N}\left(u\right)$,
$u\mapsto{}\left(M+\varphi{}L\right)u$, and
$u\mapsto{}\left(M+\varphi{}L\right)^{-1}u$ for a given $M$ and some arbitrary
scalar $\varphi$.  To achieve a low storage implementation, the
$N\left(u\right)$ and $\left(M+\varphi{}L\right)^{-1}$ implementations must
operate in-place while $\left(M+\varphi{}L\right)$ must operate out-of-place.
Given only two storage locations $a$ and $b$, each substep computation follows
algorithm~\vref{alg:substep}.

\begin{algorithm}
\label{alg:substep}
\caption{Compute one substep in the SMR91 scheme following
         equation~(\ref{eq:generaloperatormasssubstep})
         }
\begin{algorithmic}
  \REQUIRE Storage $a = u^i$;
           storage $b = N\left(u^{i-1},t_{n}+\eta_{i-1}\Delta{}t\right)$
  \STATE $b\leftarrow{}   \left(M+\Delta{}t\alpha_{i}L\right)a
                        + \Delta{}t\zeta_{i-1}\chi{}b$
  \STATE $a\leftarrow{}N\left(a,t_{n}+\eta_{i}\Delta{}t\right)$
  \STATE $b\leftarrow{}\Delta{}t\gamma_{i}\chi{}a + b$
  \STATE $b\leftarrow{}\left(M-\Delta{}t\beta_{i}L\right)^{-1}b$
  \ENSURE Storage $a = N\left(u^{i},t_{n}+\eta_{i}\Delta{}t\right)$;
          storage $b = u^{i+1}$
\end{algorithmic}
\end{algorithm}

Two possible issues arise when repeatedly using the substep algorithm: First,
during the first substep the step size $\Delta{}t$ may need to be computed
dynamically based on a stability criterion available only during the nonlinear
operator computation.  Second, it may be important to always apply each
operator against a particular storage location.  This requirement implies that
the implementation of $\left(M+\varphi{}L\right)$ must also support in-place
application.  For the second and subsequent substeps, either
$\left(M+\varphi{}L\right)$ must support a decidedly non-BLAS-like, complicated
out-of-place-apply-and-swap operation or the underlying state storage must
support a swap operation.  We choose the latter and denote the swap operation
as $a\leftrightarrow{}b$.  Under these considerations, time step computation
follows algorithm~\vref{alg:step}.

\begin{algorithm}
\caption{Compute all substeps in the SMR91 scheme following
         equation~(\ref{eq:generaloperatormasssubstep})
         }
\label{alg:step}
\begin{algorithmic}
  \renewcommand{\algorithmiccomment}[1]{\hfill{}// #1}
  \REQUIRE Storage $a = u\left(t_{n}\right) = u^{0} $;
           storage $b$ content undefined
  \STATE $b\leftarrow{}a$
  \STATE $b\leftarrow{}N\left(b,t_{n}\right)$
  \STATE Compute $\Delta{}t$ from $a=u^0$ and $b=N\left(u^0,t_{n}\right)$
  \STATE $a\leftarrow{}\left(M+\Delta{}t\alpha_{1}L\right)a$
  \STATE $a\leftarrow{}\Delta{}t \gamma_{1} \chi{} b + a$
  \STATE $a\leftarrow{}\left(M-\Delta{}t\beta_{1}L\right)^{-1}a$
  \ENSURE Storage $a = u^1$;
          storage $b = N\left(u^{0},t_{n}\right)$
  \STATE $b\leftarrow{}   \left(M+\Delta{}t\alpha_{2}L\right)a
                        + \Delta{}t\zeta_{1}\chi{}b$
  \STATE $a\leftrightarrow{}b$
  \STATE $b\leftarrow{}N\left(b,t_{n}+\eta_{2}\Delta{}t\right)$
  \STATE $a\leftarrow{}\Delta{}t \gamma_{2} \chi{} b + a$
  \STATE $a\leftarrow{}\left(M-\Delta{}t\beta_{2}L\right)^{-1}a$
  \ENSURE Storage $a = u^{2}$;
          storage $b = N\left(u^{1},t_{n}+\eta_{2}\Delta{}t\right)$
  \STATE $b\leftarrow{}   \left(M+\Delta{}t\alpha_{3}L\right)a
                        + \Delta{}t\zeta_{2}\chi{}b$
  \STATE $a\leftrightarrow{}b$
  \STATE $b\leftarrow{}N\left(b,t_{n}+\eta_{3}\Delta{}t\right)$
  \STATE $a\leftarrow{}\Delta{}t \gamma_{3} \chi{}b + a$
  \STATE $a\leftarrow{}\left(M-\Delta{}t\beta_{3}L\right)^{-1}a$
  \ENSURE Storage $a = u\left(t+\Delta{}t\right)= u^{3}$;
          storage $b = N\left(u^{2},t_{n}+\eta_{3}\Delta{}t\right)$
\end{algorithmic}
\end{algorithm}

\subsection{Combined space and time discretization}
\label{sec:combineddiscretization}

Using \eqref{eq:generaloperatormasssubstep} to advance state
$\hat{u}_{l m n}(t)$ following \eqref{eq:spatial_discretization} gives
the time-independent discrete operators
\begin{subequations}
\begin{align}
   M u\bigr|_{m n}
&= \sum_{l} B_l\!\left(y_{l'}\right)
   \hat{u}_{l m n}
\\
   \left.\tilde{L} u\right|_{m n}
&= \mathscr{L}\left(
     \sum_{l}
     B_l\!\left(y_{l'}\right)
     \hat{u}_{l m n}
   \right)
\\
   \left.\tilde{N}\!\left(u\right)\right|_{m n}
&= \underbrace{\frac{1}{N_x N_z}}_{\chi}
   \sum_{m'} \sum_{n'}
   \mathscr{N}\left(
     \sum_{m}
     \sum_{n}
     \left(
       \sum_{l} B_l\!\left(y_{l'}\right)
       \hat{u}_{l m n}
     \right)
     e^{\ii k_m x_{m'}}e^{\ii k_n z_{n'}}
   \right)
   \left(
     e^{-\ii k_{m'} x_m}e^{-\ii k_{n'} z_n}
   \right)
\end{align}
\end{subequations}
while advancing state $\hat{u}_{y_{l'} m n}(t)$ following
\eqref{eq:spatial_discretization_gridpoints} instead fixes the operators as
\begin{subequations}
\begin{align}
   M u\bigr|_{m n}
&= \hat{u}_{y_{l'} m n}
\\
   \left.\tilde{L} u\right|_{m n}
&= \mathscr{L}\left(\hat{u}_{y_{l'} m n}\right)
\\
   \left.\tilde{N}\!\left(u\right)\right|_{m n}
&= \underbrace{\frac{1}{N_x N_z}}_{\chi}
   \sum_{m'} \sum_{n'}
   \mathscr{N}\left(
     \sum_{m}
     \sum_{n}
     \hat{u}_{y_{l'} m n}
     e^{\ii k_m x_{m'}}e^{\ii k_n z_{n'}}
   \right)
   \left(
     e^{-\ii k_{m'} x_m}e^{-\ii k_{n'} z_n}
   \right)
   .
\end{align}
\end{subequations}
When following \eqref{eq:spatial_discretization} the operators $\tilde{L}$ and
$\tilde{N}$ take B-spline coefficients as input and return collocation point
values.  When following \eqref{eq:spatial_discretization_gridpoints} the
operators both take and return collocation point values.  We choose to work
with \eqref{eq:spatial_discretization} because it provides computational
advantages.  For a discussion on moving operators between these two
representations, see \textsection{}5.5 in~\cite{Boyd2001}.

\subsection{Forming discrete operators}
\label{sec:formingoperators}

Consistent with employing \eqref{eq:spatial_discretization}, discrete operators
for differentiation in the wall-normal direction map B-spline coefficients to
derivatives at wall-normal collocation points.  That is,
\begin{align}
  D^{(k)} u\bigr|_{m n}
&= \sum_{l} B^{(k)}_l\!\left(y_{l'}\right)
   \hat{u}_{l m n}
\end{align}
where the banded matrix $D^{(k)}$ is wavenumber independent.  $D^{(0)}$ is
nothing but the ``mass matrix'' $M$.  Boundary conditions may be enforced for
B-spline derivatives by using that the $k$th derivative of the function at the
first (last) collocation point depends only on the first (last) $k+1$ B-spline
coefficients.

Computing the streamwise (or spanwise) first derivative of a quantity only
requires multiplying each Fourier expansion coefficient $\hat{u}_{l m n}$ by
$\ii k_{m}$ (or~$\ii k_{n}$).  Second derivatives are found by multiplying each
$\hat{u}_{l m n}$ by $-k_{m}^2$ (or~$-k_{n}^2$).

Combining the B-spline and Fourier details for a scalar-valued $\phi$ one finds
\begin{align}
  \left.\left(\nabla{} \phi\right)\right|_{m n}
&=
  \begin{pmatrix}
    \ii k_m M \\
    D^{(1)}   \\
    \ii k_n M
  \end{pmatrix} \phi\bigr|_{m n}
\\
   \left.\left(\Delta{} \phi\right)\right|_{m n}
&= \left( -\left(k_m^2 + k_n^2\right)M + D^{(2)} \right) \phi\bigr|_{m n}
\\
  \left.\left(\nabla\nabla{} \phi\right)\right|_{m n}
&=
  \begin{pmatrix}
    -k_m^2      M       & \ii k_m D^{(1)} & -k_{m}k_{n} M   \\
    \ii k_m     D^{(1)} &         D^{(2)} & \ii k_n D^{(1)} \\
    -k_{m}k_{n} M       & \ii k_n D^{(1)} & -k_n^2 M
  \end{pmatrix} \phi\bigr|_{m n}
\end{align}
where all operators map coefficients in all three directions to wall-normal
collocation point values but streamwise and spanwise coefficients.  For a
vector field $\vec{\phi}=\trans{\begin{pmatrix}\phi_{x} & \phi_{y} &
 \phi_{z}\end{pmatrix}}$ one obtains the following:
\begin{align}
  \left.\nabla\cdot\vec{\phi}\right|_{m n}
&=
  \left.\left(\ii k_m M \phi_{x} + D^{(1)} \phi_{y} + \ii k_n M \phi_{z}\right)\right|_{m n}
\\
  \left.\nabla\vec{\phi}\right|_{m n}
&=
  \left.\begin{pmatrix}
    \ii k_m M \phi_{x} & D^{(1)} \phi_{x} & \ii k_n M \phi_{x} \\
    \ii k_m M \phi_{y} & D^{(1)} \phi_{y} & \ii k_n M \phi_{y} \\
    \ii k_m M \phi_{z} & D^{(1)} \phi_{z} & \ii k_n M \phi_{z}
  \end{pmatrix}\right|_{m n}
\\
  \left.\Delta\vec{\phi}\right|_{m n}
&=
  \left.\begin{pmatrix}
    \left( -\left(k_m^2 + k_n^2\right)M + D^{(2)} \right) \phi_{x} \\
    \left( -\left(k_m^2 + k_n^2\right)M + D^{(2)} \right) \phi_{y} \\
    \left( -\left(k_m^2 + k_n^2\right)M + D^{(2)} \right) \phi_{z}
  \end{pmatrix}\right|_{m n}
\\
  \left.\nabla\nabla\cdot\vec{\phi}\right|_{m n}
&=
  \left.\begin{pmatrix}
    - k_m^2 M \phi_{x}       + \ii k_m D^{(1)} \phi_{y} - k_{m}k_{n} M \phi_{z} \\
    \ii k_m D^{(1)} \phi_{x} + D^{(2)} \phi_{y} + \ii k_n D^{(1)} \phi_{z}      \\
    - k_{m}k_{n} M \phi_{x} + \ii k_n D^{(1)} \phi_{y} - k_n^2 M \phi_{z}
  \end{pmatrix}\right|_{m n}
\end{align}
The operator $\nabla\times\nabla\times\vec{\phi} = \nabla\nabla\cdot\vec{\phi}
- \Delta\vec{\phi}$, which may be expanded as
\begin{align}
  \left.\nabla\times\nabla\times\vec{\phi}\right|_{m n}
&=
  \left.\begin{pmatrix}
    \left(k_n^2 M - D^{(2)}\right) \phi_{x} + \ii k_m D^{(1)} \phi_{y} - k_{m}k_{n} M \phi_{z} \\
    \ii k_m D^{(1)} \phi_{x} + \left(k_n^2 + k_m^2\right)M \phi_{y} + \ii k_n D^{(1)} \phi_{z} \\
    - k_{m}k_{n} M \phi_{x} + \ii k_n D^{(1)} \phi_{y} + \left(k_m^2 M - D^{(2)}\right) \phi_{z}
  \end{pmatrix}\right|_{m n}
  ,
\end{align}
is interesting because it looks ``antidiffusive'' in the sense that all
strictly second derivative terms have signs opposite those found in the
Laplacian.  By employing the Helmholtz decomposition and modest smoothness
assumptions, one may show $\nabla\times\nabla\times\vec{\phi}$ is nothing but a
negated Laplacian acting on only the solenoidal portion of $\vec{\phi}$.

We comment that while the Fourier bases are discretely conservative, the
B-spline basis is not in general.  However, the relative discrete conservation
error in the wall normal direction is small enough that using a conservative
continuous formulation remains worthwhile.

\subsection{Time step stability criteria}
\label{sec:stabilitycriteria}

The step size $\Delta{}t$ used within the SMR91 scheme is limited by both a
convective and a diffusive stability criterion.  The time step used is taken to
be the minimum stable time step possible according to either restriction.  As
both criteria are approximate, the resulting $\Delta{}t$ is further multiplied
by a safety factor less than one.  Safety factors like $0.70-0.77$ are often
used~\cite{Venugopal2003,spalart_lowstoragerk}.

\subsubsection{Convective stability criterion}
\label{sec:convectivestability}

The convective criterion uses the maximum imaginary eigenvalue magnitude from
the Euler equations as a surrogate for the more complicated Navier--Stokes
case.  Both Kwok~\cite{Kwok2002} (equation~2.39) and Guarini~\cite{Guarini1998}
(equations~4.20 and~4.21) derive the dimensional stability result that
\begin{align}\label{eq:convectivestability}
  \pi\left(
      \frac{\left|u_{x}\right| + a}{\Delta{}x}
    + \frac{\left|u_{y}\right| + a}{\Delta{}y}
    + \frac{\left|u_{z}\right| + a}{\Delta{}z}
  \right) \Delta{}t \leq \left|\lambda_{I}\Delta{}t\right|_{\mbox{max}}
\end{align}
where $a$ is the local acoustic velocity, $u_{x}$ denotes the velocity in the $x$
direction, $\Delta{}x$ represents the local grid size in the $x$ direction, etc.
The maximum pure imaginary eigenvalue magnitude,
$\left|\lambda_{I}\Delta{}t\right|_{\mbox{max}}$, is a feature of the chosen
timestepping method.  For the SMR91 scheme,
\begin{align}
  \left|\lambda_{I}\Delta{}t\right|_{\mbox{max}} &= \sqrt{3}.
\end{align}

For nondimensional formulations like ours in which an explicit Mach number
$\mbox{Ma}=\frac{u_0}{a_0}$ appears, one must provide the velocities and the
sound speed both nondimensionalized using $u_0$.  That expressions like
$\left|u\right| + \frac{a}{\mbox{Ma}}$ are appropriate in that context can be
seen by finding the eigenvalues of the Euler equations in such a
nondimensionalization.  Using an A-stable scheme, like the implicit portion of
SMR91, to compute acoustic terms effectively sets the sound speed to zero when
computing this convective criterion.

When Venugopal~\cite{Venugopal2003} used a nearly identical convective
criterion (equation~3.10), he found the constraint to be overly conservative in
the wall-normal direction because its derivation (incorrectly) assumes a
periodic wall-normal basis.  Venugopal (section~3.2) presents a linearized
analysis taking into account the inhomogeneous nature of the wall-normal
direction.  He determined that the wall-normal imaginary eigenvalue magnitude
dropped by nearly an order of magnitude after taking into account the
inhomogeneity (i.e., using discrete operators like those in
section~\ref{sec:formingoperators}).  He concluded that using an effective
$1/\Delta{}y$ four times smaller than the nominal value was feasible
(equation~3.29).

\subsubsection{Diffusive stability criterion}
\label{sec:diffusivestability}

The diffusive criterion uses the maximum real eigenvalue magnitude from a model
diffusion equation as a surrogate for the more complicated Navier--Stokes case.
Both Kwok~\cite{Kwok2002} (equations~2.40) and Guarini~\cite{Guarini1998}
(equations~4.29 and~4.30) derive the dimensional stability result that
\begin{align}\label{eq:diffusivestability}
    \mbox{max}\!\left(
      \left|\frac{\gamma\left(\nu-\nu_{0}\right)}{\mbox{Re}\mbox{Pr}}\right|,
      \left|\frac{\nu-\nu_{0}}{\mbox{Re}}\right|,
      \left|\frac{\nu_{B}-\nu_{B0}}{\mbox{Re}}\right|
    \right)
    \pi^{2}
    \left(
        \frac{1}{\Delta{}x^{2}}
      + \frac{1}{\Delta{}y^{2}}
      + \frac{1}{\Delta{}z^{2}}
    \right)
    \Delta{}t \leq \left|\lambda_{R}\Delta_{}t\right|_{\mbox{max}}
\end{align}
where a bulk kinematic viscosity has been added to their results.  The maximum
pure real eigenvalue magnitude,
$\left|\lambda_{R}\Delta{}t\right|_{\mbox{max}}$, is a feature of the chosen
timestepping method.  For the SMR91 scheme,
\begin{align}
\left|\lambda_{R}\Delta{}t\right|_{\mbox{max}} &\approx 2.512.
\end{align}
Using an A-stable scheme, like the implicit portion of SMR91, to compute
linearized viscous terms allows subtracting the linearization reference
kinematic viscosities $\nu_0$ and $\nu_{B0}$ when computing this diffusive
criterion.  The absolute values within the maximum operations account for the
possibility that $\nu<\nu_{0}$.

Venugopal~\cite{Venugopal2003} used a nearly identical diffusive criterion
(equation~3.15).  His analysis determined that the diffusive stability
criterion was not overly conservative for a non-periodic B-spline
discretization.

\section{Numerical considerations}

Here we take the complete mathematical model from section \ref{sec:derivation}
and put it into the form which we will compute per section
\ref{sec:discretization}.  Though less clean in appearance, this section's
equations will better reflect the spectral implementation details used in
Suzerain than those given in earlier sections.

\subsection{Convective derivative operator form}

We choose to use the conservative form of the convective derivative operator,
$\nabla\cdot\left(u\otimes{}\rho{}u\right)$, instead of the skew-symmetric
form, $\frac{1}{2}u\cdot\nabla{}\rho{}u +
\frac{1}{2}\nabla\cdot{}u\otimes{}\rho{}u$.  The former is simpler to compute,
retains the conservative nature of the equations, and behaves comparably to the
latter in the incompressible case when aliasing errors are
removed~\cite{Zang1991Rotation}.  This choice may need to be revisited as the
wall-normal direction is not dealiased.

\subsection{State variable selection}
\label{state_variable_selection}

We use nondimensional density $\rho$, momentum $m=\rho{}u$, and total energy
$e=\rho{}E$ as the state variables for our computations.  Though it
eliminates division and potentially allows for fully dealiased calculations, we
do not use specific density $\sigma=1/\rho$ because it requires using a
nonconservative mass equation.  When rewritten using the state variables
the equations in section~\ref{nondim_equations} become
\begin{subequations}
\begin{align}
  \label{eq:state_continuity}
  \frac{\partial}{\partial{}t}\rho{}
&=
  - \nabla\cdot{}m
  \\
  \label{eq:state_momentum}
  \frac{\partial}{\partial{}t}m
&=
  - \nabla\cdot\left(\frac{m}{\rho}\otimes{}m\right)
  - \frac{1}{\Mach^{2}} \nabla{} p
  + \frac{1}{\Reynolds} \nabla\cdot\tau
  + f
  \\
  \label{eq:state_energy}
  \frac{\partial}{\partial{}t} e
&=
  - \nabla\cdot{}\left(e + p\right)\frac{m}{\rho}
  + \frac{1}{\Reynolds\,\Prandtl\,\left( \gamma - 1 \right)}
    \nabla\cdot\mu\nabla{} T
  + \frac{\Mach^{2}}{\Reynolds} \nabla\cdot\tau{}\frac{m}{\rho}
  + \Mach^{2} f \cdot{} \frac{m}{\rho}
  + q_{b}
\intertext{
  where the non-state quantities are fixed by
}
  \label{eq:state_pressure}
  p &= \left(\gamma-1\right) \left( e - \Mach^{2}\frac{m^2}{2\rho} \right)
  \\
  \label{eq:state_temperature}
  T &= \gamma{} \frac{p}{\rho}
  \\
  \label{eq:state_viscosity}
  \mu &= T^{\beta}
  \\
  \label{eq:state_secondviscosity}
  \lambda &= \left(\alpha- \frac{2}{3}\right) \mu
  \\
  \label{eq:state_viscousstress}
  \tau &= 2 \mu \symmetricpart{\nabla{}\frac{m}{\rho}}
        + \lambda\left(\nabla\cdot{}\frac{m}{\rho}\right) I
\end{align}
\end{subequations}
and we have employed the notation
$\symmetricpart{A}=\frac{1}{2}\left(A+\trans{A}\right)$.

\subsection{Communications overhead}
\label{sec:commoverhead}

As detailed in section~\ref{sec:combineddiscretization}, we perform time
advancement in wave space but must compute nonlinear terms in physical space.
The communications and computation cost required to convert state data from
wave space to physical space or vice versa is very high.  Consequently, we
transform back and forth only once per time integration substep.

\subsection{Velocity derivative expansions}
\label{velocity_derivative_expansions}

Because we transform to and from physical space only once per substep, we must
be able to compute derived quantity derivatives using only state derivatives.
We expand several velocity derivatives into a combination of terms each
containing derivative operators applied only to state quantities:
\begin{subequations}
\begin{align}
  \nabla\cdot\frac{m}{\rho}
  &=
  \rho^{-1}\left[ \nabla\cdot{}m - \rho^{-1}m\cdot\nabla\rho \right]
\\
  \nabla{}\frac{m}{\rho}
  &=
  \rho^{-1}\left[ \nabla{}m - \rho^{-1}{m}\otimes\nabla\rho  \right]
\\
  \symmetricpart{\nabla\frac{m}{\rho}}
  &=
  \rho^{-1}\left[
      \symmetricpart{\nabla{}m}
    - \symmetricpart{\rho^{-1}m\otimes\nabla\rho}
  \right]
\\
  \Delta\frac{m}{\rho}
  &=
 \rho^{-1}\left[
      \Delta{}m
    + \rho^{-1}\left[
          \left(
              2\rho^{-1}\left(\nabla\rho\right)^{2}
            - \Delta\rho
          \right) {m}
        - 2 \left(\nabla{}m\right)\nabla\rho
      \right]
 \right]
\\
  \nabla\nabla\cdot\frac{m}{\rho}
  &=
  \rho^{-1}\left[
        \nabla\nabla\cdot{}m
      + \rho^{-1}\left[
            \left(2\rho^{-1}\nabla\rho\cdot{}m-\nabla\cdot{}m\right)\nabla\rho
          - \left(\nabla\nabla\rho\right)m
          - \trans{\nabla{}m}\nabla\rho
        \right]
  \right]
\\
  \nabla\cdot\left(\frac{m}{\rho}\otimes{}m\right)
  &=
  \rho^{-1}\left[
      \left(\nabla{}m\right)m
      + \left(\nabla\cdot{}m - \rho^{-1}m\cdot\nabla\rho\right)m
  \right]
\end{align}
\begin{align}
  \nabla\times\nabla\times\frac{m}{\rho}
  &=
  \rho^{-1}\left[
        \nabla\times\nabla\times{}{m}
      + \rho^{-1} \left[
            \left(2\nabla{}m - \trans{\nabla{}m} \right) \nabla\rho
          + \left(\Delta\rho - 2 \rho^{-1} \left(\nabla\rho\right)^2 \right) m
        \right.
  \right.
\\ % continued...
  &\qquad\qquad\qquad\qquad\qquad
  \left.
      \left.
          - \left(\nabla\nabla\rho \right) m
          + \left(2\rho^{-1}\nabla\rho\cdot{}m-\nabla\cdot{}m\right)\nabla\rho
      \right]
  \right]
\end{align}
\end{subequations}

We note some relationships amongst the information appearing in such
derivatives:
\begin{align}\label{eq:relationships}
  \Delta\rho
  &=
  \trace\left( \nabla\nabla\rho \right)
&
  \nabla\cdot{}m
  &=
  \trace\left(\nabla{}m\right)
  =
  \trace\symmetricpart{\nabla\frac{m}{\rho}}
\end{align}

\subsection{Separation of first and second derivative operators}
\label{sec:separate_first_second_deriv}

Unlike a Fourier basis, for B-splines the repeated application of a discrete
first derivative operator gives a result that differs significantly from
applying a discrete second derivative operator.  In particular, repeated first
differentiation severely abates high frequency modes (see figures 2 and 3 in
~\cite{Kwok2001}).

Second differentiation enters equations~\eqref{eq:state_continuity},
\eqref{eq:state_momentum}, and~\eqref{eq:state_energy} through the terms
$\nabla\cdot\tau$, $\nabla\cdot\tau\frac{m}{\rho}$, and
$\nabla\cdot\mu\nabla{}T$.  We wish to compute these terms in a way that keeps
first and second derivative applications wholly separate.  Doing so will help
ensure that these three terms have the most physically correct diffusive impact
on high frequency content at a given spatial resolution.  These results will
also be used in the course of developing our implicit diffusive treatment.

We expand the three mixed order, nonlinear terms and use the symmetry of $\tau$:
\begin{align}
\label{eq:nabla_cdot_tau_expansion}
  \nabla\cdot\tau
  &=
    2 \symmetricpart{\nabla\frac{m}{\rho}}\nabla\mu
  + \mu \Delta\frac{m}{\rho}
  + \left(\mu+\lambda\right)\nabla\nabla\cdot\frac{m}{\rho}
  + \left(\nabla\cdot\frac{m}{\rho}\right)\nabla\lambda
\\
\label{eq:nabla_cdot_tau_u_expansion}
  \nabla\cdot\tau{}\frac{m}{\rho}
  &=
    \frac{m}{\rho}\cdot\left(\nabla\cdot\tau\right)
  + \trace\left( \trans{\tau}\nabla\frac{m}{\rho} \right)
\\
  \nabla\cdot\mu\nabla{}T \label{eq:mu_delta_T}
  &=
    \nabla\mu\cdot\nabla{}T
  + \mu \Delta{}T
\end{align}
One may also write
\begin{align}\label{eq:nabla_cdot_tau_expansion_alt}
  \nabla\cdot\tau
  &=
    2 \symmetricpart{\nabla\frac{m}{\rho}}\nabla\mu
  + \left(2\mu+\lambda\right) \Delta\frac{m}{\rho}
  + \left(\mu+\lambda\right)\nabla\times\nabla\times\frac{m}{\rho}
  + \left(\nabla\cdot\frac{m}{\rho}\right)\nabla\lambda
\end{align}
which highlights the isotropic diffusion of velocity within the viscous terms.

Many of the above term contains non-state derivatives which we now expand:
\begin{align}
  \nabla{}p &= (\gamma-1)\left[
        \nabla{}e + \frac{\Mach^{2}}{\rho} \left[
            \frac{m^{2}}{2\rho} \nabla\rho
          - \trans{\nabla{}m}m
        \right]
  \right]
\\
  \nabla{}T &= \frac{\gamma}{\rho}
               \left[ \nabla{}p - \frac{p}{\rho}\nabla\rho \right]
             = \rho^{-1}\left( \gamma\nabla{}p - T\nabla\rho \right)
\\
  \nabla\mu &= \beta{}T^{\beta-1}\nabla{}T
\\
  \nabla\lambda &= \left(\alpha-\frac{2}{3}\right)\nabla\mu
\\
  \Delta{}p
  &=
  \left(\gamma-1\right)\left[
      \Delta{}e
      - \frac{\Mach^{2}}{\rho}\left[
            \trace\left( \trans{\nabla{}m}\nabla{}m \right)
          + m\cdot\Delta{}m
\right.\right. \notag\\ &\qquad\qquad\qquad\qquad \left.\left. % LINE BREAK
        {}- \rho^{-1}\left[
                2\trans{\nabla{}m}m\cdot\nabla{}\rho
              + \frac{1}{2} m^2 \Delta\rho
              - \rho^{-1} m^2 \left(\nabla\rho\right)^{2}
          \right]
      \right]
  \right]
\\
  \Delta{}T
  &=
  \gamma\rho^{-1}\left[
        \Delta{}p
      - \rho^{-1}\left[
            p\Delta{}\rho
          + 2\nabla{}\rho\cdot\left( \nabla{}p - \rho^{-1}p\nabla\rho \right)
      \right]
  \right]
\end{align}

Though we could have found these expressions for only the wall-normal
direction, writing them for the complete $\nabla$ operator allows us to reuse
them later.

\subsection{Implications of a fully explicit treatment}

Though our time advance allows an implicit linear term and our Fourier basis
permits repeated first differentiation, it is useful to examine the
communication costs for a purely explicit implementation.  In this context, all
terms within equations~\eqref{eq:state_continuity}, \eqref{eq:state_momentum},
and~\eqref{eq:state_energy} are formed by $\tilde{N}$ in physical space.
Further, we do not mix derivatives of different orders.  The only linear solve
required uses a wavenumber-independent factorization of the mass matrix $M$.

Using the information in sections~\ref{velocity_derivative_expansions}
and~\ref{sec:separate_first_second_deriv} we tabulate in
table~\vref{tab:nofirstderivnonlinearcost} the specific state variable
derivatives necessary to compute each mixed-derivative nonlinear term and all
of its contributions.  From this table and the equations appearing in
section~\ref{state_variable_selection}, a fully explicit timestepping approach
could compute a single substep at the cost of converting 33 scalar fields from
wave space to physical space, forming 5 scalars representing the right hand
sides of equations~\eqref{eq:state_continuity}--\eqref{eq:state_energy}, and
converting 5 scalar fields back to wave space.

While not nearly as efficient as a hybrid implicit/explicit treatment due
to~\eqref{eq:convectivestability} and~\eqref{eq:diffusivestability}, in
conjunction with Venugopal's wall-normal modification to the convective
stability, a fully explicit treatment does allow slow progress to be made on
small problems.

%%%%%%%%%%%%%%%%%%%%%%%%%%%%%%%%%%%%%%%%%%%%%%%%%%%%%%%%%%%%%%%%%%%%%%%%%%%%%%
%%%%%%%%%%%%%%%%%%%%%%%%%%%%%%%%%%%%%%%%%%%%%%%%%%%%%%%%%%%%%%%%%%%%%%%%%%%%%%
\begin{table}[p]
\centering
\vspace{1em}
\renewcommand{\arraystretch}{1.40}   % Adds whitespace between rows
\newcommand{\cm}{\checkmark}         % For brevity in the table details
\newcommand{\cd}{\ensuremath{\cdot}} % For brevity in the table details
\begin{tabular}{r|cccc|cccccc|ccc|r}
% 001 & 002 & 003 & 004 & 005 & 006 & 007 & 008 & 009 & 011 & 012 & 013 & 014
&   1 &   3 &   1 &   6 &   3 &   1 &   6 &   9 &   3 &   3 &   1 &   3 &   1
\\
& $\rho$                                              % 01
& $\nabla\rho$                                        % 02
& $\Delta\rho$                                        % 03
& $\nabla\nabla\rho$                                  % 04
& $m$                                                 % 05
& $\nabla\cdot{}m$                                    % 06
& $\symmetricpart{\nabla{}m}$                         % 07
& $\nabla{}m$                                         % 08
& $\Delta{}m$                                         % 09
& $\nabla\nabla\cdot{}m$                              % 11
& $e$                                                 % 12
& $\nabla{}e$                                         % 13
& $\Delta{}e$                                         % 14
\\ \hline
$\nabla\cdot\frac{m}{\rho}$
% 001 & 002 & 003 & 004 & 005 & 006 & 007 & 008 & 009 & 011 & 012 & 013 & 014
& \cm & \cm &     &     & \cm & \cm &     &     &     &     &     &     &
& 8 \\
$\nabla\frac{m}{\rho}$
% 001 & 002 & 003 & 004 & 005 & 006 & 007 & 008 & 009 & 011 & 012 & 013 & 014
& \cm & \cm &     &     & \cm &     &     & \cm &     &     &     &     &
& 16 \\
$\symmetricpart{\nabla\frac{m}{\rho}}$
% 001 & 002 & 003 & 004 & 005 & 006 & 007 & 008 & 009 & 011 & 012 & 013 & 014
& \cm & \cm &     &     & \cm &     & \cm &     &     &     &     &     &
& 13 \\
$\Delta\frac{m}{\rho}$
% 001 & 002 & 003 & 004 & 005 & 006 & 007 & 008 & 009 & 011 & 012 & 013 & 014
& \cm & \cm & \cm &     & \cm &     &     & \cm & \cm &     &     &     &
& 20 \\
$\nabla\nabla\cdot\frac{m}{\rho}$
% 001 & 002 & 003 & 004 & 005 & 006 & 007 & 008 & 009 & 011 & 012 & 013 & 014
& \cm & \cm &     & \cm & \cm & \cd &     & \cm &     & \cm &     &     &
& 25 \\[1.5em]
$p$, $T$, $\mu$, $\lambda$
% 001 & 002 & 003 & 004 & 005 & 006 & 007 & 008 & 009 & 011 & 012 & 013 & 014
& \cm &     &     &     & \cm &     &     &     &     &     & \cm &     &
& 5 \\
$\nabla{}p$, $\nabla{}T$, $\nabla\mu$, $\nabla\lambda$
% 001 & 002 & 003 & 004 & 005 & 006 & 007 & 008 & 009 & 011 & 012 & 013 & 014
& \cm & \cm &     &     & \cm &     &     & \cm &     &     & \cm & \cm &
& 20 \\
$\Delta{}p$
% 001 & 002 & 003 & 004 & 005 & 006 & 007 & 008 & 009 & 011 & 012 & 013 & 014
& \cm & \cm & \cm &     & \cm &     &     & \cm & \cm &     &     &     & \cm
& 21 \\
$\Delta{}T$
% 001 & 002 & 003 & 004 & 005 & 006 & 007 & 008 & 009 & 011 & 012 & 013 & 014
& \cm & \cm & \cm &     & \cm &     &     & \cm & \cm &     & \cm & \cm & \cm
& 25 \\[1.5em]
$\tau$
% 001 & 002 & 003 & 004 & 005 & 006 & 007 & 008 & 009 & 011 & 012 & 013 & 014
& \cm & \cm &     &     & \cm & \cd & \cm &     &     &     & \cm &     &
& 14 \\[1.5em]
$\symmetricpart{\nabla\frac{m}{\rho}} \nabla\mu$
% 001 & 002 & 003 & 004 & 005 & 006 & 007 & 008 & 009 & 011 & 012 & 013 & 014
& \cm & \cm &     &     & \cm &     & \cd & \cm &     &     & \cm & \cm &
& 20 \\
$\mu\Delta\frac{m}{\rho}$
% 001 & 002 & 003 & 004 & 005 & 006 & 007 & 008 & 009 & 011 & 012 & 013 & 014
& \cm & \cm & \cm &     & \cm &     &     & \cm & \cm &     & \cm &     &
& 21 \\
$\left(\mu+\lambda\right)\nabla\nabla\cdot\frac{m}{\rho}$
% 001 & 002 & 003 & 004 & 005 & 006 & 007 & 008 & 009 & 011 & 012 & 013 & 014
& \cm & \cm &     & \cm & \cm & \cd &     & \cm &     & \cm & \cm &     &
& 26 \\
$\left(\nabla\cdot\frac{m}{\rho}\right)\nabla\lambda$
% 001 & 002 & 003 & 004 & 005 & 006 & 007 & 008 & 009 & 011 & 012 & 013 & 014
& \cm & \cm &     &     & \cm & \cd &     & \cm &     &     & \cm & \cm &
& 20 \\
$\nabla\cdot\tau$
% 001 & 002 & 003 & 004 & 005 & 006 & 007 & 008 & 009 & 011 & 012 & 013 & 014
& \cm & \cm & \cd & \cm & \cm & \cd & \cd & \cm & \cm & \cm & \cm & \cm &
& 32 \\[1.5em]
$\frac{m}{\rho}\cdot\left(\nabla\cdot\tau\right)$
% 001 & 002 & 003 & 004 & 005 & 006 & 007 & 008 & 009 & 011 & 012 & 013 & 014
& \cm & \cm & \cd & \cm & \cm & \cd & \cd & \cm & \cm & \cm & \cm & \cm &
& 32 \\
$\trace\left(\trans{\tau}\nabla\frac{m}{\rho}\right)$
% 001 & 002 & 003 & 004 & 005 & 006 & 007 & 008 & 009 & 011 & 012 & 013 & 014
& \cm & \cm &     &     & \cm & \cd & \cd & \cm &     &     & \cm &     &
& 20 \\
$\nabla\cdot\tau\frac{m}{\rho}$
% 001 & 002 & 003 & 004 & 005 & 006 & 007 & 008 & 009 & 011 & 012 & 013 & 014
& \cm & \cm & \cd & \cm & \cm & \cd & \cd & \cm & \cm & \cm & \cm & \cm &
& 32 \\[1.5em]
$\nabla\mu\cdot\nabla{}T$
% 001 & 002 & 003 & 004 & 005 & 006 & 007 & 008 & 009 & 011 & 012 & 013 & 014
& \cm & \cm &     &     & \cm &     &     & \cm &     &     & \cm & \cm &
& 20 \\
$\mu\Delta{}T$
% 001 & 002 & 003 & 004 & 005 & 006 & 007 & 008 & 009 & 011 & 012 & 013 & 014
& \cm & \cm & \cm &     & \cm &     &     & \cm & \cm &     & \cm & \cm & \cm
& 25 \\
$\nabla\cdot\mu\nabla{}T$
% 001 & 002 & 003 & 004 & 005 & 006 & 007 & 008 & 009 & 011 & 012 & 013 & 014
& \cm & \cm & \cm &     & \cm &     &     & \cm & \cm &     & \cm & \cm & \cm
& 25
\end{tabular}
\vspace{1em}
\caption{
    State variable derivatives and the relative computational cost required to
    completely compute quantities in physical space without using repeated
    first derivative applications.  A check (\checkmark) indicates that a
    quantity is required to compute the given term.  A dot ($\cdot$) indicates
    the quantity is required but it can be computed from other required
    quantities.  Costs are given relative to the cost of transforming a single
    scalar field from wave space to physical space and do not include floating
    point operations. The total cost for each term is found in the rightmost
    column of the table.
}
\label{tab:nofirstderivnonlinearcost}
\end{table}
%%%%%%%%%%%%%%%%%%%%%%%%%%%%%%%%%%%%%%%%%%%%%%%%%%%%%%%%%%%%%%%%%%%%%%%%%%%%%%
%%%%%%%%%%%%%%%%%%%%%%%%%%%%%%%%%%%%%%%%%%%%%%%%%%%%%%%%%%%%%%%%%%%%%%%%%%%%%%

\subsection{Hybrid implicit/explicit treatment}
\label{sec:imextreatment}

\subsubsection{The need for linearization}

The SMR91 time scheme requires the implicit operator $\tilde{L}$ be linear in
the state variables and time-independent.  Precious little of the
Navier--Stokes operator as written in
equations~\eqref{eq:state_continuity}--\eqref{eq:state_energy} meets these
criteria.  We must carve it up by linearizing about some reference state.  This
approach separates each quantity into an explicitly-treated nonlinear portion
plus a linear contribution that satisfies our implicit operator restrictions.

One example is $\rho^{-1}\Delta{}m = \lessreference{\rho^{-1}}\Delta{}m +
\reference{\rho^{-1}}\Delta{}m$ where $\reference{\rho^{-1}}$ indicates the
term $\rho^{-1}$ evaluated at some reference state.  At one extreme, treating a
three-dimensional reference field is prohibitively expensive but would provide
the longest possible time steps according to~\eqref{eq:convectivestability}
and~\eqref{eq:diffusivestability}.  At the other extreme, a uniform reference
value, which should be chosen from the wall as grid spacing is smallest there,
would have the smallest runtime overhead but would provide the smallest time
step gains.  A good compromise between these extremes is to employ a
one-dimensional reference state profile across the wall-normal direction.

Implicit operator implementation details becomes more complicated when ``off
diagonal'' state derivatives are treated implicitly.  By ``off diagonal'' we
mean derivatives of conserved state appearing on equations other than their
own.  For example, the term $\nabla\cdot{}m$ in~\eqref{eq:state_continuity} or
derivatives of the wall-normal momentum appearing in the streamwise portion
of~\eqref{eq:state_momentum}.  In contrast, an ``on diagonal'' example is the
divergence of total energy appearing within~\eqref{eq:state_energy}.  Handling
off-diagonal terms implicitly is better from the perspective of taking the
largest possible time step while maintaining stability but it incurs both an
associated programming and runtime overhead.

\subsubsection{Linearization of viscous terms}

Because second derivatives give rise to the most restrictive eigenvalues for
timestepping, we focus on how second order terms enter the full operator.  We
take the quantities from sections~\ref{velocity_derivative_expansions}
and~\ref{sec:separate_first_second_deriv} and expand the nonlinear coefficients on
second order terms about reference states.

Beginning with the nonlinear linearization candidates within $\nabla\cdot\tau$
from \eqref{eq:nabla_cdot_tau_expansion}:
\begin{align}
\label{eq:linearready_delta_u}
\mu\Delta\frac{m}{\rho} &=
    2\mu\rho^{-2}\left[
          \rho^{-1}m\left(\nabla\rho\right)^{2}
        - \left(\nabla{}m\right)\nabla\rho
    \right]
\\
  &{}+ \lessreference{\mu\rho^{-1}} \Delta{}m
     - \lessreference{\mu\rho^{-2}m} \Delta\rho
\\
  &{}+ \reference{\mu\rho^{-1}} \Delta{}m
     - \reference{\mu\rho^{-2}m} \Delta\rho
\\
\label{eq:linearready_grad_div_u}
\left(\mu+\lambda\right)\nabla\nabla\cdot\frac{m}{\rho} &=
   \left(\mu+\lambda\right)\rho^{-2}\left[
       \left(2\rho^{-1}\nabla\rho\cdot{}m-\nabla\cdot{}m\right)\nabla\rho
     - \trans{\nabla{}m}\nabla\rho
   \right]
\\
  &{}+ \lessreference{\left(\mu+\lambda\right)\rho^{-1}} \nabla\nabla\cdot{}m
\\
  &{}- \nabla\nabla\rho \lessreference{\left(\mu+\lambda\right)\rho^{-2}m}
\\
  &{}+ \reference{\left(\mu+\lambda\right)\rho^{-1}} \nabla\nabla\cdot{}m
     - \nabla\nabla\rho \reference{\left(\mu+\lambda\right)\rho^{-2}m}
\intertext{
    The term~$\frac{m}{\rho}\cdot\left(\nabla\cdot\tau\right)$ from
    expansion~\eqref{eq:nabla_cdot_tau_u_expansion} contains two similar
    contributions:
}
\label{eq:linearready_umu_delta_u}
\frac{m}{\rho}\cdot\mu\Delta\frac{m}{\rho} &=
    2\mu\rho^{-3}m\cdot\left[
          \rho^{-1}m\left(\nabla\rho\right)^{2}
        - \left(\nabla{}m\right)\nabla\rho
    \right]
\\
  &{}+ \lessreference{\mu\rho^{-2}m}\cdot\Delta{}m
     - \lessreference{\mu\rho^{-3}m^2} \Delta\rho
\\
  &{}+ \reference{\mu\rho^{-2}m}\cdot\Delta{}m
     - \reference{\mu\rho^{-3}m^2}\Delta\rho
\\
\label{eq:linearready_umu_grad_div_u}
\frac{m}{\rho}\cdot\left(\mu+\lambda\right)\nabla\nabla\cdot\frac{m}{\rho} &=
   \left(\mu+\lambda\right)\rho^{-3}m\cdot\left[
       \left(2\rho^{-1}\nabla\rho\cdot{}m-\nabla\cdot{}m\right)\nabla\rho
     - \trans{\nabla{}m}\nabla\rho
   \right]
\\
  &{}+ \lessreference{\left(\mu+\lambda\right)\rho^{-2}m}\cdot\nabla\nabla\cdot{}m
\\
  &{}- \trace\left[
           \trans{\nabla\nabla\rho}
           \lessreference{\left(\mu+\lambda\right)\rho^{-3}m\otimes{}m}
       \right]
\\
  &{}+ \reference{\left(\mu+\lambda\right)\rho^{-2}m}\cdot\nabla\nabla\cdot{}m
     - \trace\left[
           \trans{\nabla\nabla\rho}
           \reference{\left(\mu+\lambda\right)\rho^{-3}m\otimes{}m}
       \right]
\end{align}
Though linearizing the latter two terms is less common, not linearizing them
while linearizing the former two technically requires introducing more terms
into the stability criteria set forth in~\ref{sec:stabilitycriteria}.  We
implicitly treat the latter two terms to unify the numerics across the momentum
and energy equations and to acquire an additional iota of stability.

Finishing with the second order temperature term from
equation~\eqref{eq:mu_delta_T} and its dependencies:
\begin{align}
\Delta{}p =
  &{}- \left(\gamma-1\right)\Mach^{2}\rho^{-1}\left[
             \trace\left(\trans{\nabla{}m}\nabla{}m\right)
           - \rho^{-1}\left[
               2\trans{\nabla{}m}m\cdot\nabla{}\rho
             - \rho^{-1} m^2 \left(\nabla\rho\right)^{2}
           \right]
       \right]
\\
  &{}+ \left(\gamma-1\right)\Delta{}e
     - \left(\gamma-1\right)\Mach^{2}\rho^{-1}m\cdot\Delta{}m
     + \frac{\gamma-1}{2}\Mach^{2}\rho^{-2}m^2 \Delta\rho
\\
\label{eq:linear_ready_delta_T}
\mu\Delta{}T =
  &{}- 2\gamma\mu\rho^{-2}\nabla\rho\cdot
       \left(\nabla{}p-\rho^{-1}p\nabla\rho\right)
     + \gamma\mu\rho^{-1}\Delta{}p
     - \gamma\mu\rho^{-2}p\Delta\rho
\\
=
  &{}- 2\gamma\mu\rho^{-2}\nabla{}\rho\cdot
       \left(\nabla{}p-\rho^{-1}p\nabla\rho\right)
\\
  &{}- \gamma\left(\gamma-1\right)\Mach^{2}\mu\rho^{-2}\left[
             \trace\left(\trans{\nabla{}m}\nabla{}m\right)
           - \rho^{-1}\left[
               2\trans{\nabla{}m}m\cdot\nabla{}\rho
             - \rho^{-1} m^2 \left(\nabla\rho\right)^{2}
           \right]
       \right]
\\
  &{}+ \gamma\left(\gamma-1\right)\lessreference{\mu\rho^{-1}}\Delta{}e
     - \gamma\left(\gamma-1\right)\Mach^{2}
       \lessreference{\mu\rho^{-2}m}\cdot\Delta{}m
\\
  &{}+ \gamma\lessreference{
           \mu\rho^{-2}\left(\left(\gamma-1\right)e-2p\right)
       } \Delta\rho
\\
  &{}+ \gamma\left(\gamma-1\right)\reference{\mu\rho^{-1}}\Delta{}e
     - \gamma\left(\gamma-1\right)\Mach^{2}
       \reference{\mu\rho^{-2}m}\cdot\Delta{}m
     + \gamma\reference{
           \mu\rho^{-2}\left(\left(\gamma-1\right)e-2p\right)
       } \Delta\rho
\end{align}
The final line of each expansion contains linearized, implicit-ready portion.
Note that we will recover an explicit-only operator if we choose identically
zero reference values.

\subsubsection{Linearization of acoustic terms}

We now focus on the first order acoustic terms within the momentum and energy
equations.  In the inviscid limit of the hyperbolic Euler equations, the
pressure gives rise to the acoustic characteristics traveling at speeds
$u\pm{}a$.  Pressure is fundamentally an off-diagonal phenomenon requiring
off-diagonal implicit treatment.

Guarini~\cite{Guarini1998} (page~45) treated the pressure gradient term in the
wall-normal momentum equation and the linearized pressure work term implicitly.
The pressure gradient may be linearized as
\begin{align}
  \nabla{}p &= \left(\gamma-1\right)\Mach^{2}\left(
      \frac{1}{2} \lessreference{m^{2}\rho^{-2}}\nabla\rho
    - \trans{\nabla{}m}\lessreference{\rho^{-1}m}
  \right)
\\
&+ \left(\gamma-1\right) \nabla{}e
 + \frac{\gamma-1}{2}\Mach^{2} \reference{m^{2}\rho^{-2}}\nabla\rho
 - \left(\gamma-1\right)\Mach^{2} \trans{\nabla{}m}\reference{\rho^{-1}m}
 .
\end{align}
The total energy convection and pressure work terms may be linearized as
\begin{align}
\nabla\cdot\left(e+p\right)\frac{m}{\rho} =
   &- \left(\gamma-1\right)\mbox{Ma}^{2}\rho^{-2}m\cdot \trans{\nabla{}m}m
    + \gamma\lessreference{ \rho^{-1}m }\cdot\nabla{}e
  \\
   &+ \lessreference{
        \rho^{-1}\left(e+p\right)
      } \nabla\cdot{}m
  \\
   &+ \lessreference{
        \rho^{-2}m\left(\left(\gamma-1\right)e-2p\right)
      }\cdot\nabla\rho
  \\
   &+ \gamma\reference{
        \rho^{-1}m
      }\cdot\nabla{}e
    + \reference{
        \rho^{-1}\left(e+p\right)
      }\nabla\cdot{}m
    + \reference{
        \rho^{-2}m\left(\left(\gamma-2\right)e-2p\right)
      }\cdot\nabla\rho
.
\end{align}
We have linearized the two terms together as doing so is algebraically simple
and introduces no additional state derivatives.

\subsubsection{Linearity of the continuity equation}
\label{sec:contconv}

If off-diagonal density and momentum derivatives are computed implicitly,
implicitly treating the convective term $-\nabla\cdot{}m$ in
equation~\eqref{eq:state_continuity} is simple.  Doing so reduces by one the
number of scalar fields needing conversion from physical space to wave space.

\subsubsection{Linearization of the convective terms in the momentum equation}
\label{sec:momtconv}

Once the convective terms in equations~\eqref{eq:state_continuity}
and~\eqref{eq:state_energy} and linearized viscous terms are treated
implicitly, the incremental cost to treat the convective term in
equation~\eqref{eq:state_momentum} is small.  In particular, linearizing this
term does not increase the discrete operator bandwidth.  The linearization used
is
\begin{align}
  \nabla\cdot\left(\frac{m}{\rho}\otimes{}m\right)
&=
    \left(\nabla{}m + I \nabla\cdot{}m\right)\lessreference{\rho^{-1}m}
  - \lessreference{\rho^{-1}m\otimes\rho^{-1}m}\nabla\rho
\\
 &+ \left(\nabla{}m + I \nabla\cdot{}m\right)\reference{\rho^{-1}m}
  - \reference{\rho^{-1}m\otimes\rho^{-1}m}\nabla\rho
  .
\end{align}

Two benefits arise from this additional work.  First, simulations with
supersonic velocities at sufficiently high Reynolds number are limited by
convective stability.  Implicit treatment of the linearized convective operator
allows using a time step safety factor closer to one.  This reduces the wall
time necessary to obtain converged statistics or allows finer resolution atop
fixed computing resources.

Second, having convective terms handled implicitly in all equations replaces
$u_x$, $u_y$, and $u_z$ in criterion~\eqref{eq:convectivestability} with
$\left|u_x-u_{x0}\right|$, $\left|u_y-u_{y0}\right|$, and
$\left|u_z-u_{z0}\right|$ in a manner similar to the appearance of $\nu-\nu_0$
within criterion~\eqref{eq:diffusivestability}.  While such large time steps
cannot be taken in time-accurate simulations due to the temporal damage done to
the turbulent dynamics, such time steps will greatly accelerate time-inaccurate
simulations advancing across uninteresting transients.  For example, changing
$\Reynolds$, $\Prandtl$, or $\Mach$ often causes a lengthy transient in the
bulk energy within the domain.  Time-inaccurate simulation may be used until
the bulk energy again becomes stationary.  Of course, time-accurate
calculations must then be performed until the turbulent dynamics become
stationary prior to collecting statistics.

\subsubsection{The implicitly-treated terms}
\label{sec:implicitlytreatedterms}

Treating wall-normal acoustics implicitly requires off-diagonal coupling
between the wall-normal momentum and total energy equations.  The incremental
programming cost to further couple the streamwise and spanwise momentum
equations to the total energy equation is comparatively small.  The ability to
treat all directions implicitly for both acoustic and diffusive terms has a
higher runtime cost which should be offset by the larger time stability that
results.  Coupling the density equation reduces the communications overhead at
the expense of on-node work (see section~\ref{sec:contconv}) and further
unifies the programmatic details.  Moreover, so long as the fully coupled
implicit operations can fit in cache, increasing the on-node work to take
larger time steps (e.g. section~\ref{sec:momtconv}) helps to balance the high
communication overhead stemming from sections~\ref{sec:commoverhead}
and~\ref{sec:separate_first_second_deriv}.  Improving the computation to
communication ratio bolsters scalability.

Accordingly, we choose to implicitly treat all terms identified as candidates
in the preceding discussion.  In the full context of
equations~\eqref{eq:state_continuity}--\eqref{eq:state_energy} these terms are
\begin{align}
  \frac{\partial}{\partial{}t} \rho{} = &-\nabla\cdot{}m
\\
  \frac{\partial}{\partial{}t} m = \dots
% \underbrace{
   &+ \overleftrightarrow{c^{u\otimes{}u}} \nabla\rho
    - \left(\nabla{}m+I\nabla\cdot{}m\right)\overrightarrow{c^u}
% }_{-\nabla\cdot\left(\frac{m}{\rho}\otimes{}m\right)}
% \underbrace{
    - \frac{\gamma-1}{2} c^{u^2} \nabla\rho
    + \left(\gamma-1\right)\trans{\nabla{}m} \overrightarrow{c^u}
    - \frac{\gamma-1}{\Mach^2}\nabla{}e
% }_{-\Mach^{-1}\nabla{}p}
\\
% \underbrace{
   &- \Reynolds^{-1} \overrightarrow{c^{\nu{}u}} \Delta\rho
    - \Reynolds^{-1} \left(\alpha+\frac{1}{3}\right) \left(\nabla\nabla\rho\right) \overrightarrow{c^{\nu{}u}}
    + \Reynolds^{-1} c^{\nu} \Delta{}m
    + \Reynolds^{-1} \left(\alpha+\frac{1}{3}\right)c^{\nu} \nabla\nabla\cdot{}m
% }_{\Reynolds^{-1}\nabla\cdot\tau}
    + \dots
\\
  \frac{\partial}{\partial{}t} e = \dots
   &- \overrightarrow{c^{e}_{\nabla\rho}} \cdot\vec{\nabla}\rho
    - c^{e}_{\nabla\cdot{}m} \nabla\cdot{}m
    - \gamma \overrightarrow{c^u}\cdot\nabla{}e
    + \frac{\gamma}{\Reynolds\Prandtl\left(\gamma-1\right)}
      c^{e}_{\Delta\rho} \Delta\rho
    - \frac{\gamma\Mach^{2}}{\Reynolds\Prandtl}
      \overrightarrow{c^{\nu{}u}}\cdot\Delta{}m
    + \frac{\gamma}{\Reynolds\Prandtl}c^{\nu}\Delta{}e
\\
% \underbrace{
   &+ \frac{\Mach^2}{\Reynolds}\left(
       - c^{\nu{}u^2}\Delta\rho
       - \left(\alpha+\frac{1}{3}\right)
         \trace\left(\trans{\nabla\nabla\rho}
                     \overleftrightarrow{c^{\nu{}u\otimes{}u}}\right)
       + \overrightarrow{c^{\nu{}u}}\cdot\Delta{}m
       + \left(\alpha+\frac{1}{3}\right)
         \overrightarrow{c^{\nu{}u}}\cdot\nabla\nabla\cdot{}m
   \right)
% }_{\Mach^2\Reynolds^{-1}\nabla\cdot\tau\frac{m}{\rho}}
       + \dots
\end{align}
where some reference values have physically-motivated superscripts
\begin{align}
  \overrightarrow{c^{u}} &= \reference{\rho^{-1}m}
  = \begin{pmatrix} c^{u_x} \\ c^{u_y} \\ c^{u_z} \end{pmatrix}
&
  c^{u^2} &= \reference{m^{2}\rho^{-2}}
&
   \overleftrightarrow{c^{u\otimes{}u}}
  = \reference{\rho^{-1}m\otimes\rho^{-1}m}
  = \begin{pmatrix}
   c^{u_x u_x} & c^{u_x u_y} & c^{u_x u_z} \\
   c^{u_x u_y} & c^{u_y u_y} & c^{u_y u_z} \\
   c^{u_x u_z} & c^{u_y u_z} & c^{u_z u_z}
  \end{pmatrix}
\end{align}
\begin{align}
  c^{\nu} &= \reference{\rho^{-1}\mu}
&
  \overrightarrow{c^{\nu{}u}} &= \reference{\rho^{-2}\mu{}m}
  = \begin{pmatrix} c^{\nu{}u_x} \\ c^{\nu{}u_y} \\ c^{\nu{}u_z} \end{pmatrix}
&
  c^{\nu{}u^2} &= \reference{\rho^{-3}\mu{}m^2}
\end{align}
\begin{align}
   \overleftrightarrow{c^{\nu{}u\otimes{}u}}
  = \reference{\rho^{-3}\mu{}m\otimes{}m}
  = \begin{pmatrix}
   c^{\nu{} u_x u_x} & c^{\nu{} u_x u_y} & c^{\nu{} u_x u_z} \\
   c^{\nu{} u_x u_y} & c^{\nu{} u_y u_y} & c^{\nu{} u_y u_z} \\
   c^{\nu{} u_x u_z} & c^{\nu{} u_y u_z} & c^{\nu{} u_z u_z}
  \end{pmatrix}
\end{align}
while the remaining reference values
\begin{align}
  \overrightarrow{c^{e}_{\nabla\rho}} &= \reference{
        m\rho^{-2}\left(\left(\gamma-2\right)e-2p\right)
  }
  = \begin{pmatrix}
      c^{e_{x}}_{\nabla\rho} \\
      c^{e_{y}}_{\nabla\rho} \\
      c^{e_{z}}_{\nabla\rho}
  \end{pmatrix}
&
  c^{e}_{\nabla\cdot{}m} &= \reference{
        \rho^{-1}\left(e + p\right)
  }
&
  c^{e}_{\Delta\rho} &= \reference{
        \mu\rho^{-2}\left(\left(\gamma-1\right)e-2p\right)
  }
\end{align}
have superscripts indicating the relevant equation and
subscripts indicating the associated term.

\subsubsection{Discretization of the implicitly-treated terms}
\label{sec:discretizationofimplicitterms}

Following~\eqref{eq:generaloperatormasssubstep} and using results
from~\ref{sec:formingoperators}, we must implement $M+\varphi{}L$ for arbitrary
$\varphi$, $k_m$, and $k_n$. $L$ is nothing but the discrete form of the linear
terms from the previous section.  Notice for any reference values
$c^{\bullet}$ left-multiplying by the diagonal matrix
\begin{align}
  C^{\bullet} &= \begin{bmatrix}
   \left.c^{\bullet}\right|_{y=0} &        & 0 \\
                                  & \ddots &    \\
   0                              &        & \left.c^{\bullet}\right|_{y=L}
   \end{bmatrix}
\end{align}
scales linear operators in a way that accommodates wall-normal variations in
reference quantities.  For example, applying $C^{\nu}D^{(2)}$ rather than
$D^{(2)}$ scales the result at collocation point $y=y_l$ by
$\left.c^{\nu}\right|_{y=y_l}$.

Switching to a blocked matrix representation employing five scalar conserved
state fields, the complete, SMR91-ready discrete operator $M+\varphi{}L$ is
shown in figure~\vref{fig:discreteimplicitop}.  The representation chosen
highlights how the full operator is built from discrete operators applied to
individual state fields.  Applying the operator does not require assembling it
within a contiguous region of memory.  ``Inverting'' the operator (more
correctly, factorizing and solving for unknown coefficients given collocation
point values) does require assembly.  Employing an appropriate permutation
matrix $P$ causes the renumbered operator $P\left(M+\varphi{}L\right)\trans{P}$
to be globally banded with a small bandwidth independent of the number of
wall-normal degrees of freedom.  This $P\left(M+\varphi{}L\right)\trans{P}$
allows efficient banded factorization and back substitution.  Recent work has
shown forthcoming ``manycore'' architectures to be efficient at performing many
such conveniently parallel factorizations~\cite{Schulz2012Early}.

Strong Dirichlet boundary conditions may be enforced in the usual way.  Neumann
boundary condition implementations are less straightforward.  For simulations
with no mean bulk velocity in the spanwise direction, the reference coefficient
matrices $C^{u_z}$ and $C^{\nu{}u_z}$ may be treated as effectively zero.  This
reduces the required linear algebra.  If desired, the density equation and
density terms in the other equations may removed from the result in
figure~\ref{fig:discreteimplicitop} to greatly reduce operator assembly and
factorization overhead.  Previous work by Guarini did not treat density
implicitly~\cite{Guarini1998}.  Finally, implicitly handling only the
wall-normal directions may be accomplished by setting $k_{m}$ and $k_{n}$ to
zero.  Doing so results in a wavenumber independent operator requiring
factorization only once per substep.

\begin{sidewaysfigure}
\newcommand{\entry}[1]{}          % Provides comments for subblocks
\newcommand{\C}[2]{C^{#1}_{#2}}   % For brevity below
\newcommand{\D}[1]{D^{(#1)}}      % ditto
\newcommand{\M}{M}                % ditto
\newcommand{\g}{\gamma}           % ditto
\newcommand{\km}{k_{m}}           % ditto
\newcommand{\kn}{k_{n}}           % ditto
\newcommand{\mx}{m_{x}}           % ditto
\newcommand{\my}{m_{y}}           % ditto
\newcommand{\mz}{m_{z}}           % ditto
\newcommand{\vp}{\varphi}         % ditto
\newcommand{\subcoeff}[3]{{       % ditto
   \renewcommand{\arraystretch}{2.0}
   \begin{Bmatrix}{#1}\\{#2}\\{#3}\end{Bmatrix}
}}
\newcommand{\Ma}{\ensuremath{\mbox{\small{}Ma}}}
\renewcommand{\Pr}{\ensuremath{\mbox{\small{}Pr}}}
\renewcommand{\Re}{\ensuremath{\mbox{\small{}Re}}}
\hspace{-.04\textwidth}
{\resizebox{1.08\textwidth}{!}{\begin{minipage}[c]{\textwidth}  % SCALE-TO-FIT
\begin{align*}
\bm{\vp}
\renewcommand{\arraystretch}{9.0} % Adds whitespace between rows
\addtolength{\arraycolsep}{-.1em}
\begin{bmatrix}
% Density row
  \entry{\rho\rho}
  \subcoeff{
      \bm{\frac{1}{\vp}} % M
  }{
  }{
  }
& \entry{\rho\mx }
  \subcoeff{
    - \ii\km
  }{
  }{
  }
& \entry{\rho\my }
  \subcoeff{
  }{
    - 1
  }{
  }
& \entry{\rho\mz }
  \subcoeff{
    - \ii\kn
  }{
  }{
  }
& \entry{\rho{}e }
  0
% Streamwise momentum row
\\\entry{\mx\rho }
  \subcoeff{
    \begin{pmatrix}
        \frac{1-\g}{2}\ii\km\C{u^2}{}
      + \frac{\left(\alpha+\frac{4}{3}\right)\km^2+\kn^2}{\Re}\C{\nu{}u_x}{}
      + \frac{\alpha+\frac{1}{3}}{\Re}\km\kn\C{\nu{}u_z}{}
      \\
      + \ii\km\C{u_x u_x}{}
      + \ii\kn\C{u_x u_z}{}
    \end{pmatrix}
  }{
    - \frac{\alpha+\frac{1}{3}}{\Re}\ii\km\C{\nu{}u_y}{}
    + \C{u_x u_y}{}
  }{
    - \frac{1}{\Re}\C{\nu{}u_x}{}
  }
& \entry{\mx\mx  }
  \subcoeff{
      \bm{\frac{1}{\vp}} % M
    + \left(\g-3\right)\ii\km\C{u_x}{}
    - \ii\kn\C{u_z}{}
    - \frac{\left(\alpha+\frac{4}{3}\right)\km^2+\kn^2}{\Re}\C{\nu}{}
  }{
    - \C{u_y}{}
  }{
      \frac{1}{\Re}\C{\nu}{}
  }
& \entry{\mx\my  }
  \subcoeff{
      \left(\g-1\right)\ii\km\C{u_y}{}
  }{
    - \C{u_x}{}
    + \frac{\alpha+\frac{1}{3}}{\Re}\ii\km\C{\nu}{}
  }{
  }
& \entry{\mx\mz  }
  \subcoeff{
      \left(\g-1\right)\ii\km\C{u_z}{}
    - \ii\kn\C{u_x}{}
    - \frac{\alpha+\frac{1}{3}}{\Re}\km\kn\C{\nu}{}
  }{
  }{
  }
& \entry{\mx{}e  }
  \subcoeff{
      \frac{1-\g}{\Ma^2}\ii\km
  }{
  }{
  }
% Wall-normal momentum row
\\\entry{\my\rho }
  \subcoeff{
      \frac{\km^2+\kn^2}{\Re}\C{\nu{}u_y}{}
    + \ii\km\C{u_x u_y}{}
    + \ii\kn\C{u_y u_z}{}
  }{
      \frac{1-\g}{2}\C{u^2}{}
    - \frac{\alpha+\frac{1}{3}}{\Re}\ii\left(
         \km\C{\nu{}u_x}{} + \kn\C{\nu{}u_z}{}
      \right)
    + \C{u_y u_y}{}
  }{
    - \frac{\alpha+\frac{4}{3}}{\Re}\C{\nu{}u_y}{}
  }
& \entry{\my\mx  }
  \subcoeff{
    - \ii\km\C{u_y}{}
  }{
      \left(\g-1\right)\C{u_x}{}
    + \frac{\alpha+\frac{1}{3}}{\Re}\ii\km\C{\nu}{}
  }{
  }
& \entry{\my\my  }
  \subcoeff{
      \bm{\frac{1}{\vp}} % M
    - \ii\km\C{u_x}{}
    - \ii\kn\C{u_z}{}
    - \frac{\km^2+\kn^2}{\Re}\C{\nu}{}
  }{
      \left(\g-3\right)\C{u_y}{}
  }{
      \frac{\alpha+\frac{4}{3}}{\Re}\C{\nu}{}
  }
& \entry{\my\mz  }
  \subcoeff{
    - \ii\kn\C{u_y}{}
  }{
      \left(\g-1\right)\C{u_z}{}
    + \frac{\alpha+\frac{1}{3}}{\Re}\ii\kn\C{\nu}{}
  }{
  }
& \entry{\my{}e  }
  \subcoeff{
  }{
      \frac{1-\g}{\Ma^2}
  }{
  }
% Spanwise momentum row
\\\entry{\mz\rho}
  \subcoeff{
    \begin{pmatrix}
        \frac{1-\g}{2}\ii\kn\C{u^2}{}
      + \frac{\km^2+\left(\alpha+\frac{4}{3}\right)\kn^2}{\Re}\C{\nu{}u_z}{}
      + \frac{\alpha+\frac{1}{3}}{\Re}\km\kn\C{\nu{}u_x}{}
      \\
      + \ii\km\C{u_x u_z}{}
      + \ii\kn\C{u_z u_z}{}
    \end{pmatrix}
  }{
    - \frac{\alpha+\frac{1}{3}}{\Re}\ii\kn\C{\nu{}u_y}{}
    + \C{u_y u_z}{}
  }{
    - \frac{1}{\Re}\C{\nu{}u_z}{}
  }
& \entry{\mz\mx }
  \subcoeff{
      \left(\g-1\right)\ii\kn\C{u_x}{}
    - \ii\km\C{u_z}{}
    - \frac{\alpha+\frac{1}{3}}{\Re}\km\kn\C{\nu}{}
  }{
  }{
  }
& \entry{\mz\my }
  \subcoeff{
      \left(\g-1\right)\ii\kn\C{u_y}{}
  }{
    - \C{u_z}{}
    + \frac{\alpha+\frac{1}{3}}{\Re}\ii\kn\C{\nu}{}
  }{
  }
& \entry{\mz\mz }
  \subcoeff{
      \bm{\frac{1}{\vp}} % M
    - \ii\km\C{u_x}{}
    + \left(\g-3\right)\ii\kn\C{u_z}{}
    - \frac{\km^2+\left(\alpha+\frac{4}{3}\right)\kn^2}{\Re}\C{\nu}{}
  }{
    - \C{u_y}{}
  }{
      \frac{1}{\Re}\C{\nu}{}
  }
& \entry{\mz{}e }
  \subcoeff{
      \frac{1-\g}{\Ma^2}\ii\kn
  }{
  }{
  }
% Total energy row
\\\entry{e\rho  }
  \subcoeff{
    \begin{pmatrix}
      - \ii\left(\km\C{e_x}{\nabla\rho} + \kn\C{e_z}{\nabla\rho}\right)
      \\
      - \g\frac{\km^2+\kn^2}{\Re\Pr\left(\g-1\right)}\C{e}{\Delta\rho}
      + \frac{\Ma^2}{\Re}\left(\km^2+\kn^2\right)\C{\nu{}u^2}{}
      \\
      + \frac{\Ma^2}{\Re}\left(\alpha+\frac{1}{3}\right)\left(
              \km^2 \C{\nu{}u_x u_x}{}
          + 2 \km\kn\C{\nu{}u_x u_z}{}
          +   \kn^2 \C{\nu{}u_z u_z}{}
        \right)
    \end{pmatrix}
  }{
    - \C{e_y}{\nabla\rho}
    + \frac{\Ma^2}{\Re}\left(\alpha+\frac{1}{3}\right)\left(
       - 2\ii\km\C{\nu{}u_x u_y}{}
       - 2\ii\kn\C{\nu{}u_y u_z}{}
      \right)
  }{
      \frac{\g}{\Re\Pr\left(\g-1\right)}\C{e}{\Delta\rho}
    - \frac{\Ma^2}{\Re}\C{\nu{}u^2}{}
    - \frac{\Ma^2}{\Re}\left(\alpha+\frac{1}{3}\right)\C{\nu{}u_y u_y}{}
  }
& \entry{e\mx   }
  \subcoeff{
    \begin{pmatrix}
      \frac{\Ma^2}{\Re}\left[
          \left(\frac{\g}{\Pr}-\left(\alpha+\frac{4}{3}\right)\right)\km^2
        + \left(\frac{\g}{\Pr}-1                              \right)\kn^2
      \right] \C{\nu{}u_x}{}
      \\
      - \frac{\Ma^2}{\Re}\left(\alpha+\frac{1}{3}\right) \km\kn\C{\nu{}u_z}{}
      - \ii\km\C{e}{\nabla\cdot{}m}
    \end{pmatrix}
  }{
    \frac{\Ma^2}{\Re}\left(\alpha+\frac{1}{3}\right)\ii\km\C{\nu{}u_y}{}
  }{
    \frac{\Ma^2}{\Re}\left(1 - \frac{\g}{\Pr}\right)\C{\nu{}u_x}{}
  }
& \entry{e\my   }
  \subcoeff{
    \frac{\Ma^2}{\Re}\left(\frac{\g}{\Pr} - 1\right)
    \left(\km^2+\kn^2\right)\C{\nu{}u_y}{}
  }{
    - \C{e}{\nabla\cdot{}m}
    + \frac{\Ma^2}{\Re}\left(\alpha+\frac{1}{3}\right)\left(
          \ii\km\C{\nu{}u_x}{}
        + \ii\kn\C{\nu{}u_z}{}
      \right)
  }{
    \frac{\Ma^2}{\Re}\left[
      \left(\alpha+\frac{4}{3}\right) - \frac{\g}{\Pr}
    \right] \C{\nu{}u_y}{}
  }
& \entry{e\mz   }
  \subcoeff{
    \begin{pmatrix}
      \frac{\Ma^2}{\Re}\left[
          \left(\frac{\g}{\Pr}-1                              \right)\km^2
        + \left(\frac{\g}{\Pr}-\left(\alpha+\frac{4}{3}\right)\right)\kn^2
      \right] \C{\nu{}u_z}{}
      \\
      - \frac{\Ma^2}{\Re}\left(\alpha+\frac{1}{3}\right) \km\kn\C{\nu{}u_x}{}
      - \ii\kn\C{e}{\nabla\cdot{}m}
    \end{pmatrix}
  }{
    \frac{\Ma^2}{\Re}\left(\alpha+\frac{1}{3}\right)\ii\kn\C{\nu{}u_y}{}
  }{
    \frac{\Ma^2}{\Re}\left(1 - \frac{\g}{\Pr}\right) \C{\nu{}u_z}{}
  }
& \entry{ee     }
  \subcoeff{
      \bm{\frac{1}{\vp}} % M
    - \g\ii\left(\km\C{u_x}{} + \kn\C{u_z}{}\right)
    - \frac{\g}{\Re\Pr}\left(\km^2+\kn^2\right)\C{\nu}{}
  }{
    - \g\C{u_y}{}
  }{
      \frac{\g}{\Re\Pr}\C{\nu}{}
  }
\end{bmatrix}
\renewcommand{\arraystretch}{1.0}
\begin{bmatrix}
  \hat{\rho}_{\left(0,\,m,\,n\right)} \\
  \vdots \\
  \hat{\rho}_{\left(N_y-1,\,m,\,n\right)} \\
\\%
\\%
\\%
\\%
  \hat{\mx}_{\left(0,\,m,\,n\right)} \\
  \vdots \\
  \hat{\mx}_{\left(N_y-1,\,m,\,n\right)} \\
\\%
\\%
\\%
\\%
  \hat{\my}_{\left(0,\,m,\,n\right)} \\
  \vdots \\
  \hat{\my}_{\left(N_y-1,\,m,\,n\right)} \\
\\%
\\%
\\%
\\%
  \hat{\mz}_{\left(0,\,m,\,n\right)} \\
  \vdots \\
  \hat{\mz}_{\left(N_y-1,\,m,\,n\right)} \\
\\%
\\%
\\%
\\%
  \hat{e}_{\left(0,\,m,\,n\right)} \\
  \vdots \\
  \hat{e}_{\left(N_y-1,\,m,\,n\right)} \\
%
\end{bmatrix}
\end{align*}
\end{minipage}}}  % END SCALE-TO-FIT!
\vspace{2em}
\\
 \caption[The discrete operator $M+\varphi{}L$ used for implicit time advance]{
The complete discrete operator $M+\varphi{}L$ used for implicit time advance is
depicted.  Notice the leftmost scalar factor $\bm{\vp}$.  The $3 N_y \times
N_y$ blocked vectors surrounded by curly braces are to be ``dotted'' against
the blocked vector $
  \trans{\begin{bmatrix} \M & \D{1} & \D{2} \end{bmatrix}}
$ to form $N_y \times N_y$ subblocks.  Each of $M$, $\D{1}$, and $\D{2}$ is a
$N_y \times N_y$ banded matrix.  Reference quantities like $C^\nu$ are $N_y
\times N_y$ diagonal matrices.  The wavenumber-dependent operator takes
wall-normal B-spline coefficients to B-spline collocation point values.
 }
\label{fig:discreteimplicitop}
\end{sidewaysfigure}

\subsection{Boundary conditions}

\subsubsection{Isothermal, no slip boundary condition}

An isothermal wall is characterized by a fixed temperature and zero velocity.
One thermodynamic quantity must be allowed to vary for this boundary condition
to be well-posed.  Allowing $\rho$ to vary is simplest given our state choices.
Then~\eqref{eq:state_pressure} and~\eqref{eq:state_temperature} combined with
$m=0$ forces $e=\frac{T}{\gamma\left(\gamma-1\right)}$ for some fixed
temperature $T$.  These conditions may be built into linear operators for
implicit treatment using how B-spline basis support limits the number of
nonzero coefficients at the wall as discussed in
section~\ref{sec:formingoperators}.

\subsubsection{Nonreflecting freestream boundary conditions}

When simulating problems on semi-infinite domains, such as flat plates,
Suzerain requires nonreflecting freestream boundary conditions.  Without these,
acoustic waves generated by the flow cannot be leave the domain.   The trapped
acoustics pile up causing a non-physical partition of energy and spoiling the
simulated statistics.

Following Engquist and Majda~\cite{Engquist1977Absorbing},
Giles~\cite{Giles1988Nonreflecting,Giles1990Nonreflecting} developed localized,
approximate two-dimensional, unsteady nonreflecting boundary conditions for the
Euler equations.  We select Giles' boundary conditions over Poinsot and Lele's
``locally one-dimensional inviscid'' relations~\cite{Poinsot1992Boundary}
because other codes with similar numerics have successfully employed Giles'
conditions for our problems of interest.  While Rowley and
Colonius~\cite{Rowley2000Discretely} present higher order techniques expected
to perform better than Giles' approach, what they describe is considerably more
complex to implement.  Saxer and Giles~\cite{Saxer1993QuasiThreeDimensional}
extended the technique to three dimensions but with a presentation aimed
towards transonic axial flow turbomachinery computations.
Guarini~\cite{Guarini1998} summarizes the Cartesian extension of Giles'
approach to three spatial dimensions without reproducing the associated
analysis.  Medida~\cite{Medida2007} lucidly catalogs the intermediate results
necessary in three dimensions.

We now review Giles' approach following Guarini's presentation with
the goal of setting notation suitable for presenting and manipulating
Medida's results for nonreflecting $x$ boundaries in three-dimensional,
Cartesian coordinates.  For complete details, especially motivations
and proofs, the work of Giles, Medida, and Guarini should be consulted
in that respective order.

For the state vector
\begin{subequations}
\label{eq:eulerprim}
\begin{align}
  U &= \left\{ \rho, u, v, w, p \right\}
\end{align}
the Euler equations, using the ideal gas equation of state
\begin{align}
  \rho a^2 &= \gamma p,
\end{align}
may be written as follows:
\begin{align}
    \frac{\partial}{\partial{}t}U
+ A \frac{\partial}{\partial{}x}U
+ B \frac{\partial}{\partial{}y}U
+ C \frac{\partial}{\partial{}z}U
&= 0
\end{align}
\begin{align}
 A &= \begin{bmatrix}
        u & \rho     & 0 & 0 & 0              \\
        0 & u        & 0 & 0 & \frac{1}{\rho} \\
        0 & 0        & u & 0 & 0              \\
        0 & 0        & 0 & u & 0              \\
        0 & \gamma p & 0 & 0 & u              \\
       \end{bmatrix}
&
 B &= \begin{bmatrix}
        v & 0 & \rho     & 0 & 0              \\
        0 & v & 0        & 0 & 0              \\
        0 & 0 & v        & 0 & \frac{1}{\rho} \\
        0 & 0 & 0        & v & 0              \\
        0 & 0 & \gamma p & 0 & v              \\
       \end{bmatrix}
&
 C &= \begin{bmatrix}
        w & 0 & 0 & \rho     & 0              \\
        0 & w & 0 & 0        & 0              \\
        0 & 0 & w & 0        & 0              \\
        0 & 0 & 0 & w        & \frac{1}{\rho} \\
        0 & 0 & 0 & \gamma p & w              \\
       \end{bmatrix}
\end{align}
\end{subequations}
This system of equations identically describes the behavior of an analogous
$U^*$ whenever all of
\begin{align}
\label{eq:eulerprimnondim}
U^{*} &= \left\{
  \frac{\rho}{\rho_0},
  \frac{u}{u_0},
  \frac{v}{u_0},
  \frac{w}{u_0},
  \frac{p}{\rho_0 u_0^2}
\right\}
&
t_0 &= \frac{l_0}{u_0}
&
a_0 &= u_0
\end{align}
hold.  Therefore, all dimensional results obtained for $U$ remain unchanged
in the setting of $U^*$.

Consider perturbations
\[
\delta{}U = \left\{ \delta{}\rho, \delta{}u,
\delta{}v, \delta{}w, \delta{}p \right\}
\]
taken about some steady,
uniform reference state $\bar{U}$ so that
\[
U = \bar{U} + \delta{}U.
\]
The short-time evolution of such perturbations is governed by the linearized
Euler equations
\begin{align}
\label{eq:dimeulerperturb}
               \frac{\partial}{\partial{}t}\delta{}U
+ \bar{A} \frac{\partial}{\partial{}x}\delta{}U
+ \bar{B} \frac{\partial}{\partial{}y}\delta{}U
+ \bar{C} \frac{\partial}{\partial{}z}\delta{}U
&= 0
\end{align}
where matrices $\bar{A}$, $\bar{B}$, and $\bar{C}$ are evaluated
at $\bar{U}$.  This linearized system satisfies the prerequisites for
Giles' analysis.  Assuming a solution of the form
\begin{align}
  \delta{}U &= e^{\ii\left(
    k_x x + k_y y + k_z z - \omega t
  \right)}
  \delta\hat{U}^R
\end{align}
and substituting into the linearized equations reduces the equations to
\begin{align}
\label{eq:dimeulerreduced}
  \ii\left( - \omega I
            + k_x \bar{A}
            + k_y \bar{B}
            + k_z \bar{C}
  \right)
  \delta\hat{U}^R &= 0
\end{align}
which has nontrivial solutions only if the so-called dispersion relation
\begin{align}
  \det \left( - \omega I
              + k_x \bar{A}
              + k_y \bar{B}
              + k_z \bar{C}
       \right) &= 0
\end{align}
holds.  Defining $\lambda_x = k_x/\omega$, $\lambda_y = k_y/\omega$, and
$\lambda_z/\omega$, the dispersion relation may be equivalently expressed as
\begin{align}
\label{eq:dimeulerdisp}
  \det \left( - I
              + \lambda_x \bar{A}
              + \lambda_y \bar{B}
              + \lambda_z \bar{C}
       \right) &= 0
  .
\end{align}
Assuming $\bar{A}$ is invertible and applying
$-\left(\ii\omega{}\bar{A}\right)^{-1}$ to
equation~\eqref{eq:dimeulerreduced}, one finds an eigenvalue problem in
$\lambda_x$
\begin{align}
  \left(   \bar{A}^{-1}
         - \lambda_x I
         - \lambda_y \bar{A}^{-1} \bar{B}
         - \lambda_z \bar{A}^{-1} \bar{C}
  \right) \delta\hat{U}^R = 0
\end{align}
where $\delta\hat{U}^R$ is the eigenvector and a solution to the right null
space problem.  The signs of the associated eigenvalues, determined using the
magnitude of $\bar{u}$ relative to $\bar{a}$, are required to determine how
many characteristics are entering or exiting through the boundary.  The left
null space problem
\begin{align}
\label{eq:dimeulereigenprob}
  V^{L}
  \left(   \bar{A}^{-1}
         - \lambda_x I
         - \lambda_y \bar{A}^{-1} \bar{B}
         - \lambda_z \bar{A}^{-1} \bar{C}
  \right) &= 0
\end{align}
naturally gives rise to the associated left null vector $V^L$.  Giles,
following Engquist and Majda, used several orthogonality properties to build
the exact, nonreflecting boundary conditions
\begin{align}
\label{eq:dimeulerexact}
  V_n^L \delta{}U &= 0
\end{align}
for each $V_n^L = V^L\!\left(k_{x_n}\right)$ corresponding to \textit{either
incoming or outgoing waves}.  This exact condition is approximated using a
Taylor series in $\lambda_y$ and $\lambda_z$ for reasons of computational
tractability.  Truncating the series is equivalent to assuming waves have a
small angle of incidence to the boundary.  To first order,
\begin{align}
  \left.V_n^L\right|_{\lambda_y,\lambda_z=0}
  \delta{}U
  +
  \lambda_y
  \left.\frac{dV_n^L}{d\lambda_y}\right|_{\lambda_y,\lambda_z=0}
  \delta{}U
  +
  \lambda_z
  \left.\frac{dV_n^L}{d\lambda_z}\right|_{\lambda_y,\lambda_z=0}
  \delta{}U
  &\approx 0
  .
\end{align}
As noted by Engquist and Majda~\cite{Engquist1977Absorbing} and later expounded
upon by Trefethen and Halpern~\cite{Trefethen1986Wellposedness}, only
particular higher-order series truncations of this form lead to well-posedness.
Moreover, the straightforward application of even this first order
approximation requires either ad~hoc~\cite{Giles1988Nonreflecting,Medida2007}
or systematic~\cite{Rowley2000Discretely} modification to produce well-behaved
inflow constraints.  Multiplying through by $-\ii\omega$, Fourier transforming
in both time and space, and using that $\bar{U}$ and therefore $V_n^L$ are both
steady and uniform yields
\begin{align}
\label{eq:dimeulerapprox}
  \frac{\partial}{\partial{}t}
  V^L
  \delta{}U
  &\approx
  \frac{dV^L}{d\lambda_y}
  \frac{\partial}{\partial{}y}\delta{}U
  +
  \frac{dV^L}{d\lambda_z}
  \frac{\partial}{\partial{}z}\delta{}U
\end{align}
where the cumbersome $\lambda_y,\lambda_z=0$ and subscript $n$ are herein
and henceforth suppressed.  Inserting ${V^L}^{-1} V^L$,
\begin{align}
  \frac{\partial}{\partial{}t}
  V^L
  \delta{}U
  &\approx
  \frac{dV^L}{d\lambda_y}
  {V^L}^{-1}
  \frac{\partial}{\partial{}y}
  V^L
  \delta{}U
  +
  \frac{dV^L}{d\lambda_z}
  {V^L}^{-1}
  \frac{\partial}{\partial{}z}
  V^L
  \delta{}U
.
\end{align}
Defining characteristic variables using the action of $V^L$, \textit{viz.}
\begin{align}
  \delta{}C &= V^L \delta{}U
  ,
\end{align}
allows writing a more compact form
\begin{align}
\label{eq:dimeulerapproxchar}
  \frac{\partial}{\partial{}t}
  \delta{}C
  &\approx
  B^G
  \frac{\partial}{\partial{}y}
  \delta{}C
  +
  C^G
  \frac{\partial}{\partial{}z}
  \delta{}C
\end{align}
employing the notation
\begin{align}
\label{eq:dimeulerapproxBG}
  B^G
&=
  \frac{dV^L}{d\lambda_y}
  {V^L}^{-1}
&
  C^G
&=
  \frac{dV^L}{d\lambda_z}
  {V^L}^{-1}
\end{align}
where the superscript $G$ is meant to suggest ``Giles''.  This last statement
intermixes the constraints for inflow and outflow conditions.  While the
submatrices vary depending on the choices made to express $V^L$ and the
magnitude of $\bar{u}$ relative to $\bar{a}$, in complete generality the matrix
equation~\eqref{eq:dimeulerapproxchar} may be partitioned as
\begin{align}
\label{eq:dimeulerapproxcharpart}
\renewcommand*{\arraystretch}{1.2}
  \frac{\partial}{\partial{}t}
  \left[\begin{array}{c}\delta{}C_{I}\\\hline\delta{}C_{O}\end{array}\right]
  &\approx
\renewcommand*{\arraystretch}{1.2}
  \left[\begin{array}{c} B^G_{I} \\ \hline B^G_{O} \end{array}\right]
  \frac{\partial}{\partial{}y}
  \delta{}C
  +
  \left[\begin{array}{c} C^G_{I} \\ \hline C^G_{O} \end{array}\right]
  \frac{\partial}{\partial{}z}
  \delta{}C
\end{align}
where subscripts $I$ and $O$ denote inflow or outflow condition submatrices,
respectively.  Only the conditions arising from one set of submatrices are to
be applied at any location.  The ``other'' submatrices are defined to be zero
depending on the use case at hand:
\begin{align}
  B^G &=
\renewcommand*{\arraystretch}{1.2}
  \left[\begin{array}{c} B^G_{I} \\ \hline 0       \end{array}\right]
  &
  C^G &=
\renewcommand*{\arraystretch}{1.2}
  \left[\begin{array}{c} C^G_{I} \\ \hline 0       \end{array}\right]
  &
  &\left(\text{inflow boundary}\right)
\\
  B^G &=
\renewcommand*{\arraystretch}{1.2}
  \left[\begin{array}{c} 0       \\ \hline B^G_{O} \end{array}\right]
  &
  C^G &=
\renewcommand*{\arraystretch}{1.2}
  \left[\begin{array}{c} 0       \\ \hline C^G_{O} \end{array}\right]
  &
  &\left(\text{outflow boundary}\right)
\end{align}
Notationally, it will later be convenient to have a projection
\begin{align}
\label{eq:PG}
\renewcommand*{\arraystretch}{1.2}
  P^G
  &=
  \left[\begin{array}{c} P^G_{I} \\ \hline P^G_{O} \end{array}\right]
\end{align}
defined to preserve only the characteristic(s) constrained by a boundary
condition with the ``other'' submatrix again being zero.  This definition
implies
\begin{align}
\label{eq:PGidempotence}
  P^G B^G &= B^G
  &
  P^G C^G &= C^G
.
\end{align}
Notice $I - P^G$ recovers the characteristics \emph{not} constrained by the
boundary condition For some $\bar{U}$ possessing an agreed upon relationship
between $\bar{u}$ and $\bar{a}$, specifying $V^L$ and these partitioned
submatrices concretely states a Giles-like nonreflecting $x$ boundary condition
for the Euler equations.

In section 5.8 of his thesis, Medida presents two such concrete nonreflecting
boundary condition specifications for subsonic inflows and outflows where $0 <
\bar{u} < \bar{a}$.  Medida's equations~(5.78) and~(5.79) specify the
transformations to and from characteristic variables:
\begin{align}
\label{eq:dimeulerapproxmedidaproj}
  V^L &= \left[\begin{array}{ccccc}
    -\bar{a}^2 & 0                   & 0                  & 0                  & 1 \\ \noalign{\smallskip}
    0          & 0                   & \bar{\rho} \bar{a} & 0                  & 0 \\ \noalign{\smallskip}
    0          & 0                   & 0                  & \bar{\rho} \bar{a} & 0 \\ \noalign{\smallskip}
    0          & \bar{\rho} \bar{a}  & 0                  & 0                  & 1 \\ \noalign{\smallskip}
    0          & -\bar{\rho} \bar{a} & 0                  & 0                  & 1 \\ \noalign{\smallskip}
  \end{array}\right]
&
  {V^L}^{-1} &= \left[\begin{array}{ccccc}
    -\frac{1}{\bar{a}^2} & 0                            & 0                            & \frac{1}{2\bar{a}^2}          & \frac{1}{2\bar{a}^2}            \\ \noalign{\smallskip}
    0                    & 0                            & 0                            & \frac{1}{2\bar{\rho} \bar{a}} & - \frac{1}{2\bar{\rho} \bar{a}} \\ \noalign{\smallskip}
    0                    & \frac{1}{\bar{\rho} \bar{a}} & 0                            & 0                             & 0                               \\ \noalign{\smallskip}
    0                    & 0                            & \frac{1}{\bar{\rho} \bar{a}} & 0                             & 0                               \\ \noalign{\smallskip}
    0                    & 0                            & 0                            & \frac{1}{2}                   & \frac{1}{2}                     \\ \noalign{\smallskip}
  \end{array}\right]
\end{align}
Direct computation shows
\[
  \det V^L = -2\bar{\rho}^3\bar{a}^5
\]
and so $V^L$ is always nonsingular for a realizable reference state.  Medida's
equations~(5.82) and~(5.83) provide one pair of inflow and outflow conditions
for which reflection coefficients were not reported:
\begin{align}
\label{eq:dimeulerapproxmedidaopt1}
  B^G_1 &= \left[\begin{array}{ccccc}
    0 & 0                         & 0       & 0                           & 0                         \\ \noalign{\smallskip}
    0 & \bar{v}                   & 0       & \frac{\bar{a} + \bar{u}}{2} & \frac{\bar{a}-\bar{u}}{2} \\ \noalign{\smallskip}
    0 & 0                         & \bar{v} & 0                           & 0                         \\ \noalign{\smallskip}
    0 & \frac{\bar{a}-\bar{u}}{2} & 0       & \bar{v}                     & 0                         \\ \noalign{\smallskip} \hline
    0 & \bar{u}                   & 0       & 0                           & \bar{v}                   \\ \noalign{\smallskip}
  \end{array}\right]
&
  C^G_1 &= \left[\begin{array}{ccccc}
    0 & 0       & 0                         & 0                         & 0                         \\ \noalign{\smallskip}
    0 & \bar{w} & 0                         & 0                         & 0                         \\ \noalign{\smallskip}
    0 & 0       & \bar{w}                   & \frac{\bar{a}+\bar{u}}{2} & \frac{\bar{a}-\bar{u}}{2} \\ \noalign{\smallskip}
    0 & 0       & \frac{\bar{a}-\bar{u}}{2} & \bar{w}                   & 0                         \\ \noalign{\smallskip} \hline
    0 & 0       & \bar{u}                   & 0                         & \bar{w}                   \\ \noalign{\smallskip}
  \end{array}\right]
\end{align}
where partitioning has been used to delineate inflow versus outflow submatrices
within $B^G$ and $C^G$.  The characteristic-preserving projection operator is
\begin{align}
\label{eq:PGmedida}
  P^G
  &= \left[\begin{array}{ccccc}
    1 & 0 & 0 & 0 & 0 \\
    0 & 1 & 0 & 0 & 0 \\
    0 & 0 & 1 & 0 & 0 \\
    0 & 0 & 0 & 1 & 0 \\ \hline
    0 & 0 & 0 & 0 & 1 \\
  \end{array}\right]
.
\end{align}
We assume, but have not verified, analysis like that presented in section~3.7.4
of Giles~\cite{Giles1988Nonreflecting} extends to Medida's analogous results.
At the inflow, the outgoing pressure wave produces no reflected entropy or
vorticity waves and generates a fourth-order pressure reflection.  At the
outflow, the outgoing entropy and vorticity waves produce no reflections while
the outgoing pressure wave produces a second order reflection.  In contrast,
Medida's equation~(5.84), which arises from modifying the already-well-posed
outflow condition to match the well-posed inflow condition, provides different
outflow submatrices:
\begin{align}
\label{eq:dimeulerapproxmedidaopt2}
  B^G_2 &= \left[\begin{array}{ccccc}
    0 & 0                         & 0       & 0                           & 0                         \\ \noalign{\smallskip}
    0 & \bar{v}                   & 0       & \frac{\bar{a} + \bar{u}}{2} & \frac{\bar{a}-\bar{u}}{2} \\ \noalign{\smallskip}
    0 & 0                         & \bar{v} & 0                           & 0                         \\ \noalign{\smallskip}
    0 & \frac{\bar{a}-\bar{u}}{2} & 0       & \bar{v}                     & 0                         \\ \noalign{\smallskip} \hline \noalign{\smallskip}
    0 & \frac{\bar{a}+\bar{u}}{2} & 0       & 0                           & \bar{v}                   \\ \noalign{\smallskip}
  \end{array}\right]
&
  C^G_2 &= \left[\begin{array}{ccccc}
    0 & 0       & 0                         & 0                         & 0                         \\ \noalign{\smallskip}
    0 & \bar{w} & 0                         & 0                         & 0                         \\ \noalign{\smallskip}
    0 & 0       & \bar{w}                   & \frac{\bar{a}+\bar{u}}{2} & \frac{\bar{a}-\bar{u}}{2} \\ \noalign{\smallskip}
    0 & 0       & \frac{\bar{a}-\bar{u}}{2} & \bar{w}                   & 0                         \\ \noalign{\smallskip} \hline \noalign{\smallskip}
    0 & 0       & \frac{\bar{a}+\bar{u}}{2} & 0                         & \bar{w}                   \\ \noalign{\smallskip}
  \end{array}\right]
\end{align}
At the outflow, these matrices cause outgoing pressure waves to produce fourth
and first order reflections, respectively.  As Giles
notes~\cite{Giles1990Nonreflecting}, this second outflow condition is
preferable only when one knows there will be no outgoing vorticity wave.  This
should be the case for some of our problems of interest.

In section~4.3 of his thesis, Guarini~\cite{Guarini1998} proved the linear
structure of the Euler equations admits a straightforward translation of Giles'
boundary conditions to another set of state variables $V$ with steady, uniform
reference state $\bar{V}$ and therefore perturbations
\[
  \delta{}V = V - \bar{V}.
\]
The corresponding coordinate transformation Jacobian matrix is
\[
   S = \frac{\partial{}U}{\partial{}V}
   .
\]
In this new setting, Guarini rewrote the exact nonreflecting
conditions~\eqref{eq:dimeulerexact} as
\begin{align}
  \left(V^L S\right) \delta{}V &= 0
\end{align}
which causes the approximate condition~\eqref{eq:dimeulerapprox} to become
\begin{align}
\label{eq:dimeulertransform}
  V^L
  S
  \frac{\partial}{\partial{}t}
  \delta{}V
  &\approx
  \frac{dV^L}{d\lambda_y}
  S
  \frac{\partial}{\partial{}y}
  \delta{}V
  +
  \frac{dV^L}{d\lambda_z}
  S
  \frac{\partial}{\partial{}z}
  \delta{}V
  .
\end{align}
Using notation from the compact representation~\eqref{eq:dimeulerapproxBG},
\begin{align}
\label{eq:dimeulertransformcharnot}
  V^L S
  \frac{\partial}{\partial{}t}
  \delta{}V
  &\approx
  B^G V^L S
  \frac{\partial}{\partial{}y}
  \delta{}V
  +
  C^G V^L S
  \frac{\partial}{\partial{}z}
  \delta{}V
\end{align}
is the simplest form for applying Medida's $x$ boundary condition matrices to
alternative state variables.

The particular coordinate transformation we require maps the nondimensional
primitive state $U^*$ satisfying requirements~\eqref{eq:eulerprimnondim} to the
conserved state $V^*$ nondimensionalized per sections~\ref{sec:nondimrefq}
and~\ref{nondim_equations}:
\begin{align}
\label{eq:eulerconsnondim}
V^{*}
&= \left\{
  \frac{\rho}{\rho_0},
  \frac{\rho u}{\rho_0 u_0},
  \frac{\rho v}{\rho_0 u_0},
  \frac{\rho w}{\rho_0 u_0},
  \frac{\rho E}{\rho_0 a_0^2}
\right\}
= \left\{
    \rho^{*},
  \,\rho^{*} u^{*},
  \,\rho^{*} v^{*},
  \,\rho^{*} w^{*},
  \,\rho^{*} E^{*}
\right\}
&
a^{*} &= \frac{a}{a_0}
&
\Mach &= \frac{u_0}{a_0}
&
t_0 &= \frac{l_0}{u_0}
\end{align}
Observing several relationships between $U^*$ and $V^*$ with care to
distinguish between $u_0$ and $a_0$
\begin{align}
  \frac{\rho}{\rho_0} &= \rho^*
&
  \frac{u}{u_0} &=
  \frac{\frac{\rho{}u}{\rho_0u_0}}{\frac{\rho}{\rho_0}}
  =
  \frac{\rho^{*}u^{*}}{\rho^{*}}
&
  \frac{v}{u_0} &= \frac{\rho^{*}v^*}{\rho^*}
&
  \frac{w}{u_0} &= \frac{\rho^{*}w^*}{\rho^*}
&
 \frac{p}{\rho_0 u_0^2} &= \frac{\rho a^2}{\gamma \rho_0 u_0^2}
                         = \frac{\rho^{*} {a^{*}}^2}{\gamma\Mach^2}
\end{align}
\begin{align}
 \frac{p}{\rho_0 u_0^2}
&=
   \frac{\gamma-1}{\rho_0 u_0^2} \rho E
 + \frac{1-\gamma}{2 \rho_0 u_0^2 \rho}\left(
           \left(\rho{}u\right)^2
         + \left(\rho{}v\right)^2
         + \left(\rho{}w\right)^2
   \right)
=
   \frac{\gamma-1}{\Mach^2} \rho^{*} E^{*}
 + \frac{1-\gamma}{2 \rho^{*}}\left(
           \left(\rho^{*}u^{*}\right)^2
         + \left(\rho^{*}v^{*}\right)^2
         + \left(\rho^{*}w^{*}\right)^2
   \right)
\end{align}
aids computing the Jacobian matrix evaluated at some $\bar{V}^{*}$,
\begin{align}
S &= \left[\begin{array}{ccccc}
    1      % 11
  & 0      % 12
  & 0      % 13
  & 0      % 14
  & 0      % 15
  \\ \noalign{\medskip}
    - \frac{\bar{u}^*}{\bar{\rho}^*} % 21
  & \frac{1}{\bar{\rho}^*}           % 22
  & 0                                % 23
  & 0                                % 24
  & 0                                % 25
  \\ \noalign{\medskip}
    - \frac{\bar{v}^*}{\bar{\rho}^*} % 31
  & 0                                % 32
  & \frac{1}{\bar{\rho}^*}           % 33
  & 0                                % 34
  & 0                                % 35
  \\ \noalign{\medskip}
    - \frac{\bar{w}^*}{\bar{\rho}^*} % 41
  & 0                                % 42
  & 0                                % 43
  & \frac{1}{\bar{\rho}^*}           % 44
  & 0                                % 45
  \\ \noalign{\medskip}
    \frac{{\bar{a}^*}^2}{\gamma\Mach^2} % 51
  & \left(1 - \gamma\right) \bar{u}^*   % 52
  & \left(1 - \gamma\right) \bar{v}^*   % 53
  & \left(1 - \gamma\right) \bar{w}^*   % 54
  & \frac{\gamma-1}{\Mach^2}            % 55
  \\ \noalign{\medskip}
\end{array}\right]
.
\end{align}
%Examining $\rho = \rho{}u/u$ closely confirms the slightly
%unintuitive zeros in the first row of $S$ because
%\[
%    \frac{\partial}{\partial\rho{}u} \rho
%  = \frac{\partial}{\partial\rho{}u} \frac{\rho{}u}{u\!\left(\rho,\rho{}u\right)}
%  = \frac{1}{u\!\left(\rho,\rho{}u\right)}
%  - \frac{\rho{}u}{u^2\!\left(\rho,\rho{}u\right)}
%    \frac{\partial}{\partial\rho{}u}u\!\left(\rho,\rho{}u\right)
%  = \frac{1}{u}
%  - \frac{\rho{}u}{u^2}\frac{1}{\rho}
%  = 0
%.
%\]
As expected, the transformation is nonsingular for realizable fields because
\[
  \det S = \frac{\gamma-1}{\Mach^2 {\bar{\rho}^{*}}^3}.
\]
The inverse is
\begin{align}
S^{-1} &= \left[\begin{array}{ccccc}
  1                                                                                          & 0                              & 0                              & 0                              & 0                        \\
  \bar{u}^{*}                                                                                & \bar{\rho}^*                   & 0                              & 0                              & 0                        \\
  \bar{v}^{*}                                                                                & 0                              & \bar{\rho}^*                   & 0                              & 0                        \\
  \bar{w}^{*}                                                                                & 0                              & 0                              & \bar{\rho}^*                   & 0                        \\
  \Mach^2\left(\bar{u}^2+\bar{v}^2+\bar{w}^2\right)+\frac{{\bar{a}^{*}}^2}{\gamma(1-\gamma)} & \Mach^2 \bar{\rho}^* \bar{u}^* & \Mach^2 \bar{\rho}^* \bar{v}^* & \Mach^2 \bar{\rho}^* \bar{w}^* & \frac{\Mach^2}{\gamma-1}
\end{array}\right]
.
\end{align}
Medida's matrices $V^L$, $B^G$, and $C^G$ derived for $U$ remain valid for
nondimensional $U^*$ possessing sound speed $a/u_0$.  When reusing these
matrices for $V^*$ every sound speed must be scaled by $1/\Mach$ because
\[
  \frac{\bar{a}}{u_0} = \frac{a_0 \bar{a}^*}{u_0} = \frac{\bar{a}^*}{\Mach}.
\]

Thus far nonreflecting $x$ boundary conditions in physical space have been
presented.  Suzerain requires rotating these results to handle nonreflecting
$y$ boundaries followed by transforming the constraints into coefficient space.
Defining
\begin{align}
  x' &= z &
  y' &= x &
  z' &= y
\intertext{
induces the following relationships:
}
  u &= v' &
  v &= w' &
  w &= u'
\\
  \frac{\partial}{\partial{}x} &= \frac{\partial}{\partial{}y'} &
  \frac{\partial}{\partial{}y} &= \frac{\partial}{\partial{}z'} &
  \frac{\partial}{\partial{}z} &= \frac{\partial}{\partial{}x'}
\end{align}
The perturbed state vector entries may be reordered more conventionally by
defining $R^Y$ and $\delta{}V'$ per
\begin{align}
  \delta{}V
  &= \begin{bmatrix}
       \delta\rho     \\
       \delta\rho{}v' \\
       \delta\rho{}w' \\
       \delta\rho{}u' \\
       \delta\rho{}E  \\
     \end{bmatrix}
  = R^Y \delta{}V'
  = \begin{bmatrix}
      1 & 0 & 0 & 0 & 0 \\
      0 & 0 & 1 & 0 & 0 \\
      0 & 0 & 0 & 1 & 0 \\
      0 & 1 & 0 & 0 & 0 \\
      0 & 0 & 0 & 0 & 1 \\
    \end{bmatrix}
    \begin{bmatrix}
      \delta\rho     \\
      \delta\rho{}u' \\
      \delta\rho{}v' \\
      \delta\rho{}w' \\
      \delta\rho{}E  \\
    \end{bmatrix}
  .
\end{align}
Substituting these details into equation~\eqref{eq:dimeulertransformcharnot}
produces the desired nonreflecting $y$ boundary condition,
\[
\label{eq:dimeulertransformcharnotYwieldy}
\left.\left[
  V^L S
\right]\right|_{\bar{u}=\bar{v}', \bar{v}=\bar{w}', \bar{w}=\bar{u}'}
  R^Y
  \frac{\partial}{\partial{}t}
  \delta{}V'
\approx
\left.\left[
  C^G V^L S
\right]\right|_{\bar{u}=\bar{v}', \bar{v}=\bar{w}', \bar{w}=\bar{u}'}
  R^Y
  \frac{\partial}{\partial{}x'}
  \delta{}V'
  +
\left.\left[
  B^G V^L S
\right]\right|_{\bar{u}=\bar{v}', \bar{v}=\bar{w}', \bar{w}=\bar{u}'}
  R^Y
  \frac{\partial}{\partial{}z'}
  \delta{}V'
.
\]
Suppressing the primes and matrix evaluation details makes the above result
much more wieldy,
\[
\label{eq:dimeulertransformcharnotYphys}
  \left[V^L S\right]
  R^Y
  \frac{\partial}{\partial{}t}
  \delta{}V
\approx
  \left[C^G V^L S\right]
  R^Y
  \frac{\partial}{\partial{}x}
  \delta{}V
  +
  \left[B^G V^L S\right]
  R^Y
  \frac{\partial}{\partial{}z}
  \delta{}V
.
\]
Transforming to Fourier space gives a linear condition almost suitable for
implicit advance per section~\ref{sec:timediscretization},
\begin{align}
\label{eq:dimeulertransformcharnotYwave}
  \left[V^L S\right]
  R^Y
  \frac{\partial}{\partial{}t}
  \hat{V}
&\approx
  \ii k_x
  \left[C^G V^L S\right]
  R^Y
  \hat{V}
  +
  \ii k_z
  \left[B^G V^L S\right]
  R^Y
  \hat{V}
.
\end{align}
This last result holds identically for boundary collocation points and boundary
coefficients because the boundary value for a B-spline basis expansion is
nothing but the boundary coefficient.  Notice that when $k_x=k_z=0$ the
relevant characteristics within the mean state are held constant in time.

While constraint~\eqref{eq:dimeulertransformcharnotYwave} fixes the
characteristics arriving from beyond the boundary, it does not evolve the
remaining ones determined by the simulation domain.  Returning to the time
discretization of section~\ref{sec:timediscretization},
equation~\eqref{eq:timediscretization} evolves coefficients more or less like
\begin{align}
\label{eq:coeffspaceevolve}
  \frac{\partial}{\partial{}t} \hat{V} &= L \hat{V} + N(\hat{V})
  .
\end{align}
At the nonreflecting $y$ boundary, projecting the evolution into characteristic
space yields
\begin{align}
  \left[V^L S\right] R^Y
  \frac{\partial}{\partial{}t} \hat{V} &=
  \left[V^L S\right] R^Y
  \left(
    L \hat{V}
    +
    N(\hat{V})
  \right)
  .
\end{align}
Updating only the unconstrained characteristics using $P^G$ as defined
\vpageref{eq:PG},
\begin{align}
  \left(I - P^G\right)
  \left[V^L S\right] R^Y
  \frac{\partial}{\partial{}t} \hat{V}
&=
  \left(I - P^G\right)
  \left[V^L S\right] R^Y
  \left(
    L \hat{V}
    +
    N(\hat{V})
  \right)
  .
\end{align}
Applying $P^G$ to both sides of
equation~\eqref{eq:dimeulertransformcharnotYwave}, simplifying
using~\eqref{eq:PGidempotence}, and adding that to the above equality,
\begin{align}
  \left[V^L S\right]
  R^Y
  \frac{\partial}{\partial{}t}
  \hat{V}
&\approx
  \ii k_x
  \left[C^G V^L S\right]
  R^Y
  \hat{V}
  +
  \ii k_z
  \left[B^G V^L S\right]
  R^Y
  \hat{V}
\\
&{}+
  \left(I - P^G\right)
  \left[V^L S\right] R^Y
  \left(
    L \hat{V}
    +
    N(\hat{V})
  \right)
.
\end{align}
Moving the nonsingular characteristic projection to the right hand side and
collecting linear terms,
\begin{align}
\label{eq:dimeulertransformevolve}
  \frac{\partial}{\partial{}t}
  \hat{V}
&\approx
\overbrace{
  {R^Y}^{-1}
  \left[V^L S\right]^{-1}
  \left(
    \left( \ii k_x \left[C^G\right] + \ii k_z \left[B^G\right] \right)
    \left[V^L S\right] R^Y
    +
    \left(I-P^G\right)
    \left[V^L S\right] R^Y
    L
  \right)
}^{L^G}
  \hat{V}
\\
&{}+
\underbrace{
  {R^Y}^{-1}
  \left[V^L S\right]^{-1}
  \left(I - P^G\right)
  \left[V^L S\right] R^Y
  N(\hat{V})
}_{N^G\left(\hat{V}\right)}
.
\end{align}
Further auxiliary definitions could improve the previous equation's brevity but
they would obfuscate the physics.  Evidently, Giles' conditions can be
shoehorned into section~\ref{sec:imextreatment}'s hybrid implicit/explicit
Runge--Kutta framework by modifying the action of any existing global operators
$L$ and $N$ to obtain the boundary-specific $L^G$ and $N^G$ behavior specified
by equation~\eqref{eq:dimeulertransformevolve}.

While the derivation of this section's results has assumed a steady $\bar{V}$,
in practice $\bar{V}$ is not constant.  For example, the spatially-averaged,
instantaneous wall-normal outflow velocity generated by a homogenized boundary
layer problem may be unknown and will vary in time.  The mean state at the
beginning of the first Runge--Kutta substep should be taken as the ``steady''
$\bar{V}$ for the direction of the time step.  The mean state should be evolved
according to equation~\eqref{eq:dimeulertransformevolve} modified to account
for any auxiliary forcing constraints (e.g. holding freestream streamwise
momentum constant at each substep).

Finally, Guarini~\cite{Guarini1998} mentions that Poinsot and
Lele~\cite{Poinsot1992Boundary} suggest two additional viscous conditions be
supplied when conditions like these are applied to the Navier--Stokes
equations.  These are currently omitted but that choice may need to be
revisited.  Finally, Guarini rotated the nonreflecting boundary to match the
nonorthogonal coordinate system used for his spatially-homogenized boundary
layer.  Personal communications with Victor Topalian regarding his
temporally-homogenized work suggests reorienting the nonreflecting coordinate
system is unnecessary.  However, modest grid stretching near the freestream and
other ``solution conditioning'' tools like low pass filtering have been
employed in his work.  This choice, too, might need to be reviewed.

\subsubsection{Isothermal, transpiring wall boundary condition}

TODO  Reuse nonreflecting freestream conditions with pegged inflow momentum and
total energy?

\subsection{Imposing a favorable streamwise pressure gradient}

TODO

\section{In support of Favre-averaged Navier--Stokes modeling}
\label{sec:supportFANS}

The Favre-averaged Navier--Stokes (FANS) equations are often used to estimate
the mean effects of turbulence.  The unclosed FANS equations require modeling
approximations to be solvable.  Statistics gathered from Suzerain's solution of
the Navier--Stokes equations may be used to inform the development and
application of FANS closures.  Extensive background may be found in
books by Chassaing et.~al.~\cite{Chassaing2010} or Smits and
Dussauge~\cite{SmitsDussauge2005}.

The material in this section borrows liberally (and often literally) from work
by Todd Oliver~\cite{OliverFANSModels2011}.  It departs from that particular
document in that it employs Suzerain's constitutive relationships, avoids
introducing customary assumptions about the relative importance of unclosed
terms, accounts for forcing, and nondimensionalizes the results.


\subsection{Reynolds- and Favre-averages}
\label{sec:averaging}

The Reynolds average is simply the usual mean of a random variable.  Consider a
generic flow variable $q$.  The value, $q(x, y, z, t)$, of this variable at a
particular point in space, $(x, y, z)$, and time, $t$, is a random variable.
Assuming that the probability density function for $q(x, y, z, t)$ is given by
$\pi_q(V; x, y, z, t)$, the Reynolds average is defined by
%
\begin{equation}
\label{eqn:reynoldsAvg}
\bar{q}(x, y, z, t) \equiv \int V \pi_q(V; x, y, z, t) \,\mathrm{d} V.
\end{equation}
%
The Favre average is defined as the density-weighted average.  Thus,
denoting the fluid density by $\rho(x,y,z, t)$, the Favre average of
$q(x,y,z, t)$ is
%
\begin{equation*}
\tilde{q}(x,y,z, t) \equiv \frac{ \overline{\rho q}(x,y,z, t) }{ \bar{\rho}(x,y,z, t) }.
\end{equation*}
%
For the remainder of this section to make sense mathematically, it is
assumed that both the Reynolds and Favre averages are well-defined for
any required flow variable, $q$.  That is, the integral on the
right-hand side of (\ref{eqn:reynoldsAvg}) exists whenever required,
and the Reynolds-averaged density, $\bar{\rho}$, is positive
everywhere.

In the following, the flow variables will be decomposed into mean and
fluctuating parts.  Specifically, the fluctuations about the
mean---denoted by $(\cdot)'$ and $(\cdot)''$ for the Reynolds and
Favre averages, respectively---are defined by the following
relationships:
%
\begin{align*}
q' &\equiv q - \bar{q}, \\
q'' &\equiv q - \tilde{q}.
\end{align*}
%
Using the linearity of the Reynolds average and the fact that
$\bar{q}$ and $\tilde{q}$ are deterministic, not random, variables, it
is straightforward to see that
%
\begin{gather*}
\overline{q'} = \overline{q - \bar{q}} = \bar{q} - \bar{q} =  0, \\
\widetilde{q''} = \widetilde{q - \tilde{q}} = \tilde{q} - \tilde{q} = 0.
\end{gather*}
%
Furthermore,
%
\begin{equation*}
\overline{\rho q''} = \bar{\rho} \widetilde{q''} = 0.
\end{equation*}
%
However, in general,
%
\begin{equation*}
\overline{q''} = \overline{q - \tilde{q}} = \bar{q} - \tilde{q} \neq 0.
\end{equation*}
%

Wherever necessary, realizations of random fields of flow quantities
are assumed to be differentiable in both time and space so that Reynolds
averaging and differentiation commute.  For example,
%
\begin{equation*}
\overline{ \nabla{}u } = \nabla\bar{u}.
\end{equation*}
%
This commutativity is used to develop the FANS equations.  In contrast, Favre
averaging and differentiation do not, in general, commute:
\begin{align}
  \rho \nabla q &= \rho \nabla q
\\
   \rho \widetilde{\nabla{}q} + \rho \left(\nabla{}q\right)''
&=
   \rho \nabla \tilde{q} + \rho \nabla{}q''
\\
     \bar{\rho} \widetilde{\nabla{}q}
&=
     \bar{\rho} \nabla{\tilde{q}}
   + \overline{\rho \nabla{}q''}
\\
&=
     \bar{\rho} \nabla{\tilde{q}}
   - \overline{q''\nabla\rho}
\end{align}
We have adopted the common convention that taking Favre fluctuations
$\left(\cdot\right)''$ has higher precedence than $\nabla\left(\cdot\right)$.
Rearranging to better examine the difference between $\widetilde{\nabla{}q}$
and $\nabla\tilde{q}$ in terms of mean quantities,
\begin{align}
  \label{eq:favremeancommute}
  \widetilde{\nabla{}q}
  -
  \nabla{\tilde{q}}
&=
  \widetilde{\nabla{}q''}
= - \frac{{\overline{q''\nabla\rho}}}{\bar{\rho}}
= \frac{\tilde{q}\nabla\bar{\rho}}{\bar{\rho}}
  - \frac{\overline{q\nabla\rho}}{\bar{\rho}}
.
\end{align}
This lack of commutativity not problematic as it is not required to derive the
FANS equations.  It does, however, slightly complicate the mean constitutive
relationships.  The fluctuating gradient and the gradient of the fluctuations
differ according to
\begin{align}
  \label{eq:favrefluctcommute}
  \left(\nabla{}q\right)'' - \nabla{}q'' &= - \widetilde{\nabla{}q''}
.
\end{align}
In some circumstances, the difference between quantities written using a
fluctuating gradient and the gradient of the fluctuations can vanish.
One useful example is
\begin{align}
  \label{eq:favrefluctexample}
\widetilde{f''\left(\nabla{}g\right)''}
&=
\overline{\rho{}f''\left(\nabla{}g\right)''}
=
\overline{\rho{}f''\left(\nabla{}g'' - \widetilde{\nabla{}g''}\right)}
=
\overline{\rho{}f''\nabla{}g''}
- \overline{\rho{}f''}\widetilde{\nabla{}g''}
=
\widetilde{f''\nabla{}g''}
.
\end{align}

\subsection{The dimensional Favre-averaged Navier--Stokes equations}

\subsubsection{Derivation}

Recall that equations~\eqref{eq:cons_mass}, \eqref{eq:cons_momentum},
and~\eqref{eq:cons_energy} may be written as
\begin{align}
    \frac{\partial}{\partial{}t}\rho
&=
  - \nabla\cdot\rho{}u
\\
    \frac{\partial{}}{\partial{}t}\rho{}u
&=
  - \nabla\cdot(u\otimes{}\rho{}u)
  - \nabla{}p + \nabla\cdot{}\tau + f
\\
    \frac{\partial}{\partial{}t} \rho{}E
&=
  - \nabla\cdot{}\rho{}Hu
  + \nabla\cdot{}\tau{}u
  - \nabla\cdot{}q_{s}
  + f\cdot{}u
  + q_b
\end{align}
where we have employed total enthalpy $H$ to reduce the number of terms in the
energy equation.

A lengthy algebraic procedure (detailed in section~2
of~\cite{OliverFANSModels2011}) produces exact equations governing the
evolution of mean conserved quantities $\bar{\rho}$, $\overline{\rho{}u}=
\bar{\rho}\tilde{u}$, and $\overline{\rho{}E} = \bar{\rho}\tilde{E}$:
\begin{subequations}\label{eq:unclosedfansequations}
\begin{align}
    \frac{\partial}{\partial{}t}\bar{\rho}
 =
 &- \nabla\cdot\bar{\rho}\tilde{u}
\\
    \frac{\partial{}}{\partial{}t}\bar{\rho}\tilde{u}
 =
 &- \nabla\cdot(\tilde{u}\otimes\bar{\rho}\tilde{u})
  - \nabla{}\bar{p}
  + \nabla\cdot\left(
        \bar{\tau}
      - \bar{\rho}\widetilde{u''\otimes{}u''}
    \right)
  + \bar{f}
\\
  \frac{\partial}{\partial{}t} \bar{\rho}\tilde{E}
 =
 &- \nabla\cdot{}\bar{\rho}\tilde{H}\tilde{u}
  + \nabla\cdot\left(
        \left(
            \bar{\tau}
          - \bar{\rho} \widetilde{u''\otimes{}u''}
        \right) \tilde{u}
      - \frac{1}{2}\bar{\rho}\widetilde{{u''}^{2}u''}
      + \overline{\tau{}u''}
    \right)
\\
 &- \nabla\cdot\left(
        \bar{q}_s
      + \bar{\rho} \widetilde{h''u''}
    \right)
  + \bar{f}\cdot\tilde{u}
  + \overline{f\cdot{}u''}
  + \bar{q}_{b}
\end{align}
\end{subequations}
Several correlations impact the evolution of mean quantities: the Reynolds
stress $-\bar{\rho}\widetilde{u''\otimes{}u''}$, the Reynolds heat flux
$\bar{\rho} \widetilde{h''u''}$, turbulent transport
$-\frac{1}{2}\bar{\rho}\widetilde{{u''}^{2}u''}$, turbulent work
$\overline{\tau{}u''}$, and the forcing-velocity correlation
$\overline{f\cdot{}u''}$.  The Reynolds stress and heat flux augment the
viscous stress and heat flux, respectively.  The turbulent transport and work
terms represent transport of the turbulent kinetic energy density $k$, defined
below, and viscous stress work due to turbulent velocity fluctuations,
respectively.

We now average the perfect gas relations from section~\ref{sec:constitutive}.
The Reynolds average of~\eqref{eq:perfectgaseos} gives
\begin{align}
  \bar{p} &= R\overline{\rho{}T} = \bar{\rho}R\tilde{T}
\end{align}
while the Favre average of~\eqref{eq:perfectgasenthalpy} gives both
\begin{align}
 \tilde{H} &= \tilde{E} + R \tilde{T}
&
 \tilde{h} &= \frac{\gamma{}R\tilde{T}}{\gamma-1}.
\end{align}
The turbulent kinetic energy density
\begin{align}
  k &= \frac{1}{2}\widetilde{{u''}^2}
 \end{align}
arises from averaging the total energy given by
\eqref{eq:perfectgastotalenergy}:
\begin{align}
  \rho{} E
&=
  \frac{R}{\gamma-1} \rho{}T + \frac{1}{2}\rho{} u^{2}
\\
&=
  \frac{R}{\gamma-1} \rho{}\left( \tilde{T} + T'' \right)
+ \frac{1}{2}\rho{} \left( \tilde{u} + u'' \right)^2
\\
  \overline{\rho{}E}
&=
  \frac{R}{\gamma-1} \bar{\rho} \tilde{T}
+ \frac{1}{2}\bar{\rho} \tilde{u}^2
+ \frac{1}{2}\overline{\rho{}{u''}^2}
\\
  \tilde{E}
&=
  \frac{R}{\gamma-1} \tilde{T}
+ \frac{1}{2} \tilde{u}^2
+ k
\end{align}

An exact equation may be derived for the evolution of $k$ (details in section~5
of~\cite{OliverFANSModels2011})
\begin{align}
\label{eq:fanstke1}
    \frac{\partial{}}{\partial{}t}\bar{\rho}k
 =
 &- \nabla\cdot\bar{\rho}k\tilde{u}
  - \bar{\rho} \widetilde{u''\otimes{}u''} : \nabla\tilde{u}
  - \bar{\rho} \epsilon
  + \nabla\cdot\left(
        -\frac{1}{2}\bar{\rho}\widetilde{{u''}^{2}u''}
      + \overline{\tau{}u''}
    \right)
\\
 &- \overline{u''}\cdot\nabla\bar{p}
  - \nabla\cdot\overline{p' u''}
  + \overline{p' \nabla\cdot{}u''}
  + \overline{f\cdot{}u''}
\end{align}
where $A:B$ denotes $\trace \left(\trans{A} B\right)$. The dissipation rate
density $\epsilon$, which governs the conversion rate from $k$ to mean internal
energy, is defined by
\begin{align}
  \bar{\rho} \epsilon &= \overline{\tau : \nabla{}u''}
.
\end{align}
As page~216 of Lele~\cite{Lele1994Compressibility} suggests, expanding $h$,
averaging, removing the mean state from both sides, and applying perfect gas
assumptions demonstrates the exact relationship
\begin{align}
  \overline{u''}
&=
  \frac{\widetilde{T''u''}}{\tilde{T}} - \frac{\overline{p'u''}}{\bar{p}}
.
\end{align}
Substituting $h''$ everywhere for $T''$, noting $\bar{p}/\tilde{h} =
\frac{\gamma-1}{\gamma}\bar{\rho}$, and differentiating one obtains
\begin{align}
  \overline{p'u''}
&=
  \frac{\gamma-1}{\gamma} \bar{\rho} \widetilde{h''u''}
- \bar{p} \overline{u''}
\\
  \nabla\cdot \overline{p'u''}
&=
  \frac{\gamma-1}{\gamma} \nabla\cdot \bar{\rho} \widetilde{h''u''}
- \bar{p}\nabla\cdot\overline{u''}
- \overline{u''}\cdot\nabla{}\bar{p}
.
\end{align}
Rearranging the above result to mimic terms within~\eqref{eq:fanstke1}
\begin{align}
  - \overline{u''}\cdot\nabla\bar{p}
  - \nabla\cdot\overline{p'u''}
&=
  \bar{p}\nabla\cdot\overline{u''}
- \frac{\gamma-1}{\gamma} \nabla\cdot \bar{\rho} \widetilde{h''u''}
\end{align}
allows us to trade an occurrence of $\overline{p'u''}$ for the Reynolds heat
flux in the exact $k$ equation:
\begin{align}
\label{eq:fanstke}
    \frac{\partial{}}{\partial{}t}\bar{\rho}k
 =
 &- \nabla\cdot\bar{\rho}k\tilde{u}
  - \bar{\rho} \widetilde{u''\otimes{}u''} : \nabla\tilde{u}
  - \bar{\rho} \epsilon
  + \nabla\cdot\left(
        -\frac{1}{2}\bar{\rho}\widetilde{{u''}^{2}u''}
      + \overline{\tau{}u''}
    \right)
\\
 &+ \bar{p}\nabla\cdot\overline{u''}
  - \frac{\gamma-1}{\gamma} \nabla\cdot\bar{\rho} \widetilde{h''u''}
  + \overline{p' \nabla\cdot{}u''}
  + \overline{f\cdot{}u''}
\end{align}
The trade reduces by one the number of correlations appearing in the $k$
equation which do not appear in the mean continuity, momentum, or energy
equations.  It also, as Lele suggests, encourages thermodynamic consistency
when working with pressure correlation information.

Returning to the constitutive relations, combining~\eqref{eq:tauSmub}
and~\eqref{eq:secondviscosityclaw} we obtain
\begin{align}
  \tau
&= 2 \mu{} S + \alpha \mu \left( \nabla\cdot{}u \right) I.
\end{align}
Using the kinematic viscosity $\nu = \mu / \rho$ and averaging we find
\begin{align}
   \tilde{S}
&=
     \frac{1}{2}\left(
       \widetilde{\nabla{}u} + \trans{\widetilde{\nabla{}u}}
     \right)
   - \frac{1}{3}\left(\widetilde{\nabla\cdot{}u}\right) I
\\
  \bar{\tau}
&=
    2 \bar{\mu}\tilde{S}
  + 2 \bar{\rho} \widetilde{\nu''S''}
  + \alpha \bar{\mu} \widetilde{\nabla\cdot{}u} I
  + \alpha \bar{\rho} \widetilde{\nu''\left(\nabla\cdot{}u\right)''} I
.
\end{align}
By~\eqref{eq:favrefluctexample},
$\widetilde{\nu''\left(\nabla\cdot{}u\right)''}$ may also be written
$\widetilde{\nu''\nabla\cdot{}u''}$ while $\widetilde{\nu''S''}$ is equivalent
to a version using the deviatoric part of the strain rate of the fluctuating
velocity field.  Many FANS closure approximations neglect correlations between
the kinematic viscosity and velocity derivatives.  Many assume $\alpha=0$.
Accepting those approximations would eliminate the second through fourth terms
in $\bar{\tau}$.  Making the ubiquitous closure approximations
$\widetilde{\nabla{}u} + \trans{\widetilde{\nabla{}u}} \approx \nabla\tilde{u}
+ \trans{\nabla\tilde{u}}$ and
$\widetilde{\nabla{}\cdot{}u}\approx\nabla\cdot\tilde{u}$ are equivalent to
neglecting $\widetilde{\nabla{}u''} + \trans{\widetilde{\nabla{}u''}}$ and
$\widetilde{\nabla{}\cdot{}u''}$ per~\eqref{eq:favremeancommute}.

To find $\bar{q}_s$ we combine~\eqref{eq:fourierlaw} and our assumption of a
constant Prandtl number
\begin{align}
  q_{s} &= - \kappa \nabla{} T
     = - \frac{\kappa}{C_p} \nabla{}h
     = - \frac{\mu}{\Prandtl} \nabla{}h
\end{align}
and again employ $\nu$ when averaging to obtain
\begin{align}
  \bar{q}_s
&= - \frac{1}{\Prandtl}\left(
                \bar{\mu}\widetilde{\nabla{}h}
              + \bar{\rho} \widetilde{\nu''\left(\nabla{}h\right)''}
            \right)
.
\end{align}
Again, by~\eqref{eq:favrefluctexample},
$\widetilde{\nu''\left(\nabla{}h\right)''}$ may also be written
$\widetilde{\nu''\nabla{}h''}$.  Again, making the ubiquitous closure
assumption $\widetilde{\nabla{}h}\approx\nabla\tilde{h}$ is equivalent to
neglecting $\widetilde{\nabla{}h''}$ per~\eqref{eq:favremeancommute}.
Straightforward averaging applied to~\eqref{eq:powerlawviscosity} produces
\begin{align}
   \bar{\rho}\tilde{\nu}
 = \bar{\mu}
&= \mu_0 \overline{\left(\frac{T}{T_0}\right)^\beta}
\end{align}
which is not computable given only Favre-averaged state.  One commonly accepted
simplification is taking $\overline{\mu\left(T\right)} \approx
\mu\left(\tilde{T}\right)$.

\subsubsection{Summary}

The Favre-averaged equations of interest are:
\begin{subequations}
\begin{align}
    \frac{\partial}{\partial{}t}\bar{\rho}
=
 &- \nabla\cdot\bar{\rho}\tilde{u}
\\
    \frac{\partial{}}{\partial{}t}\bar{\rho}\tilde{u}
 =
 &- \nabla\cdot(\tilde{u}\otimes\bar{\rho}\tilde{u})
  - \nabla{}\bar{p}
  + \nabla\cdot\left(
        \bar{\tau}
      - \bar{\rho} \widetilde{u''\otimes{}u''}
    \right)
  + \bar{f}
\\
    \frac{\partial}{\partial{}t} \bar{\rho}\tilde{E}
 =
 &- \nabla\cdot{}\bar{\rho}\tilde{H}\tilde{u}
  + \nabla\cdot\left(
        \left(
            \bar{\tau}
          - \bar{\rho} \widetilde{u''\otimes{}u''}
        \right) \tilde{u}
      - \frac{1}{2}\bar{\rho}\widetilde{{u''}^{2}u''}
      + \overline{\tau{}u''}
    \right)
\\
 &- \nabla\cdot\left(
        \bar{q}_s
      + \bar{\rho} \widetilde{h''u''}
    \right)
  + \bar{f}\cdot\tilde{u}
  + \overline{f\cdot{}u''}
  + \bar{q}_b
\\
    \frac{\partial{}}{\partial{}t}\bar{\rho}k
=
 &- \nabla\cdot\bar{\rho}k\tilde{u}
  - \bar{\rho} \widetilde{u''\otimes{}u''} : \nabla\tilde{u}
  - \bar{\rho} \epsilon
  + \nabla\cdot\left(
        -\frac{1}{2}\bar{\rho} \widetilde{{u''}^{2}u''}
      + \overline{\tau{}u''}
    \right)
\\
 &+ \bar{p}\nabla\cdot\overline{u''}
  - \frac{\gamma-1}{\gamma} \nabla\cdot\bar{\rho} \widetilde{h''u''}
  + \overline{p' \nabla\cdot{}u''}
  + \overline{f\cdot{}u''}
\end{align}
The equations are augmented by the following relationships:
\begin{align}
  \bar{p} &= \bar{\rho}R\tilde{T}
&
   \bar{\rho}\tilde{\nu} =
   \bar{\mu}
&= \mu_0 \overline{\left(\frac{T}{T_0}\right)^\beta}
&
  k &= \frac{1}{2}\widetilde{{u''}^2}
&
  \bar{\rho} \epsilon &= \overline{\tau : \nabla{}u''}
\end{align}
\begin{align}
  \tilde{E}
&=
  \frac{R}{\gamma-1} \tilde{T}
+ \frac{1}{2} \tilde{u}^2
+ k
&
  \tilde{H}
&=
  \tilde{E}
+ R \tilde{T}
&
  \tilde{h} &= \frac{\gamma{}R\tilde{T}}{\gamma-1}
&
  \bar{q}_s
&= - \frac{1}{\Prandtl}\left(
                \bar{\mu}\widetilde{\nabla{}h}
              + \bar{\rho} \widetilde{\nu''\left(\nabla{}h\right)''}
            \right)
\end{align}
\begin{align}
   \tilde{S}
&=
     \frac{1}{2}\left(
       \widetilde{\nabla{}u} + \trans{\widetilde{\nabla{}u}}
     \right)
   - \frac{1}{3}\left(\widetilde{\nabla\cdot{}u}\right) I
&
   \bar{\tau}
&=  2 \bar{\mu}\tilde{S}
  + 2 \bar{\rho} \widetilde{\nu''S''}
  + \alpha \bar{\mu} \widetilde{\nabla\cdot{}u} I
  + \alpha \bar{\rho} \widetilde{\nu''\left(\nabla\cdot{}u\right)''} I
\end{align}
\end{subequations}
One may exactly compute the mean state evolution given the following
information:
\begin{samepage}
\begin{align}
&\bar{\rho}
&
&\tilde{u}
&
&\tilde{E}
&
&\bar{\mu}
&
&\bar{f}
&
&\bar{q}_b
&
&k
&
&\epsilon
&
&\overline{u''}
&
&\symmetricpart{\widetilde{\nabla{}u}}
\end{align}
\begin{align}
&\overline{f\cdot{}u''}
&
&\overline{\tau{}u''}
&
&\overline{p'\nabla\cdot{}u''}
&
&-\widetilde{u''\otimes{}u''}
&
&-\frac{1}{2}\widetilde{{u''}^{2}u''}
&
&\widetilde{h''u''}
&
&\widetilde{\nu''S''}
&
&\widetilde{\nu''\left(\nabla\cdot{}u\right)''}
&
&\widetilde{\nu''\left(\nabla{}h\right)''}
\end{align}
\end{samepage}
Other ways to minimally capture the mean thermodynamic state, e.g. $\bar{\rho}$
and $\tilde{T}$, could have been chosen.  The information above are natural
choices following section~\ref{state_variable_selection}.  Favre-fluctuating
correlation densities (e.g.  $\widetilde{h''u''}$ are adopted for notational
brevity instead of their informationally equivalent Reynolds-averaged
correlations (e.g.  $\overline{\rho{}h''u''}$).

\subsection{The nondimensional Favre-averaged Navier--Stokes equations}
\label{sec:nondimfans}

The dimensional FANS equations from the last section need to be
nondimensionalized.   The reference quantity selections made in
section~\ref{sec:nondimrefq} are used and are augmented by
\begin{align}
  k_0 &= u_{0}^2
&
  \epsilon_0 &= \frac{u_{0}^2}{t_0}
\end{align}
Superscript star notation is suppressed as all terms
are dimensionless.  The results are:
\begin{subequations}
\begin{align}
    \frac{\partial}{\partial{}t}\bar{\rho}
=
 &- \nabla\cdot\bar{\rho}\tilde{u}
\\
    \frac{\partial{}}{\partial{}t}\bar{\rho}\tilde{u}
=
 &- \nabla\cdot(\tilde{u}\otimes\bar{\rho}\tilde{u})
  - \frac{1}{\Mach^2}\nabla{}\bar{p}
  + \nabla\cdot\left(
        \frac{\bar{\tau}}{\Reynolds}
      - \bar{\rho} \widetilde{u''\otimes{}u''}
    \right)
  + \bar{f}
\\
  \frac{\partial}{\partial{}t} \bar{\rho}\tilde{E}
=
 &- \nabla\cdot\bar{\rho}\tilde{H}\tilde{u}
  + \Mach^{2} \nabla\cdot\left(
        \left(
            \frac{\bar{\tau}}{\Reynolds}
          - \bar{\rho} \widetilde{u''\otimes{}u''}
        \right) \tilde{u}
      - \frac{1}{2}\bar{\rho}\widetilde{{u''}^{2}u''}
      + \frac{\overline{\tau{}u''}}{\Reynolds}
    \right)
\\
 &+ \frac{1}{\gamma-1} \nabla\cdot\left(
      \frac{
         \bar{\mu} \widetilde{\nabla{}T}
       + \bar{\rho} \widetilde{\nu'' \left(\nabla{}T\right)''}
      }{\Reynolds\Prandtl}
      - \bar{\rho} \widetilde{T''u''}
    \right)
  + \Mach^{2} \left(
        \bar{f}\cdot\tilde{u}
      + \overline{f\cdot{}u''}
    \right)
  + \bar{q}_b
\\
    \frac{\partial{}}{\partial{}t}\bar{\rho}k
=
 &- \nabla\cdot\bar{\rho}k\tilde{u}
  - \bar{\rho} \widetilde{u''\otimes{}u''} : \nabla\tilde{u}
  - \frac{\bar{\rho} \epsilon}{\Reynolds}
  + \nabla\cdot\left(
        -\frac{1}{2}\bar{\rho} \widetilde{{u''}^{2}u''}
      + \frac{\overline{\tau{}u''}}{\Reynolds}
    \right)
\\
 &+ \frac{1}{\Mach^2} \left(
        \bar{p}\nabla\cdot\overline{u''}
      + \overline{p' \nabla\cdot{}u''}
      - \frac{1}{\gamma} \nabla\cdot\bar{\rho} \widetilde{T''u''}
    \right)
  + \overline{f\cdot{}u''}
\end{align}
The equations are augmented by the following nondimensional relationships:
\begin{align}
  \bar{p} &= \frac{\bar{\rho} \tilde{T}}{\gamma}
&
   \bar{\rho}\tilde{\nu} =
   \bar{\mu}
&= \overline{T^\beta}
&
  k &= \frac{1}{2}\widetilde{{u''}^2}
&
  \bar{\rho} \epsilon &= \overline{\tau : \nabla{}u''}
\end{align}
\begin{align}
  \tilde{E}
&=
  \frac{\tilde{T}}{\gamma\left(\gamma-1\right)}
  + \Mach^2 \left( \frac{1}{2}\tilde{u}^2 + k
  \right)
&
  \tilde{H}
&=
  \tilde{E} + \frac{\tilde{T}}{\gamma}
&
  \tilde{h} &= \frac{\tilde{T}}{\gamma-1}
\end{align}
\begin{align}
   \tilde{S}
&=
     \frac{1}{2}\left(
       \widetilde{\nabla{}u} + \trans{\widetilde{\nabla{}u}}
     \right)
   - \frac{1}{3}\left(\widetilde{\nabla\cdot{}u}\right) I
&
   \bar{\tau}
&=  2 \bar{\mu}\tilde{S}
  + 2 \bar{\rho} \widetilde{\nu''S''}
  + \alpha \bar{\mu} \widetilde{\nabla\cdot{}u} I
  + \alpha \bar{\rho} \widetilde{\nu''\left(\nabla\cdot{}u\right)''} I
\end{align}
\end{subequations}
where $\Reynolds$, $\Mach$, and $\Prandtl$ are defined as in
section~\ref{sec:nondimrefq}.  One may exactly compute the nondimensional mean
state evolution given the following statistical quantities:
\begin{samepage}
\begin{align}
&\bar{\rho}
&
&\tilde{u}
&
&\tilde{E}
&
&\bar{\mu}
&
&\bar{f}
&
&\bar{q}_b
&
&k
&
&\epsilon
&
&\overline{u''}
&
&\symmetricpart{\widetilde{\nabla{}u}}
\end{align}
\begin{align}
&\overline{f\cdot{}u''}
&
&\overline{\tau{}u''}
&
&\overline{p'\nabla\cdot{}u''}
&
&-\widetilde{u''\otimes{}u''}
&
&-\frac{1}{2}\widetilde{{u''}^{2}u''}
&
&\widetilde{T''u''}
&
&\widetilde{\nu''S''}
&
&\widetilde{\nu''\left(\nabla\cdot{}u\right)''}
&
&\widetilde{\nu''\left(\nabla{}T\right)''}
\end{align}
\end{samepage}

Notice that the heat flux $\bar{q}_s$ has been merged into the mean energy
equation to better mimic \eqref{eq:nondim_energy}.  Enthalpy-based correlations
have been replaced by temperature-based correlations.  Notice also that the
choice of $p_0$ implies nondimensional $p$ includes a factor of $\gamma$
relative to the dimensional quantity.  Where possible, nondimensional
coefficients have been pulled out of the constitutive relationships and pushed
into the equations (for example, the factor $1/\Reynolds$ arising from the
dimensional definition of $\bar{\rho}\epsilon$).

Some locally computed quantities, e.g. the local Mach number, local Reynolds
number, or local eddy viscosity, require rescaling within to account for
nondimensionalization.  Other quantities, e.g. the local turbulent Prandtl
number, do not.  Writing the dimensional definition and re-expressing all
dimensional terms as a nondimensional value multiplied by a reference value
allows determining any appropriate multiplicative factors.

\subsection{Sampling logistics}

Statistical quantities are obtained by sampling from a well-resolved,
stationary simulation.  Mean quantity samples may be computed on-the-fly.
Fluctuating quantity samples, because they must be taken relative to an unknown
true mean, are computed by combining mean quantity samples following the rules
in section~\ref{sec:averaging}.

Sampling the following mean quantities is sufficient to compute
the statistical quantities listed in section~\ref{sec:nondimfans}:
\begin{samepage}
\begin{align}
&\bar{\rho}
&
&\overline{\rho{}u}
&
&\overline{\rho{}E}
&
&\bar{\mu}
&
&\bar{f}
&
&\bar{q}_b
&
&\bar{u}
&
&\symmetricpart{\overline{\rho\nabla{}u}}
&
&\overline{\rho\nabla{}T}
&
&\overline{\tau:\nabla{}u}
&
&\overline{f\cdot{}u}
&
&\bar{\tau}
\end{align}
\begin{align}
&\overline{\tau{}u}
&
&\overline{p\nabla\cdot{}u}
&
&\overline{\rho{}u\otimes{}u}
&
&\overline{\rho{}u\otimes{}u\otimes{}u}
&
&\overline{\rho{}Tu}
&
&\overline{\mu{}S}
&
&\overline{\mu\nabla\cdot{}u}
&
&\overline{\mu\nabla{}T}
\end{align}
\end{samepage}
After averaging across the homogeneous streamwise $x$ and spanwise $z$
directions, each sample (ignoring tensor order) is a one-dimensional,
instantaneous profile varying only along the wall-normal B-spline direction.
The amount of data gathered may be reduced by exploiting symmetries in
$\bar{\tau}$, $\overline{\rho{}u\otimes{}u}$,
$\overline{\rho{}u\otimes{}u\otimes{}u}$, and $\overline{\mu{}S}$.

The instantaneous samples are combined according to the following ordered
sequence of computations to obtain the desired quantities:
{ \allowdisplaybreaks[1]
\begin{align}
  \tilde{u} &= \bar{\rho}^{-1} \overline{\rho{}u}
\\
  \tilde{E} &= \bar{\rho}^{-1} {\overline{\rho{}E}}
\\
  \widetilde{u\otimes{}u} &= \bar{\rho}^{-1} \overline{\rho{}u\otimes{}u}
\\
  \widetilde{u''\otimes{}u''} &=
  \widetilde{u\otimes{}u} - \tilde{u}\otimes\tilde{u}
\\
  k &= \frac{1}{2} \trace \widetilde{u''\otimes{}u''}
\\
  \tilde{T} &= \gamma\left(\gamma-1\right)\left(
      \tilde{E} - \Mach^2\left(\frac{1}{2}\tilde{u}^2 + k\right)
  \right)
\\
  \tilde{H} &= \tilde{E} + \frac{\tilde{T}}{\gamma}
\\
  \bar{p} &= \frac{\bar{\rho}\tilde{T}}{\gamma}
\\
  \overline{\tau:\nabla{}u''} &=
  \overline{\tau:\nabla{}u} - \bar{\tau}:\nabla\tilde{u}
\\
  \epsilon &= \bar{\rho}^{-1} \overline{\tau:\nabla{}u''}
\\
  \overline{u''} &= \bar{u} - \tilde{u}
\\
  \overline{f\cdot{}u''} &= \overline{f\cdot{}u} - \bar{f}\cdot{}\tilde{u}
\\
  \overline{\tau{}u''} &= \overline{\tau{}u} - \bar{\tau}\tilde{u}
\\
  \overline{p'\nabla\cdot{}u''}
  &= \overline{p\nabla\cdot{}u}
   - \bar{p}\nabla\cdot\bar{u}
\\
  \widetilde{u\otimes{}u\otimes{}u}
  &= \bar{\rho}^{-1} \overline{\rho{}u\otimes{}u\otimes{}u}
\end{align}
}

Expressions for computing $\widetilde{u''\otimes{}u''\otimes{}u''}$ and
$-\frac{1}{2}\widetilde{{u''}^{2}u''}$ are derived in stages using index
notation and the summation convention.  Using the identities
\begin{align}
  \widetilde{u_{i}u_{j}''}
&=
  \widetilde{u_{i}u_{j}} - \tilde{u}_{i}\tilde{u}_{j}
\\
  \widetilde{u_{i}u_{j}u_{k}''}
&=
  \widetilde{u_{i}u_{j}u_{k}} - \widetilde{u_{i}u_{j}}\tilde{u}_{k}
\shortintertext{
and ignoring any symmetry-related simplifications, we obtain
}
  \widetilde{u_{i}''u_{j}''u_{k}''}
  &= \bar{\rho}^{-1} \overline{\rho{}\left(u_{i}-\tilde{u}_{i}\right)
                                     \left(u_{j}-\tilde{u}_{j}\right)
                                     u_{k}''}
\\
  &= \widetilde{u_{i}u_{j}u_{k}''}
   - \tilde{u}_i \widetilde{u_{j}u_{k}''}
   - \tilde{u}_j \widetilde{u_{i}u_{k}''}
%%\\
%%  &= \widetilde{u_{i}u_{j}u_{k}}
%%   - \widetilde{u_{i}u_{j}}\tilde{u}_{k}
%%   - \tilde{u}_{i} \widetilde{u_{j} u_{k}}
%%   + \tilde{u}_{i} \tilde{u}_{j}\tilde{u}_{k}
%%   - \tilde{u}_{j} \widetilde{u_{i}u_{k}}
%%   + \tilde{u}_{j} \tilde{u}_{i}\tilde{u}_{k}
\\
  &=   \widetilde{u_{i}u_{j}u_{k}}
   -   \widetilde{u_{i}u_{j}}\tilde{u}_{k}
   -   \tilde{u}_{i} \widetilde{u_{j} u_{k}}
   -   \tilde{u}_{j} \widetilde{u_{i}u_{k}}
   + 2 \tilde{u}_{i} \tilde{u}_{j}\tilde{u}_{k}
.
\shortintertext{
Contracting the first two indices and relabelling the third index,
}
  \widetilde{u_{i}''u_{i}''u_{j}''}
  &= \widetilde{u_{i}u_{i}u_{j}}
   - \widetilde{u_{i}u_{i}}\tilde{u}_{j}
   - 2 \tilde{u}_{i} \widetilde{u_{i} u_{j}}
   + 2 \tilde{u}_{i} \tilde{u}_{i}\tilde{u}_{j}
   .
\shortintertext{
Reverting to direct notation and employing the symmetry of
$\widetilde{u\otimes{}u}$, we arrive at
}
  \widetilde{{u''}^{2}u''}
&=
      \widetilde{u^{2}u}
  -   \widetilde{u^{2}}\tilde{u}
  - 2 \widetilde{u\otimes{}u}\tilde{u}
  + 2 \tilde{u}^2 \tilde{u}
\end{align}
where $\widetilde{u^{2}u}$ and $\widetilde{u^2}$ may be computed by contracting
$\widetilde{u\otimes{}u\otimes{}u}$ and $\widetilde{u\otimes{}u}$,
respectively.

Continuing the ordered sequence of computations:
\begin{align}
  \widetilde{Tu} &= \bar{\rho}^{-1} \overline{\rho{}Tu}
\\
  \widetilde{T''u''} &= \widetilde{Tu} - \tilde{T}\tilde{u}
\\
  \tilde{\nu} &= \bar{\rho}^{-1} \overline{\mu}
\\
  \symmetricpart{\widetilde{\nabla{}u}}
  &= \bar{\rho}^{-1} \symmetricpart{\overline{\rho\nabla{}u}}
\\
  \tilde{S} &= \symmetricpart{\widetilde{\nabla{}u}}
   - \frac{1}{3} \trace\symmetricpart{\widetilde{\nabla{}u}} I
\\
  \widetilde{\nu''S''}
  &= \bar{\rho}^{-1} \overline{\mu{}S} - \tilde{\nu}\tilde{S}
\\
  \widetilde{\nu''\left(\nabla\cdot{}u\right)''}
  &= \bar{\rho}^{-1} \overline{\mu\nabla\cdot{}u}
   - \tilde{\nu}\trace\symmetricpart{\widetilde{\nabla{}u}}
\\
  \widetilde{\nu''\left(\nabla{}T\right)''}
  &= \bar{\rho}^{-1} \overline{\mu\nabla{}T}
   - \tilde{\nu} \bar{\rho}^{-1} \overline{\rho\nabla{}T}
\end{align}

\subsection{Quantifying statistical quantity convergence}
\label{sec:quantconvergence}

TODO


%%%%%%%%%%%%%%%%%%%%%%%%%%%%%%%%%%%%%%%%%%%%%%%%%%%%%%%%%%%%%%%%%%%%
%%%%%%%%%%%%%%%%%%%%%%%%%%% Bibliography %%%%%%%%%%%%%%%%%%%%%%%%%%%
%%%%%%%%%%%%%%%%%%%%%%%%%%%%%%%%%%%%%%%%%%%%%%%%%%%%%%%%%%%%%%%%%%%%
\bibliographystyle{amsplain}
\bibliography{references}


%%%\appendix
%%%
%%%\section{Miscellaneous}
%%%
%%%\subsection{The Variable Density Equations}
%%%
%%%The limits of the compressible Navier--Stokes equations as the Mach number
%%%$\Mach = u_{0} / a_{0} \to 0$ while allowing finite temperature and density
%%%fluctuations are called the variable density equations.

\end{document}
