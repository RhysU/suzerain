\documentclass[letterpaper,11pt,nointlimits,reqno]{amsart}

% Packages
\usepackage{accents}
\usepackage{array}
\usepackage{algorithm}
\usepackage{algorithmic}
\usepackage{amsfonts}
\usepackage{amsmath}
\usepackage{amssymb}
\usepackage{booktabs}
\usepackage{cancel}
\usepackage{enumerate}
\usepackage{fancyhdr}
\usepackage{fullpage}
\usepackage{ifthen}
\usepackage{lastpage}
\usepackage{latexsym}
\usepackage{listings}
\usepackage{mathrsfs}
\usepackage{mathtools}
\usepackage[numbers,sort&compress]{natbib}
\usepackage{parskip}
\usepackage{pstricks}
\usepackage{setspace}
\usepackage{txfonts}
\usepackage{units}
\usepackage{varioref}
\usepackage{wrapfig}


% Line Spacing
\singlespacing

% Set appropriate header/footer information on each page
\fancypagestyle{plain}{
    \fancyhf{}
    \renewcommand{\headheight}{2.0em}
    \renewcommand{\headsep}{0.75em}
    \renewcommand{\headrulewidth}{1.0pt}
    \renewcommand{\footrulewidth}{0pt}
    \lhead{
        Two-point correlations computations using Suzerain's discretization
    }
    \rhead{
        Page \thepage{} of \pageref{LastPage}
    }
}
\pagestyle{plain}

% Document-specific commands
\newcommand{\ii}{\ensuremath{\mathrm{i}}}
\newcommand{\htrans}[1]{{#1}^{\ensuremath{\mathsf{H}}}}
\newcommand{\trans}[1]{{#1}^{\ensuremath{\mathsf{T}}}}
\newcommand{\OO}[1]{\operatorname{O}\left(#1\right)}

\begin{document}

In this document we detail how to compute two-point correlations.  The material
is based upon \citet[\textsection{}6.4--5]{Pope2000Turbulent} and the notation
is changed to make it consistent with Suzerain's documentations.  Where
necessary, these computations are specialized for Suzerain's discretization.  In
this discretization, any instantaneous, real-valued field
$u\!\left(x,y,z\right)$ on the spatial domain
$\left[-\frac{L_x}{2},\frac{L_x}{2}\right] \times{} [0,L_y] \times{}
\left[-\frac{L_z}{2},\frac{L_z}{2}\right]$ is discretized as
\begin{align}
  u^h(x,y,z)
&=
  \sum_{l=0}^{N_y - 1}
  \sum_{m=-\frac{N_x}{2}}^{\frac{N_x}{2}-1}
  \sum_{n=-\frac{N_z}{2}}^{\frac{N_z}{2}-1}
  \hat{u}_{l m n}
  B_l\!\left(y\right)
  e^{\ii\frac{2\pi{}m}{L_x}x}
  e^{\ii\frac{2\pi{}n}{L_z}z}
=
  \sum_{l}\sum_{m}\sum_{n}
  \hat{u}_{l m n}B_l\!\left(y\right)e^{\ii k_m x}e^{\ii k_n z}
\end{align}
where $k_m = 2\pi{}m/L_x$, $k_n = 2\pi{}n/L_z$, and $B_l\!\left(y\right)$ are a
B-spline basis for some order and some knot selection.

\section{Two-point correlation
         \citep[\textsection{}6.3]{Pope2000Turbulent}}

In real space, the two-point spatial correlation of two discrete fields, $u^h$ and
$v^h$ at taken at position $\vec{x}$ and at time $t$ and is defined as 
%
\begin{equation}
\begin{aligned}
R_{ij} \left( \vec{x}, \vec{r}, t \right) 
  &=   \overline{ 
       u^h \left( \vec{x}+ \vec{r}, t \right)  
       v^h \left( \vec{x}         , t \right) 
       }
\end{aligned}
\end{equation}
%
where $\vec{r}$ is the distance vector from the position where $v$ is measured
to where $u$ is measured. For statistically steady flow, with homogenity in 
both $x$ and $z$, the expression reduces to  
%
\begin{equation}
\begin{aligned}
R_{ij} \left( y, \vec{r} \right) 
  &=   \overline{
       u^h \left( \vec{x}+ \vec{r}, t \right)
       v^h \left( \vec{x}         , t \right) 
       }
\end{aligned}
\end{equation}
%
where we will be interested mainly in $\vec{r}$ taken along the $x$ and 
$z$ directions, and the expectation for any $y$ can be computed over 
$xz$-planes over several flow fields. Specifically, the expectation 
for any $\vec{r}_{xz} = r_x \hat{i} + r_z \hat{k}$ at time $t_i$ (implied and
removed from the computation for simplicity) is given by
%
\begin{align}
R_{uv} \left( y, \vec{r}_{xz} \right) 
&=
  \frac{1}{L_x L_z}
  \int_{-\frac{L_x}{2}}^{\frac{L_x}{2}}
  \int_{-\frac{L_z}{2}}^{\frac{L_z}{2}}
  u^h\left( \vec{x}+ \vec{r}_{xz}\right)
  v^h\left( \vec{x}              \right) 
  \,d\!z \,d\!x
\\ &=
  \frac{1}{L_x L_z}
  \int_{-\frac{L_x}{2}}^{\frac{L_x}{2}}
  \int_{-\frac{L_z}{2}}^{\frac{L_z}{2}}
  \left(
    \sum_{l}\sum_{m}\sum_{n}
    \hat{u}_{l m n}B_l\!\left(y\right)e^{\ii k_m (x+r_x)}e^{\ii k_n (z+r_z)}
  \right)
  \left(
    \sum_{l^\prime}\sum_{m^\prime}\sum_{n^\prime}
    \hat{v}_{l^\prime m^\prime n^\prime}^*
    B_{l^\prime}\!\left(y\right)e^{-\ii k_{m^\prime} x}e^{-\ii k_{n^\prime} z}
  \right)
  \,d\!z \,d\!x
\\ &=
  \frac{1}{L_x L_z}
  \sum_{l}
  B_l\!\left(y\right)
  \sum_{l^\prime}
  B_{l^\prime}\!\left(y\right)
  \sum_{m}
  \sum_{m^\prime}
  \int_{-\frac{L_x}{2}}^{\frac{L_x}{2}}
  \left(
    e^{\ii k_m (x+r_x)}
    e^{-\ii k_{m^\prime} x}
    \sum_{n}
    \sum_{n^\prime}
    \hat{u}_{l m n}
    \hat{v}_{l^\prime m^\prime n^\prime}^*
    e^{\ii k_n r_z}
    \underbrace{
    \int_{-\frac{L_z}{2}}^{\frac{L_z}{2}}
    \left(
      e^{\ii k_n z}
      e^{-\ii k_{n^\prime} z}
    \right)
    \,d\!z
    }_{L_z \, \textrm{if} \, n=n^\prime, \,
       0  \, \textrm{otherwise}}
  \right)
  \,d\!x
\\ &=
  \frac{1}{L_x}
  \sum_{l}
  B_l\!\left(y\right)
  \sum_{l^\prime}
  B_{l^\prime}\!\left(y\right)
  \sum_{m}
  \sum_{m^\prime}
  \sum_{n}
  \hat{u}_{l m n}
  \hat{v}_{l^\prime m^\prime n}^*
  e^{\ii k_m r_x}
  e^{\ii k_n r_z}
  \left(
  \underbrace{
  \int_{-\frac{L_x}{2}}^{\frac{L_x}{2}}
    e^{\ii k_m x+}
    e^{-\ii k_{m^\prime} x}
    \,d\!x
  }_{L_x \, \textrm{if} \, m=m^\prime, \,
       0  \, \textrm{otherwise}}
  \right)
\\ &=
  \sum_{l}
  B_l\!\left(y\right)
  \sum_{l^\prime}
  B_{l^\prime}\!\left(y\right)
  \sum_{m}
  \sum_{n}
  \hat{u}_{l m n}
  \hat{v}_{l^\prime m n}^*
  e^{\ii k_m r_x}
  e^{\ii k_n r_z}
\\ &=
  \sum_{m}
  \sum_{n}
  \left(
  \underbrace{
  \sum_{l}
  B_l\!\left(y\right)
  \hat{u}_{l m n}
  \sum_{l^\prime}
  B_{l^\prime}\!\left(y\right)
  \hat{v}_{l^\prime m n}^*
  }_{\hat{R}_{uv}\left( y, \vec{k}_{xz} \right)}
  \right)
  e^{\ii k_m r_x}
  e^{\ii k_n r_z}
\end{align}
%
The two-point correlation at points $y_j$ reduces to
%
\begin{align}
R_{uv} \left( y_j, \vec{r}_{xz} \right) 
  &=
  \sum_{m}
  \sum_{n}
  \left(
  \left(M \hat{u}_{m n}  \right)\bigr|_{y=y_j}
  \left(M \hat{v}_{m n}^*\right)\bigr|_{y=y_j}
  \right)
  e^{\ii k_m r_x}
  e^{\ii k_n r_z}
\end{align}
%
In Suzerain we will be computing and storing the two-point correlations 
along the $x$- and $z$-directions only. 

%FIXME
Discuss implementation details.


\section{Velocity spectra
         \citep[\textsection{}6.5]{Pope2000Turbulent}}

%FIXME: Review this one, in particular the claimed generalization
Here we consider a generalization of the definiton given in Pope
\citep[\textsection{}6.3]{Pope2000Turbulent}, where the fields are considered to
be the velocity components: the one-dimensional spectra 
$E_{u,v}(y,\vec{k}_{xz})$ are defined to be twice the one-dimensional Fourier
transform of $R_{u,v}(y,\vec{k}_{xz})$:
%
\begin{equation}
E_{uv} = \frac{1}{\pi} \int_{-\infty}^{\infty} R_{uv}(y,\vec{k}_{xz}) 
         \exp^{-i \vec{k}_{xz} \vec{r}_xz} \,d\!r_{xz}
\end{equation}
%


\newcommand*{\doi}[1]{\href{http://dx.doi.org/\detokenize{#1}}{doi: #1}}
\bibliographystyle{plainnat}
\bibliography{references}

\end{document}
