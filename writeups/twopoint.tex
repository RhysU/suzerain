\documentclass[letterpaper,11pt,nointlimits,reqno]{amsart}

% Packages
\usepackage{accents}
\usepackage{array}
\usepackage{algorithm}
\usepackage{algorithmic}
\usepackage{amsfonts}
\usepackage{amsmath}
\usepackage{amssymb}
\usepackage{booktabs}
\usepackage{cancel}
\usepackage{enumerate}
\usepackage{fancyhdr}
\usepackage{fullpage}
\usepackage{ifthen}
\usepackage{lastpage}
\usepackage{latexsym}
\usepackage{listings}
\usepackage{mathrsfs}
\usepackage{mathtools}
\usepackage[numbers,sort&compress]{natbib}
\usepackage{parskip}
\usepackage{pstricks}
\usepackage{setspace}
\usepackage{todonotes}
\usepackage{txfonts}
\usepackage{units}
\usepackage{varioref}
\usepackage{wrapfig}

\mathtoolsset{showonlyrefs,showmanualtags}

% Line Spacing
\singlespacing

% Set appropriate header/footer information on each page
\fancypagestyle{plain}{
    \fancyhf{}
    \renewcommand{\headheight}{2.0em}
    \renewcommand{\headsep}{0.75em}
    \renewcommand{\headrulewidth}{1.0pt}
    \renewcommand{\footrulewidth}{0pt}
    \lhead{
        Two-point correlations computations using Suzerain's discretization
    }
    \rhead{
        Page \thepage{} of \pageref{LastPage}
    }
}
\pagestyle{plain}

% Document-specific commands
\newcommand{\ii}{\ensuremath{\mathrm{i}}}
\newcommand{\htrans}[1]{{#1}^{\ensuremath{\mathsf{H}}}}
\newcommand{\trans}[1]{{#1}^{\ensuremath{\mathsf{T}}}}
\newcommand{\OO}[1]{\operatorname{O}\left(#1\right)}

\begin{document}

In this document we detail how to compute two-point correlations.  The material
is based upon \citet[\textsection{}6.4--5]{Pope2000Turbulent} and the notation
is changed to make it consistent with Suzerain's documentations.  Where
necessary, these computations are specialized for Suzerain's discretization.  In
this discretization, any instantaneous, real-valued field
$u\!\left(x,y,z\right)$ on the spatial domain
$\left[-\frac{L_x}{2},\frac{L_x}{2}\right] \times{} [0,L_y] \times{}
\left[-\frac{L_z}{2},\frac{L_z}{2}\right]$ is discretized as
\begin{align}
  u^h(x,y,z)
&=
  \sum_{l=0}^{N_y - 1}
  \sum_{m=-\frac{N_x}{2}}^{\frac{N_x}{2}-1}
  \sum_{n=-\frac{N_z}{2}}^{\frac{N_z}{2}-1}
  \hat{u}_{l m n}
  B_l\!\left(y\right)
  e^{\ii\frac{2\pi{}m}{L_x}x}
  e^{\ii\frac{2\pi{}n}{L_z}z}
=
  \sum_{l}\sum_{m}\sum_{n}
  \hat{u}_{l m n}B_l\!\left(y\right)e^{\ii k_m x}e^{\ii k_n z}
\end{align}
where $k_m = 2\pi{}m/L_x$, $k_n = 2\pi{}n/L_z$, and $B_l\!\left(y\right)$ are a
B-spline basis for some order and some knot selection.

\section{Two-point correlation
         \citep[\textsection{}6.3]{Pope2000Turbulent}}

\subsection{As a function of separation in both $x$ and $z$}
The two-point correlation of complex-valued $u$ and $v$ is
\begin{align}
R_{uv} \left( \vec{x}, \vec{r}, t \right)
  &= \frac{%
        \int_\Omega
        u^\ast\left(\vec{x}          , t\right)
        v     \left(\vec{x} + \vec{r}, t\right)
        \,\mathrm{d}\vec{x}
     }{%
        \int_\Omega
        \,\mathrm{d}\vec{x}
     }
\end{align}
where $t$ is the time, $\vec{r}$ the separation, and $\ast$ denotes complex
conjugation.  $R_{uv}$ is nothing but the spatial cross-correlation, related to
but different from the convolution, normalized by the measure of the domain.
Expanding position vector $\vec{x}$ and separation vector $\vec{r}$ into their
scalar components and invoking stationary,
\begin{align}
R_{uv} \left( x, r_x, y, r_y, z, r_z \right)
  &= \frac{%
        \int_\Omega
        u^\ast\left( x      , y      , z       \right)
        v     \left( x + r_x, y + r_y, z + r_z \right)
        \,\mathrm{d}\vec{x}
     }{%
        \int_\Omega
        \,\mathrm{d}\vec{x}
    }.
\end{align}
When the $x$ and $z$ directions are homogeneous and only $r_y=0$ is of interest
at some fixed $y_j$,
\begin{align}
R_{uv} \left( r_x, y_j, r_z \right)
  &= \frac{1}{L_x L_z}
     \int_{-\frac{L_x}{2}}^{\frac{L_x}{2}}
     \int_{-\frac{L_z}{2}}^{\frac{L_z}{2}}
       u^\ast\left( x      , y_j, z       \right)
       v     \left( x + r_x, y_j, z + r_z \right)
     \,\mathrm{d}z
     \,\mathrm{d}x.
\end{align}
Substituting $(u^h)^\ast$ and $v^h$, introducing shorthand $\phi_{y_j} =
\sum_{l} B_l\left(y_j\right) \phi_l$ denoting evaluating the B-spline basis
expansion at point $y_j$, using Fourier orthogonality, and simplifying,
\begin{align}
R_{uv} \left( r_x, y_j, r_z \right)
  &= \frac{1}{L_x L_z}
     \int_{-\frac{L_x}{2}}^{\frac{L_x}{2}}
     \int_{-\frac{L_z}{2}}^{\frac{L_z}{2}}
        \left(
         \sum_{m}\sum_{n}
         \hat{u}^\ast_{y_j m n}
         e^{-\ii k_m x}e^{-\ii k_n z}
       \right)
       \left(
         \sum_{m^\prime}\sum_{n^\prime}
         \hat{v}_{y_j m^\prime n^\prime}
         e^{\ii k_m^\prime \left(x+r_x\right)}e^{\ii k_n^\prime \left(z+r_z\right)}
       \right)
     \,\mathrm{d}z
     \,\mathrm{d}x
\\
  &= \frac{1}{L_x L_z}
     \sum_{m}
     e^{\ii k_m^\prime r_x}
     \sum_{m^\prime}
     \int_{-\frac{L_x}{2}}^{\frac{L_x}{2}}
     e^{-\ii k_m x} e^{\ii k_m^\prime x}
     \sum_{n}
     e^{\ii k_n^\prime r_z}
     \sum_{n^\prime}
     \int_{-\frac{L_z}{2}}^{\frac{L_z}{2}}
     e^{-\ii k_n z} e^{\ii k_n^\prime z}
     \hat{u}^\ast_{y_j m n}
     \hat{v}_{y_j m^\prime n^\prime}
     \,\mathrm{d}z
     \,\mathrm{d}x
\\
  &= \frac{1}{L_x L_z}
     \sum_{m}
     e^{\ii k_m^\prime r_x}
     \sum_{m^\prime}
     L_x \delta_{m m^\prime}
     \sum_{n}
     e^{\ii k_n^\prime r_z}
     \sum_{n^\prime}
     L_z \delta_{n n^\prime}
     \hat{u}^\ast_{y_j m n}
     \hat{v}_{y_j m^\prime n^\prime}
\\
  &= \sum_{m^\prime}
     \sum_{n^\prime}
     \hat{u}^\ast_{y_j m^\prime n^\prime}
     \hat{v}_{y_j m^\prime n^\prime}
     e^{\ii k_m^\prime r_x}
     e^{\ii k_n^\prime r_z}
    \label{eq:twopointxz_physical}
\end{align}
Relabeling $r_x = x$, $r_z = z$, and applying the Fourier transform in $x$ and
$z$, by orthogonality one then sees
\begin{align}
\left(\hat{R}_{uv}\right)_{y_j m n}
    =
    \frac{1}{L_x L_z}
    \int_{-\frac{L_x}{2}}^{\frac{L_x}{2}}
    \int_{-\frac{L_z}{2}}^{\frac{L_z}{2}}
    R_{uv} \left( x, y_j, z \right)
    e^{-\ii k_m x}
    e^{-\ii k_n z}
    \,\mathrm{d}z
    \,\mathrm{d}x
    &=
    \hat{u}^\ast_{y_j m n}
    \hat{v}_{y_j m n}.
    \label{eq:twopointxz_wave}
\end{align}
The Fourier representation of the two-point correlation function evaluated at
collocation point $y_j$ is nothing but the product of the Fourier
representations of $u$ and $v$.

\subsection{As a function of separation in only $x$}

Choosing $r_z = 0$ and relabeling $r_x = x$ and $n^\prime = n$ within
\eqref{eq:twopointxz_physical}, the two-point correlation as a
function of only separation in $x$ is
\begin{align}
R_{uv} \left( x, y_j, 0 \right)
  &= \sum_{m^\prime}
     \sum_{n}
     \hat{u}^\ast_{y_j m^\prime n}
     \hat{v}_{y_j m^\prime n}
     e^{\ii k_m^\prime x}.
    \label{eq:twopointx_physical}
\end{align}
Fourier transforming in $x$,
\begin{align}
\left(\hat{R}_{uv}\right)_{y_j m 0}
   =
    \frac{1}{L_x}
    \int_{-\frac{L_x}{2}}^{\frac{L_x}{2}}
    R_{uv} \left( x, y_j, 0 \right)
    e^{-\ii k_m x}
    \,\mathrm{d}x
  &=
    \sum_{n}
    \hat{u}^\ast_{y_j m n}
    \hat{v}_{y_j m n}
    \label{eq:twopointx_wave}
\end{align}
where the summation is over $n \in \left\{-N_z/2, \dots, 0, \dots,
N_z/2-1\right\}$.

\subsection{As a function of separation in only $z$}

Choosing $r_x = 0$ and relabeling $r_z = z$ and $m^\prime = m$ within
\eqref{eq:twopointxz_physical}, the two-point correlation as a
function of only separation in $z$ is
\begin{align}
R_{uv} \left( 0, y_j, z \right)
  &= \sum_{m}
     \sum_{n^\prime}
     \hat{u}^\ast_{y_j m n^\prime}
     \hat{v}_{y_j m n^\prime}
     e^{\ii k_n^\prime z}.
    \label{eq:twopointz_physical}
\end{align}
Fourier transforming in $z$,
\begin{align}
\left(\hat{R}_{uv}\right)_{y_j 0 n}
   =
    \frac{1}{L_z}
    \int_{-\frac{L_z}{2}}^{\frac{L_z}{2}}
    R_{uv} \left( 0, y_j, z \right)
    e^{-\ii k_n z}
    \,\mathrm{d}x
  &=
    \sum_{m}
    \hat{u}^\ast_{y_j m n}
    \hat{v}_{y_j m n}
    \label{eq:twopointz_wave}
\end{align}
where the summation is over $m \in \left\{-N_x/2, \dots, 0, \dots,
N_x/2-1\right\}$.

\subsection{Accounting for Hermitian symmetry}

When $u$ and $v$ are real-valued so is $R_uv$.  By utilizing the Hermitian
symmetry present in coefficients $\hat{u}_{y_j m n}$ and $\hat{v}_{y_j m n}$,
\eqref{eq:twopointx_wave} and \eqref{eq:twopointz_wave} may be computed at
reduced cost.  For example, exploiting $x$ symmetry when computing
\eqref{eq:twopointz_wave} yields
%
\small
\begin{align}
  \sum_{m=-\frac{N_x}{2}}^{\frac{N_x}{2}-1}
  \hat{u}^\ast_{y_j m n} \hat{v}_{y_j m n}
&=
    \sum_{m=0}^{\frac{N_x}{2}-1}
    \hat{u}^\ast_{y_j m n} \hat{v}_{y_j m n}
    +
    \sum_{m=-\frac{N_x}{2}}^{-1}
    \hat{u}^\ast_{y_j m n} \hat{v}_{y_j m n}
\\ &=
    \sum_{m=0}^{\frac{N_x}{2}-1}
    \hat{u}^\ast_{y_j m n} \hat{v}_{y_j m n}
    +
    \sum_{m=1}^{\frac{N_x}{2}}
    \hat{u}_{y_j m n} \hat{v}^\ast_{y_j m n}
\\ &=
    \hat{u}^\ast_{y_j 0 n} \hat{v}_{y_j 0 n}
    +
    \sum_{m=1}^{\frac{N_x}{2}-1}
    \left[
      \hat{u}^\ast_{y_j m n} \hat{v}_{y_j m n}
      +
      \hat{u}_{y_j m n} \hat{v}^\ast_{y_j m n}
    \right]
    +
    \hat{u}_{y_j \frac{N_x}{2} n} \hat{v}^\ast_{y_j \frac{N_x}{2} n}.
\\ &=
      \left(
        \operatorname{Re}
        \hat{u}_{y_j 0 n}
      \right)
      \left(
        \operatorname{Re}
        \hat{v}_{y_j 0 n}
      \right)
    +
    2
    \sum_{m=1}^{\frac{N_x}{2}-1}
    \left[
      \left(
        \operatorname{Re}
        \hat{u}_{y_j m n}
      \right)
      \left(
        \operatorname{Re}
        \hat{v}_{y_j m n}
      \right)
      +
      \left(
        \operatorname{Im}
        \hat{u}_{y_j m n}
      \right)
      \left(
        \operatorname{Im}
        \hat{v}_{y_j m n}
      \right)
    \right]
    +
      \left(
        \operatorname{Re}
        \hat{u}_{y_j \frac{N_x}{2} n}
      \right)
      \left(
        \operatorname{Re}
        \hat{v}_{y_j \frac{N_x}{2} n}
      \right)
\end{align}
\normalsize
%
where the $0$ and $N_x / 2$ products simplify because they lack imaginary
content.

\section{One-dimensional spectra
         \citep[\textsection{}6.5]{Pope2000Turbulent}}

%FIXME: Review this one, in particular the claimed generalization
\todo[inline]{Incorrect generalization and transform scaling-- see Redmine \#2998 update 6}
Here we consider a generalization (in the sense that we will apply the formula
to any two fields) of the definiton for one-dimensional velocity spectra given
in Pope \citep[\textsection{}6.3]{Pope2000Turbulent}, where the fields are any
two velocity components.  The one-dimensional spectra along the
$\vec{r}_{xz1}$-direction  $\hat{E}_{u,v}(y_j,\vec{k}_{xz1})$ are defined to be twice
the one-dimensional Fourier transform of $R_{u,v}(y_j,\vec{r}_{xz1})$:
%
\begin{equation}
\hat{E}_{uv}(y_j,\vec{k}_{xz1}) = \frac{1}{\pi} \int_{-\infty}^{\infty}
                           R_{uv}(y_j,\vec{r}_{xz1})
                           e^{-i \vec{k}_{xz1} \vec{r}_{xz1}} \,d\!r_{xz1}
\end{equation}
%

\newcommand*{\doi}[1]{\href{http://dx.doi.org/\detokenize{#1}}{doi: #1}}
\bibliographystyle{plainnat}
\bibliography{references}

\end{document}
