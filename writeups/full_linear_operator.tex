\documentclass[letterpaper,11pt,nointlimits,reqno,draft]{amsart}

% Avoid "Too many math alphabets used in version normal" issue
\newcommand\hmmax{0}
\newcommand\bmmax{0}

% Load the color package first to avoid option clashes
\usepackage[usenames,dvipsnames,svgnames,table]{xcolor}

% General packages
\usepackage{accents}
\usepackage{algorithm}
\usepackage{algorithmic}
\usepackage{amsfonts}
\usepackage{fixltx2e,amsmath}
\usepackage{amssymb}
\usepackage{bm}
\usepackage{enumerate}
\usepackage{fancyhdr}
\usepackage{floatpag}
\usepackage{fullpage}
\usepackage[final]{graphicx}
\usepackage{ifthen}
\usepackage{lastpage}
\usepackage{latexsym}
\usepackage{mathrsfs}
\usepackage{mathtools}
\usepackage[numbers,sort&compress]{natbib}
\usepackage{parskip}
\usepackage{pstricks}
\usepackage{rotating}
\usepackage{setspace}
\usepackage{txfonts}
\usepackage{units}
\usepackage{varioref}
\usepackage{wrapfig}
\usepackage{yhmath}

\usepackage[obeyDraft,textsize=scriptsize]{todonotes}

% In conjunction with -shell-escape, automatically convert EPS to PDF
\usepackage{epstopdf}
\epstopdfsetup{outdir=./,suffix=-generated,update,verbose}
\epstopdfDeclareGraphicsRule{.eps}{pdf}{.pdf}{%
    epstopdf --outfile=\OutputFile \space `kpsewhich \space "\SourceFile"`
}

% Hyperref package must be last otherwise the contents are jumbled
% hypertexnames disabled to fix links pointing to incorrect locations
\usepackage[colorlinks=true,
            linkcolor=blue,
            urlcolor=blue,
            citecolor=blue,
            final,
            hypertexnames=false]{hyperref}

% Fix Todonotes wrongly placed in the margin
\setlength{\marginparwidth}{2cm}

% Environment sidewaysfigure from rotating plays poorly with amsart class
% Fix per http://www.latex-community.org/forum/viewtopic.php?f=4&t=1742
\setlength\rotFPtop{0pt plus 1fil}

\mathtoolsset{showonlyrefs,showmanualtags}
%%% \allowdisplaybreaks[1] % Allow grouped equations to be split across pages

% Permit \eqref inside moving environments
% http://tex.stackexchange.com/questions/61764/eqref-in-captions-with-mathtools
\MakeRobust{\eqref}

% Line Spacing
\singlespacing

% Increase table of contents depth
\setcounter{tocdepth}{4}

% Simplify headings on floating pages
\floatpagestyle{plain}
\rotfloatpagestyle{empty}

% Document-specific commands
\newcommand{\ii}{\ensuremath{\mathrm{i}}}
\newcommand{\trans}[1]{{#1}^{\ensuremath{\mathsf{T}}}}
\newcommand{\Knudsen}[1][]{\ensuremath{\mbox{Kn}_{#1}}}
\newcommand{\Mach}[1][]{\ensuremath{\mbox{Ma}_{#1}}}
\newcommand{\Reynolds}[1][]{\ensuremath{\mbox{Re}_{#1}}}
\newcommand{\Prandtl}[1][]{\ensuremath{\mbox{Pr}_{#1}}}
\newcommand{\reference}[1]{\ensuremath{\left\{#1\right\}_{0}}}
\newcommand{\lessreference}[1]
  {\ensuremath{\left({#1}-\reference{#1}\right)}}
\newcommand{\symmetricpart}[1]
  {\ensuremath{\operatorname{sym}\left(#1\right)}}
\DeclareMathOperator{\trace}{tr}
\newcommand{\Ssd}{\ensuremath{\mathcal{S}}} % source term from slow derivative
\newcommand{\Cs}{\ensuremath{\mathcal{C}}}  % source term from integral constraints

\begin{document}

Our full matrix M,
\begin{equation}
\tiny
M = 
\left(
\begin{array}{cccccc}
 v & -\frac{v \rho _{\alpha }}{\rho } & 0 & \frac{\rho _{\alpha }}{\rho } & 0 & 0 \\
 0 & v & 0 & 1 & 0 & 0 \\
 0 & -u v & v & u & 0 & 0 \\
 (\gamma -1) \left(\frac{1}{2} \left(u^2+v^2+w^2\right)-e_{\alpha }\right)+T \left(R_{\alpha }-R\right) & -v^2+R T+\left(\frac{1}{2} \left(u^2+v^2+w^2\right)-e\right) (\gamma
   -1) & -u (\gamma -1) & 2 v-v (\gamma -1) & -w (\gamma -1) & \gamma -1 \\
 0 & -v w & 0 & w & v & 0 \\
 v \left((\gamma -1) \left(\frac{1}{2} \left(u^2+v^2+w^2\right)-e_{\alpha }\right)+T \left(R_{\alpha }-R\right)\right) & v \left(R T+\left(\frac{1}{2}
   \left(u^2+v^2+w^2\right)-e\right) (\gamma -1)\right)-\text{vH} & -u v (\gamma -1) & H-v^2 (\gamma -1) & -v w (\gamma -1) & (\gamma -1) v+v
\end{array}
\right)
\end{equation}

From this, we construct a linearized matrix of the form,
\begin{equation}
\tiny
L = 
\left(
\begin{array}{cccccc}
 0 & 0 & 0 & 0 & 0 & 0 \\
 0 & v & 0 & 0 & 0 & 0 \\
 0 & -u v & 0 & u & 0 & 0 \\
 0 & -v^2+R T+\left(\frac{1}{2} \left(u^2+v^2+w^2\right)-e\right) (\gamma -1)+\frac{\left((\gamma -1) \left(\frac{1}{2} \left(u^2+v^2+w^2\right)-e_{\alpha }\right)+T
   \left(R_{\alpha }-R\right)\right) \rho _{\alpha }}{\rho } & -u (\gamma -1) & v-v (\gamma -1) & -w (\gamma -1) & \gamma -1 \\
 0 & -v w & 0 & w & 0 & 0 \\
 0 & v \left(R T+\left(\frac{1}{2} \left(u^2+v^2+w^2\right)-e\right) (\gamma -1)\right)-\text{vH} & -u v (\gamma -1) & H-v^2 (\gamma -1) & -v w (\gamma -1) & v (\gamma -1)
\end{array}
\right)
\end{equation}

The next page details how this matrix is actually organized in
suzerain. 

\newpage
\begin{sidewaysfigure}
\newcommand{\entry}[1]{}          % Provides comments for subblocks
\newcommand{\C}[2]{C^{#1}_{#2}}   % For brevity below
\newcommand{\D}[1]{D^{(#1)}}      % ditto
\newcommand{\M}{M}                % ditto
\newcommand{\g}{\gamma}           % ditto
\newcommand{\km}{k_{m}}           % ditto
\newcommand{\kn}{k_{n}}           % ditto
\newcommand{\mx}{m_{x}}           % ditto
\newcommand{\my}{m_{y}}           % ditto
\newcommand{\mz}{m_{z}}           % ditto
\newcommand{\vp}{\varphi}         % ditto
\newcommand{\subcoeff}[3]{{       % ditto
   \renewcommand{\arraystretch}{2.0}
   \begin{Bmatrix}{#1}\\{#2}\\{#3}\end{Bmatrix}
}}
\newcommand{\Ma}{\ensuremath{\mbox{\small{}Ma}}}
\renewcommand{\Pr}{\ensuremath{\mbox{\small{}Pr}}}
\renewcommand{\Re}{\ensuremath{\mbox{\small{}Re}}}
\newcommand{\ind}[1]{\textcolor{BrickRed}{#1}}  % Wavenumber independent term
\hspace{-.04\textwidth}
{\resizebox{1.08\textwidth}{!}{\begin{minipage}[c]{\textwidth}  % SCALE-TO-FIT
\begin{align*}
\ind{\bm{\vp}}
\renewcommand{\arraystretch}{1.0} % Adds whitespace between rows
\addtolength{\arraycolsep}{-.1em}
\begin{bmatrix}
%
% species density row
%
% is this here?
  \entry{\rho_{\alpha}\rho_{\alpha}}
  \subcoeff{
 \ind{\frac{1}{\vp}}
  }{
  }{
  }
& \entry{\rho\rho_{\alpha} }
 \ind{0}
& \entry{\rho\mx }
 \ind{0}
& \entry{\rho\my }
 \ind{0}
& \entry{\rho\mz }
 \ind{0}
& \entry{\rho{}e }
  \ind{0}
%
% Density row
%
 \\\entry{\rho_{\alpha}\rho}
 \ind{0}
& \entry{\rho\rho }
  \subcoeff{
 \ind{\frac{1}{\vp}}
  }{
 \ind{v}
  }{
  }
& \entry{\rho\mx }
 \ind{0}
& \entry{\rho\my }
 \ind{0}
  % \subcoeff{
  % }{
  %   \ind{- 1}
  % }{
  % }
& \entry{\rho\mz }
 \ind{0}
& \entry{\rho{}e }
  \ind{0}
% Streamwise momentum row
\\\entry{\mx\rho_{\alpha} }
 \ind{0}
& \entry{\mx\rho }
  \subcoeff{
  }{
 \ind{-uv}
  }{
  }
& \entry{\mx\mx  }
  \subcoeff{
      \ind{\bm{\frac{1}{\vp}}} % M
  }{
  }{
    \ind{  \frac{1}{\Re}\C{\nu}{}}
  }
& \entry{\mx\my  }
  \subcoeff{
  }{
    \ind{u}
  }{
  }
& \entry{\mx\mz  }
 0
& \entry{\mx{}e  }
 0
% Wall-normal momentum row
\\\entry{\my\rho_{\alpha} }
 \ind{0}
& 
\entry{\my\rho }
  \subcoeff{
  }{
      \ind{\frac{1-\g}{2}\C{u^2}{}}
    \ind{+} \ind{\C{u_y u_y}{}}
  }{
  }
& \entry{\my\mx  }
  \subcoeff{
  }{
      \ind{\left(\g-1\right)\C{u_x}{}}
  }{
  }
& \entry{\my\my  }
  \subcoeff{
      \ind{\bm{\frac{1}{\vp}}} % M
  }{
      \ind{\left(\g-3\right)\C{u_y}{}}
  }{
      \ind{\frac{\alpha+\frac{4}{3}}{\Re}\C{\nu}{}}
  }
& \entry{\my\mz  }
  \subcoeff{
  }{
      \ind{\left(\g-1\right)\C{u_z}{}}
  }{
  }
& \entry{\my{}e  }
  \subcoeff{
  }{
      \ind{\frac{1-\g}{\Ma^2}}
  }{
  }
% Spanwise momentum row
\\\entry{\mz\rho_{\alpha} }
 \ind{0}
& 
\entry{\mz\rho}
  \subcoeff{
  }{
    \ind{-vw}
  }{
  }
& \entry{\mz\mx }
 0
& \entry{\mz\my }
  \subcoeff{
  }{
    \ind{w}
  }{
  }
& \entry{\mz\mz }
  \subcoeff{
      \ind{\bm{\frac{1}{\vp}}} % M
  }{
 0 
  }{
    \ind{  \frac{1}{\Re}\C{\nu}{}}
  }
& \entry{\mz{}e }
  \subcoeff{
  }{
  }{
  }
% Total energy row
\\\entry{e\rho_{\alpha} }
 \ind{0}
& 
\entry{e\rho  }
  \subcoeff{
  }{
    \ind{- \C{e_y}{\nabla\rho}}
  }{
    \ind{
    }
  }
& \entry{e\mx   }
  \subcoeff{
  }{
  }{
  }
& \entry{e\my   }
  \subcoeff{
  }{
    \ind{- \C{e}{\nabla\cdot{}m}}
  }{
  }
& \entry{e\mz   }
  \subcoeff{
  }{
  }{
  }
& \entry{ee     }
  \subcoeff{
      \ind{\bm{\frac{1}{\vp}}} % M
  }{
    \ind{- \g\C{u_y}{}}
  }{
    \ind{  \frac{\g}{\Re\Pr}\C{\nu}{}}
  }
\end{bmatrix}
\renewcommand{\arraystretch}{0.6}
\begin{bmatrix}
  \hat{\rho_{\alpha}}_{\left(0,\,m,\,n\right)} \\
  \vdots \\
  \hat{\rho_{\alpha}}_{\left(N_y-1,\,m,\,n\right)} \\
\\%
\\%
\\%
\\%
  \hat{\rho}_{\left(0,\,m,\,n\right)} \\
  \vdots \\
  \hat{\rho}_{\left(N_y-1,\,m,\,n\right)} \\
\\%
\\%
\\%
\\%
  \hat{\mx}_{\left(0,\,m,\,n\right)} \\
  \vdots \\
  \hat{\mx}_{\left(N_y-1,\,m,\,n\right)} \\
\\%
\\%
\\%
\\%
  \hat{\my}_{\left(0,\,m,\,n\right)} \\
  \vdots \\
  \hat{\my}_{\left(N_y-1,\,m,\,n\right)} \\
\\%
\\%
\\%
\\%
  \hat{\mz}_{\left(0,\,m,\,n\right)} \\
  \vdots \\
  \hat{\mz}_{\left(N_y-1,\,m,\,n\right)} \\
\\%
\\%
\\%
\\%
  \hat{e}_{\left(0,\,m,\,n\right)} \\
  \vdots \\
  \hat{e}_{\left(N_y-1,\,m,\,n\right)} \\
%
\end{bmatrix}
\end{align*}
\end{minipage}}}  % END SCALE-TO-FIT!
\vspace{2em}
\\
\caption[The discrete operator $M+\varphi{}L$ used for implicit time advance]
{
    The complete discrete operator $M+\varphi{}L$ used for implicit time advance is
    depicted.  Notice the leftmost scalar factor $\bm{\vp}$.  The $3 N_y \times
    N_y$ blocked vectors surrounded by curly braces are to be ``dotted'' against
    the blocked vector $ \trans{\begin{bmatrix} \M & \D{1} & \D{2} \end{bmatrix}} $
    to form $N_y \times N_y$ subblocks.  Each of $M$, $\D{1}$, and $\D{2}$ is a
    $N_y \times N_y$ banded matrix. The complex-valued, wavenumber-dependent
    operator takes wall-normal B-spline coefficients to B-spline collocation point
    values.
}
\label{fig:discreteimplicitop}
\end{sidewaysfigure}

\end{document}