\documentclass[letterpaper,11pt,nointlimits,reqno]{amsart}

% Packages
\usepackage{accents}
\usepackage{algorithm}
\usepackage{algorithmic}
\usepackage{amsfonts}
\usepackage{amsmath}
\usepackage{amssymb}
\usepackage{cancel}
\usepackage{enumerate}
\usepackage{fancyhdr}
\usepackage{fullpage}
\usepackage{ifthen}
\usepackage{lastpage}
\usepackage{latexsym}
\usepackage{mathrsfs}
\usepackage{mathtools}
\usepackage[numbers,sort&compress]{natbib}
\usepackage{parskip}
\usepackage{pstricks}
\usepackage{setspace}
\usepackage{txfonts}
\usepackage{units}
\usepackage{varioref}
\usepackage{wrapfig}

% Hyperref package must be last otherwise the contents are jumbled
% hypertexnames disabled to fix links pointing to incorrect locations
\usepackage[hypertexnames=false,final]{hyperref}

\mathtoolsset{showonlyrefs,showmanualtags}
\allowdisplaybreaks[1] % Allow grouped equations to be split across pages

% Line Spacing
\singlespacing

% Set appropriate header/footer information on each page
\fancypagestyle{plain}{
    \fancyhf{}
    \renewcommand{\headheight}{2.0em}
    \renewcommand{\headsep}{0.75em}
    \renewcommand{\headrulewidth}{1.0pt}
    \renewcommand{\footrulewidth}{0pt}
    \lhead{
        Suzerain isothermal flat plate treatment
    }
    \rhead{
        Page \thepage{} of \pageref{LastPage}
    }
}
\pagestyle{plain}

% Document-specific commands
\newcommand{\Mach}[1][]{\mbox{Ma}_{#1}}
\newcommand{\Reynolds}[1][]{\mbox{Re}_{#1}}
\newcommand{\Cov}[2]{\ensuremath{\operatorname{Cov}\left[{#1},{#2}\right]}}
\newcommand{\Var}[1]{\ensuremath{\operatorname{Var}\left[{#1}\right]}}

\begin{document}

This document describes the treatment of the isothermal freestream boundary for
a compressible, nondimensional flat plate problem computed using Suzerain.

\section{Ideal scenario definition}

The scenario has a mean freestream characterized by a Mach number
$\Mach[\infty]$ and Reynolds number $\Reynolds[\infty]$:
\begin{align}
  \Mach[\infty]{}
  &= \Mach{} \overline{\,\frac{u}{a}\,}
   = \Mach{} \overline{\,\frac{u}{\sqrt{T}}\,}
   = \Mach\, \overline{u T^{-1/2}}
\\
  \Reynolds[\infty]{}
  &= \Reynolds{} \overline{\,\frac{\rho u \delta\,} {\mu}}
   = \Reynolds{} \overline{\,\frac{\rho u \delta\,} {T^\beta}}
   = \Reynolds\, \overline{\rho u \delta T^{-\beta}}
\end{align}
Here, $\Mach$ and $\Reynolds$ are Suzerain code parameters and $\delta$ is the
boundary layer thickness.  The constitutive laws
\begin{align}
  a &= \sqrt{T}
&\mu &= {T}^\beta
\end{align}
have been used to re-express the sound speed and viscosity using temperature
because we desire
\begin{align}
  \label{eq:ideal_T}
  \bar{T} &= T_\infty
\end{align}
for some given constant $T_\infty$.  Defining and substituting
\begin{align}
  \Mach[\infty]{} &= \Mach
& \Reynolds[\infty]{} &= \Reynolds
& \delta &= 1
\end{align}
produces two statements about mean freestream nondimensional state implied by
the scenario definition:
\begin{align}
  \label{eq:ideal_u}
  1 &= \overline{u T^{-1/2}}
\\
  \label{eq:ideal_rhou}
  1 &= \overline{\rho u T^{-\beta}}
\end{align}
Examining $\Mach[\infty]{}^2$ one can likewise obtain
\begin{align}
  \label{eq:ideal_u2}
  1 &= \overline{u^2 T^{-1}}
\end{align}
which constraints the streamwise contribution to the kinetic energy.  Ideally,
one would implement an isothermal freestream by strongly enforcing
\eqref{eq:ideal_T}, \eqref{eq:ideal_u}, and~\eqref{eq:ideal_rhou}.
Unfortunately, exactly achieving these three constraints is difficult as the
spectral nature of Suzerain and its state choices make constraining only
$\overline{\rho}$, $\overline{\rho{}u}$, and $\overline{\rho{}E}$ simple.

\section{Approximate scenario definition}

All of \eqref{eq:ideal_u}, \eqref{eq:ideal_rhou}, and~\eqref{eq:ideal_u2}
possess the form $\overline{c T^p}$ for spatially-varying $c$ and constant $p$.
A second-order expansion about $\bar{c}$ and $\bar{T}$ yields
\begin{align}
  c T^p &= \bar{c} \bar{T}^p
         + \left(c - \bar{c}\right) \bar{T}^p
         + \left(T - \bar{T}\right) \bar{c} p \bar{T}^{p-1}
\\      &+ \frac{1}{2} p \bar{T}^{p-1} \left(c - \bar{c}\right)\left(T - \bar{T}\right)
         + \frac{1}{2} \bar{c} p \left(p-1\right) \bar{T}^{p-2} \left(T - \bar{T}\right)^2
         + \dots
\end{align}
where the $\mathcal{O}\left[c - \bar{c}\right]^2$ does not appear because it
has leading coefficient zero.  Averaging,
\begin{align}
  \overline{c T^p} &= \bar{c} \bar{T}^p
                    + \frac{1}{2} p \bar{T}^{p-1} \Cov{c}{T}
                    + \frac{1}{2} \bar{c} p \left(p-1\right) \bar{T}^{p-2} \Var{T}
                    + \dots
.
\end{align}
Therefore, making the approximation
\begin{align}
  \label{eq:damage}
  \overline{c T^p} &\approx \bar{c} \bar{T}^p
\end{align}
does not require $\Var{c} = 0$ but it does force neglecting $\Cov{c}{T}$, $\Var{T}$,
and all higher moments.

We apply \eqref{eq:damage} followed by~\eqref{eq:ideal_T} to conditions
\eqref{eq:ideal_u}, \eqref{eq:ideal_rhou}, and~\eqref{eq:ideal_u2} to obtain
three approximate relationships:
\begin{align}
  \label{eq:approx_u}
  1 &\approx \bar{u} T_\infty^{-1/2}
  & &\implies
  & \bar{u}_\infty &\approx \sqrt{T_\infty}
\\
  \label{eq:approx_rhou}
  1 &\approx \overline{\rho u} T_\infty^{-\beta}
  & &\implies
  & \overline{\rho u}_\infty &\approx T_\infty^\beta
\\
  \label{eq:approx_u2}
  1 &\approx \overline{u^2} T_\infty^{-1}
  & &\implies
  & \overline{u^2}_\infty &\approx T_\infty
\end{align}
These statements are not unreasonable as $\Cov{u}{T}$, $\Cov{\rho u}{T}$, and
$\Var{T}$ should be small at the freestream.

Because~\eqref{eq:approx_u} constrains something other than mean conserved
state, it is difficult to implement.  Statements $\eqref{eq:approx_u}$ and
$\eqref{eq:approx_rhou}$ may be combined into an approximate constraint for
$\bar{\rho}_\infty$:
\begin{equation}
  \bar{\rho}_\infty
  =
  \frac{\bar{\rho}_\infty \bar{u}_\infty}
       {\bar{u}_\infty}
  \approx
  \frac{\bar{\rho}_\infty \bar{u}_\infty + \Cov{\rho_\infty}{u_\infty}}
       {\bar{u}_\infty}
  =
  \frac{\overline{\rho u}_\infty}
       {\bar{u}_\infty}
  \approx
  \frac{T_\infty^{-1/2}}{T_\infty^{-\beta}}
  =
  T_\infty^{\beta-1/2}
\end{equation}
Notice the useful identity
\begin{equation}
  \overline{xy} = \overline{xy} - \bar{x} \bar{y} + \bar{x} \bar{y}
                = \Cov{x}{y} + \bar{x} \bar{y}
\end{equation}
was employed and $\Cov{\rho_\infty}{u_\infty}$ neglected to produce this
result.




%%Expanding $c T^p$ in a Taylor series about mean values $\bar{c}$ and
%%$\bar{T}$,
%%\begin{align}
%%  c T^p &= \bar{c} \bar{T}^p
%%\\      &+ \bar{T}^p \left(c - \bar{c}\right)
%%         + p \bar{c} \bar{T}^{p-1} \left(T - \bar{T}\right)
%%\\      &+ \frac{1}{2} p \left(p-1\right) \bar{c} \bar{T}^{p-2} \left(T-\bar{T}\right)^2
%%\\      &+ p \bar{T}^{p-1} \left(c - \bar{c}\right) \left(T - \bar{T}\right)
%%\\      &+ \frac{1}{2} p \left(p-1\right)         \bar{T}^{p-2} \left(c - \bar{c}\right) \left(T-\bar{T}\right)^2
%%\\      &+ \mathcal{O}\left[c-\bar{c}\right]^3
%%         + \mathcal{O}\left[T-\bar{T}\right]^3
%%.
%%\intertext{Averaging, }
%%  \overline{c T^p} &= \bar{c} \bar{T}^p
%%\\      &+ \frac{1}{2} p \left(p-1\right) \bar{c} \bar{T}^{p-2} \Var{T}
%%\\      &+ p \bar{T}^{p-1} \Cov{c}{T}
%%\\      &+ \frac{1}{2} p \left(p-1\right)         \bar{T}^{p-2} \overline{\left(c - \bar{c}\right) \left(T-\bar{T}\right)^2}
%%\\      &+ \mathcal{O}\overline{\left[c-\bar{c}\right]}^3
%%         + \mathcal{O}\overline{\left[T-\bar{T}\right]}^3
%%.
%%\end{align}

%% REHASH BELOW HERE
\section{Rehash below here}


% Based on Rhys' personal notes dated 9 May 2013

Freestream Mach number $\Mach[\infty]$ and Reynolds number $\Reynolds[\infty]$
may be related to Suzerain code parameters $\Mach$ and $\Reynolds$ and
nondimensional state as follows:
\begin{align}
  \Mach[\infty]{}
  &=         \frac{u_\infty     }{a_\infty     }
   = \Mach{} \frac{u_\infty^\ast}{a_\infty^\ast}
&
  \Reynolds[\infty]{}
  &=             \frac{\rho_\infty      u_\infty      \delta     }
                      {\mu_\infty     }
   = \Reynolds{} \frac{\rho_\infty^\ast u_\infty^\ast \delta^\ast}
                      {\mu_\infty^\ast}
\end{align}
Here, $\delta$ and $\delta^\ast$ are, respectively, a dimensional and
nondimensional boundary layer thickness.
Defining
\begin{align}
  \Mach[\infty]{} &= \Mach
& \Reynolds[\infty]{} &= \Reynolds,
& \delta^\ast &= 1
\end{align}
implies
\begin{align}
  u_\infty^\ast   &= a_\infty^\ast,
&
  \rho_\infty^\ast u_\infty^\ast &= \mu_\infty^\ast
.
\end{align}
In this nondimensional setting the constitutive laws
\begin{align}
  a^\ast &= \sqrt{T^\ast}
&\mu^\ast &= {T^\ast}^\beta
\end{align}
may be combined with the prior expressions to obtain the following:
\begin{align}
  \label{eq:u_target}
  u_\infty^\ast &= \sqrt{T_\infty^\ast}
\\
  \label{eq:rho_target}
  \rho_\infty^\ast &= \frac{\mu_\infty^\ast}{u_\infty^\ast}
                    = \frac{{T_\infty^\ast}^\beta}{\sqrt{T_\infty^\ast}}
                    = {T_\infty^\ast}^{\left(\beta-\frac{1}{2}\right)}
\\
  \label{eq:mx_target}
  \rho_\infty^\ast u_\infty^\ast &= {T_\infty^\ast}^\beta
\\
  \label{eq:ke_target}
  \rho_\infty^\ast {u_\infty^\ast}^2
&=
  {T_\infty^\ast}^{\left(\beta + \frac{1}{2}\right)}
\end{align}
Enforcing these three nondimensional constraints at the freestream will achieve
a target $\Mach$ and $\Reynolds$ provided that temperature $T_\infty^\ast$ is
somehow constrained.

\section{Obtaining freestream total energy}

Unlike the momentum \eqref{eq:mx_target} and density \eqref{eq:rho_target}
constraints, there is seemingly no crisp way to obtain a clean, constant target
value for the total energy which enforces an isothermal-in-the-mean freestream.
By the \emph{nondimensional} equation of state,
\begin{align}
  \rho E &= \frac{p}{\gamma-1}
          + \frac{\Mach^2}{2}\left(\rho{}u^2 + \rho{}v^2 + \rho{}w^2\right)
\\
         &= \frac{\rho T}{\gamma\left(\gamma-1\right)}
          + \frac{\Mach^2}{2}\left(\rho{}u^2 + \rho{}v^2 + \rho{}w^2\right)
.
\end{align}
Considering the freestream and both using \eqref{eq:rho_target} and
\eqref{eq:ke_target},
\begin{align}
  \left(\rho E\right)_\infty
         &= \frac{\rho_\infty T_\infty}{\gamma\left(\gamma-1\right)}
          + \frac{\Mach^2}{2}\left(  \rho_\infty u_\infty^2
                                   + \rho_\infty v_\infty^2
                                   + \rho_\infty w_\infty^2\right)
\\
         &= \frac{T_\infty^{\beta+\frac{1}{2}}}{\gamma\left(\gamma-1\right)}
          + \frac{\Mach^2}{2}\left(  T_\infty^{\beta+\frac{1}{2}}
                                   + \rho_\infty v_\infty^2
                                   + \rho_\infty w_\infty^2\right)
\\
\label{eq:e_target}
         &= \left(  \frac{1}{\gamma\left(\gamma-1\right)}
                  + \frac{\Mach^2}{2} \right)
             T_\infty^{\beta+\frac{1}{2}}
          + T_\infty^{\beta-\frac{1}{2}} \left(v_\infty^2 + w_\infty^2\right)
.
\end{align}

\section{Averaging}

Wishing to permit fluctuations about a nominal state, we weaken
\eqref{eq:rho_target}, \eqref{eq:mx_target}, and \eqref{eq:e_target} to be
statements only about mean nondimensional state at the freestream:
\begin{align}
   \bar{\rho}_\infty &= \overline{{T_\infty}^{\beta-\frac{1}{2}}}
\\
                     &\approx {\bar{T}_\infty}^{\beta-\frac{1}{2}}
\\
  \overline{\rho u}_\infty &= \overline{{T_\infty}^\beta}
\\
                           &\approx {\bar{T}_\infty}^\beta
\\
  \overline{\rho E}_\infty
         &= \left(  \frac{1}{\gamma\left(\gamma-1\right)}
                  + \frac{\Mach^2}{2} \right)
            \overline{T_\infty^{\beta+\frac{1}{2}}}
          + \overline{T_\infty^{\beta-\frac{1}{2}} v_\infty^2}
          + \overline{T_\infty^{\beta-\frac{1}{2}} w_\infty^2}
\\
         &\approx \left(  \frac{1}{\gamma\left(\gamma-1\right)}
                        + \frac{\Mach^2}{2} \right)
            \bar{T}_\infty^{\beta+\frac{1}{2}}
          + \bar{T}_\infty^{\beta-\frac{1}{2}}
            \left(\bar{v}_\infty^2 + \bar{w}_\infty^2\right)
\end{align}
The approximations made in the final statement for $\overline{\rho E}_\infty$
are appreciably more severe than those in the ones for $\bar{\rho}_\infty$ and
$\overline{\rho u}_\infty$.

\section{BACKUP: Constraining Freestream Total Energy Evolution}

% Based on Rhys' personal notes dated 8 May 2013

Observe
\begin{align}
  \overline{xy} &= \overline{xy}
\\              &= \overline{xy} - \bar{x} \bar{y} + \bar{x} \bar{y}
\\              &= \Cov{x}{y} + \bar{x} \bar{y}
.
\end{align}
Examining mean \emph{nondimensional} state evolution,
\begin{align}
  \partial_t \overline{\rho u} &= \partial_t \Cov{\rho}{u}
                                + \partial_t \bar{\rho} \bar{u}
\\
                               &= \partial_t \Cov{\rho}{u}
                                + \bar{\rho} \partial_t \bar{u}
                                + \bar{u}    \partial_t \bar{\rho}
\end{align}
As $\partial_t \Cov{\rho_\infty}{u_\infty} = 0$ and $\partial_t \bar{u}_\infty
= 0$ for a stationary, constant freestream,
\begin{align}
  \partial_t \overline{\rho u}_\infty
  &\approx \bar{u}_\infty \partial_t \bar{\rho}_\infty
\end{align}
Similarly,
\begin{align}
  \partial_t \overline{\rho w}_\infty
  &\approx \bar{w}_\infty \partial_t \bar{\rho}_\infty,
&
  \partial_t \overline{\rho E}_\infty
  &\approx \bar{E}_\infty \partial_t \bar{\rho}_\infty
  .
\end{align}
Now,
\begin{align}
  \bar{E}
  &= \frac{\bar{T}}{\gamma\left( \gamma-1 \right)}
   + \frac{\Mach^2}{2}\left(\overline{u^2}+\overline{v^2}+\overline{w^2}\right)
\\
  &= \frac{\bar{T}}{\gamma\left( \gamma-1 \right)}
   + \frac{\Mach^2}{2}\left(\bar{u}^2+\bar{v}^2+\bar{w}^2
                            +\Var{u} +\Var{v} +\Var{w}\right)
\\
  &= \frac{\bar{T}}{\gamma\left( \gamma-1 \right)}
   + \frac{\Mach^2}{2}\left(\bar{u}^2+\bar{v}^2+\bar{w}^2+2\bar{k}\right)
\end{align}
In a non-turbulent freestream, $\bar{k}_\infty\approx{}0$ suggests
using the approximation
\begin{align}
  \partial_t \overline{\rho E}_\infty
&\approx
  \left(
         \frac{\bar{T}_\infty}{\gamma\left( \gamma-1 \right)}
       + \frac{\Mach^2}{2}\left(  \bar{u}_\infty^2
                                 +\bar{v}_\infty^2
                                 +\bar{w}_\infty^2\right)
  \right)
  \partial_t \bar{\rho}_\infty
\end{align}
which permits using known $T_\infty$ and only mean velocity information
to enforce an approximately isothermal freestream condition via the
total energy equation.

\end{document}
