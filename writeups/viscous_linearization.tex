\documentclass[10pt]{article}

% useful packages
\usepackage{amssymb, amsmath, amsfonts, amsthm, wasysym} % math
\usepackage{epsfig, subfigure} % graphics
\usepackage{fullpage, url, color, parskip} % misc

% commands I like
\newcommand{\mbb}[1]{\mathbb{#1}}
\newcommand{\mbf}[1]{\mathbf{#1}}
\newcommand{\sbf}[1]{\boldsymbol{#1}}
\newcommand{\mcal}[1]{\mathcal{#1}}
\newcommand{\mfk}[1]{\mathfrak{#1}}
\newcommand{\pp}[2]{\frac{\partial #1}{\partial #2}}
\newcommand{\dd}[2]{\frac{d #1}{d #2}}
\newcommand{\rarrow}{\rightarrow}
\newcommand{\Rarrow}{\Rightarrow}
\newcommand{\LRarrow}{\Leftrightarrow}
\newcommand{\jump}[1]{\llbracket #1 \rrbracket}
\newcommand{\avg}[1]{\{ #1 \}}
\def\etal{{\it et al.~}}
\newcommand{\vvvert}{|\kern-1pt|\kern-1pt|}
\newcommand{\enorm}[1]{\vvvert #1 \vvvert}
\newcommand{\ud}{\,\mathrm{d}}
\newcommand{\pdv}[2]{{\frac{\partial{#1}}{\partial{#2}}}}
\newcommand{\sa}{\nu_{\mathrm{sa}}}

\newcommand{\symmetricpart}[1]
  {\ensuremath{\operatorname{sym}\left(#1\right)}}

\DeclareMathOperator{\trace}{tr}
\newcommand{\Ssd}{\ensuremath{\mathcal{S}}} % source term due to slow derivative

\newcommand{\trans}[1]{{#1}^{\ensuremath{\mathsf{T}}}}

%\newcommand{\red}[1]{\color{red}{#1}}
\newcommand{\myred}[1]{{\color{red} #1}}

% title and author
\title{Linearization of the Viscous Operator in Suzerain}
\author{Nicholas Malaya \\
        Rhys Ulerich \\         
	Todd Oliver \\ }

% \thanks{Institute for Computational
% Engineering and Sciences, The University of Texas at Austin,
% Austin, TX 78712 (nick@ices.utexas.edu)}}


% lets rock and roll
\begin{document}
\maketitle

%-------------------------------------------------
\section{Problem Statement}

We are attempting to linearize our reacting flow equations around some
reference state.  That is, decomposing
\begin{equation}
\partial_t u = R(u) = \left(R(u) - Lu\right) + Lu
\end{equation}
so that the linear operator $L$ consists of only reference coefficients times
differential operators.  For instance, supposing $R(\rho, m) = \Delta m / \rho$
we would construct $L$ per
\begin{equation}
 \frac{1}{\rho}\Delta m = (\frac{1}{\rho}-\frac{1}{\rho_0})\Delta m +
  \frac{1}{\rho_0}\Delta m
\end{equation}
We will perform this procedure on the Navier-Stokes equations. Let's
start with modifications to the current suzerain implementation. 

\section{Modifications to current Suzerain capabilities}

Second differentiation enters through the terms
$\nabla\cdot\tau$, $\nabla\cdot\tau\frac{m}{\rho}$, and
$\nabla\cdot\kappa\nabla{}T$.  We wish to compute these terms in a way that keeps
first and second derivative applications wholly separate.  Doing so will help
ensure that these three terms have the most physically correct diffusive impact
on high frequency content at a given spatial resolution.  These results will
also be used in the course of developing our implicit diffusive treatment.


Expanding the three mixed order, nonlinear terms and using the symmetry of
$\tau$:
\begin{align}
\label{eq:nabla_cdot_tau_expansion}
  \nabla\cdot\tau
  &=
    2 \symmetricpart{\nabla\frac{m}{\rho}}\nabla\mu
  + \mu \myred{\Delta\frac{m}{\rho}}
  + \left(\mu+\lambda\right)\nabla\nabla\cdot\frac{m}{\rho}
  + \left(\nabla\cdot\frac{m}{\rho}\right)\nabla\lambda
\\
\label{eq:nabla_cdot_tau_u_expansion}
  \nabla\cdot\tau{}\frac{m}{\rho}
  &=
    \frac{m}{\rho}\cdot\left(\myred{\nabla\cdot\tau}\right)
  + \trace\left( \trans{\tau}\nabla\frac{m}{\rho} \right)
\\
  \nabla\cdot\kappa\nabla{}T \label{eq:mu_delta_T}
  &=
    \nabla\kappa\cdot\nabla{}T
  + \kappa \myred{\Delta{}T}
\end{align}
The terms highlighted in red must be treated implicitly. The first,
$\Delta\frac{m}{\rho}$, is identical to the treatment currently in
suzerain. Likewise, $\nabla\cdot\tau$ must be identical to the treatment
in suzerain, for an argument identical to the previous discussion.

Finally, $\Delta{}T$. This can be expressed as, 
\begin{equation}
\Delta T = \nabla^2 T = \nabla \cdot (\nabla T) = \nabla \cdot
 (\frac{\partial T}{\partial U} \nabla U) =  \nabla (\frac{\partial
 T}{\partial U} )\cdot \nabla U + \frac{\partial T}{\partial U} \nabla^2 U  
\end{equation}
(U is conserved state, e.g. $U=\{\rho,\rho u,\rho v, \rho w, \rho E
\}$) This requires $\frac{\partial T}{\partial U}$. T is functionally
dependent on the energy and species concentrations, e.g.
\begin{equation}
 T = f(e,Y)
\end{equation}
Therefore, the derivative must be,
\begin{equation}
 \frac{\partial T}{\partial U} = \frac{\partial f}{\partial
  e}\frac{\partial e}{\partial U} + \frac{\partial f}{\partial
  Y}\frac{\partial Y}{\partial U}
\end{equation}

%
% 
%
\subsection{$\frac{\partial f}{\partial e}$}

The raw computation of T is attained by inverting the following equation
for a given species concentration,
\begin{equation}
e = e_{tr}(T,Y) + e_{ve}(T,Y) + e_{0}(Y)
\end{equation}
Thus, if we perturb e by $\delta e$ and examine the effect of T for
fixed Y, and expand as a Taylor Series,
\begin{align}
e + \delta e &= e_{tr}(T + \delta T,Y) + e_{ve}(T+\delta T,Y) + e_{0}(Y) \\
 &\approx e_{tr}(T,Y) + \frac{\partial e_{tr}}{\partial T} \delta T +
 e_{ve}(T,Y)+ \frac{\partial e_{ve}}{\partial T} \delta T + e_{0}(Y)
\end{align}
Therefore, 
\begin{align}
\delta e &\approx \left( \frac{\partial e_{tr}}{\partial T} +
 \frac{\partial e_{ve}}{\partial T} \right) \delta T \\ 
 &= c_v(T,Y)
\end{align}
and
\begin{equation}
 \frac{\partial f}{\partial e} = \frac{\partial T}{\partial e} = \frac{1}{\partial c_v}
\end{equation}

%
% 
%
\subsection{$\frac{\partial f}{\partial Y_i}$}

Now, we perturb $Y_i$ as well as T, and again expand as a Taylor
series. 
\begin{align}
 e &= e_{tr}(T+\delta T, Y + \delta Y ) + e_{ve}(T+\delta T, Y + \delta Y ) + e_{0}(Y + \delta Y) \\
 &\approx e_{tr}(T,Y) +  e_{ve}(T,Y) + e_{0}(Y) \\
 & + \frac{\partial e_{tr}}{\partial T} \delta T + \frac{\partial e_{tr}}{\partial Y_i} \delta Y_i
   + \frac{\partial e_{ve}}{\partial T} \delta T + \frac{\partial e_{ve}}{\partial Y_i} \delta Y_i 
   + \frac{\partial e_{o}}{\partial Y_i} \delta Y_i 
\end{align}
Subtracting off e leaves only,
\begin{equation}
0 \approx \left( \frac{\partial e_{tr}}{\partial T} + \frac{\partial
	   e_{ve}}{\partial T} \right) \delta T + 
           \left( \frac{\partial e_{tr}}{\partial Y_i} + \frac{\partial
	    e_{ve}}{\partial Y_i} + \frac{\partial
	    e_{0}}{\partial Y_i} \right) \delta Y_i
\end{equation}
The first term is the heat capacity at constant volume ($C_v(T,Y_i)$) and the
second is the internal energy of the ith species ($e_i(T)$), e.g. 
\begin{equation}
0 \approx C_v(T,Y_i) \delta T + 
          e_i(T)     \delta Y_i
\end{equation}
Solving for $\frac{\partial T}{\partial Y_i}$ gives, 
\begin{align}
 C_v \delta T = - e_i \delta Y_i \\
 \frac{\partial T}{\partial Y_i} = - \frac{e_i}{C_v}
\end{align}
%
% 
%
\subsection{$\frac{\partial e}{\partial U}$}

%
% 
%
\subsection{$\frac{\partial Y}{\partial U}$}


% Finally, $\Delta{}T$. This will be identical to the current treatment in
% suzerain if $\gamma$ is constant, e.g. $T=\gamma \frac{p}{\rho}$
% \begin{equation}
%  \Delta{}T = \Delta{}\left(\gamma \frac{p}{\rho}\right) = \gamma \Delta{}\left(\frac{p}{\rho}\right)
% \end{equation}
% In the event $\gamma$ is not constant, then we will require the following modification
% \begin{equation}
%  \Delta{}T = \Delta{}\left(\gamma \beta \right) = \Delta{}\beta + 2
%   \nabla \gamma \nabla \beta + \gamma \Delta{} \beta
% \end{equation}
% We omit treatment of the middle non-second derivative term, and
% therefore our expression becomes
% \begin{equation}
%  \Delta{}T = \myred{\beta \Delta{}\gamma}  + \gamma \Delta{} \beta
% \end{equation}
% Where $\beta = \frac{p}{\rho}$, and the term in red is a new addition to
% Suzerain. 

\section{Species Equations}

Linearization for species mass fraction equations and the
energy equation will need to be added to suzerain. 

Species Equations:
\begin{equation}
 \frac{\partial}{\partial t} \rho_s + \frac{\partial}{\partial
  x_i}\left(\rho_s u_i\right) = \frac{\partial}{\partial x_i}\left(\rho D_s
  \frac{\partial c_s}{\partial x_i} \right) + \dot \omega_s
\end{equation}

Energy Equation:
\begin{equation}
\frac{\partial}{\partial t} \rho E + \frac{\partial }{\partial x_j}\left(\rho
u_j H\right) = \frac{\partial }{\partial x_i}\left(\tau_{ij}u_j\right) - \frac{\partial q_j
}{\partial x_j} + \frac{\partial }{\partial x_i}\left(\rho \sum^{N_s}_{s=1}
h_s D_s \frac{\partial c_s}{\partial x_i} \right)
\end{equation}

%\myred{Might mention the role of the dilluter and why various species-related
%summations have varying limits}.


%
% Species Equations
%
%
\subsection{Species Equations}
Let us begin with the species equations. We start with the convective
term, $\nabla \cdot (\rho_s u)$, and replace $u$ with $\frac{m}{\rho}$
\begin{equation}
  \nabla \cdot (\rho_s u) = u \cdot \nabla \rho_s + \rho_s \cdot \nabla
   u 
\end{equation}
\newline
\newline
The only other term we need consider here is 
the diffusivity. This term, in Gibbs notation, is
\begin{equation}
  \nabla \cdot (\rho_s D_s \nabla c_s)
\end{equation}
Note that $c_s$, the mass fractions, are defined as:
$\frac{\rho_s}{\rho} = 
c_s$.  It can therefore be shown that,
\begin{equation}
  \nabla c_s = \nabla (\frac{\rho_s}{\rho}) = \nabla (\rho_s \rho^{-1})
             = \rho^{-1} \nabla \rho_s - \rho^{-2} \rho_s \nabla \rho
\end{equation}
In other words,
\begin{equation}
  \label{eq:speciesdiffexpansion}
  \nabla \cdot (\rho_s D_s \nabla c_s) = \nabla \cdot \rho D_s (\rho^{-1} \nabla \rho_s - \rho^{-2} \rho_s \nabla \rho)
\end{equation}
We distribute the $\rho^{-1}$ and expand several of the derivatives,
\begin{equation}
  \nabla D_s \cdot \nabla \rho_s + D_s \Delta \rho_s - \nabla \cdot (D_s \rho^{-1} \rho_s \nabla \rho) - D_s \rho^{-1} \rho_s \Delta \rho - \nabla(D_s \rho^{-1} \rho_s) \cdot \nabla \rho
\end{equation}
Thus, the second and third terms ($D_s \Delta \rho_s$ and $D_s \rho^{-1}
\rho_s \Delta \rho$) are components of the linear operator. 




%
% energy equation
%
%

\subsection{Energy Equation}

Let's perform a similar analysis for the Energy Equation:
\begin{equation}
\frac{\partial}{\partial t} \rho E + \frac{\partial }{\partial x_j}\left(\rho
u_j H\right) = \frac{\partial }{\partial x_i}\left(\tau_{ij}u_j\right) -
\frac{\partial }{\partial x_j}q_j + \frac{\partial }{\partial x_i}\left(\rho
\sum^{N_s}_{s=1} h_s D_s \frac{\partial}{\partial x_i} c_s\right)
\end{equation}

We will consider each term individually. Let's start with $\nabla \cdot (\rho u H)$. A simple expansion of this provides:
\begin{equation}
  \nabla \cdot (\rho u H) = H \nabla \cdot (\rho u) + \rho u \nabla H
\end{equation}
Likewise, $\nabla \cdot (\tau u)$ can be expanded as,
\begin{equation}
 \nabla \cdot (\tau u) = u \cdot \nabla \tau + \tau \cdot \nabla u
\end{equation}
where $\nabla \tau$ is defined in section 3.5 of the Suzerain documentation and
$\nabla u$ rewritten as $\nabla \frac{m}{\rho}$.

The heat flux term ($\nabla \cdot q$) is also a simple expansion of the
derivatives, using the assumption of Fourier's Law ($q=-k\nabla T$):
\begin{equation}
  \nabla \cdot q = \nabla k \cdot \nabla T + k \Delta T
\end{equation}
Finally, the more tricky operator is:
\begin{equation}
  \nabla \cdot (\rho \sum_s h_s \nabla c_s)
\end{equation}
Analogously to the expansion of $\nabla \cdot (\rho_s D_s \nabla c_s)$
in equation~\eqref{eq:speciesdiffexpansion},
\begin{align}
  \nabla \cdot (\rho \sum_s h_s \nabla c_s) &= \nabla \cdot (\rho \sum_s h_s D_s ( \rho^{-1} \nabla \rho_s - \rho^{-2} \rho_s \nabla \rho))
\intertext{
As before, we distribute $\rho$,
}
  &= \nabla \cdot (\sum_s h_s D_s \nabla \rho_s - \sum_s h_s D_s \rho^{-1} \rho_s \nabla \rho)
\intertext{
Next, we distribute the divergence operator,
}
  &= \sum_s \nabla h_s D_s \nabla \rho_s + \sum_s h_s \nabla D_s \nabla \rho_s + \sum_s h_s D_s \Delta \rho_s - \nabla \cdot (\sum_s h_s D_s \rho^{-1} \rho_s \nabla \rho)
\end{align}
Our final result is:
\begin{equation}
  = \sum_s \nabla (h_s D_s) \cdot \nabla \rho_s + \sum_s h_s D_s \Delta \rho_s - \sum_s h_s D_s \rho^{-1} \rho_s \Delta \rho - \nabla (\sum_s h_s D_s \rho^{-1} \rho_s) \cdot \nabla \rho
\end{equation}
%We will use the second and third terms in the linear operator. 
We are only interested in linearizing terms on the main diagonal,
e.g. $\frac{\partial A}{\partial t} = \frac{\partial A}{\partial
x_i}$. Thus, for energy only E would be considered, and we hav no second
derivative terms remaining that are a function of E. Therefore, no
additional modifications are required. 


\end{document}
