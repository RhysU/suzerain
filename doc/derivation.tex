\documentclass[letterpaper,11pt,nointlimits]{amsart}

% Packages
\usepackage{amsfonts}
\usepackage{amsmath}
\usepackage{amssymb}
\usepackage{cancel}
\usepackage{enumerate}
\usepackage{fancyhdr}
\usepackage{fullpage}
\usepackage{ifthen}
\usepackage{lastpage}
\usepackage{latexsym}
\usepackage{mathtools}
\usepackage{pstricks}
\usepackage{setspace}
\usepackage{units}
\usepackage{txfonts}
\usepackage{wrapfig}

\mathtoolsset{showonlyrefs,showmanualtags}
\allowdisplaybreaks[1] % Allow grouped equations to be split across pages

% Line Spacing
\singlespacing

% Set appropriate header/footer information on each page
\fancypagestyle{plain}{
    \fancyhf{}
    \renewcommand{\headheight}{2.0em}
    \renewcommand{\headsep}{0.75em}
    \renewcommand{\headrulewidth}{1.0pt}
    \renewcommand{\footrulewidth}{0pt}
    \lhead{
        Suzerain model derivation
    }
    \rhead{
        Page \thepage{} of \pageref{LastPage}
    }
}
\pagestyle{plain}

% Paragraph spacing
\setlength{\parindent}{0em}
\setlength{\parskip}{2.0ex plus 0.75ex minus 0.75ex}

% Document-specific commands
\newcommand{\trans}[1]{{#1}^{\ensuremath{\mathsf{T}}}}
\DeclareMathOperator{\trace}{tr}

\begin{document}

\section{Model derivation}

Here we derive the mathematical model in use.  Special attention is
paid to the origins of all conservation laws and constitutive relations
employed.  The model will nondimensionalized after derivation is complete.

\subsection{Conservation laws}

\subsubsection{Reynolds transport theorem}

Consider a time-varying control volume $\Omega$ with surface
$\partial\Omega$ and unit outward normal $\hat{n}$.  For any 
scalar, vector, or tensor field quantity
$T$, Leibniz' theorem states
\begin{align}
  \label{eq:rtt}
  \frac{d}{dt}\int_{\Omega(t)}T(x,t)\,dV
  &=
  \int_{\Omega}\frac{\partial}{\partial{}t}T\,dV
  +
  \int_{\partial\Omega} \hat{n}\cdot{}w T\,dA
  =
  \int_{\Omega}\frac{\partial}{\partial{}t}T+\nabla\cdot{}wT\,dV
\end{align}
where $w$ is the velocity of $\partial\Omega$.  When $\Omega$ follows
a fixed set of fluid particles, $w$ becomes the fluid velocity $u$.

\subsubsection{Mass continuity} 
Since mass $M=\int_{\Omega} \rho\,dV$
and mass conservation requires $\frac{d}{dt}M=0$,
\begin{align}
  0 = \frac{d}{dt}M 
  = \frac{d}{dt}\int_{\Omega} \rho\,dV
  =
  \int_{\Omega}\frac{\partial}{\partial{}t}\rho+\nabla\cdot{}u\rho{}\,dV.
\end{align}
Because the result must hold for any control volume, we obtain
\begin{align}
  \label{eq:cons_mass}
  \frac{\partial}{\partial{}t}\rho+\nabla\cdot\rho{}u &= 0
  .
\end{align}

\subsubsection{Momentum equation} 
Separating total force into surface forces and an intrinsic body force
\begin{align}
  \sum{}F
  &= 
     \int_{\partial\Omega} f_\text{surface} \, dA 
   + \int_{\Omega} f_\text{body} \, dV 
  = 
     \int_{\partial\Omega} \sigma \hat{n} \, dA 
  +  \int_{\Omega} f_\text{body} \, dV 
  =  \int_{\Omega} \nabla\cdot\sigma + f_\text{body} \, dV 
\end{align}
where $\sigma$ is the Cauchy stress tensor.  Examining
momentum $I=\int_{\Omega} \rho{}u\,dV$ and its conservation
$\frac{d}{dt}I=\sum{}F$,
\begin{align}
    \int_{\Omega}\frac{\partial{}}{\partial{}t}\rho{}u
  + \nabla\cdot(u\otimes{}\rho{}u)\,dV
&= \int_{\Omega} \nabla\cdot\sigma + f_\text{body} \, dV 
.
\end{align}
Because the control volume may be arbitrary, 
\begin{align}
  \frac{\partial{}}{\partial{}t}\rho{}u + \nabla\cdot(u\otimes{}\rho{}u) 
&= \nabla\cdot\sigma + f_\text{body}
.
\end{align}
We further separate pressure $p$ and viscous contributions $\tau$ to
the Cauchy stress tensor so that $\sigma = -p I + \tau$,
\begin{align}
\frac{\partial{}}{\partial{}t}\rho{}u + \nabla\cdot(u\otimes{}\rho{}u) 
&= -\nabla{}p + \nabla\cdot{}\tau + f_\text{body}
.
\end{align}
Lastly, observing that $u\otimes\rho{}u = \frac{1}{\rho}\rho{}u\otimes\rho{}u$
is symmetric,
\begin{align}
\label{eq:cons_momentum}
\frac{\partial{}}{\partial{}t}\rho{}u 
+ \frac{1}{2}\nabla\cdot(u\otimes{}\rho{}u + \rho{}u\otimes{}u) 
&= -\nabla{}p + \nabla\cdot{}\tau + f_\text{body}
.
\end{align}

\subsubsection{Energy equation} 
Lumping internal and kinetic energy into an intrinsic density $e$, 
the energy $E$ is
\begin{align}
  E &= \int_{\Omega} \rho{}e \, dV 
  .
\end{align}
Treating heat input $Q$ as a surface phenomenon described by an outward
heat flux $q$, 
\begin{align}
  Q 
  &= 
  -\int_{\partial\Omega}\hat{n}\cdot{}q\,dA
  =
  -\int_{\Omega}\nabla\cdot{}q\,dV
  .
\end{align}
Power input $P=F\cdot{}v$ accounts for surface stress work and body force work
to give
\begin{align}
  P 
  &= 
    \int_{\partial\Omega} \sigma{}\hat{n} \cdot{} u \, dA 
  + \int_{\Omega} f_\text{body} \cdot{} u \, dV 
  = \int_{\Omega} \nabla\cdot{}\sigma{}u + f_\text{body} \cdot{} u \, dV 
  .
\end{align}
Demanding energy conservation $\frac{d}{dt}E=Q+P$,
\begin{align}
  \int_{\Omega}\frac{\partial}{\partial{}t} \rho{}e
  +
  \nabla\cdot{}u\rho{}e
  \,dV
&=
  - \int_{\Omega}\nabla\cdot{}q\,dV
  + \int_{\Omega} \nabla\cdot\sigma{}u + f_\text{body} \cdot{} u \, dV 
  .
\end{align}
Again, since the control volume was arbitrary, 
\begin{align}
  \frac{\partial}{\partial{}t} \rho{}e
  +
  \nabla\cdot{}\rho{}eu
&=
  - \nabla\cdot{}q + \nabla\cdot\sigma{}u + f_\text{body} \cdot{} u 
  .
\end{align}
After splitting $\sigma$'s pressure and viscous stress contributions we have
\begin{align}
  \label{eq:cons_energy}
  \frac{\partial}{\partial{}t} \rho{}e
  +
  \nabla\cdot{}\rho{}eu
&=
  - \nabla\cdot{}q 
  - \nabla\cdot{}pu
  + \nabla\cdot{}\tau{}u
  + f_\text{body} \cdot{} u 
  .
\end{align}

\subsection{Constitutive relations and other assumptions}

\subsubsection{Perfect gas}

We assume our fluid is a thermally and calorically perfect gas governed by
\begin{align}
  \label{eq:perfectgaseos}
  p &= \rho{} R T
\end{align}
where $R$ is the gas constant. The constant volume $C_{v}$ specific heat,
constant pressure specific heat $C_{p}$, and acoustic velocity $a$ 
relationships follow:
\begin{align}
  \label{eq:perfectgasrelations}
  \gamma &= \frac{C_{p}}{C_{v}}
  &
  C_{v} &= \frac{R}{\gamma - 1}
  &
  C_{p} &= \frac{\gamma{}R}{\gamma-1}
  &
  R &= C_{p} - C_{v}
  &
  a^{2} = \gamma{}RT
\end{align}
We assume $\gamma$ and therefore $C_{v}$ and $C_{p}$ are constant.
The total (internal and kinetic) energy density is
\begin{align}
  \label{eq:perfectgastotalenergy}
  e &= C_{v} T + \frac{u\cdot{}u}{2}
     = \frac{RT}{\gamma-1} + \frac{u\cdot{}u}{2}
  .
\end{align}
See a gas dynamics reference, e.g. Liepmann \& Roshko 1957, for more details.

\subsubsection{Newtonian fluid}

If we seek a constitutive law for the viscous stress tensor $\tau$
using only velocity information, the principle of material frame
indifference implies that uniform translation (given by velocity $u$)
and solid-body rotation (given by the skew-symmetric rotation tensor
$\omega=\frac{1}{2}\left( \nabla{}u-\trans{\nabla{}u} \right)$)
may not influence $\tau$.  Considering contributions only up to the
gradient of velocity, extensional strain (dilatation) and shear strain
effects may depend on only the symmetric rate-of-deformation tensor
$\varepsilon=\frac{1}{2}\left( \nabla{}u+\trans{\nabla{}u}\right)$
and its principal invariants.

Assuming $\tau$ is isotropic and depends linearly upon only $\varepsilon$,
we can express it as
\begin{align}
\tau_{ij} 
  &= c_{ijmn} \varepsilon_{mn}
  \\
  &= \left( A \delta_{ij} \delta_{mn} 
          + B \delta_{im} \delta_{jn} 
          + C \delta_{in} \delta_{jm}
     \right) \varepsilon_{mn}
  &
  &\text{for some }A, B, C\in\mathbb{R}
  \\
  &= A \delta_{ij} \varepsilon_{mm} + B\varepsilon_{ij} + C\varepsilon_{ji}
  \\
  &= A \delta_{ij} \varepsilon_{mm} + \left( B+C \right)\varepsilon_{ji}
  \\
  &= 2 \mu \varepsilon_{ij} + \lambda\delta_{ij}\nabla\cdot{}u
\end{align}
where $\mu=\frac{1}{2}\left( B + C \right)$ is the viscosity
and $\lambda=A$ is the second viscosity.
Reverting to direct notation we have
\begin{align}
\tau 
  &= 2 \mu \varepsilon + \lambda \left( \nabla\cdot{}u \right) I
  \\
\label{eq:taunewt}
&=   \mu \left( \nabla{}u + \trans{\nabla{}u} \right) 
   + \lambda \left( \nabla\cdot{}u \right) I
\end{align}

\subsubsection{Stokes hypothesis}

We generally assume the second viscosity $\lambda=-\frac{2}{3}\mu$.
However, because we anticipate separately maintaining $\lambda$ being
useful, we will not combine $\mu$ and $\lambda$ terms in the model.

\subsubsection{Power law viscosity}

We assume that viscosity varies only with temperature according to
\begin{align}
  \label{eq:powerlawviscosity}
  \frac{\mu}{\mu_{0}}=\left(\frac{T}{T_{0}}\right)^{\beta}
\end{align}
where $\mu_{0}$ and $T_{0}$ are suitable reference values.  This
relationship models air well for temperatures up to several thousand
degrees K.  See Svehla's 1962 NASA technical report R-132.

\subsubsection{Fourier's equation}

We neglect the transport of energy by molecular diffusion, chemical
reactions, and radiative heat transfer.  We seek a relation between
the heat flux $q$ and the temperature $T$.  The principle of 
frame indifference implies we may only use the temperature gradient
so that
\begin{align}
  \label{eq:fouriertensorlaw}
  q &= \underline{\kappa} \cdot \nabla{} T
\end{align}
where $\underline{\kappa}$ is a thermal conductivity tensor.  
Consistent with our assumption that $\tau$ is isotropic, we assume
$\underline{\kappa}$ is isotropic to obtain 
\begin{align}
  \label{eq:fourierlaw}
  q &= - \kappa \nabla{} T
\end{align}
where $\kappa$ is the scalar thermal conductivity.  We introduce the
negative sign so that heat flows from hot to cold when $\kappa>0$.

\subsubsection{Constant Prandtl number}

We assume the Prandtl number $Pr = \frac{\mu{}C_{p}}{\kappa}$ is constant.
Because $C_{p}$ is constant the ratio $\frac{\mu}{\kappa}$ must be
constant.  The viscosity and thermal conductivity must either grow at
identical rates or they must grow according to an inverse relationship.
The latter is not observed in practice for our class of fluids, and
so we assume
\begin{align}
  \frac{\mu}{\mu_{0}} = \frac{\kappa}{\kappa_{0}}
  .
  \label{eq:mukappa}
\end{align}

\subsubsection{Body force}

We retain the body force term $f_{\text{body}}$ and allow it to vary
in all spatial directions and across time.  Retaining body force will
simplify using the method of manufactured solutions for implementation
verification.  We drop the subscript and subsequently denote this quantity
as $f$.

\subsubsection{Dimensional equations}

By combining the conservation laws with our constitutive relations 
and assumptions, we arrive at the dimensional model
\begin{subequations}\label{eq:dimensionalmodel}
\begin{align}
  \label{eq:dim_continuity}
  \frac{\partial}{\partial{}t}\rho
&= 
  - \nabla\cdot\rho{}u 
  \\
  \label{eq:dim_momentum}
  \frac{\partial{}}{\partial{}t}\rho{}u 
&= 
  - \frac{1}{2}\nabla\cdot(u\otimes{}\rho{}u + \rho{}u\otimes{}u) 
  -\nabla{} p
  + \nabla\cdot{} \tau
  + f
  \\
  \label{eq:dim_energy}
  \frac{\partial}{\partial{}t} \rho{}e
&=
  - \nabla\cdot{}\rho{}eu
  + \nabla\cdot{} \frac{\kappa_{0}}{\mu_{0}} \mu \nabla{} T
  - \nabla\cdot{} p u
  + \nabla\cdot{}\tau{} u
  + f \cdot{} u 
\intertext{where terms in the right hand side make use of}
  \label{eq:dim_temperature}
  T &= \left( \frac{\gamma-1}{R} \right)\left( e - \frac{u\cdot{}u}{2} \right)
  \\
  \label{eq:dim_pressure}
  p &= \rho{} R T
  \\
  \label{eq:dim_viscosity}
  \mu &= \mu_{0} \left( \frac{T}{T_{0}} \right)^{\beta} 
  \\
  \label{eq:dim_viscousstress}
  \tau &=   \mu \left( \nabla{}u + \trans{\nabla{}u} \right) 
          + \lambda \left( \nabla\cdot{}u \right) I
  .
\end{align}
\end{subequations}

\subsection{Nondimensionalization}

\subsubsection{Reference quantity introduction}

We rewrite the dimensional equations using nondimensional variables
combined with arbitrary reference quantities.  For each dimensional
quantity in the dimensional model we introduce a nondimensional variable
or operator denoted by a superscript star, e.g. $\nabla^{*}$.

We introduce $t^{*}=\frac{t}{t_{0}}$ and $x^{*}=\frac{x}{x_{0}}$ for some
reference $t_{0}$ and $x_{0}$.  This induces the following relationships:
\begin{align}
  \frac{\partial{}}{\partial{}t} 
  &= 
  \frac{\partial{}}{\partial{}t^{*}} 
  \frac{\partial{}t^{*}}{\partial{}t} 
  =
  \frac{1}{t_{0}}\frac{\partial}{\partial{}t^{*}}
  &
  \frac{\partial{}}{\partial{}x} 
  &= 
  \frac{\partial{}}{\partial{}x^{*}} 
  \frac{\partial{}x^{*}}{\partial{}x} 
  =
  \frac{1}{x_{0}}\frac{\partial}{\partial{}x^{*}}
  &
  \nabla
  &=
  \hat{e}_{i} \frac{\partial{}}{\partial{}x_{i}} 
  =
  \hat{e}_{i} \frac{1}{x_{0}} \frac{\partial}{\partial{}x^{*}_{i}}
  =
  \frac{1}{x_{0}} \nabla^{*}
  \label{eq:nondim_derivops}
\end{align}

We introducing other nondimensional quantities (e.g. $\rho^{*} =
\frac{\rho}{\rho_{0}}$) and use them to reexpress the model
\begin{subequations}\label{eq:dimwithref_model}
\begin{align}
  \label{eq:dimwithref_continuity}
  \frac{\rho_{0}}{t_{0}} \frac{\partial}{\partial{}t^{*}}\rho^{*}
&= 
- \frac{\rho_{0}u_{0}}{x_{0}} \nabla^{*}\cdot\rho^{*}u^{*}
  \\
  \label{eq:dimwithref_momentum}
  \frac{\rho_{0}u_{0}}{t_{0}} \frac{\partial{}}{\partial{}t^{*}}\rho^{*}u^{*}
&= 
  - \frac{1}{2}
    \frac{\rho_{0}u_{0}^{2}}{x_{0}}
    \nabla^{*}\cdot(u^{*}\otimes{}\rho^{*}u^{*} + \rho^{*}u^{*}\otimes{}u^{*}) 
  - \frac{p_{0}}{x_{0}} \nabla^{*} p^{*}
  + \frac{\tau_{0}}{x_{0}} \nabla^{*}\cdot{} \tau^{*}
  + f_{0} f^{*}
  \\
  \label{eq:dimwithref_energy}
  \frac{\rho_{0}e_{0}}{t_{0}} \frac{\partial}{\partial{}t^{*}} \rho^{*}e^{*}
&=
  - \frac{\rho_{0}e_{0}u_{0}}{x_{0}} \nabla^{*} \cdot{}\rho^{*}e^{*}u^{*}
  + \frac{\kappa_{0}T_{0}}{x_{0}^{2}} 
    \nabla^{*}\cdot{} \mu^{*} \nabla^{*} T^{*}
  - \frac{p_{0}u_{0}}{x_{0}} \nabla^{*}\cdot{} p^{*} u^{*}
  + \frac{\tau_{0}u_{0}}{x_{0}} \nabla^{*}\cdot{}\tau^{*} u^{*}
  + f_{0}u_{0} f \cdot{} u 
\intertext{where terms in the right hand side are determined by}
  \label{eq:dimwithref_temperature}
  T^{*} &= \frac{1}{T_{0}} \left( \frac{\gamma-1}{R} \right)
           \left( e_{0} e^{*} - u_{0}^{2} \frac{u^{*}\cdot{}u^{*}}{2} \right)
  \\
  \label{eq:dimwithref_pressure}
  p^{*} &= \frac{\rho_{0}RT_{0}}{p_{0}} \rho^{*} T^{*}
  \\
  \label{eq:dimwithref_viscosity}
  \mu^{*} &= \left( T^{*} \right)^{\beta} 
  \\
  \label{eq:dimwithref_viscousstress}
\tau^{*} &= \frac{\mu_{0}u_{0}}{x_{0} \tau_{0}} \left[ 
      \mu^{*} \left( \nabla^{}u^{*} + \trans{\nabla^{*}u^{*}} \right) 
      + \lambda^{*} \left( \nabla^{*}\cdot{}u^{*} \right) I
    \right]
  .
\end{align}
\end{subequations}








\subsubsection{Reference quantity choices}

We choose a reference length $l_{0}$, temperature $T_{0}$,
and density $\rho_{0}$.  These choices fix other dimensional
reference quantities:
\begin{align*}
  a_{0} &= \sqrt{\gamma{}RT_{0}}
  \\
  t_{0} &= \frac{l_{0}}{a_{0}}
  \\
  \mu_{0} &= \rho_{0} a_{0} l_{0}
  \\
  \kappa_{0} &= \frac{\rho_{0} a_{0}^{3} l_{0}}{T_{0}}
\end{align*}

Using the above constitutive relations, these two choices fix our
nondimensional, starred quantities according to:
\begin{align}
  T &= T_{*} T_{0}
  \\
  \rho &= \rho_{*} \rho_{0}
  \\
  p &= p_{*} p_{0} = p_{*} \rho_{0} R T_{0}
  \\
  u &= u_{*} a_{0} = u_{*} \sqrt{\gamma{}RT_{0}}
  \label{eq:refquantities}
\end{align}

\end{document}
