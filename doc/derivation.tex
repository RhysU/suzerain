\documentclass[letterpaper,11pt,nointlimits]{amsart}

% Packages
\usepackage{amsfonts}
\usepackage{amsmath}
\usepackage{amssymb}
\usepackage{cancel}
\usepackage{enumerate}
\usepackage{fancyhdr}
\usepackage{fullpage}
\usepackage{ifthen}
\usepackage{lastpage}
\usepackage{latexsym}
\usepackage{mathtools}
\usepackage{pstricks}
\usepackage{setspace}
\usepackage{txfonts}
\usepackage{wrapfig}

\mathtoolsset{showonlyrefs,showmanualtags}
\allowdisplaybreaks[1]

% Line Spacing
\singlespacing

% Set appropriate header/footer information on each page
\fancypagestyle{plain}{
    \fancyhf{}
    \renewcommand{\headheight}{2.0em}
    \renewcommand{\headsep}{0.75em}
    \renewcommand{\headrulewidth}{1.0pt}
    \renewcommand{\footrulewidth}{0pt}
    \lhead{
        Suzerain model derivation
    }
    \rhead{
        Page \thepage{} of \pageref{LastPage}
    }
}
\pagestyle{plain}

% Paragraph spacing
\setlength{\parindent}{0em}
\setlength{\parskip}{2.0ex plus 0.75ex minus 0.75ex}

% Document-specific commands
\newcommand{\trans}[1]{{#1}^{\ensuremath{\mathsf{T}}}}
\DeclareMathOperator{\trace}{tr}

\begin{document}

\section{Model derivation}

Here we derive the mathematical model in use.  Special attention is
paid to the origins of all conservation laws and constitutive relations
employed.  The model will nondimensionalized after derivation is complete.

\subsection{Conservation equations}

\subsubsection{Reynolds transport theorem}

Consider a time-varying control volume $\Omega$ with surface
$\partial\Omega$ and unit outward normal $\hat{n}$.  For any 
scalar, vector, or tensor field quantity
$T$, Leibniz' theorem states
\begin{align}
  \label{eq:rtt}
  \frac{d}{dt}\int_{\Omega(t)}T(x,t)\,dV
  &=
  \int_{\Omega}\frac{\partial}{\partial{}t}T\,dV
  +
  \int_{\partial\Omega} \hat{n}\cdot{}w T\,dA
  =
  \int_{\Omega}\frac{\partial}{\partial{}t}T+\nabla\cdot{}wT\,dV
\end{align}
where $w$ is the velocity of $\partial\Omega$.  When $\Omega$ follows
a fixed set of fluid particles, $w$ becomes the fluid velocity $u$.

\subsubsection{Mass continuity} 
Since mass $M=\int_{\Omega} \rho\,dV$
and mass conservation requires $\frac{d}{dt}M=0$,
\begin{align}
  0 = \frac{d}{dt}M 
  = \frac{d}{dt}\int_{\Omega} \rho\,dV
  =
  \int_{\Omega}\frac{\partial}{\partial{}t}\rho+\nabla\cdot{}u\rho{}\,dV.
\end{align}
Because the result must hold for any control volume, we obtain
\begin{align}
  \label{eq:cons_mass}
  \frac{\partial}{\partial{}t}\rho+\nabla\cdot\rho{}u &= 0
  .
\end{align}

\subsubsection{Momentum equation} 
Separating total force into surface forces and an intrinsic body force
\begin{align}
  \sum{}F
  &= 
     \int_{\partial\Omega} f_\text{surface} \, dA 
   + \int_{\Omega} f_\text{body} \, dV 
  = 
     \int_{\partial\Omega} \sigma \hat{n} \, dA 
  +  \int_{\Omega} f_\text{body} \, dV 
  =  \int_{\Omega} \nabla\cdot\sigma + f_\text{body} \, dV 
\end{align}
where $\sigma$ is the Cauchy stress tensor.  Examining
momentum $I=\int_{\Omega} \rho{}u\,dV$ and its conservation
$\frac{d}{dt}I=\sum{}F$,
\begin{align}
    \int_{\Omega}\frac{\partial{}}{\partial{}t}\rho{}u
  + \nabla\cdot(u\otimes{}\rho{}u)\,dV
&= \int_{\Omega} \nabla\cdot\sigma + f_\text{body} \, dV 
.
\end{align}
Because the control volume may be arbitrary, 
\begin{align}
  \frac{\partial{}}{\partial{}t}\rho{}u + \nabla\cdot(u\otimes{}\rho{}u) 
&= \nabla\cdot\sigma + f_\text{body}
.
\end{align}
We further separate pressure $p$ and viscous contributions $\tau$ to
the Cauchy stress tensor so that $\sigma = -p I + \tau$,
\begin{align}
\label{eq:cons_momentum}
\frac{\partial{}}{\partial{}t}\rho{}u + \nabla\cdot(u\otimes{}\rho{}u) 
&= -\nabla{}p + \nabla\cdot{}\tau + f_\text{body}
.
\end{align}

\subsubsection{Energy equation} 
Lumping internal and kinetic energy into an intrinsic density $e$, 
the energy $E$ is
\begin{align}
  E &= \int_{\Omega} \rho{}e \, dV 
  .
\end{align}
Treating heat input $Q$ as a surface phenomenon described by an outward
heat flux $q$, 
\begin{align}
  Q 
  &= 
  -\int_{\partial\Omega}\hat{n}\cdot{}q\,dA
  =
  -\int_{\Omega}\nabla\cdot{}q\,dV
  .
\end{align}
Power input $P=F\cdot{}v$ accounts for surface stress work and body force work
to give
\begin{align}
  P 
  &= 
    \int_{\partial\Omega} \sigma{}\hat{n} \cdot{} u \, dA 
  + \int_{\Omega} f_\text{body} \cdot{} u \, dV 
  = \int_{\Omega} \nabla\cdot{}\sigma{}u + f_\text{body} \cdot{} u \, dV 
  .
\end{align}
Demanding energy conservation $\frac{d}{dt}E=Q+P$,
\begin{align}
  \int_{\Omega}\frac{\partial}{\partial{}t} \rho{}e
  +
  \nabla\cdot{}u\rho{}e
  \,dV
&=
  - \int_{\Omega}\nabla\cdot{}q\,dV
  + \int_{\Omega} \nabla\cdot\sigma{}u + f_\text{body} \cdot{} u \, dV 
  .
\end{align}
Again, since the control volume was arbitrary, 
\begin{align}
  \frac{\partial}{\partial{}t} \rho{}e
  +
  \nabla\cdot{}\rho{}eu
&=
  - \nabla\cdot{}q + \nabla\cdot\sigma{}u + f_\text{body} \cdot{} u 
  .
\end{align}
After splitting $\sigma$'s pressure and viscous stress contributions we have
\begin{align}
  \label{eq:cons_energy}
  \frac{\partial}{\partial{}t} \rho{}e
  +
  \nabla\cdot{}\rho{}eu
&=
  - \nabla\cdot{}q 
  - \nabla\cdot{}pu
  + \nabla\cdot{}\tau{}u
  + f_\text{body} \cdot{} u 
  .
\end{align}

\subsection{Constitutive relations and other assumptions}

\subsubsection{Perfect gas}

We assume our fluid is a thermally and calorically perfect gas governed by
\begin{align}
  \label{eq:perfectgaseos}
  p &= \rho{} R T
\end{align}
where $R$ is the gas constant. The constant volume $c_{v}$ specific heat,
constant pressure specific heat $c_{p}$, and speed of sound $a$ relationships 
follow:
\begin{align}
  \label{eq:perfectgasrelations}
  \gamma &= \frac{c_{p}}{c_{v}}
  &
  c_{v} &= \frac{R}{\gamma - 1}
  &
  c_{p} &= \frac{\gamma{}R}{\gamma-1}
  &
  R &= c_{p} - c_{v}
  &
  a^{2} = \gamma{}RT
\end{align}
We assume $\gamma$ and therefore $c_{v}$ and $c_{p}$ are constant.
The total (internal and kinetic) energy density is given by
\begin{align}
  \label{eq:perfectgastotalenergy}
  e &= c_{v} T + \frac{u\cdot{}u}{2}
     = \frac{RT}{\gamma-1} + \frac{u\cdot{}u}{2}
  .
\end{align}
See a gas dynamics reference, e.g. Liepmann \& Roshko 1957, for more details.

\subsubsection{Newtonian fluid}

If we seek a constitutive law for the viscous stress tensor $\tau$
using only velocity information, the principle of material frame
indifference implies that uniform translation (given by velocity $u$)
and solid-body rotation (given by the skew-symmetric rotation tensor
$\omega=\frac{1}{2}\left( \nabla{}u-\trans{\nabla{}u} \right)$)
may not influence $\tau$.  Considering contributions only up to the
gradient of velocity, extensional strain (dilatation) and shear strain
effects may depend on only the symmetric rate-of-deformation tensor
$\varepsilon=\frac{1}{2}\left( \nabla{}u+\trans{\nabla{}u}\right)$
and its principal invariants.

Assuming $\tau$ is isotropic and depends linearly upon only $\varepsilon$,
we can express it as
\begin{align}
\tau_{ij} 
  &= c_{ijmn} \varepsilon_{mn}
  \\
  &= \left( A \delta_{ij} \delta_{mn} 
          + B \delta_{im} \delta_{jn} 
          + C \delta_{in} \delta_{jm}
     \right) \varepsilon_{mn}
  &
  &\text{for some }A, B, C\in\mathbb{R}
  \\
  &= A \delta_{ij} \varepsilon_{mm} + B\varepsilon_{ij} + C\varepsilon_{ji}
  \\
  &= A \delta_{ij} \varepsilon_{mm} + \left( B+C \right)\varepsilon_{ji}
  \\
  &= 2 \mu \varepsilon_{ij} + \lambda\delta_{ij}\nabla\cdot{}u
\end{align}
where $\mu=\frac{1}{2}\left( B + C \right)$ is the viscosity
and $\lambda=A$ is the second viscosity.
Reverting to direct notation we have
\begin{align}
\tau 
  &= 2 \mu \varepsilon + \lambda \left( \nabla\cdot{}u \right) I
  \\
\label{eq:taunewt}
&=   \mu \left( \nabla{}u + \trans{\nabla{}u} \right) 
   + \lambda \left( \nabla\cdot{}u \right) I
\end{align}

\subsubsection{Stokes hypothesis}

We assume $\lambda=-\frac{2}{3}\mu$ so that
\begin{align}
\tau &= \mu \left( 
    \nabla{}u + \trans{\nabla{}u} - \frac{2}{3} (\nabla\cdot{}u) I
  \right) 
\end{align}

\subsubsection{Power law viscosity}

We assume that viscosity varies only with temperature according to
\begin{align}
  \label{eq:powerlawviscosity}
  \frac{\mu}{\mu_{0}}=\left(\frac{T}{T_{0}}\right)^{\beta}
\end{align}
where $\mu_{0}$ and $T_{0}$ are suitable reference values.  This
relationship models air well for temperatures up to several thousand
degrees K.  See Svehla's 1962 NASA technical report R-132.

\subsubsection{Fourier's equation}

We neglect the transport of energy by molecular diffusion, chemical
reactions, and radiative heat transfer.  We seek a relation between
the heat flux $q$ and the temperature $T$.  The principle of 
frame indifference implies we may only use the temperature gradient
so that
\begin{align}
  \label{eq:fouriertensorlaw}
  q &= \underline{\kappa} \cdot \nabla{} T
\end{align}
where $\underline{\kappa}$ is a thermal conductivity tensor.  
Consistent with our assumption that $\tau$ is isotropic, we assume
$\underline{\kappa}$ is isotropic to obtain 
\begin{align}
  \label{eq:fourierlaw}
  q &= - \kappa \nabla{} T
\end{align}
where $\kappa$ is the scalar thermal conductivity.  We introduce the
negative sign so that heat flows from hot to cold when $\kappa>0$.

\subsubsection{Constant Prandtl number}

We assume the Prandtl number $Pr = \frac{\mu{}c_{p}}{\kappa}$ is constant.
This implies $\kappa=\mu$ when combined with our perfect gas assumptions.

\subsubsection{No body force}

We set $f_{\text{body}} = 0$.

\end{document}
