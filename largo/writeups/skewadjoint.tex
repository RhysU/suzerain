\documentclass[letterpaper,reqno,11pt]{amsart}

% Packages
\usepackage{accents}
\usepackage{amsfonts}
\usepackage{amsmath}
\usepackage{amssymb}
\usepackage{enumerate}
\usepackage{fancyhdr}
\usepackage{fullpage}
\usepackage{lastpage}
\usepackage{latexsym}
\usepackage{mathtools}
\usepackage{pstricks}
\usepackage{setspace}
\usepackage{txfonts}

\mathtoolsset{showmanualtags,showonlyrefs}
\allowdisplaybreaks[1]

% Line Spacing
\singlespacing

% Set appropriate header/footer information on each page
\fancypagestyle{plain}{
    \fancyhf{}
    \renewcommand{\headheight}{2.0em}
    \renewcommand{\headsep}{0.75em}
    \renewcommand{\headrulewidth}{1.0pt}
    \renewcommand{\footrulewidth}{0pt}
    \lhead{
      Skew-adjoint form of the convective derivative operator
    }
    \rhead{
        Page \thepage{} of \pageref{LastPage}
    }
}
\pagestyle{plain}

% Paragraph spacing
\setlength{\parindent}{0em}
\setlength{\parskip}{2.0ex plus 0.75ex minus 0.75ex}

% Document-specific commands
\newcommand{\vecnabla}{\ensuremath{\vec{\nabla}}}

\begin{document}

For fixed, real-valued $\vec{u}$, consider the convective linear operator
\begin{align}
  L\vec{f} &\coloneqq
      \frac{1}{2}\vec{u}\cdot\vecnabla{}\vec{f}
    + \frac{1}{2}\vecnabla\cdot\vec{u}\otimes\vec{f}.
\end{align}
With appropriate assumptions about smoothness, the domain, and the boundary
conditions, $L$ is skew-adjoint in the $L^2\left(\Omega\right)$ inner product,
that is $ \int_{\Omega} L\vec{f}\cdot\vec{g}^{\ast}\,d\!V = -\int_{\Omega}
\vec{f}\cdot{}\left(L\vec{g}\right)^{\ast} \,d\!V$ where
$\left(\cdot\right)^\ast$ denotes complex conjugation.  Skew-adjointness
implies that all the operator's eigenvalues are imaginary.  In conjunction with
a first-order time derivative, a skew-adjoint operator propagates a signal
without growth or decay.

Showing that $L$ is skew-adjoint requires two integration by parts formulas.
For simplicity, we develop both using index notation in a Cartesian frame. They
do hold in generalized coordinates.  The first result is
\begin{align}
  \int_{\Omega}
    \left(\vec{u}\cdot\vecnabla\vec{f}\right)\cdot{}\vec{g}^{\ast}
    \,d\!V
  &=
  \int_{\Omega} g_i^{\ast} u_j \partial\!x_{i} f_j \,d\!V
  \\
  &=
    \int_{\Omega} \partial\!x_{i} g_i^{\ast} u_j f_j \,d\!V
  - \int_{\Omega} f_j \partial\!x_{i} g_i^{\ast} u_j \,d\!V
  \\
  &=
    \int_{\Omega}
      \vecnabla\cdot\left(\vec{u}\cdot\vec{f}\right)\vec{g}^{\ast}
      \,d\!V
  - \int_{\Omega}
      \vec{f}\cdot\left(\vecnabla\cdot\vec{u}\otimes\vec{g}^{\ast}\right)
      \,d\!V
  \\
  &=
    \int_{\partial\Omega}
      \left(\vec{u}\cdot\vec{f}\right)\vec{g}^{\ast}\cdot\hat{n}
      \,d\!S
  - \int_{\Omega}
      \vec{f}\cdot\left(\vecnabla\cdot\vec{u}\otimes\vec{g}^{\ast}\right)
      \,d\!V
  \\
  &=
    \int_{\partial\Omega}
      \left(\vec{g}^{\ast}\otimes\vec{f}\right)\vec{u}\cdot\hat{n}
      \,d\!S
  - \int_{\Omega}
      \vec{f}\cdot\left(\vecnabla\cdot\vec{u}\otimes\vec{g}^{\ast}\right)
      \,d\!V
  .
\end{align}
The second result is
\begin{align}
  \int_{\Omega}
    \left(\vecnabla\cdot\vec{u}\otimes\vec{f}\right)\cdot\vec{g}^{\ast}
    \,d\!V
  &=
  \int_{\Omega} g_i^{\ast} \partial\!x_{j} u_i f_j \,d\!V
  \\
  &=
    \int_{\Omega} \partial\!x_{j} g_i^{\ast} u_i f_j \,d\!V
  - \int_{\Omega} u_i f_j \partial\!x_{j} g_i^{\ast} \,d\!V
  \\
  &=
    \int_{\Omega}
      \vecnabla\cdot\left(\vec{u}\cdot\vec{g}^{\ast}\right)\vec{f}
      \,d\!V
  - \int_{\Omega}
      \vec{f}\cdot\left(\vec{u}\cdot\vecnabla\vec{g}^{\ast}\right)
      \,d\!V
  \\
  &=
    \int_{\partial\Omega}
      \left(\vec{u}\cdot\vec{g}^{\ast}\right)\vec{f}\cdot\hat{n}
      \,d\!S
  - \int_{\Omega}
      \vec{f}\cdot\left(\vec{u}\cdot\vecnabla\vec{g}^{\ast}\right)
      \,d\!V
  \\
  &=
    \int_{\partial\Omega}
      \left(\vec{f}\otimes\vec{g}^{\ast}\right)\vec{u}\cdot\hat{n}
      \,d\!S
  - \int_{\Omega}
      \vec{f}\cdot\left(\vec{u}\cdot\vecnabla\vec{g}^{\ast}\right)
      \,d\!V
  .
\end{align}
Lastly, we depend on $\vec{u}$ being real-valued so that
\begin{align}
  \left(L\vec{g}\right)^{\ast}
  &=
      \frac{1}{2}\vec{u}^{\ast}\cdot\vecnabla{}\vec{g}^{\ast}
    + \frac{1}{2}\vecnabla\cdot\vec{u}^{\ast}\otimes\vec{g}^{\ast}
  =
      \frac{1}{2}\vec{u}\cdot\vecnabla{}\vec{g}^{\ast}
    + \frac{1}{2}\vecnabla\cdot\vec{u}\otimes\vec{g}^{\ast}
  = L\vec{g}^{\ast}
  .
\end{align}

By employing the above three results and combining the boundary terms, we see
\begin{align}
  \int_{\Omega} L\vec{f}\cdot\vec{g}^{\ast}\,d\!V
  &=
  - \int_{\Omega} \vec{f}\cdot{}\left(L\vec{g}\right)^{\ast} \,d\!V
  + \int_{\partial\Omega} \frac{1}{2}\left[
        \left(\vec{f}\otimes\vec{g}^{\ast}\right)
      + \left(\vec{f}\otimes\vec{g}^{\ast}\right)^{\mathrm{T}}
    \right] \vec{u}\cdot\hat{n} \,d\!S
  .
\end{align}
Beyond the smoothness requirements necessary to apply integration by parts, the
boundary integral must also vanish for the operator to be self-adjoint.

\nocite{Zang1991Rotation}
\nocite{Boyd2001Page213}
\bibliographystyle{amsplain}
\bibliography{references}

\end{document}
