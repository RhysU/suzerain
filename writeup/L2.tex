\documentclass[letterpaper,11pt,nointlimits,reqno]{amsart}

% Packages
\usepackage{accents}
\usepackage{algorithm}
\usepackage{algorithmic}
\usepackage{amsfonts}
\usepackage{amsmath}
\usepackage{amssymb}
\usepackage{cancel}
\usepackage{enumerate}
\usepackage{fancyhdr}
\usepackage{fullpage}
\usepackage{ifthen}
\usepackage{lastpage}
\usepackage{latexsym}
\usepackage{mathrsfs}
\usepackage{mathtools}
\usepackage{pstricks}
\usepackage{setspace}
\usepackage{txfonts}
\usepackage{units}
\usepackage{varioref}
\usepackage{wrapfig}

\mathtoolsset{showonlyrefs,showmanualtags}
\allowdisplaybreaks[1] % Allow grouped equations to be split across pages

% Line Spacing
\singlespacing

% Set appropriate header/footer information on each page
\fancypagestyle{plain}{
    \fancyhf{}
    \renewcommand{\headheight}{2.0em}
    \renewcommand{\headsep}{0.75em}
    \renewcommand{\headrulewidth}{1.0pt}
    \renewcommand{\footrulewidth}{0pt}
    \lhead{
        $L_2$ norm within Suzerain's discretization
    }
    \rhead{
        Page \thepage{} of \pageref{LastPage}
    }
}
\pagestyle{plain}

% Paragraph spacing
\setlength{\parindent}{0em}
\setlength{\parskip}{2.0ex plus 0.75ex minus 0.75ex}

% Document-specific commands
\newcommand{\ii}{\ensuremath{\mathrm{i}}}

\begin{document}

We wish to compute the $L_2$ norm of an instantaneous, real-valued field
$u\!\left(x,y,z\right)$ on the spatial domain
$\left[-\frac{L_x}{2},\frac{L_x}{2}\right] \times{} [0,L_y] \times{}
\left[-\frac{L_z}{2},\frac{L_z}{2}\right]$ which has been discretized as
\begin{align}
u^h(x,y,z)
&=
  \sum_{l=0}^{N_y - 1}
  \sum_{m=-\frac{N_x}{2}}^{\frac{N_x}{2}-1}
  \sum_{n=-\frac{N_z}{2}}^{\frac{N_z}{2}-1}
  \hat{u}_{l,m,n}
  B_l\!\left(y\right)
  e^{\ii\frac{2\pi{}m}{L_x}x}
  e^{\ii\frac{2\pi{}n}{L_z}z}
=
  \sum_{l}\sum_{m}\sum_{n}
  \hat{u}_{l,m,n}B_l\!\left(y\right)e^{\ii k_m x}e^{\ii k_n z}
\end{align}
where $k_m = 2\pi{}m/L_x$, $k_n = 2\pi{}n/L_z$, and $B_l\!\left(y\right)$ are a
B-spline basis for some order and knot selection.

With appropriate smoothness assumptions, by direct computation we find
\begin{align}
  \left|\left|u\left(x,y,z\right)\right|\right|_{2}^{2}
&=
  \int_{-\frac{L_x}{2}}^{\frac{L_x}{2}}
  \int_0^{L_y}
  \int_{-\frac{L_z}{2}}^{\frac{L_z}{2}}
  u^{2}\!\left(x,y,z\right)
  \,d\!z \,d\!y \,d\!x
\\ &=
  \int_0^{L_y}
  \int_{-\frac{L_x}{2}}^{\frac{L_x}{2}}
  \int_{-\frac{L_z}{2}}^{\frac{L_z}{2}}
  u\!\left(x,y,z\right)
  u^{\ast}\!\left(x,y,z\right)
  \,d\!z \,d\!x \,d\!y
\\ &=
  \int_0^{L_y}
  \int_{-\frac{L_x}{2}}^{\frac{L_x}{2}}
  \int_{-\frac{L_z}{2}}^{\frac{L_z}{2}}
  \left(
    \sum_{l}\sum_{m}\sum_{n}
    \hat{u}_{l,m,n}B_l\!\left(y\right)e^{\ii k_m x}e^{\ii k_n z}
  \right)
  \left(
    \sum_{l\prime}\sum_{m\prime}\sum_{n\prime}
    \hat{u}_{l\prime,m\prime,n\prime}
    B_{l\prime}\!\left(y\right)e^{-\ii k_{m\prime} x}e^{-\ii k_{n\prime} z}
  \right)
  \,d\!z \,d\!x \,d\!y
\\ &=
  \sum_{l}
  \sum_{l\prime}
  \int_0^{L_y}
  \left(
    B_{l\prime}\!\left(y\right)
    B_l\!\left(y\right)
    \sum_{m}
    \sum_{m\prime}
    \int_{-\frac{L_x}{2}}^{\frac{L_x}{2}}
    \left(
      e^{\ii k_m x}
      e^{-\ii k_{m\prime} x}
      \sum_{n}
      \sum_{n\prime}
      \int_{-\frac{L_z}{2}}^{\frac{L_z}{2}}
      \left(
        e^{\ii k_n z}
        e^{-\ii k_{n\prime} z}
        \hat{u}_{l,m,n}
        \hat{u}_{l\prime,m\prime,n\prime}
      \right)
      \,d\!z
    \right)
    \,d\!x
  \right)
  \,d\!y
\\ &=
  \sum_{l}
  \sum_{l\prime}
  \int_0^{L_y}
  \left(
    B_{l\prime}\!\left(y\right)
    B_l\!\left(y\right)
    \sum_{m}
    \sum_{m\prime}
    \int_{-\frac{L_x}{2}}^{\frac{L_x}{2}}
    \left(
      e^{\ii k_m x}
      e^{-\ii k_{m\prime} x}
      L_z \sum_{n} \hat{u}_{l,m,n} \hat{u}_{l\prime,m\prime,n}
    \right)
    \,d\!x
  \right)
  \,d\!y
\\ &=
  \sum_{l}
  \sum_{l\prime}
  \int_0^{L_y}
  \left(
    B_{l\prime}\!\left(y\right)
    B_l\!\left(y\right)
    L_x L_z \sum_{m} \sum_{n} \hat{u}_{l,m,n} \hat{u}_{l\prime,m,n}
  \right)
  \,d\!y
\\ &=
   L_x L_z \sum_{m} \sum_{n}
  \sum_{l}
  \hat{u}_{l,m,n}
  \sum_{l\prime}
  \hat{u}_{l\prime,m,n}
  \int_0^{L_y}
  B_{l\prime}\!\left(y\right)
  B_l\!\left(y\right)
  \,d\!y
\end{align}

\end{document}
